\chapter{Recommender Systems}
\section{\cite{dinerstein2018consumer}: Consumer Price Search and Platform Design in Internet Commerce}
There is a redesign of eBay:
\begin{itemize}
    \item Before the redesign: ranking according to a relevance algorithm.
    \item After the redesign: first prompting consumers to identify an exact product, then comparing seller listings of that product  head-to-head, ranked (mostly) by price.
\end{itemize}

\section{Conceptual Framework}
Trade-off between guiding consumers to their most desired products and strengthening seller  incentives to provide better product attributes.

There are $J$ sellers. Each product $j$ has a fixed vector of attributes $x_j$ and a posted price $p_j$ (determined by the seller). Each consumer $i$ arrives at the platform is defined by a vector of characteristics $\xi_i$ ($\sim F$). Each consumer has a unit demand and decides which product to purchase, or not to purchase at all. Consumer $i$'s utility from product $j$ is given by $u(x_j,p_j;\xi_i)$.

The platform sets an awareness/visibility function $a_{ij}\in[0,1]$, where $a_{ij}$ is the probability that product $j$ is being considered by consumer $i$.
\begin{assumption}
    $a_{ij}=a_j=a(p_j,x_j;p_{-j},x_{-j})$ for all $i$. That is, the platform presents products to consumers based on their prices and attributes, but does not discriminate presentation across consumers (may generate discrimination ex post).
\end{assumption}
The platform charges sellers a transaction fee T   and a fraction  t  of the transaction price.