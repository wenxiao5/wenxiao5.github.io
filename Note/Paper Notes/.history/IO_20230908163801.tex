\documentclass[11pt]{elegantbook}
\definecolor{structurecolor}{RGB}{40,58,129}
\linespread{1.6}
\setlength{\footskip}{20pt}
\setlength{\parindent}{0pt}
\newcommand{\argmax}{\operatornamewithlimits{argmax}}
\newcommand{\argmin}{\operatornamewithlimits{argmin}}
\elegantnewtheorem{proof}{Proof}{}{Proof}
\elegantnewtheorem{claim}{Claim}{prostyle}{Claim}
\DeclareMathOperator{\col}{col}
\title{\textbf{Industrial Organization Papers}}
\author{Wenxiao Yang}
\institute{Haas School of Business, University of California Berkeley}
\date{2023}
\setcounter{tocdepth}{2}
\cover{cover.JPEG}
\extrainfo{All models are wrong, but some are useful.}

% modify the color in the middle of titlepage
\definecolor{customcolor}{RGB}{9,119,119}
\colorlet{coverlinecolor}{customcolor}
\usepackage{cprotect}

\addbibresource[location=local]{reference.bib} % bib

\begin{document}

\maketitle
\frontmatter
\tableofcontents
\mainmatter


\chapter{Berry, S. T. (1994). Estimating discrete-choice models of product differentiation.}
Based on
\begin{enumerate}[$\circ$]
    \item Berry, S. T. (1994). Estimating discrete-choice models of product differentiation. \textit{The RAND Journal of Economics}, 242-262.
\end{enumerate}


\section{Problem}
The problem of estimating supply-and-demand models in markets with product differentiation.


\section{What's New?}
\begin{enumerate}
    \item Discrete-choice model.
    \item Unobserved demand factors are considered.
    \item Estimation by "inverting" the market-share equation to find the implied mean levels of utility for each good.
\end{enumerate}
\begin{note}
    \textbf{The problem of previous empirical model:} a system of $N$ goods gives $N^2$ elasticizes to estimate.
\end{note}
This paper put some structure based by making assumptions on consumer utility. It uses the aggregation of consumers' individual choice to estimate demand.

\section{Model}
\textbf{Data:} The econometrician is assumed to observe the market outcomes of price and quantities sold by each firm.

\begin{enumerate}
    \item There are $R$ independent markets.
    \item There are $N_r$ firms in market $r$, with each firm producing one product. (For simplicity, we omit $r$ in following notations).
    \item For product $j$, the observed characteristics are denoted by $z_{j}=(x_j,w_j)\in \mathbb{R}^k$.\\ Specifically, $z_j$ includes two parts:
    \subitem characteristics that affect demand $x_j$;
    \subitem characteristics that affect marginal cost $w_j$.
    \item For product $j$, the unobserved characteristics are denoted by $(\xi_j,\omega_j)\in \mathbb{R}^k$.\\ Specifically, $(\xi_j,\omega_j)$ includes two parts:
    \subitem unobserved characteristics that affect demand $\xi_j$;
    \subitem unobserved characteristics that affect marginal cost $\omega_j$.\\
    Unobserved characteristics are mean independent of observed characteristics and independent across markets.
\end{enumerate}

































\end{document}