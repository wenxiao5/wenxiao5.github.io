\documentclass[11pt]{elegantbook}
\definecolor{structurecolor}{RGB}{40,58,129}
\linespread{1.6}
\setlength{\footskip}{20pt}
\setlength{\parindent}{0pt}
\newcommand{\argmax}{\operatornamewithlimits{argmax}}
\newcommand{\argmin}{\operatornamewithlimits{argmin}}
\elegantnewtheorem{proof}{Proof}{}{Proof}
\elegantnewtheorem{claim}{Claim}{prostyle}{Claim}
\DeclareMathOperator{\col}{col}
\title{\textbf{Industrial Organization Papers}}
\author{Wenxiao Yang}
\institute{Haas School of Business, University of California Berkeley}
\date{2023}
\setcounter{tocdepth}{2}
\cover{cover.JPEG}
\extrainfo{All models are wrong, but some are useful.}

% modify the color in the middle of titlepage
\definecolor{customcolor}{RGB}{9,119,119}
\colorlet{coverlinecolor}{customcolor}
\usepackage{cprotect}

\addbibresource[location=local]{reference.bib} % bib

\begin{document}

\maketitle
\frontmatter
\tableofcontents
\mainmatter


\chapter{Berry, S. T. (1994). Estimating discrete-choice models of product differentiation.}
Based on
\begin{enumerate}[$\circ$]
    \item Berry, S. T. (1994). Estimating discrete-choice models of product differentiation. \textit{The RAND Journal of Economics}, 242-262.
\end{enumerate}


\section{Problem}
The problem of estimating supply-and-demand models in markets with product differentiation.


\section{What's New?}
\begin{note}
    \textbf{The problem of previous empirical model:} a system of $N$ goods gives $N^2$ elasticizes to estimate.
\end{note}
\begin{enumerate}
    \item Discrete-choice model.
    \item Unobserved demand factors are considered.
    \item Estimation by "inverting" the market-share equation to find the implied mean levels of utility for each good.
\end{enumerate}



































\end{document}