\documentclass[11pt]{elegantbook}
\definecolor{structurecolor}{RGB}{40,58,129}
\linespread{1.6}
\setlength{\footskip}{20pt}
\setlength{\parindent}{0pt}
\newcommand{\argmax}{\operatornamewithlimits{argmax}}
\newcommand{\argmin}{\operatornamewithlimits{argmin}}
\elegantnewtheorem{proof}{Proof}{}{Proof}
\elegantnewtheorem{claim}{Claim}{prostyle}{Claim}
\DeclareMathOperator{\col}{col}
\title{\textbf{Industrial Organization Papers}}
\author{Wenxiao Yang}
\institute{Haas School of Business, University of California Berkeley}
\date{2023}
\setcounter{tocdepth}{2}
\cover{cover.JPEG}
\extrainfo{All models are wrong, but some are useful.}

% modify the color in the middle of titlepage
\definecolor{customcolor}{RGB}{9,119,119}
\colorlet{coverlinecolor}{customcolor}
\usepackage{cprotect}

\addbibresource[location=local]{reference.bib} % bib

\begin{document}

\maketitle
\frontmatter
\tableofcontents
\mainmatter


\chapter{Berry, S. T. (1994). Estimating discrete-choice models of product differentiation.}
Based on
\begin{enumerate}[$\circ$]
    \item Berry, S. T. (1994). Estimating discrete-choice models of product differentiation. \textit{The RAND Journal of Economics}, 242-262.
\end{enumerate}


\section{Problem}
The problem of estimating supply-and-demand models in markets with product differentiation.


\section{What's New?}
\begin{enumerate}
    \item Discrete-choice model.
    \item Unobserved demand factors are considered.
    \item Estimation by "inverting" the market-share equation to find the implied mean levels of utility for each good.
\end{enumerate}
\begin{note}
    \textbf{The problem of previous empirical model:} a system of $N$ goods gives $N^2$ elasticizes to estimate.
\end{note}
This paper put some structure based by making assumptions on consumer utility. It uses the aggregation of consumers' individual choice to estimate demand.

\section{Model}
\textbf{Data:} The econometrician is assumed to observe the market outcomes of price and quantities sold by each firm.
\subsection{Settings}
\begin{enumerate}
    \item There are $R$ independent markets.
    \item There are $N_r$ firms in market $r$, with each firm producing one product. (For simplicity, we omit $r$ in following notations).
    \item For product $j$, the observed characteristics are denoted by $z_{j}=(x_j,w_j)\in \mathbb{R}^k$.\\ Specifically, $z_j$ includes two parts:
    \subitem characteristics that affect demand $x_j$;
    \subitem characteristics that affect marginal cost $w_j$.
    \item For product $j$, the unobserved characteristics are denoted by $(\xi_j,\omega_j)\in \mathbb{R}^k$.\\ Specifically, $(\xi_j,\omega_j)$ includes two parts:
    \subitem unobserved characteristics that affect demand $\xi_j$;
    \subitem unobserved characteristics that affect marginal cost $\omega_j$.\\
    Unobserved characteristics are mean independent of observed characteristics and independent across markets.
    \item Price of product $j$ is denoted by $p_j$.
    \item \textbf{Discrete Choice Model:} the utility of consumer $i$ for product $j$: $$u_{ij}=U(x_j,\xi_j,p_j,v_i,\theta_d)$$
    \subitem $v_i$ captures the consumer $i$'s consumer-specific terms that are not observed by the econometrician;
    \subitem $\theta_d$ are demand parameters.\\
    Consider a simple random coefficients specification for utility,
    \begin{equation}
        \begin{aligned}
            u_{ij}&=U(x_j,\xi_j,p_j,v_i=\tilde{\beta}_i,\theta_d=\alpha)\\
            &=x_j\tilde{\beta}_i-\alpha p_j+\xi_j+\epsilon_{ij}
        \end{aligned}
        \tag{(2)}
        \label{(2)}
    \end{equation}
    where $\epsilon_{ij}$ represents the distribution of consumer preferences about this mean, and $\alpha$ is invariant across consumers (although not necessary).\\
    \textbf{Random coefficients:} (to avoid unreasonable substitution effects) Decompose consumer $i$'s taste parameter for characteristic $k$ as
    \begin{equation}
        \begin{aligned}
            \tilde{\beta}_{ik}=\beta_k+\sigma_k\xi_{ik}
        \end{aligned}
        \tag{(3)}
        \label{(3)}
    \end{equation}
    where $\beta_k$ is the mean level of taste for characteristic $k$ and $\xi_{ik}$ has mean zero.\\
    Combing \ref{(2)} and \ref{(3)}, we can write
    \begin{equation}
        \begin{aligned}
            u_{ij}&=\sum_{k}x_{jk}\tilde{\beta}_{ik}-\alpha p_j+\xi_j+\epsilon_{ij}\\
            &=\sum_{k}x_{jk}\left(\beta_k+\sigma_k\xi_{ik}\right)-\alpha p_j+\xi_j+\epsilon_{ij}\\
            &=x_j\beta-\alpha p_j+\xi_j+\sum_{k}x_{jk}\sigma_k\xi_{ik}+\epsilon_{ij}\\
            &=x_j\beta-\alpha p_j+\xi_j+v_{ij}
        \end{aligned}
        \nonumber
    \end{equation}
    with $v_{ij}=\sum_{k}x_{jk}\sigma_k\xi_{ik}+\epsilon_{ij}$, which has mean zero.\\
    The mean utility level of product $j$ is
    \begin{equation}
        \begin{aligned}
            \delta_j\equiv x_j\beta-\alpha p_j+\xi_j
        \end{aligned}
        \tag{(5)}
        \label{(5)}
    \end{equation}
    Then, the utility of consumer $i$ for product $j$ can be written as $$u_{ij}=\delta_j+v_{ij}$$
    \item \textbf{Discrete-choice Market Share Function:}
    \subitem Define the set of consumer unobservables that lead to the consumption of good $j$ as $$A_j(\boldsymbol{\delta})\triangleq \{\boldsymbol{v}_i: \delta_j+v_{ij}>\delta_k+v_{ik}, \forall k\neq j\}$$
    Then, the market share of $j$ is the probability that $\boldsymbol{v}_i$ falls into the region $A_j(\boldsymbol{\delta})$, $P(\boldsymbol{v}_i\in A_j(\boldsymbol{\delta}))$.\\
    Given $\boldsymbol{x},\boldsymbol{p},\boldsymbol{\xi}, \boldsymbol{\theta}$, and the distribution of $\boldsymbol{v}$ follows $F(\cdot,\boldsymbol{x},\sigma)$ (p.d.f $f$), the market share is
    \begin{equation}
        \begin{aligned}
            \natural_j(\boldsymbol{\delta}(\boldsymbol{x},\boldsymbol{p},\boldsymbol{\xi}),\boldsymbol{x},\boldsymbol{\theta})=\int_{A_j(\boldsymbol{\delta})}f(\boldsymbol{v},\boldsymbol{x},\sigma_v)d \boldsymbol{v}
        \end{aligned}
        \nonumber
    \end{equation}
    The measure of consumers in a market is denoted by $M$ (which is assumed to be observed). The observed output quantity of the firm is
    \begin{equation}
        \begin{aligned}
            q_j=M \natural_j(\boldsymbol{x},\boldsymbol{\xi},\boldsymbol{p},\theta_d)
        \end{aligned}
        \nonumber
    \end{equation}
    In addition to competing products $j=1,...,N$, there is also an outside good $j=0$.
\end{enumerate}

































\end{document}