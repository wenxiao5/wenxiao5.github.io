\documentclass[11pt]{elegantbook}
\usepackage{graphicx}
%\usepackage{float}
\definecolor{structurecolor}{RGB}{40,58,129}
\linespread{1.6}
\setlength{\footskip}{20pt}
\setlength{\parindent}{0pt}
\newcommand{\argmax}{\operatornamewithlimits{argmax}}
\newcommand{\argmin}{\operatornamewithlimits{argmin}}
\elegantnewtheorem{proof}{Proof}{}{Proof}
\elegantnewtheorem{claim}{Claim}{prostyle}{Claim}
\DeclareMathOperator{\col}{col}
\title{Applied Metrics Papers}
\author{Wenxiao Yang}
\institute{Haas School of Business, University of California Berkeley}
\date{2024}
\setcounter{tocdepth}{2}
\extrainfo{All models are wrong, but some are useful.}

\cover{cover}

% modify the color in the middle of titlepage
\definecolor{customcolor}{RGB}{32,178,170}
\colorlet{coverlinecolor}{customcolor}
\usepackage{cprotect}

\bibliographystyle{apalike_three}

\begin{document}
\maketitle

\frontmatter
\tableofcontents

\mainmatter

\chapter{Identification of Prediction Errors}
\section{\cite{rambachan2024identifying}: Identifying Prediction Mistakes in Observational Data}


\subsection{Expected Utility Maximization at Accurate Beliefs}
A decision maker (DM) makes a binary choice $c\in\{0,1\}$ for each individual, which is summarized by characteristics $x\in \mathcal{X}$ and an unknown outcome $y^*\in \mathcal{Y}$ (observable when $c=1$).

These three variables are summarized by a joint distribution, $(X,C,Y^*)\sim P(\cdot)$. We assume finite full support of $x$, i.e. there is a $\delta>0$ such that $P(x):=P(X=x)\geq \delta,\forall x\in \mathcal{X}$. As the $Y^*$ is only observable when $C=1$. We define
\begin{equation}
    \begin{aligned}
        Y:=C\cdot Y^*
    \end{aligned}
    \nonumber
\end{equation}
The observable data is the joint distribution $(X,C,Y)\sim P(\cdot)$.












\appendix










\bibliography{ref_paper}




















\end{document}