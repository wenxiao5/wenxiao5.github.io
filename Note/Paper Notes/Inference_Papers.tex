\documentclass[11pt]{elegantbook}
\usepackage{graphicx}
%\usepackage{float}
\definecolor{structurecolor}{RGB}{40,58,129}
\linespread{1.6}
\setlength{\footskip}{20pt}
\setlength{\parindent}{0pt}
\newcommand{\argmax}{\operatornamewithlimits{argmax}}
\newcommand{\argmin}{\operatornamewithlimits{argmin}}
\elegantnewtheorem{proof}{Proof}{}{Proof}
\elegantnewtheorem{claim}{Claim}{prostyle}{Claim}
\DeclareMathOperator{\col}{col}
\title{Applied Metrics Papers}
\author{Wenxiao Yang}
\institute{Haas School of Business, University of California Berkeley}
\date{2024}
\setcounter{tocdepth}{2}
\extrainfo{All models are wrong, but some are useful.}

\cover{cover}

% modify the color in the middle of titlepage
\definecolor{customcolor}{RGB}{32,178,170}
\colorlet{coverlinecolor}{customcolor}
\usepackage{cprotect}

\bibliographystyle{apalike_three}

\begin{document}
\maketitle

\frontmatter
\tableofcontents

\mainmatter

\chapter{Identification of Prediction Errors}
\section{\cite{rambachan2024identifying}: Identifying Prediction Mistakes in Observational Data}
Uncovering systematic prediction mistakes in empirical settings is challenging because
\begin{enumerate}
    \item the decision maker's preferences and
    \item the information set
\end{enumerate}
are unknown to us.

\subsection{Expected Utility Maximization at Accurate Beliefs}
A decision maker (DM) makes a binary choice $c\in\{0,1\}$ for each individual, which is summarized by characteristics $x\in \mathcal{X}$ and an unknown outcome $y^*\in \mathcal{Y}$ (observable when $c=1$).

\begin{example}[ (Pretrial Release)]
    A judge decides whether to detain ore release defendants $C\in\{0,1\}$. The outcome $Y^*\in\{0,1\}$ is whether a defendant would fail to appear in court if released. $X$ is the recorded information of the defendant.
\end{example}

\begin{example}[ (Medical Testing and Diagnosis)]
    $C\in\{0,1\}$ is whether to conduct a test. $Y^*\in\{0,1\}$ is whether the patient had a heart attack. $X$ is the recorded information of the patient.
\end{example}

\begin{example}[ (Hiring)]
    $C\in\{0,1\}$ is whether to hire a candidate. $Y^*$  is a vector of on-the-job productivity measures. $X$ is the recorded information of the candidate.
\end{example}


These three variables are summarized by a joint distribution, $(X,C,Y^*)\sim P(\cdot)$. We assume finite full support of $x$, i.e. there is a $\delta>0$ such that $P(x):=P(X=x)\geq \delta,\forall x\in \mathcal{X}$. As the $Y^*$ is only observable when $C=1$. We define
\begin{equation}
    \begin{aligned}
        Y:=C\cdot Y^*
    \end{aligned}
    \nonumber
\end{equation}
The observable data is the joint distribution $(X,C,Y)\sim P(\cdot)$. The DM's conditional choice probabilities are
\begin{equation}
    \begin{aligned}
        \pi_c(x):=P(C=c|X=x),c\in\{0,1\},x\in \mathcal{X}
    \end{aligned}
    \nonumber
\end{equation}
The observable conditional outcome probabilities are
\begin{equation}
    \begin{aligned}
        P_1(y^*\mid x):=P(Y^*=y^*\mid C=1, X=x),y^*\in \mathcal{Y},x\in \mathcal{X}
    \end{aligned}
    \nonumber
\end{equation}
The $P_0(y^*\mid x)$ and the true outcome probabilities $P(y^*\mid x)$ are not identified due to the missing-data problem.

\begin{note}
    In the main context of paper: (i). The decision maker makes a binary choice $c\in\{0,1\}$ for each individual; (ii). The decision maker's choice does not have a direct causal effect on the out-come.
\end{note}






\appendix










\bibliography{ref_paper}




















\end{document}