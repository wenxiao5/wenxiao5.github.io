%!TEX program = xelatex
\documentclass[11pt,a4paper]{article}
\usepackage[utf8]{inputenc}
\usepackage[T1]{fontenc}
\usepackage{authblk}
\usepackage{ctex}
\usepackage{tikz}
\usepackage{pgfplots}
\usepackage{verbatim}
\usepackage{amsfonts}
\usepackage{amsmath}
\usepackage{amsthm}
\usepackage{indentfirst}
\usepackage{amssymb}
\setlength{\parindent}{0pt}
\usetikzlibrary{shapes,snakes}
\newcommand{\argmax}{\operatornamewithlimits{argmax}}
\newcommand{\argmin}{\operatornamewithlimits{argmin}}
\DeclareMathOperator{\col}{col}
\usepackage{booktabs}
\newtheorem{theorem}{Theorem}
\newtheorem{note}{Note}
\newtheorem{definition}{Definition}
\newtheorem{proposition}{Proposition}
\newtheorem{lemma}{Lemma}
\newtheorem{example}{Example}
\newtheorem{corollary}{Corollary}
\usepackage{graphicx}
\usepackage{geometry}
\usepackage{hyperref}
\newcommand{\code}{	exttt}
\geometry{a4paper,scale=0.8}
\title{STAT R note}
\author[*]{Wenxiao Yang}
\affil[*]{Department of Mathematics, University of Illinois at Urbana-Champaign}
\date{2021}


\usepackage{listings}
\usepackage{xcolor}

\lstset{numbers=left,numberstyle=\tiny,keywordstyle=\color{blue},commentstyle=\color[cmyk]{1,0,1,0},frame=single,escapeinside=``,extendedchars=false,xleftmargin=2em,xrightmargin=2em,aboveskip=1em,tabsize=4,showspaces=false}






\begin{document}
\maketitle
\tableofcontents
\newpage

\section{Basic}
\subsection{q-value of $\chi^2_n$}
\begin{lstlisting}[language=R]
qchisq(0.95, n)
\end{lstlisting}
\subsection{读取数据 txt (galton)}
\begin{lstlisting}[language=R]
galton <- read.table("Galton.txt", header=TRUE)
\end{lstlisting}
\subsection{读取数据 csv (bikeshares)}
\begin{lstlisting}[language=R]
bikeshares <-  read.csv("BikeShares.csv", header=TRUE)
#分隔符为";"
whitewines.data<-read.csv("whitewines.csv",sep=";",header = TRUE)
\end{lstlisting}
\subsection{查看数据维度 (bikeshares)}
\begin{lstlisting}[language=R]
dim(bikeshares)
## [1] 17414    10
\end{lstlisting}

\subsection{数据中删除列 (bikeshares)}
\begin{lstlisting}[language=R]
# We remove columns 1,  7, 8, 9, 10:
bikeshares.reg = bikeshares[,c(-1,-7,-8,-9,-10)] #-i即删除i列
head(bikeshares.reg)
##   cnt  t1  t2   hum wind_speed
## 1 182 3.0 2.0  93.0        6.0
## 2 138 3.0 2.5  93.0        5.0
## 3 134 2.5 2.5  96.5        0.0
## 4  72 2.0 2.0 100.0        0.0
## 5  47 2.0 0.0  93.0        6.5
## 6  46 2.0 2.0  93.0        4.0
\end{lstlisting}


\subsection{数据“列”处理:赋值,条件选中 (galton)}
\begin{lstlisting}[language=R]
# Define the Adjusted Heigth Variable (according to Galton)
galton$AH <- galton$Height
galton$AH[galton$Gender=="F"]<-galton$Height[galton$Gender=="F"]*1.08
galton$MP <- (galton$Father + 1.08*galton$Mother)/2
head(galton)
##   Family Father Mother Gender Height Kids     AH    MP
## 1      1   78.5   67.0      M   73.2    4 73.200 75.43
## 2      1   78.5   67.0      F   69.2    4 74.736 75.43
## 3      1   78.5   67.0      F   69.0    4 74.520 75.43
## 4      1   78.5   67.0      F   69.0    4 74.520 75.43
## 5      2   75.5   66.5      M   73.5    4 73.500 73.66
## 6      2   75.5   66.5      M   72.5    4 72.500 73.66
\end{lstlisting}

\subsection{查看数据类型 (bikeshare)}
\begin{lstlisting}[language=R]
class(numeric(n.iter))
## [1] "numeric"
class(bikeshares.reg)
## [1] "data.frame"
\end{lstlisting}


\subsection{data.frame}
\subsubsection{修改列名}
\begin{lstlisting}[language=R]
names(myCR) = c("t1","hum");
\end{lstlisting}
\subsubsection{data.frame列拼接cbind() (bikeshares)}
\begin{lstlisting}[language=R]
bikeshare.mlr1$fitted[1:5]
## 1         2         3         4         5 
## 158.12967 152.85747  42.50091 -77.95731 126.47427 
bikeshare.mlr1$residuals[1:5]
## 1         2         3         4         5 
## 23.87033 -14.85747  91.49909 149.95731 -79.47427 
cbind(bikeshare.mlr1$fitted[1:5], bikeshare.mlr1$residuals[1:5])
##        [,1]      [,2]
## 1 158.12967  23.87033
## 2 152.85747 -14.85747
## 3  42.50091  91.49909
## 4 -77.95731 149.95731
## 5 126.47427 -79.47427
\end{lstlisting}
\subsubsection{data.frame行拼接rbind() (bikeshares)}
\begin{lstlisting}[language=R]
rbind(bikeshare.mlr1$fitted[1:5], bikeshare.mlr1$residuals[1:5])
## 1         2        3         4         5
## [1,] 158.12967 152.85747 42.50091 -77.95731 126.47427
## [2,]  23.87033 -14.85747 91.49909 149.95731 -79.47427
\end{lstlisting}
\subsubsection{data.frame抽样}
\begin{lstlisting}[language=R]
head(bikeshares.reg)
##      cnt t1  t2  hum     wind_speed
## 1	182	3.0	2.0	93.0	6.0
## 2	138	3.0	2.5	93.0	5.0
## 3	134	2.5	2.5	96.5	0.0
## 4	72	2.0	2.0	100.0	0.0
## 5	47	2.0	0.0	93.0	6.5
## 6	46	2.0	2.0	93.0	4.0
bikeshares.reg[sample(5), c(3,4)] #前五行(第3,4列)中随机抽样
##      t2  hum
## 2	2.5	93.0
## 5	0.0	93.0
## 3	2.5	96.5
## 1	2.0	93.0
## 4	2.0	100.0
\end{lstlisting}

\subsection{集体求均值}
\begin{lstlisting}[language=R]
apply(bikeshares.reg,2,mean)
## cnt         t1         t2        hum wind_speed 
## 1143.10164   12.46809   11.52084   72.32495   15.91306 
\end{lstlisting}





\subsection{numeric}
\subsubsection{numeric(k): 生成k个0的numeric}
\begin{lstlisting}[language=R]
numeric(5)
## [1] 0 0 0 0 0
class(numeric(5))
## [1] "numeric"
\end{lstlisting}
\subsubsection{numeric 数值修改}
\begin{lstlisting}[language=R]
A=numeric(5)
A[1]=2
A
## [1] 2 0 0 0 0
\end{lstlisting}
\subsection{matrix}
\subsubsection{data.frame 转成 matrix}
\begin{lstlisting}[language=R]
M=data.matrix(X)
\end{lstlisting}

\subsubsection{修改列名}
\begin{lstlisting}[language=R]
colnames(x)=c("t1", "t2", "hum")
\end{lstlisting}

\subsubsection{去掉矩阵 列/行 的名字}
\begin{lstlisting}[language=R]
rownames(A)<-NULL
colnames(A)<-NULL
\end{lstlisting}

\subsubsection{自己创建matrix}
\begin{lstlisting}[language=R]
A=matrix(1:12,nrow=3,ncol=4)
A
##      [,1] [,2] [,3] [,4]
## [1,]    1    4    7   10
## [2,]    2    5    8   11
## [3,]    3    6    9   12
\end{lstlisting}
\subsubsection{Transpose of matrix 转置矩阵}
\begin{lstlisting}[language=R]
t(A)
##       [,1] [,2] [,3]
## [1,]    1    2    3
## [2,]    4    5    6
## [3,]    7    8    9
## [4,]   10   11   12
\end{lstlisting}
\subsubsection{Multiplication of matrix 矩阵乘法}
\begin{lstlisting}[language=R]
A%*%t(A)
##     [,1] [,2] [,3]
## [1,]  166  188  210
## [2,]  188  214  240
## [3,]  210  240  270
\end{lstlisting}
\subsubsection{解$Ax=b$: solve(A,b)}
Solve $ax=b$
\begin{lstlisting}[language=R]
A.data=data.frame(a=c(1,43,765,9),b=c(2,455,787,2),
                c=c(213,434,67,24),d=c(672,332,7,123))
A=data.matrix(A.data)
b=matrix(c(1,10,8,9))
solve(A,b)
##      [,1]
## a -0.6723499
## b  0.7380811
## c -0.9034118
## d  0.2866412
\end{lstlisting}

\subsubsection{矩阵行列式 : det()}
\begin{lstlisting}[language=R]
det(A)
\end{lstlisting}
\subsubsection{生成对角阵:diag(1,2,3,4)}
\begin{lstlisting}[language=R]
diag(c(1,2,3,4))
##     [,1] [,2] [,3] [,4]
## [1,]    1    0    0    0
## [2,]    0    2    0    0
## [3,]    0    0    3    0
## [4,]    0    0    0    4
\end{lstlisting}
\subsubsection{提取对角线上的元素:diag()}
\begin{lstlisting}[language=R]
diag(A)
## [1]   1 455  67 123
\end{lstlisting}
\subsubsection{特征值和特征向量:eigen()}
\begin{lstlisting}[language=R]
eigen(A)
## eigen() decomposition
## $values
## [1]  962.54862 -533.15335  195.96895   20.63578
## 
## $vectors
##             [,1]        [,2]       [,3]       [,4]
## [1,] -0.18050353 -0.31476395  0.7098847  0.5218457
## [2,] -0.65689212 -0.36245740 -0.6561850 -0.5428550
## [3,] -0.73165231  0.87683413  0.2141936  0.6319310
## [4,] -0.02441547 -0.02664961  0.1400217 -0.1834356
\end{lstlisting}

\subsubsection{逆矩阵 solve(A)}
\begin{lstlisting}[language=R]
solve(A)
##      [,1]          [,2]          [,3]        [,4]
## [1,]  0.015470466 -0.0038533021  0.0023771584 -0.07425607
## [2,] -0.016656510  0.0038972675 -0.0011449021  0.08054712
## [3,]  0.019498924 -0.0018420827  0.0012737127 -0.10163107
## [4,] -0.004665816  0.0005780095 -0.0004038514  0.03208420
\end{lstlisting}

\subsubsection{列或行的函数处理 $apply(A,1/2,func)$}
\begin{lstlisting}[language=R]
apply(A,1,mean) #1表示对行求均值
apply(A,2,mean) #2表示对列求均值
apply(x,2,sd)
apply(x,2,var)
\end{lstlisting}


\section{Simple Linear Regression}
\subsection{拟合slr (galton)}
\begin{lstlisting}[language=R]
# Simple Linear Regression
slr.fit <- lm(AH ~ MP, data=galton)
summary(slr.fit)
## 
## Call:
## lm(formula = AH ~ MP, data = galton)
## 
## Residuals:
##     Min      1Q  Median      3Q     Max 
## -9.4947 -1.4779  0.0995  1.5175  9.1262 
## 
## Coefficients:
##             Estimate Std. Error t value Pr(>|t|)    
## (Intercept) 18.76698    2.84062   6.607 6.74e-11 ***
## MP           0.72906    0.04102  17.772  < 2e-16 ***
## ---
## Signif. codes:  0 '***' 0.001 '**' 0.01 '*' 0.05 '.' 01 ' ' 1
## 
## Residual standard error: 2.233 on 896 degrees of freedom
## Multiple R-squared:  0.2606, Adjusted R-squared:  02598 
## F-statistic: 315.9 on 1 and 896 DF,  p-value: < 2.2e-16
\end{lstlisting}

\subsection{Summary中提取R-square (galton)}
\begin{lstlisting}[language=R]
summary(slr.fit)$r.square
\end{lstlisting}
\subsection{Summary中提取coefficients (galton)}
\begin{lstlisting}[language=R]
galton.coef = summary(slr.fit)$coef
galton.coef
##               Estimate Std. Error   t value     Pr(>|t|)
## (Intercept) 18.7669821 2.84062068  6.606648 6.735528e-11
## MP           0.7290562 0.04102226 17.772211 9.224505e-61

galton.coef[2,1]
galton.coef[2,3]  ## 提取t-test
\end{lstlisting}

\subsection{回归中提取 degrees of freedom (galton)}
\begin{lstlisting}[language=R]
slr.fit$df
## [1] 896
\end{lstlisting}












\subsection{Hypothesis test}
\subsubsection{p-value of t-test (galton)}
\begin{lstlisting}[language=R]
# pt(t-statistics, df)
# $H_0:\beta_1=0$, 由于检验0对称,我们需要乘2
2*pt(-galton.coef[2,1]/galton.coef[2,2], 896)
## [1] 9.224505e-61
\end{lstlisting}

\subsubsection{Critical value of $\alpha=0.05$ in t(n)}
\begin{lstlisting}[language=R]
qt(.05, n)
## -1.644941
\end{lstlisting}

\subsubsection{ANOVA(F-test) (HW1)}
\begin{lstlisting}[language=R]
grade.anova=anova(slr.fit)
grade.anova
## Analysis of Variance Table
##
## Response: final
##              Df  SumSq  MeanSq  F value Pr(>F)
## QuizAverage   1  69812   69812  423.19 < 2.2e-16 ***
## Residuals   380  62687     165
## ---
## Signif. codes:  0 '***' 0.001 '**' 0.01 '*' 0.05 '.' 0.1 ' ' 1

grade.anova[1,4]   ## 提取F- value from ANOVA Table
\end{lstlisting}

\subsubsection{p-value of F-test (HW1)}
\begin{lstlisting}[language=R]
pf(grade.anova[1,4],df1=1,df2=380,lower.tail = FALSE)
# lower.tail: if TRUE (default), probabilities are P[X $\leq$ x],
# otherwise, P[X > x].
\end{lstlisting}

\subsubsection{Critical value of $\alpha=0.05$ in F(p,n)}
\begin{lstlisting}[language=R]
qf(.05, p, n, lower.tail = FALSE)
\end{lstlisting}

\subsection{Confidence interval 置信区间 (HW1)}
\begin{lstlisting}[language=R]
confint(slr.fit, 'QuizAverage', level=0.9)
##            5 %      95 %
##QuizAverage 0.7880018 0.9253306
\end{lstlisting}







\subsection{Prediction}
\subsubsection{模型带入数据 (galton)}
\begin{lstlisting}[language=R]
predict(slr.fit, newdata=data.frame(MP=70))
##        1 
## 69.80092
\end{lstlisting}
\subsubsection{Confidence interval (HW1) $
\hat{\beta}_{0}+\hat{\beta}_{1} x^{\star} \pm T_{n-2}(\alpha / 2) \hat{\sigma} \sqrt{\frac{1}{n}+\frac{\left(x^{\star}-\bar{x}\right)^{2}}{S_{x x}}}
$}
\begin{lstlisting}[language=R]
predict(slr.fit,newdata = data.frame(QuizAverage=85),
interval = 'confidence', level=0.9)
##        fit     lwr      upr
## 1 76.7638 75.44682 78.08077
\end{lstlisting}
\subsubsection{Prediction interval (HW1) $
\hat{\beta}_{0}+\hat{\beta}_{1} x^{\star} \pm T_{n-2}(\alpha / 2) \hat{\sigma} \sqrt{1+\frac{1}{n}+\frac{\left(x^{\star}-\bar{x}\right)^{2}}{S_{x x}}}
$}
\begin{lstlisting}[language=R]
predict(slr.fit,newdata = data.frame(QuizAverage=85),
interval = 'prediction', level=0.9)
##        fit      lwr      upr
## 1 76.7638 55.54486 97.98273
\end{lstlisting}


\section{Multiple Linear Regression}
\subsection{拟合mlr (bikeshares)}
\begin{lstlisting}[language=R]
bikeshare.mlr1 = lm(cnt ~ t1 + t2 + hum + wind_speed,
                                        data=bikeshares.reg )
summary(bikeshare.mlr1)
## 
## Call:
## lm(formula = cnt ~ t1 + t2 + hum + wind_speed,
##                                      data = bikeshares.reg)
## 
## Residuals:
##     Min      1Q  Median      3Q     Max 
## -1970.1  -602.7  -252.7   332.6  6007.4 
## 
## Coefficients:
##              Estimate Std. Error t value Pr(>|t|)    
## (Intercept) 2582.5618    64.7237  39.901  < 2e-16 ***
## t1            66.1963     9.4206   7.027 2.19e-12 ***
## t2           -18.2313     7.7565  -2.350 0.018762 *  
## hum          -27.5645     0.5865 -46.999  < 2e-16 ***
## wind_speed    -3.8435     0.9899  -3.883 0.000104 ***
## ---
## Signif. codes:  0 '***' 0.001 '**' 0.01 '*' 0.05 '.' 0.1 ' ' 1
## 
## Residual standard error: 936 on 17409 degrees of freedom
## Multiple R-squared:  0.2562, Adjusted R-squared:  0.256 
## F-statistic:  1499 on 4 and 17409 DF,  p-value: < 2.2e-16
\end{lstlisting}

\subsection{Update regression, add or delete predictor}
\begin{lstlisting}[language=R]
rat.lm_body = update(rat.lm, ~ liver+dose)
rat.lm_body = lm(Y~liver+dose, data = rat)
# 两者等价
\end{lstlisting}










\subsection{回归中提取residuals, fitted values (bikeshare)}
\begin{lstlisting}[language=R]
bikeshare.mlr1$res
bikeshare.mlr$residuals
bikeshare.mlr$fitted.values
\end{lstlisting}
\subsection{Summary中提取F-test statitic}
\begin{lstlisting}[language=R]
summary(bikeshare.mlr1)$fstat
## value    numdf   dendf
## 1499.07  4.00    17409.00
summary(bikeshare.mlr1)$fstat[1]
## 1499.07
\end{lstlisting}
\subsubsection{得到RSS: $\sum_{i=1}^nr_i^2$}
\begin{lstlisting}[language=R]
sum(bikeshare.mlr1$res^2) #方法1
deviance(bikeshare.mlr1)  #方法2
\end{lstlisting}



\subsection{Correlation matrix (bikeshares) $cor()$}
\begin{lstlisting}[language=R]
cor(bikeshares.reg[,-1]) #这里[,-1]是不想算第一列
##                    t1          t2        hum  wind_speed
## t1          1.0000000  0.98834422 -0.4477810  0.14547097
## t2          0.9883442  1.00000000 -0.4034951  0.08840854
## hum        -0.4477810 -0.40349514  1.0000000 -0.28778917
## wind_speed  0.1454710  0.08840854 -0.2877892  1.00000000
\end{lstlisting}

\begin{lstlisting}[language=R]
round(cor(seatpos), dig=2)
# 打印出来的数据保留两位小数
\end{lstlisting}
\subsection{Plot all pairs of variables}
\begin{lstlisting}[language=R]
pairs(rat)
\end{lstlisting}
\begin{center}\begin{figure}[htbp]
  \centering
  \includegraphics[scale=0.5]{rat1.png}
  \caption{}
  \label{}
\end{figure}\end{center}






\subsection{Partial $F-$Tests (bikeshare)}
\begin{lstlisting}[language=R]
bikeshare.mlr.full = lm(cnt ~ t1 + t2+ hum + wind_speed,
                        data=bikeshares.reg ) #先回归full model
bikeshare.mlr.reduced = lm(cnt ~ hum + wind_speed ,
                        data=bikeshares.reg ) #回归reduced model
anova(bikeshare.mlr.reduced, bikeshare.mlr.full)
                            #do the partial F-test by "anova(.)"
## Analysis of Variance Table
## 
## Model 1: cnt ~ hum + wind_speed
## Model 2: cnt ~ t1 + t2 + hum + wind_speed
##   Res.Df        RSS Df Sum of Sq      F    Pr(>F)
## 1  17411 1.6103e+10
## 2  17409 1.5250e+10  2 853010396 486.88 < 2.2e-16 ***
## ---
## Signif. codes:  0 '***' 0.001 '**' 0.01 '*' 0.05 '.' 0.1 ' ' 1
\end{lstlisting}
Sum of Square $853010396$ 是 $RSS_0-RSS_\alpha=1.6103e+10-1.5250e+10=853010396$\\
我们也可以按照公式算:
\begin{lstlisting}[language=R]
rss.full = sum(bikeshare.mlr.full$res^2)
# You can also compute it with
# deviance(bikeshare.mlr.full)
rss.reduced = sum(bikeshare.mlr.reduced$res^2)
# deviance(bikeshare.mlr.reduced)
Fstat = (rss.reduced - rss.full)/2/(rss.full/17409)
Fstat
## [1] 486.8763
1-pf(Fstat, 2, 17409)
## [1] 0
\end{lstlisting}

\subsection{Permutation Tests (bikeshares)}
\[\begin{cases}
    &H_0: bikeshares \sim humidity + windspeed\\
    &H_{\alpha}: bikeshares \sim RealTemp + FeelsLikeTemp + humidity + windspeed
    \end{cases}\]
If \textit{RealTemp} and \textit{FeelsLikeTemp} are insignificant (Under $H_0$), the F-statstic of regression model will not be affected by switching the orders of these two data. Then new F-statistic will be equal(or less) to the old. i.e. High new F-statistic is more extreme than $H_0$. So lower $p-$value will support $H_\alpha:$ \textit{RealTemp} and \textit{FeelsLikeTemp} are significant.
\begin{lstlisting}[language=R]
n.iter = 2000;
fstats = numeric(n.iter);
for(i in 1:n.iter){
  newbikes = bikeshares.reg;
  newbikes[, c(3,4)] = bikeshares.reg[sample(17414), c(3,4)];
  ge = lm(cnt ~ t1 + t2+ hum + wind_speed, data=newbikes);
  fstats[i] = summary(ge)$fstat[1]
}

# Estimated p-value
length(fstats[fstats > summary(bikeshare.mlr.full)$fstat[1]])/n.iter
## [1] 0
\end{lstlisting}

\subsection{Confidence/Prediction Interval}
\subsubsection{Estimators' Confidence Interval}
\begin{lstlisting}[language=R]
confint(bikeshare.mlr)
##                   2.5 %      97.5 %
## (Intercept) 2543.679114 2766.669772
## t1            41.516598   47.099480
## hum          -28.984942  -26.739780
## wind_speed    -4.941603   -1.262794
confint(bikeshare.mlr, 't1', level=0.99)
##       0.5 %   99.5 %
## t1 40.63932 47.97676
\end{lstlisting}

\subsubsection{Estimators' Confidence regions}
\begin{lstlisting}[language=R]
library(ellipse)
library(ggplot2)
CR95 = ellipse(bikeshare.mlr, c(2,3))
CR99 = ellipse(bikeshare.mlr, c(2,3), level=0.99)
CR998 = ellipse(bikeshare.mlr, c(2,3), level=0.998)
# Plot Confidence Regions for column 2,3
dim(CR95)
## [1] 100   2
head(CR95)
##            t1       hum
## [1,] 47.25426 -26.67754
## [2,] 47.13012 -26.63239
## [3,] 46.99462 -26.59219
## [4,] 46.84830 -26.55710
## [5,] 46.69175 -26.52728
## [6,] 46.52561 -26.50282
\end{lstlisting}
\begin{lstlisting}[language=R]
myCR = rbind(CR95, CR99, CR998);
# 行连接
myCR = data.frame(myCR); 
names(myCR) = c("t1","hum"); 
myCR[, 'level']=as.factor(c(rep(0.95, dim(CR95)[1]), 
                              rep(0.99, dim(CR99)[1]), 
                              rep(0.998, dim(CR998)[1])));
# 添加列‘level', 给各行根据精度赋值

ggplot(data=myCR, aes(x=t1, y=hum, colour=level)) + 
  geom_path(aes(linetype=level), size=1.5) + 
  geom_point(x=coef(bikeshare.mlr)[2], y=coef(bikeshare.mlr)[3]
  , shape=3, size=3, colour='red') + 
  geom_point(x=0, y=0, shape=1, size=3, colour='red') 
\end{lstlisting}
\begin{center}\begin{figure}[htbp]
    \centering
    \includegraphics[scale=0.5]{9599998.png}
    \caption{}
    \label{}
\end{figure}\end{center}

\subsubsection{Confidence Interval for new observation}
\begin{lstlisting}[language=R]
x=data.frame(t(meanvalue))
predict.lm(bikeshare.mlr,x,interval="confidence",level=0.95)
##        fit      lwr      upr
## 1 1143.102 1129.198 1157.006
\end{lstlisting}
\subsubsection{Prediction Interval for new observation}
\begin{lstlisting}[language=R]
predict.lm(bikeshare.mlr,x,interval="prediction",level=0.95)
##        fit       lwr      upr
## 1 1143.102 -691.7461 2977.949
\end{lstlisting}

\subsection{Unusual Observation}
\subsubsection{Leverage Points}
\begin{lstlisting}[language=R]
lev=influence(bikeshare.mlr)$hat
# H matrix 的对角上的所有元素
newlev = lev[lev>2*p/n]
# 找出所有high leverage points
bikeshares.reg[lev > 2*p/n,]
# 筛选出bikeshares中high leverage points的项
\end{lstlisting}

\subsubsection{Half-norm Plot}
Designed to identify unusually large values and assess positive data.\\
Plot the data against the positive normal quantiles. Specifically,\\
1. Sort the data:
$$x_{[1]} \leq ... \leq x_{[n]}.$$
2. Compute the quantiles:
$$u_i=\Phi^{-1}(\frac{n+i}{2n+1})$$
3. Plot $x_{[i]}$ against $u_i$.
\begin{lstlisting}[language=R]
library(faraway)
halfnorm(newlev, 6, labs=as.character(1:length(newlev)),
 ylab="Leverages")
# 6是nlab, 即给几个点标注
\end{lstlisting}
\begin{center}\begin{figure}[htbp]
  \centering
  \includegraphics[scale=0.3]{week402.png}
  \caption{}
  \label{}
\end{figure}\end{center}

\subsubsection{Standardized Residuals, Studentized Residuals, $rstandard()$, $rstudent()$}
\begin{lstlisting}[language=R]
rstandard(model)
rstudent(model)
\end{lstlisting}


\subsubsection{Outliers}
\begin{lstlisting}[language=R]
# Compute Studentized Residuals
jack=rstudent(bikeshare.mlr);
# The critical value WITH Bonferroni correction is
qt(.05/(2*n), n-p-1) 
## [1] -4.681361
# The critical value WITHOUT Bonferroni correction is
qt(.05/2, n-p-1)
## [1] -1.9601
# Sort the residuals indescending order to find outliers (if any)
sort(abs(jack), decreasing=TRUE)[1:10]
##     4462     5130     5139     4471    15888     5140
## 6.408782 5.665958 5.499140 5.317999 4.807279 4.787554
##    15217    15385    16727    14905
## 4.746059 4.738005 4.661289 4.522918
\end{lstlisting}
As we can see here, we have 8 outliers, i.e. the values that are higher (in absolute value) of the critical T distribution value with Bonferroni correction ($|-4.681361|$). These are observations: \#4462, \#5130, \#5139, \#4471, \#15888, \#5140, \#15217, \#15385.

\subsection{High influential points}
\begin{lstlisting}[language=R]
# Compute Cook's Distance
cook = cooks.distance(bikeshare.mlr)
# Extract max Cook's Distance
max(cook)
## [1] 0.005641587
which.max(cook)
## 4471
# Prepare a Half Normal Plot of Cook's distances
halfnorm(cook, 6, labs=as.character(1:length(cook)),
 ylab="Cook's distances")
\end{lstlisting}
\begin{center}\begin{figure}[htbp]
  \centering
  \includegraphics[scale=0.3]{week403.png}
  \caption{}
  \label{}
\end{figure}\end{center}

\subsection{Diagnostics}
\subsubsection{Checking Homoskedasticity Graph
}
\begin{lstlisting}[language=R]
plot(bikeshare.mlr, which=1)
\end{lstlisting}
\begin{center}\begin{figure}[htbp]
  \centering
  \includegraphics[scale=0.2]{d1}
  \caption{}
  \label{}
\end{figure}\end{center}
Which is same as
\begin{lstlisting}[language=R]
plot(bikeshare.mlr$fitted.values, bikeshare.mlr$residuals)
\end{lstlisting}
\begin{center}\begin{figure}[htbp]
  \centering
  \includegraphics[scale=0.2]{d2.png}
  \caption{}
  \label{}
\end{figure}\end{center}

\subsubsection{Breusch-Pagan Test}
\begin{lstlisting}[language=R]
library(lmtest)
bptest(bikeshare.mlr)
## 
##  studentized Breusch-Pagan test
## 
## data:  bikeshare.mlr
## BP = 133.29, df = 3, p-value < 2.2e-16
\end{lstlisting}
We can also perform the BP test by hand:
\begin{lstlisting}[language=R]
tmp.fit = lm(bikeshare.mlr$res^2 ~ t1 +  hum + wind_speed,
 data=bikeshares.reg )
summary(tmp.fit)$r.sq*dim(bikeshares.reg)[1]
\end{lstlisting}

\subsubsection{Checking Normality Graph}
\textbf{QQ-Plot}\\
\begin{lstlisting}[language=R]
plot(bikeshare.mlr, which=2)
\end{lstlisting}
\begin{center}\begin{figure}[htbp]
  \centering
  \includegraphics[scale=0.2]{d3.png}
  \caption{}
  \label{}
\end{figure}\end{center}
\textbf{Histogram}
\begin{lstlisting}[language=R]
hist(bikeshare.mlr$residuals)
\end{lstlisting}
\begin{center}\begin{figure}[htbp]
  \centering
  \includegraphics[scale=0.2]{d4.png}
  \caption{}
  \label{}
\end{figure}\end{center}

\subsubsection{Shapiro test}
\begin{lstlisting}[language=R]
shapiro.test(residuals(bikeshare.mlr))
\end{lstlisting}
\subsubsection{Kolmogorov-Smirnov test}
\begin{lstlisting}[language=R]
ks.test(residuals(bikeshare.mlr), y=pnorm)
## 
##  One-sample Kolmogorov-Smirnov test
## 
## data:  residuals(bikeshare.mlr)
## D = 0.63627, p-value < 2.2e-16
## alternative hypothesis: two-sided
\end{lstlisting}
The p-value is low, which implies that the normality assumption is not satisfied either.

\subsubsection{Checking Serial Dependence: Durbin Watson test}
\begin{lstlisting}[language=R]
library(lmtest)
dwtest(lm.sales)
##
##  Durbin-Watson test
##
## data:  lm.sales
## DW = 0.73473, p-value = 0.0001748
## alternative hypothesis: true autocorrelation is greater than 0
\end{lstlisting}




\subsubsection{Checking the Linearity Assumption with Partial Regression Plots}
Test $t1$
\begin{lstlisting}[language=R]
bikeshare.mlr = lm(cnt ~ hum + wind_speed, data=bikeshares.reg)
bikeshare.mlr.t1 = lm(t1 ~ hum + wind_speed, data=bikeshares.reg)
plot(bikeshare.mlr.t1$residuals, bikeshare.mlr$residuals)
\end{lstlisting}
\begin{center}\begin{figure}[htbp]
  \centering
  \includegraphics[scale=0.2]{d5}
  \caption{}
  \label{}
\end{figure}\end{center}

\subsubsection{Box Cox Transformations}
First we need to make sure each $y_i>0$:
\begin{lstlisting}[language=R]
min(bikeshares.reg$cnt) # this is the min value in the y's
## [1] 0
which(bikeshares.reg$cnt==0)
# this is the location of the min value
## [1] 2016
bikeshares.reg$cnt[2016]=0.01
# we replace the min with a small positive value
min(bikeshares.reg$cnt)
# we checke whether the 0 value was replaced 
# by the small positive number
## [1] 0.01
\end{lstlisting}
Now, we are ready to apply the \textit{boxcox} function:
\begin{lstlisting}[language=R]
bikes.transformation = boxcox(bikeshare.mlr,
lambda=seq(-2, 2, length=400))
\end{lstlisting}
which also same as
\begin{lstlisting}[language=R]
boxcox(bikeshare.mlr,plotit=T) # plotit=T is the default setting
\end{lstlisting}
\begin{center}\begin{figure}[htbp]
  \centering
  \includegraphics[scale=0.2]{d6}
  \caption{}
  \label{}
\end{figure}\end{center}
改变范围
\begin{lstlisting}[language=R]
boxcox(bikeshare.mlr,plotit=T,lambda=seq(0.1,1,by=0.05)) # zoom-in
\end{lstlisting}
\begin{center}\begin{figure}[htbp]
  \centering
  \includegraphics[scale=0.2]{d7}
  \caption{}
  \label{}
\end{figure}\end{center}
Find the $\lambda$ that maximizes the $Log-likelihood$.
\begin{lstlisting}[language=R]
names(bikes.transformation)
## [1] "x" "y"
bikes.transformation$x[1:10]
##  [1] -2.000000 -1.989975 -1.979950 -1.969925 -1.959900 -1.949875 -1.939850
##  [8] -1.929825 -1.919799 -1.909774
bikes.transformation$y[1:10]
##  [1] -372422.1 -370582.3 -368742.9 -366904.0 -365065.6 -363227.6 -361390.0
##  [8] -359553.0 -357716.4 -355880.2
bikes.transformation$x[bikes.transformation$y == 
max(bikes.transformation$y)] # lambda.hat
## [1] 0.2656642
\end{lstlisting}
$$\hat{\lambda}=0.2656642$$
Construct a Confidence Interval for $\lambda$ as follows:
$$\{\lambda: L(\lambda)>L(\hat{\lambda})-\frac{1}{2}\chi_1^2(1-\alpha)  \}$$
\begin{lstlisting}[language=R]
tmp=bikes.transformation$x[bikes.transformation$y > 
      max(bikes.transformation$y) - qchisq(0.95, 1)/2];
range(tmp) # 95% CI. Read Chapter 9 in the Faraway textbook for details.
## [1] 0.2556391 0.2756892
\end{lstlisting}



\subsubsection{Summary of Diagnostic Plots}
\begin{lstlisting}[language=R]
fit=lm(Y~., data=rat)
par(mfrow=c(2,2))
plot(fit)
\end{lstlisting}
\begin{center}\begin{figure}[htbp]
  \centering
  \includegraphics[scale=0.3]{rat2.png}
  \caption{}
  \label{}
\end{figure}\end{center}


\subsection{Collinearity}
\begin{lstlisting}[language=R]
library(faraway)
# 提取数据seatpos
data(seatpos)
attach(seatpos)

# Fit the FULL model
position.full=lm(hipcenter ~ ., seatpos)
x = model.matrix(position.full)[,-1]
# remove the column that corresponds to the intercept
\end{lstlisting}

\subsubsection{Standardized each colum of $X$}
\begin{lstlisting}[language=R]
x = model.matrix(position.full)[,-1] #去除第一列(即intercept)
x = x - matrix(apply(x,2, mean), 38,8, byrow=TRUE)
x = x / matrix(apply(x, 2, sd), 38,8, byrow=TRUE)
apply(x,2,mean)
##           Age        Weight       HtShoes            Ht        Seated 
## -2.193512e-17  2.810252e-16  9.566280e-16  1.941574e-16 -1.073010e-15 
##           Arm         Thigh           Leg 
## -1.070022e-16  8.909895e-17 -9.114182e-17
apply(x,2,var)
##     Age  Weight HtShoes      Ht  Seated     Arm   Thigh     Leg 
##       1       1       1       1       1       1       1       1
\end{lstlisting}



\subsubsection{Condition number of the $\mathbf{X}^T \mathbf{X}$ matrix}
\begin{lstlisting}[language=R]
e = eigen(t(x) %*% x) # compute the eigenvalues
sqrt(e$val[1]/e$val[8])
## [1] 59.7662
\end{lstlisting}
The condition number is 59.77, larger than 30, so we conclude that collinearity is present.

\subsubsection{Variance Inflation Factor (VIF)}
\begin{lstlisting}[language=R]
# Variance Inflation Factor (VIF)
round(vif(x), dig=2)
##     Age  Weight HtShoes      Ht  Seated     Arm   Thigh     Leg 
##    2.00    3.65  307.43  333.14    8.95    4.50    2.76    6.69
\end{lstlisting}
\begin{lstlisting}[language=R]
sqrt(307.43)
## [1] 17.53368
\end{lstlisting}
Note that the se for the coef associated with $HtShoes$ is $17.5$ times larger than it would have been without collinearity.

\subsubsection{Pairwise correlations and partial F-tests}
\begin{lstlisting}[language=R]
cor(Seated+Thigh, Ht)
## [1] 0.9389819
cor(Seated+Leg, Ht)
## [1] 0.965607
cor(Seated+Arm, Ht)
## [1] 0.9465523
\end{lstlisting}
\begin{lstlisting}[language=R]
position.red1 = lm(hipcenter ~ Age + Weight + Ht + Seated, data=seatpos)
position.red2 = lm(hipcenter ~ Ht, data=seatpos)
anova(position.red2, position.red1)
## Analysis of Variance Table
##
## Model 1: hipcenter ~ Ht
## Model 2: hipcenter ~ Age + Weight + Ht + Seated
##   Res.Df   RSS Df Sum of Sq      F Pr(>F)
## 1     36 47616
## 2     33 44774  3    2841.6 0.6981 0.5599
\end{lstlisting}
Based on the $F-$test provided in the $ANOVA$ table, we conclude that the reduced model with $Ht$ as the only variable is better than the model that includes $Age, Weight, Ht$ and $Seated$.



\section{Time Series}
\subsection{First Order Autoregressive Model}
\begin{lstlisting}[language=R]
library(nlme)
lm.sales.cor = gls(company_sales~industry_sales,
correlation = corAR1(form= ~ index), data=sales)

summary(lm.sales.cor)
## Generalized least squares fit by REML
##   Model: company_sales ~ industry_sales 
##   Data: sales 
##         AIC       BIC   logLik
##   -31.74311 -28.18162 19.87156
## 
## Correlation Structure: AR(1)
##  Formula: ~index 
##  Parameter estimate(s):
## Phi 
##   1 
## 
## Coefficients:
##                     Value Std.Error  t-value p-value
## (Intercept)    -0.3189197 2041.6945 -0.00016  0.9999
## industry_sales  0.1684878    0.0051 33.06272  0.0000
## 
##  Correlation: 
##                (Intr)
## industry_sales 0     
## 
## Standardized residuals:
##           Min            Q1           Med            Q3           ##           Max
## -9.036061e-05 -4.156415e-05 -3.013053e-06  8.080346e-05  
##  1.091922e-04
## 
## Residual standard error: 2041.694 
## Degrees of freedom: 20 total; 18 residual
\end{lstlisting}





\section{Polynomials Regression}
\subsection{Orthogonal Polynomials}
\begin{lstlisting}[language=R]
  poly(.)
\end{lstlisting}
\subsection{B-Splines Basis}
\begin{lstlisting}[language=R]
  bs(x, df, knots, degree=3, intercept=FALSE)
  # x是数据
  # df是输出的design matrix的columns数,和真正的df无关
  # intercept=FALSE, df=真df-1
  # intercept=TRUE, df=真df
  # knots=k,代表k是那个唯一的knot,所以knot数是1,无论k多大
  new.knots= c(1/6, 3/6, 5/6)
  bs(x, knots=new.knots, intercept=TRUE)
  bs(x, knots=quantile(x, c(1/3,2/3)), intercept=TRUE)
\end{lstlisting}

\subsection{Natural Cubic Splines}
\begin{lstlisting}[language=R]
  ns(x, df, knots, Boundary.knots, degree=3, intercept=FALSE)
  # knots只表示interior knots,还有俩boundary knots。
  # 所以 真df=#knots+2
  # 其他一样
  ns(x, knots=new.knots, Boundary.knots=c(0,1), intercept=TRUE)
\end{lstlisting}



\section{Categorical ANOVA}
\subsection{Effect tests}
When the levels of the categorical variable are in text (instead of number), R assigns 0 and 1 in alphabetical order: 0 first and 1 second.

\begin{lstlisting}[language=R]
  quest.full=lm(rate~lot.size*color,quest.data)
  anova(quest.full)
  ##Analysis of Variance Table
  ##Response: rate
  ##               Df Sum Sq Mean Sq F value  Pr(>F)  
  ##lot.size        1 43.226  43.226  7.1024 0.01765 *
  ##color           1 20.052  20.052  3.2947 0.08955 .
  ##lot.size:color  1  0.166   0.166  0.0272 0.87111  
  ##Residuals      15 91.293   6.086                  
  ##---
  ##Signif. codes:  0 ‘***’ 0.001 ‘**’ 0.01 ‘*’ 0.05 ‘.’ 0.1 ‘ ’ 1
\end{lstlisting}
第一行 intercept only vs. intercept+lot.size\\
第二行 intercept+lot.size vs. intercept+lot.size+color\\
第三行 intercept+lot.size+color vs. intercept+lot.size+color+lot.size*color

\subsection{ANOVA Type III}
This type tests for the presence of an effect given that both the other effects are in the model.
\begin{center}\begin{figure}[htbp]
  \centering
  \includegraphics[scale=0.3]{type3}
  \caption{}
  \label{}
\end{figure}\end{center}






\section{Variation Selection}
\subsection{Leap and Bounds method}
Use function $regsubsets$ from library leaps to evaluate different scores for sub-sets of models up to size p (including the intercept).
\begin{lstlisting}[language=R]
  library(leaps)
  Hitters.leaps=regsubsets(Salary~.,data=data.reg,nvmax=16)
  rs=summary(Hitters.leaps)
  rs$adjr2
  rs$which[which.max(rs$adjr2),]
  rs$cp
  rs$bic
  n=dim(data.reg)[1]
  m=2:17
  Aic=n*log(rs$rss/n)+2*m
\end{lstlisting}
\subsection{Searching algorithm based on AIC and BIC}
Use function step from the $stats$ library to apply searching algorithms based on the AIC (default) or BIC criteria (k = log(n)). The option direction=both uses the Stepwise searching algorithm. You can also use the options: $direction=forward$ and $direction=backward$.
\begin{lstlisting}[language=R]
  step(full.model, direction="both")
  step(full.model, direction="both", k=log(n))
  We can also use direction=forward and direction=backward
\end{lstlisting}



\section{Shrinkage Methods}
\subsection{PCR, PCA}
Function prcomp can be used to calculate the PCs and extract the $\lambda$’s squared-roots (sdev) and eigenvectors (rotation) of the variance-covariance matrix:
\begin{center}\begin{figure}[htbp]
  \centering
  \includegraphics[scale=0.5]{prcomp}
  \caption{}
  \label{}
\end{figure}\end{center}

The pcr function (principal component regression) from the pls package has useful features for prediction and cross-validation. We can easily calculate the RMSE for the training set and the testing set.
\begin{lstlisting}[language=R]
  library(pls)
  modpcr<-pcr(fat ~ ., data=trainmeat,ncomp=50)
  #summary(modpcr)
  #RMSE with 4 PCAs
  rmse(predict(modpcr,ncom=4),trainmeat$fat) 
  ## [1] 4.064745
  rmse(predict(modpcr,testmeat,ncomp=4),testmeat$fat)
  ## [1] 4.533982
\end{lstlisting}
You can use the function RMSEP instead, to select the number of PC’s that minimize the 10-fold Cross-Validation error. The resulting Cross-Validation error is < 2.5
\begin{lstlisting}[language=R]
  set.seed(123)
  # Minimize RMSE using function RMSEP
  pcrmse<-RMSEP(modpcr,newdata=testmeat)
  plot(pcrmse)  
\end{lstlisting}
\begin{center}\begin{figure}[htbp]
  \centering
  \includegraphics[scale=0.3]{fat}
  \caption{}
  \label{}
\end{figure}\end{center}



\section{One/Two Way ANOVA}
\subsection{Pairwise comparisons}
Construct $90\%$ family confidence intervals for all pairwise comparisons of classroom environments.
\begin{center}\begin{figure}[htbp]
  \centering
  \includegraphics[scale=0.3]{aov}
  \caption{}
  \label{}
\end{figure}\end{center}



\section{ Experimental Designs}
\subsection{ Paired t-test}
\begin{lstlisting}[language=R]
t.test(shoes$A-shoes$B)
\end{lstlisting}

\subsection{}
We use the drop1 function instead of anova, because of the lack of orthogonality due to the incompleteness of the design.
\begin{lstlisting}[language=R]
  lmodbibd <- lm(gain~block+treat, rabbit)
  drop1(lmodbibd, test="F")
  ## Single term deletions
  ## 
  ## Model:
  ## gain ~ block + treat
  ##        Df Sum of Sq    RSS     AIC F value    Pr(>F)    
  ## <none>              150.77  78.437                      
  ## block   9    595.74 746.51 108.426  6.5854 0.0007602 ***
  ## treat   5    158.73 309.50  90.013  3.1583 0.0381655 *  
  ## ---
  ## Signif. codes:  0 '***' 0.001 '**' 0.01 '*' 0.05 '.' 0.1 ' ' 1
\end{lstlisting}

































\section{画图}

\subsection{$2\times2$的画布}
\begin{lstlisting}[language=R]
par(mfrow=c(2,2))
\end{lstlisting}
\subsection{plot点图,接上节 (bikeshares)}
\begin{lstlisting}[language=R]
par(mfrow=c(2,2))
# Plot of t1 vs. cnt
plot(bikeshares.reg$t1, bikeshares.reg$cnt, xlab="Real
Temperature in C", ylab="New Bike Shares")
# Plot of t2 vs. cnt
plot(bikeshares.reg$t2, bikeshares.reg$cnt, xlab=" Feels
Like Temperature in C", ylab="New Bike Shares")
# Plot of t1 vs. t2
plot(bikeshares.reg$t1, bikeshares.reg$t2, xlab=" Feels
Like Temperature in C", ylab="Real Temperature in C")
# Plot of hum vs. t1
plot(bikeshares.reg$hum, bikeshares.reg$t1, xlab="Humidity",
ylab="Real Temperature in C")
\end{lstlisting}
\begin{center}\begin{figure}[htbp]
    \centering
    \includegraphics[scale=0.3]{Bike.png}
    \caption{}
    \label{}
\end{figure}\end{center}

\subsection{ggplot}
\begin{lstlisting}[language=R]
library(ggplot2)
\end{lstlisting}
\subsubsection{Plot the regression line along with the connected “point-wise” confidence intervals (galton)}
\begin{lstlisting}[language=R]
library(ggplot2)
ggplot(galton, aes(MP,AH)) + geom_point() + geom_smooth(method=lm)
\end{lstlisting}
\begin{center}\begin{figure}[htbp]
    \centering
    \includegraphics[scale=0.5]{plot1.png}
    \caption{}
    \label{}
\end{figure}\end{center}
\subsubsection{给颜色取名,竖直的线,坐标label}
\begin{lstlisting}[language=R]
# Form the data frame for plotting
ggplot(data=NULL, aes(x=0:56)) + 
  geom_line(aes(y=myCI[,1], colour="LSfit"), size=1) + 
  geom_line(aes(y=myCI[,2], colour="90% CI"), size=1) +
  geom_line(aes(y=myCI[,3], colour="90% CI"), size=1) +
  geom_line(aes(y=myPI[,2], colour="90% PI"), size=1, linetype=2)+
  geom_line(aes(y=myPI[,3], colour="90% PI"), size=1, linetype=2)+ 
  scale_colour_manual("", values=c("LSfit" = "black",
                               "90% CI" = "blue", 
                               "90% PI"="red"))+ 
  xlab("wind_speed") +ylab("bike shares")+ 
  geom_vline(xintercept = mean(bikeshares.reg$wind_speed),
  colour="purple", size=1, linetype=3)
\end{lstlisting}
\begin{center}\begin{figure}[htbp]
    \centering
    \includegraphics[scale=0.5]{week501.png}
    \caption{}
    \label{}
\end{figure}\end{center}

































































\end{document}