\chapter{Predicting Long-term Outcomes}
\section{Surrogate Index: \cite{athey2019surrogate}}
Define two samples $N_E$ (Experimental) and $N_O$ (Observational), $N=N_E+N_O$, where we use $P_i\in\{O,E\}$ as a binary indicator of whether an individual $i$ is in $N_E$ or $N_O$.

For each individual, we have
\begin{center}
    \begin{tabular}{ccc}
        \hline
            $X_i$& vector & pre-treatment covariates for each unit\\
        \hline
            $W_i$& $\mathbf{W}\in\{0,1\}$ & binary treatment for each unit\\
        \hline
            $Y_i$& scalar & primary outcome (unobserved if $P_i=E$)\\
            $S_i$& vector &surrogates (intermediate outcomes)\\
        \hline
    \end{tabular}
\end{center}
The primary outcome $Y_i$ is unobservable for individuals in $N_E$ (i.e., in experimental sample, $P_i=E$).

Following the potential outcome framework or Rubin Causal Model, each individual has two pairs of potential outcomes:
\begin{equation}
    \begin{aligned}
        Y_i\equiv Y_i(W_i)=\left\{\begin{matrix}
            Y_i(0),& W_i=0\\
            Y_i(1),& W_i=1
        \end{matrix}\right.,\ S_i\equiv S_i(W_i)=\left\{\begin{matrix}
            S_i(0),& W_i=0\\
            S_i(1),& W_i=1
        \end{matrix}\right.
    \end{aligned}
    \nonumber
\end{equation}
Overall, the units are characterized by $(Y_i(0),Y_i(1),S_i(0),S_i(1),X_i,W_i,P_i)$.