\documentclass[11pt]{elegantbook}
\usepackage{graphicx}
%\usepackage{float}
\definecolor{structurecolor}{RGB}{40,58,129}
\linespread{1.6}
\setlength{\footskip}{20pt}
\setlength{\parindent}{0pt}
\newcommand{\argmax}{\operatornamewithlimits{argmax}}
\newcommand{\argmin}{\operatornamewithlimits{argmin}}
\elegantnewtheorem{proof}{Proof}{}{Proof}
\elegantnewtheorem{claim}{Claim}{prostyle}{Claim}
\DeclareMathOperator{\col}{col}
\title{Choice Modeling}
\author{Wenxiao Yang}
\institute{Haas School of Business, University of California Berkeley}
\date{2024}
\setcounter{tocdepth}{2}
\extrainfo{All models are wrong, but some are useful.}

\cover{cover.png}

% modify the color in the middle of titlepage
\definecolor{customcolor}{RGB}{32,178,170}
\colorlet{coverlinecolor}{customcolor}
\usepackage{cprotect}

\addbibresource[location=local]{reference.bib} % bib

\begin{document}
\maketitle

\frontmatter
\tableofcontents

\mainmatter






\chapter{}
\section{Random Utility Models}
Let $j=1,...,J_i$ index the ``inside goods'' available to consumer $i$ while $j=0$ denotes the outside good. A consumer's choice set is characterized by $J_i$ and a set $\chi_i$, which may include
\begin{enumerate}[$\circ$]
    \item observed characteristics of consumer $i$,
    \item observed characteristics of goods (including prices),
    \item observed characteristics of the local market,
    \item and characteristics of the market or goods that are unobserved to the researcher.
\end{enumerate}







\end{document}