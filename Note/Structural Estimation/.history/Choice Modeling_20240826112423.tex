\documentclass[11pt]{elegantbook}
\usepackage{graphicx}
%\usepackage{float}
\definecolor{structurecolor}{RGB}{40,58,129}
\linespread{1.6}
\setlength{\footskip}{20pt}
\setlength{\parindent}{0pt}
\newcommand{\argmax}{\operatornamewithlimits{argmax}}
\newcommand{\argmin}{\operatornamewithlimits{argmin}}
\elegantnewtheorem{proof}{Proof}{}{Proof}
\elegantnewtheorem{claim}{Claim}{prostyle}{Claim}
\DeclareMathOperator{\col}{col}
\title{Choice Modeling}
\author{Wenxiao Yang}
\institute{Haas School of Business, University of California Berkeley}
\date{2024}
\setcounter{tocdepth}{2}
\extrainfo{All models are wrong, but some are useful.}

\cover{cover.png}

% modify the color in the middle of titlepage
\definecolor{customcolor}{RGB}{32,178,170}
\colorlet{coverlinecolor}{customcolor}
\usepackage{cprotect}

\addbibresource[location=local]{reference.bib} % bib

\begin{document}
\maketitle

\frontmatter
\tableofcontents

\mainmatter






\chapter{Random Utility Models}
\section{Random Utility Models}
Let $j=1,...,J_i$ index the ``inside goods'' available to consumer $i$ while $j=0$ denotes the outside good. A consumer's choice set is characterized by $J_i$ and a set $\chi_i$, which may include
\begin{enumerate}[$\circ$]
    \item observed characteristics of consumer $i$,
    \item observed characteristics of goods (including prices),
    \item observed characteristics of the local market,
    \item and characteristics of the market or goods that are unobserved to the researcher.
\end{enumerate}
Each consumer $i$ has a (conditional indirect) utility $u_{ij}$ for good $j$. Consumer knows her utilities for all goods and chooses the good with the highest utility.

In this model, the heterogeneity of consumer preferences is modeled by random utilities:
\begin{definition}[Random Utility Model]
    Given the choice set $(J_i, \chi_i)$, each consumer's utility vector $(u_{ij})_{j=0,1,...,J_i}$ is an independent draw from a joint distribution $F_u(\cdot\mid J_i, \chi_i)$.
\end{definition}
Since only the ordinal ranking of goods matters for a consumer's behavior, we can normalize the location and scale of each consumer's utility vector without loss of generality. We assume that ``ties'' ($u_{ij}=u_{ik}$ for some $j\neq k$) occur with probability zero in the distribution $F_u(\cdot\mid J_i, \chi_i)$. Then, we can represent consumer $i$'s choice with the vector $(q_{i1},...,q_{iJ_i})$, where
\begin{equation}
    \begin{aligned}
        q_{ij}=\mathbf{1}\{u_{ij}\geq u_{ik},\forall k\}
    \end{aligned}
    \nonumber
\end{equation}
The consumer-specific choice probabilities are then given by
\begin{equation}
    \begin{aligned}
        s_{ij}:=\mathbb{E}[q_{ij}\mid J_i,\chi_i]=\int_{\mathcal{A}_{ij}}d F_u\left(u_{i0},...,u_{iJ_i}\mid J_i, \chi_i\right)
    \end{aligned}
    \nonumber
\end{equation}
where $\mathcal{A}_{ij}=\{(u_{i0},...,u_{iJ_i})\in\mathbb{R}^{J_i+1}\mid u_{ij}\geq u_{ik},\forall k\}$.


\section{The Canonical Model}
Discrete choice demand models are frequently formulated using a parametric random utility specification such as
\begin{equation}
    \begin{aligned}
        u_{ijt}=x_{jt}\beta_{it}-\alpha_{it}p_{jt}+\xi_{jt}+\epsilon_{ijt}
    \end{aligned}
    \nonumber
\end{equation}
for $j>0$, with $u_{i0t}=\epsilon_{i0t}$.




















\end{document}