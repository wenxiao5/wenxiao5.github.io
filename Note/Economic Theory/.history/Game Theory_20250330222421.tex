\documentclass[11pt]{elegantbook_2}
\usepackage{graphicx}
%\usepackage{float}
\definecolor{structurecolor}{RGB}{40,58,129}
\linespread{1.6}
\setlength{\footskip}{20pt}
\setlength{\parindent}{0pt}
\newcommand{\argmax}{\operatornamewithlimits{argmax}}
\newcommand{\argmin}{\operatornamewithlimits{argmin}}
\elegantnewtheorem{claim}{Claim}{prostyle}{Claim}
\DeclareMathOperator{\col}{col}
\title{Game Theory}
\author{Wenxiao Yang}
\institute{Haas School of Business, University of California Berkeley}
\date{2025}
\setcounter{tocdepth}{2}
\extrainfo{Mind offline, notes online.}

\cover{HZ.jpg}

% modify the color in the middle of titlepage
\definecolor{customcolor}{RGB}{250,255,240}
\colorlet{coverlinecolor}{customcolor}
\usepackage{cprotect}


\bibliographystyle{apalike_three}

\begin{document}
\maketitle

\frontmatter
\tableofcontents

\mainmatter




\chapter{Game Theory}
Based on
\begin{enumerate}[$\circ$]
    \item "Kreps, D. M., \& Sobel, J. (1994). Signalling. \textit{Handbook of game theory with economic applications}, 2, 849-867."
    \item Mas-Colell, Whinston, and Green, Microeconomic Theory, Oxford University Press (1995).
    \item UIUC ECON 530 21Fall, Nolan H. Miller
    \item UC Berkeley ECON 201A 23Fall, 201B 24Spring
    \item UC Berkeley MATH 272 23Fall, Alexander Teytelboym
    \item  Jehle, G., Reny, P.: Advanced Microeconomic Theory. Pearson, 3rd ed. (2011). Ch. 6.
    \item MIT 14.16 Strategy and Information, Mihai Manea
\end{enumerate}



\section{Basic Game Theory}
\subsection{Action and Domination Theorem}
Let $A$ be the finite set of possible actions and $\Omega$ be the finite set of possible states. A function can map the action and state to a value, $u(a,\omega)$. It can be represented by $\vec{u}(a)=\{u(a,\omega)\}_{\omega\in\Omega}$. It is common in game theory to assume the utility function is given or known.

A \textbf{mixed action} is a probability distribution over $A$, $\sigma\in\Delta(A)$.

A \textbf{belief} of the agent is a probability distribution over $\Omega$, $\mu\in\Delta(\Omega)$.

\begin{definition}[Optimal and Justifiable Mixed Action]
    %\normalfont
    A mixed action $\sigma\in\Delta(A)$ is \textbf{optimal} given $\mu\in\Delta(\Omega)$ if $$\mathbb{E}_\mu u(\sigma,\tilde{\omega})\geq \mathbb{E}_\mu u(\sigma',\tilde{\omega}),\ \forall \sigma'\in \Delta(A)$$
    A mixed action $\sigma\in\Delta(A)$ is \textbf{justifiable} if it is optimal for some belief $\mu\in\Delta(\Omega)$.
\end{definition}

\begin{definition}[Dominant and Dominated Action]
    %\normalfont
    A mixed action $\sigma\in\Delta(A)$ is \textbf{dominant} if $$u(\sigma,\omega)>u(\sigma',\omega),\ \forall \omega\in \Omega, \sigma'\in \Delta(A),\sigma\neq\sigma'$$
    A mixed action $\sigma\in\Delta(A)$ is \textbf{dominated} if $$u(\sigma,\omega)<u(\sigma',\omega),\ \forall \omega\in \Omega, \text{ and for some } \sigma'\in \Delta(A)$$
    In this case we say $\sigma'$ dominates $\sigma$.
\end{definition}

\begin{theorem}[Domination Theorem: Justifiable $=$ Not Dominated]
    A mixed action is justifiable \underline{if and only if} it is not dominated.
\end{theorem}
\begin{proof}
    $\Rightarrow$ is easily proved by the definition. We focus on proving $\Leftarrow$:
    
    Let $\mathcal{U}=\{\vec{u}(\sigma):\sigma\in\Delta(A)\}$ and $\sigma^*$ be an undominated mixed action. Then, we have $\mathcal{U}\cap(\{\vec{u}(\sigma^*)\}+\mathbb{R}_{++}^\Omega)=\emptyset$. Because $\mathcal{U}$ and $\{\vec{u}(\sigma^*)\}+\mathbb{R}_{++}^\Omega$ are disjoint, convex, and nonempty, we can use the Separating Hyperplane Theorem \ref{SHT}: $\exists p\in \mathbb{R}^\Omega,p\neq 0$ such that $p\cdot a\leq p\cdot b, \forall a\in\mathcal{U}, b\in (\{\vec{u}(\sigma^*)\}+\mathbb{R}_{++}^\Omega)$.

    \begin{claim}
        $p\cdot \vec{u}(\sigma)\leq p\cdot \vec{u}(\sigma^*), \forall \sigma\in\Delta(A)$.
    \end{claim}
    \begin{proof}
        For any positive number $m$, $\vec{u}(\sigma^*)+(\frac{1}{m},....,\frac{1}{m})\in \{\vec{u}(\sigma')\}+\mathbb{R}_{++}^\Omega$. So, for any $\sigma\in\Delta(A)$, $p\cdot \vec{u}(\sigma)\leq p\cdot\left(\vec{u}(\sigma^*)+(\frac{1}{m},....,\frac{1}{m})\right)$. By taking limit, $p\cdot \vec{u}(\sigma^*)=\lim_{m \rightarrow \infty}p\cdot\left(\vec{u}(\sigma^*)+(\frac{1}{m},....,\frac{1}{m})\right)\geq p\cdot \vec{u}(\sigma)$.
    \end{proof}
    \begin{claim}
        $p>0$.
    \end{claim}
    \begin{proof}
        Prove by the contradiction. Suppose $p_\omega<0$ for some $\omega\in\Omega$. Let $z=(\epsilon,...,\epsilon)+M\mathbb{1}_\omega, M>0,\epsilon>0$. So, $\vec{u}(\sigma^*)+z\in (\{\vec{u}(\sigma^*)\}+\mathbb{R}_{++}^\Omega)$. We have $p\cdot\vec{u}(\sigma^*)\leq p\cdot (\vec{u}(\sigma^*)+z)$ by the result of SHT. There is a contradiction since $p_\omega<0$. So, we have $p\geq 0$. Because $p\neq 0$, $p>0$ is proved.
    \end{proof}
    Finally, we normalize $p$ to $\mu=\frac{1}{\sum_{\omega}p_\omega}p$. Then, $\sigma^*$ is optimal for the belief $\mu$, which means $\sigma^*$ is justifiable.
\end{proof}










\subsection{Extensive Game}
\begin{definition}[History]
    %\normalfont
    The sequences of actions are called \textbf{histories}. $h'=(\underbrace{a_1,...,a_n}_{h: \text{prefix of }h'},a_{n+1},...)\in H$. We call $h'$ is the \textbf{continuation} of $h$. $h$ is a \textbf{terminal} of $H$ if there is no continuation of $h$ in $H$. ($\emptyset\in H$.)
\end{definition}

\begin{definition}[Extensive form Perfect Information Game]
    %\normalfont
    Am extensive form game with prefect information is defined as $G=\{N,A,H,Z,P,O,o,\succ_{n\in N}\}$, where $N$ is the set of players, $A$ is the set of actions, $H$ is the set of all histories, $Z$ is the set of all histories that are terminals, $P:H/Z \rightarrow N$ is a mapping from a non-terminal histories to a player (who moves after a non-terminal history), $O$ is the set of outcomes, and $o$ is a function from $Z$ to $O$.\\
    A PIG is \underline{finite horizon} if there is a bound on the length of its histories.
\end{definition}


We denote $A(h)$ as the actions available to player $P(h)$ after a history $h$.

Let $H_i=\{h\in H/Z:i=P(h)\}$ be the set of histories that player $i$ moves after.

\begin{definition}[Strategy]
    %\normalfont
    A \textbf{strategy} is defined as a function $s_i:H_i \rightarrow A$ for which $s_i(h)\in A(h),\forall h\in H_i$. Let $S_i$ be the set of all strategies available to the player $i$. A \textbf{strategy profile} is a collection of strategy $s=(s_i)_{i\in N}$.
\end{definition}



\begin{definition}[Subgame]
%\normalfont
    A \textbf{subgame} of a PIG $G=\{N,A,H,Z,P,O,o,\succ_{n\in N}\}$ is a game (a PIG) that starts after a given finite history $h\in H$. Formally, the subgame $G(h)$ associated with $h=(h_1,...,h_n)\in H$ is $G(h)=\{N,A,H_h,Z,P_h,O,o_h,\succ_{n\in N}\}$, where
    \begin{equation}
        \begin{aligned}
            H_h=\{(a_1,a_2,...):(h_1,...,h_n,a_1,a_2,...)\in H\}\\
            o_h(h')=o(hh'), P_h(h')=P(hh')
        \end{aligned}
        \nonumber
    \end{equation}
    A strategy $s$ of $G$ defines a strategy $s_h$ of $G(h)$ by $s_h(h')=s(hh')$.
\end{definition}

\begin{definition}[Subgame Perfect Equilibrium (SPNE)]
    %\normalfont
    A \textbf{subgame perfect equilibrium}[SPNE/SPE] of $G$ is a strategy profile $\sigma^*$ such that for every subgame $G(h)$ it holds that $\sigma^*$ (more precisely, its restriction to $H_h$) is a Nash equilibrium of $\Gamma(h)$.
\end{definition}
Over time $t=1,...,T$, the set of actions is $A^T$.
\begin{lemma}
    $\sigma^*$ is a SPNE \underline{iff} its restriction $\sigma^*(\cdot|h)$ is SPNE of $\Gamma(h)$, for all subgames $\Gamma(h)$.
\end{lemma}

\subsection{Infinite Horizon Games}
\begin{definition}[Finite Horizon]
    %\normalfont
    An extensive-form game has a \textbf{finite horizon} if there is a bound on the length of any history in $Z$.
\end{definition}
In these games, SPNE can be found through ``generalized backwards induction.''

\begin{definition}[Continuous At Infinity Utility Function]
    %\normalfont
    A utility function $v:A^T \rightarrow \mathbb{R}$ is \textbf{continuous at infinity} if
    \begin{equation}
        \begin{aligned}
            \lim_{t \rightarrow \infty}\sup\{|v(\hat{h})-v(h)|:h,\hat{h}\in A^T\}=0
        \end{aligned}
        \nonumber
    \end{equation}
    That is, eventually, a negligible effect on utility.
\end{definition}

\begin{definition}[Discounted Utility]
%\normalfont
We focus on additively separable models with a sub-utility $u:A \rightarrow \mathbb{R}$ that is the same for each $t$. We model a \textbf{discounted utility} by assuming a discount factor $\delta\in[0,1)$:
\begin{equation}
    \begin{aligned}
        v(a_1,...)=(1-\delta)\sum_{t=1}^\infty \delta^{t-1}u(a_t)
    \end{aligned}
    \nonumber
\end{equation}
\end{definition}
Can think of $\delta$ as the probability that the game will go on for another period, and $1-\delta$ the prob that the game will end.

"Ending" events are independent: A geometric distribution.

The probability that the game ends at time $t$ is $(1 - \delta)\delta^{t-1}$ and $\sum(1-\delta)\delta^t u(a_t)$ becomes expected utility.

\begin{proposition}
    $v(a_1,...)=(1-\delta)\sum_{t=1}^\infty \delta^{t-1}u(a_t)$ is continuous at infinity.
\end{proposition}


\subsection{Repeated Games}
Let $G_0=\left(N,\{A_i:i\in N\},\{u_i:i\in N\}\right)$ be a normal-form game. We let $A=\Pi_{i\in N}A_i$. We only consider games in which $A$ is compact and each $u_i$ is continuous.

Let $T$ be the number of periods (or repetitions) of the repeated game. $T$ can be either finite or infinite. $G_0$ is the stage game.

\begin{definition}[Repeated Game]
    %\normalfont
    A $T$-repeated, $\delta$-discounted, game of $G_0$ is an extensive form game $\Gamma=\left(N,A,H,P,\{u_i\}_{i\in N},\delta\right)$, where
    \begin{enumerate}
        \item $P(h)=N$ for all $h\in H/Z$.
        \item The set of histories $H$ has terminal histories $Z=A^T$; recall that $Z$
        uniquely determines $H$ as the set of all prefixes of $Z$.
        \item $v_i:Z \rightarrow \mathbb{R}$ is a payoff function of the form
        \begin{equation}
            \begin{aligned}
                v((a_t)_{t=1}^T)=\left\{\begin{matrix}
                    (1-\delta)\sum_{t=1}^\infty \delta^{t-1}u(a_t),&\delta\in(0,1)\\
                    \sum_{t=1}^T u_i(a_t),&\delta=1
                \end{matrix}\right.
            \end{aligned}
            \nonumber
        \end{equation}
    \end{enumerate}
\end{definition}
Given a strategy profile $s$, we obtain a terminal history $z \in Z$, and (abusing notation) write $v_i(s)$ for $v_i(z)$.

We call $(v_1(s),...,v_n(s))$ the \textbf{payoff profile} associated with $s$.

%The set $\mathcal{F}=\{(u_1(v),...,u_n(v)):v\in\Delta(A)\}$ is the set of \textbf{feasible payoffs}, which is a convex hull of $\{(u_1(a),...,u_n(a)):a\in A\}$.

\begin{definition}[(Profitable) One-Stage Deviation]
    %\normalfont
    A \textbf{one-stage deviation} from $\sigma$ by $i$ is a strategy $\sigma'_i$ that coincides with $\sigma_i$ everywhere except at a single history.\\
    A \textbf{profitable one-stage deviation} from $\sigma$ by $i$ is a strategy $\sigma'_i$ that coincides with $\sigma_i$ at a single history $h$, and for which $v_i(\sigma'_i,\sigma_{-i}|h)>v_i(\sigma_i,\sigma_{-i}|h)$.
\end{definition}

\begin{proposition}[SPNE$\Leftrightarrow$No Profitable One-Stage Deviation]
    Let $\Gamma$ be a discounted, infinitely repeated game. Then a strategy-profile is a SPNE of $\Gamma$ \underline{iff} no player has a profitable one-stage deviation.
\end{proposition}

\begin{corollary}[NE$\Rightarrow$SPNE]
    If $s$ is a pure-strategy Nash equilibrium of $G_0$, then the strategy profile in which each player chooses $s_i$ after any history, is a SPNE of infinitely-repeated discounted $G_0$.
\end{corollary}

\begin{proposition}[Nash Reversion Folk Theorem]
    Let $\sigma$ be a Nash equilibrium of the finite stage game $G_0$. Let $v$ be a feasible payoff profile s.t. $u_i(\sigma) < v_i$ for all players $i$. Then there is $\underline{\delta}$ s.t. if $\delta\geq \underline{\delta}$ then there exists a SPNE of infinitely repeated $G_0$ with discount factor $\delta$ in which players' equilibrium payoffs is the profile $v$.
\end{proposition}
No Nash equilibrium that has higher payoffs for all players than Nash equilibrium can be obtained with relatively high discount factor in an infinite game.


\section{Refinement}
\subsection{Trembling-hand Perfect Equilibrium}
\begin{definition}[$\epsilon$-Constrained Equilibrium]
    An \textbf{$\epsilon$-constrained equilibrium} of a strategic form game is a mixed strategy profile $\sigma^\epsilon$ such that there exists $\bar{\epsilon}: \bigcup_{i \in I} S_i \rightarrow(0, \epsilon)$ such that for each player $i$,
    \begin{equation}
        \begin{aligned}
            \sigma^\epsilon_i\in\argmax_{\sigma_i \textnormal{ s.t. }\sigma_i(s_i)\geq \bar{\epsilon}(s_i)} u_i(\sigma_i,\sigma^\epsilon_{-i})
        \end{aligned}
        \nonumber
    \end{equation}
\end{definition}


\begin{definition}[Trembling-hand Perfect Equilibrium]
    A strategy profile $\sigma$ is a \textbf{trembling-hand perfect equilibrium} of a strategic form game if it is a limit of $\epsilon$-constrained equilibria as $\epsilon \rightarrow 0$.
\end{definition}

\begin{note}
    Trembling-hand perfect equilibrium implies a SPE, while a SPE is not always a trembling-hand perfect equilibrium.
\end{note}

This doesn't imply backward induction. (We do not rationalize the ``trembling-hand''.)


\subsection{Forward Induction: Burning Money (Ben-Polath and Dekel, 1992)}
Consider a game as following:
\begin{center}
    \begin{tabular}{ccc}
        \hline
            &$A_2$ &$B_2$\\
        \hline
            $A_1$& 9,6 & 0,4\\
            $B_1$& 4,0 & 6,9\\
        \hline
    \end{tabular}
\end{center}
Except the actions, the Player 1 can choose to Burn or Not Burn (i.e., lose $2.5$ units of utility), and the Player 2 can see the Player 1's action.

There are four possible strategies of Player 1:
\begin{enumerate}[(S1).]
    \item Burn and $A_1$
    \item Burn and $B_1$
    \item Not Burn and $A_1$
    \item Not Burn and $B_1$
\end{enumerate}
The potential payoffs of playing (S2) are $1.5$ and $3.5$, which is dominated by playing (S4). Then, if the Player 1 chooses Burn, he must play (S1). Thus, the Player 2 plays $A_2$, which gives $(6.5,6)$. Therefore, the Player 1 chooses Burn (i.e., (S1)) can dominate (S4). Now, we only remain two possible strategies of Player 1, (S1) and (S3). (S1) is dominated by (S3), so (S3) is the optimal strategy of Player 1.

\chapter{Signaling}
\section{Signaling Game}
\subsection{Canonical Game}
\begin{definition}[Canonical Game]
    %\normalfont
    \begin{enumerate}
        \item There are two players: $\mathbf{S}$ (sender) and $\mathbf{R}$ (receiver).
        \item $\mathbf{S}$ holds more information than $\mathbf{R}$: the value of some random variable $t$ with support $\mathcal{T}$. (We say that $t$ is the \textbf{type} of $\mathbf{S}$)
        \item Prior belief of $\mathbf{R}$ concerning $t$ are given by a probability distribution $\rho$ over $\mathcal{T}$ (common knowledge)
        \item $\mathbf{S}$ sends a \textbf{signal $s\in \mathcal{S}$} to $\mathbf{R}$ drawn from a signal set $\mathcal{S}$.
        \item $\mathbf{R}$ receives this signal, and then takes an \textbf{action} $a\in \mathcal{A}$ drawn from a set $\mathcal{A}$ (which could depend on the signal $s$ that is sent).
        \item $\mathbf{S}$'s payoff is given by a function $u: \mathcal{T}\times \mathcal{S} \times \mathcal{A} \rightarrow \mathbb{R}$ and $\mathbf{R}$'s payoff is given by a function $v: \mathcal{T}\times \mathcal{S} \times \mathcal{A} \rightarrow \mathbb{R}$.
    \end{enumerate}
\end{definition}

\subsection{Nash Equilibrium}
\begin{definition}[Strategy]
    %\normalfont
    A \textbf{behavior strategy} for $\mathbf{S}$ is given by a function $\sigma: \mathcal{T}\times\mathcal{S} \rightarrow [0,1]$ such that $\sum_s \sigma(t,s)$ for each $t$.\\
    A \textbf{behavior strategy} for $\mathbf{R}$ is given by a function $\alpha: \mathcal{S}\times\mathcal{A} \rightarrow [0,1]$ such that $\sum_a \alpha(s,a)$ for each $t$.
\end{definition}

\begin{definition}[Nash Equilibrium]
    %\normalfont
    Behavior strategies $\alpha$ and $\sigma$ form a \textbf{Nash equilibrium} if and only if
    \begin{enumerate}
        \item For all $t\in \mathcal{T}$,
        \begin{center}
            $\sigma(t,s)>0$ implies $\sum_a \alpha(s,a)u(t,s,a) = \max_{s'\in \mathcal{S}}\left(\sum_a \alpha(s',a)u(t,s',a)\right)$
        \end{center}
        \item For each $s\in \mathcal{S}$ such that $\sum_{t}\sigma(t,s)\rho(t)>0$,
        \begin{center}
            $\alpha(s,a)>0$ implies $\sum_{t}\mu(t;s)v(t,s,a) = \max_{a'}\sum_{t}\mu(t;s)v(t,s,a')$
        \end{center}
        where $\mu(t;s)$ is the $\mathbb{R}$'s posterior belief about $t$ given $s$, $\mu(t;s)=\frac{\sigma(t,s)\rho(t)}{\sum_{t'}\sigma(t',s)\rho(t')}$ if $\sum_t\sigma(t,s)\rho(t)>0$ and $\mu(t;s)=0$ otherwise.
    \end{enumerate}
\end{definition}

\begin{definition}[Separating \& Pooling Equilibrium]
    %\normalfont
    An equilibrium $(\sigma,\alpha)$ is called a \textbf{separating} equilibrium if each type $t$ sends different signals; i.e., the set $\mathcal{S}$ can be partitioned into (disjoint) sets $\{\mathcal{S}_t; t\in \mathcal{S}\}$ such that $\sigma(t, \mathcal{S}_t) = 1$. An equilibrium $(\sigma,\alpha)$ is called a \textbf{pooling} equilibrium if there is a single signal $s^*$ that is sent by all types; i.e., $\sigma(t, s^*) = 1$ for all $t\in \mathcal{T}$.
\end{definition}


\subsection{Single-crossing}

\subsubsection{Situation over real line}
Consider the situation that $\mathcal{T},\mathcal{S},\mathcal{A}\subseteq \mathbb{R}$ and $\geq$ is the usual "greater than or equal to" relationship.

\begin{enumerate}
    \item We let $\Delta \mathcal{A}$ denote the set of probability distributions on $\mathcal{A}$.
    \item For each $s\in \mathcal{S}$ and $\mathcal{T}'\subseteq \mathcal{T}$, we let $\Delta\mathcal{A}(s,T')$ be the set of mixed strategies that are the best responses by $\mathbf{R}$ to $s\in \mathcal{S}$ for some probability distribution with support $\mathcal{T}'$.
    \item For $\alpha\in \Delta\mathcal{A}$, we write $u(t,s,\alpha)\triangleq \sum_{a\in \mathcal{A}}u(t,s,a)\alpha(a)$.
\end{enumerate}

\begin{definition}[Single-crossing]
    %\normalfont
    The data of the game are said to satisfy the \textbf{single-crossing property} if the following holds: If $t\in \mathcal{T}$, $(s,\alpha)\in \mathcal{S}\times \Delta\mathcal{A}$ and $(s',\alpha')\in \mathcal{S}\times \Delta\mathcal{A}$ are such that $\alpha\in \Delta\mathcal{A}(s,\mathcal{T})$, $\alpha'\in \Delta\mathcal{A}(s',\mathcal{T})$, $s>s'$ and $u(t,s,\alpha)\geq u(t,s',\alpha')$, then for all $t'\in T$ such that $t'>t$, $u(t',s,\alpha)\geq u(t',s',\alpha')$.
\end{definition}

\section{Adverse Selection}
Consider a labor market that has many identical firms. In competitive equilibrium, firms' profits are $0$. Firms are price (wage) takers, risk-neutral, and CRS. There are continuum of workers with productivity levels $\theta\in\left[\underline{\theta},\overline{\theta}\right]$ (Assume workers work if it is indifferent for them between employment and non-employment).
\begin{enumerate}
    \item $\theta\sim F$, $F(\cdot)$ is a c.d.f. over $\left[\underline{\theta},\overline{\theta}\right]$.
    \item $N$ is the total mass of workers.
    \item Type $\theta$ worker has a reservation utility $r(\theta)$.
\end{enumerate}

\begin{enumerate}[$\circ$]
    \item Suppose the competitive equilibrium wages are $\theta=w^*(\theta)$.
    \item An allocation is denoted by $I:\left[\underline{\theta},\overline{\theta}\right] \rightarrow \{0,1\}$, where $I(\theta)=0$ denotes $\theta$ is unemployed and $I(\theta)=1$ denotes $\theta$ is employed.
    \item Aggregate welfare = sum of utilities of all participants
    \begin{equation}
        \begin{aligned}
            =N\int_{\underline{\theta}}^{\overline{\theta}} \left[I(\theta)\times\theta+[1-I(\theta)]r(\theta)\right]dF(\theta)
        \end{aligned}
        \nonumber
    \end{equation}
    Then we have the optimal allocation satisfies
    \begin{equation}
        \begin{aligned}
            I^*(\theta)\left\{\begin{matrix}
                =1,&\theta>r(\theta)\\
                \in\{0,1\}&\theta=r(\theta)\\
                =0,&\theta<r(\theta)
            \end{matrix}\right.
        \end{aligned}
        \nonumber
    \end{equation}
\end{enumerate}
In the asymmetric information case,
\begin{definition}
%\normalfont
$w$ is CE wage if $w=\mathbb{E}[\theta|r(\theta)\leq w]$.
\end{definition}

\subsection{Adverse Selection}
\begin{assumption}
    \begin{enumerate}[({A}1).]
        \item $r$ is strictly increasing in $\theta$.
        \item $F(\cdot)$ has a strictly positive density, $F(\theta)>0, \forall \theta\in \left[\underline{\theta},\overline{\theta}\right]$.
        \item $r(\theta)\leq\theta$ (outside option is worse than productivity, i.e., full employment is optimal).
    \end{enumerate}
\end{assumption}

\begin{lemma}
    Under A1-A3, $\Phi(w):=\mathbb{E}[\theta|r(\theta)\leq w]$ is well-defined, continuous, and non-decreasing.
\end{lemma}

Hence, there exists underemployment, which makes $1^{st}$ welfare theorem fails. There may exist multiple CEs, where the one with the highest wage Pareto dominates others.

\begin{example}
    Suppose $\theta\in[0,2]$, $F(\theta)=\frac{\theta}{2}$, $f(\theta)=\frac{1}{2}$, $r(\theta)=\alpha\theta,\alpha\in(0,1)$.
    \begin{equation}
        \begin{aligned}
            \mathbb{E}[\theta|r(\theta)\leq w]=\mathbb{E}\left[\theta|\theta \leq \frac{w}{\alpha}\right]=\left\{\begin{matrix}
                1,&w\geq 2\alpha\\
                \frac{1}{F\left(\frac{w}{\alpha}\right)}\int_0^{\frac{w}{\alpha}}\theta f(\theta)d\theta=\frac{w}{2\alpha},&w\leq 2\alpha
            \end{matrix}\right.
        \end{aligned}
        \nonumber
    \end{equation}
    CEs are given by $\mathbb{E}[\theta|r(\theta)\leq w]=w$. $w^*=0$ is always CE and $w^*=1$ is CE if $\alpha\leq\frac{1}{2}$.
\end{example}

\subsection{Game Theoretical Approach}
\begin{enumerate}
    \item Suppose there are two firms setting wages simultaneously.
    \item Workers observe the wages in stage 1 and make an employment decision.
\end{enumerate}
Let $W^*$ be the set of CE wages and $w^*:=\max W^*$.
\begin{lemma}\label{lemma:ad_l2}
    $\forall w'\in\left(w^*,\overline{\theta}\right]$: $\mathbb{E}[\theta|r(\theta)\leq w']<w'$.
\end{lemma}
\begin{proof}
    Suppose by the contradiction that $\exists w'\in \left(w^*,\overline{\theta}\right]$ s.t. $\mathbb{E}[\theta|r(\theta)\leq w']\geq w'$. Since $\mathbb{E}[\theta|r(\theta)\leq \overline{\theta}]<\overline{\theta}$, there must exist a $w''\in [w',\overline{\theta})$ s.t. $\mathbb{E}[\theta|r(\theta)\leq w'']=w''$ by intermediate value theorem, which contradicts to the definition of $w^*$.
\end{proof}

\begin{proposition}
    \begin{enumerate}[(i).]
        \item If $w^*>r(\underline{\theta})$ and $\exists \epsilon>0$ s.t. $\mathbb{E}[\theta|r(\theta)\leq w']>w',\forall w'\in \left(w^*-\epsilon,w^*\right)$. Then, there is a unique SPE where both firms set wage $=w^*$.
        \item If $w^*=r(\underline{\theta})$ (complete market shutdown at $w^*$), there are multiple SPE that all give the same outcome as complete market shutdown where both firms set wage $=w^*$.
    \end{enumerate}
\end{proposition}
\begin{proof}
    \begin{lemma}\label{lemma:p1}
        In all SPE, firms make zero profits.
    \end{lemma}
    \begin{proof}
        Suppose not, i.e., at least one firm makes strictly positive profits. Then, the total profits of firms $1\&2$, $$\Pi=M(\bar{w})\left[\mathbb{E}[\theta|r(\theta)\leq\bar{w}]-\bar{w}\right]>0$$
        where $\bar{w}$ is the max wage set by the two firms and $M(\bar{w})$ is the mass of workers willing to work at $\bar{w}$. At least one firm, $i$, makes profit $\leq\frac{\Pi}{2}$. Then, $i$'s profits from setting $\bar{w}+\delta$, with $\delta \rightarrow 0^+$, is higher:
        \begin{equation}
            \begin{aligned}
                &M(\bar{w}+\delta)\left[\mathbb{E}[\theta|r(\theta)\leq\bar{w}+\delta]-\bar{w}-+\delta\right]\\
                \geq &M(\bar{w})\left[\mathbb{E}[\theta|r(\theta)\leq\bar{w}+\delta]-\bar{w}-+\delta\right] \rightarrow \Pi \textnormal{ as }\delta \rightarrow 0
            \end{aligned}
            \nonumber
        \end{equation}
        Hence, the $i$ has incentive to deviate.
    \end{proof}
    \begin{lemma}\label{lemma:p2}
        In all SPE, firm $i$ sets $w_i\leq w^*, i\in\{1,2\}$.
    \end{lemma}
    \begin{proof}
        Directly given by Lemma \ref{lemma:ad_l2} and Lemma \ref{lemma:p1}.
    \end{proof}
    \begin{enumerate}[(i):]
        \item In SPE, no firm $i$ sets $w_i<w^*$: suppose $w_i<w^*$ and let $j\neq i$, take any $w'_j$ s.t. $w'_j\in\left(w_i,w^*\right)$ and $w'_j>w^*-\epsilon$. Then, $j$ gets profit: $M(w'_j)\left[\mathbb{E}[\theta|r(\theta)\leq w'_j]-w'_j\right]>0$ (by Case (i)'s conditions).
        \item By Lemma \ref{lemma:p2}, both firms set $w_i\leq w^*=r(\underline{\theta})$. Check that $\{(w_1,w_2):w_1,w_2\leq w^*\}$ is SPE wage profiles.
    \end{enumerate}
\end{proof}

\subsection{Planner Intervention}
Planner can't observe the true type $\theta$.

The planner's tools:
\begin{enumerate}
    \item Take over the firms.
    \item $w_e$, employment wage.
    \item $w_u$, unemployment wage.
\end{enumerate}
s.t. budget balanced.

\begin{definition}[Constrained Efficient]
    %\normalfont
    A CE $w$ is \textbf{constrained efficient} if it cannot be Pareto improved upon by an intervention by the planner.
\end{definition}

\begin{proposition}[$w^*:=\max W^*$ is constrained efficient]
    Let $W^*$ be the set of CE wages. $w^*:=\max W^*$ is constrained efficient.
\end{proposition}
\begin{proof}
    Note that both firms are making zero profits by the Lemma \ref{lemma:p1}. Any CE wage $w\neq w^*$ can be Pareto improved by $\{w_e=w^*,w_u=0\}$. Then, we prove $w^*$ can't be Pareto improved.
    \begin{enumerate}
        \item Case 1: if $w^*$ gives full-employment in CE, then $w^*$ is Pareto efficient.
        \item Case 1: suppose $w^*$ doesn't give full-employment in CE.
        
        Consider taking an intervention $w_e\&w_u$. Then, $\{\theta:r(\theta)+w_u\leq w_e\}=[\underline{\theta},\hat{\theta}]$ for some $\hat{\theta}\in[\underline{\theta},\overline{\theta}]$ such that
        \begin{equation}
            \begin{aligned}
                r(\hat{\theta})+w_u=w_e
            \end{aligned}
            \label{con:1}
        \end{equation}
        The budget balanced gives
        \begin{equation}
            \begin{aligned}
                w_e F(\hat{\theta})+w_u (1-F(\hat{\theta}))=\int_{\underline{\theta}}^{\hat{\theta}}\theta d F(\theta)
            \end{aligned}
            \label{con:2}
        \end{equation}
        Plug \eqref{con:1} into \eqref{con:2}:
        \begin{equation}
            \begin{aligned}
                \left\{\begin{matrix}
                    &w_u(\hat{\theta})=\int_{\underline{\theta}}^{\hat{\theta}}\theta d F(\theta)-r(\hat{\theta})F(\hat{\theta})=F(\hat{\theta})\left(\mathbb{E}[\theta|\theta\leq\hat{\theta}]-r(\hat{\theta})\right)\\
                    &w_e(\hat{\theta})=\int_{\underline{\theta}}^{\hat{\theta}}\theta d F(\theta)+r(\hat{\theta})(1-F(\hat{\theta}))
                \end{matrix}\right.
            \end{aligned}
            \nonumber
        \end{equation}
        Let $\theta^*$ be s.t. $r(\theta^*)=w^*$. Because $w^*$ is a CE price, $\mathbb{E}[\theta|\theta\leq\theta^*]=r(\theta^*)=w^*$. So, CE with $w^*$ can be implemented by $w_u(\theta^*)=0$ and $w_e(\theta^*)=w^*$.
        \begin{enumerate}
            \item If $\hat{\theta}<\theta^*$. $\underline{\theta}$ is worse off under the intervention since $w_e(\hat{\theta})<w^*$.
            \item If $\hat{\theta}>\theta^*$. $\overline{\theta}$ is worse off under the intervention since $w_u(\hat{\theta})=F(\hat{\theta})\left(\mathbb{E}[\theta|\theta\leq\hat{\theta}]-r(\hat{\theta})\right)<0$ by the Lemma \ref{lemma:ad_l2}
        \end{enumerate}
    \end{enumerate}
\end{proof}


\subsection{Signaling}\label{sec:signaling}
Suppose the worker $\theta\in[\underline{\theta},\overline{\theta}]$ can properly and costlessly reveal his type to the firms. Then,
\begin{lemma}
    All workers revel their types.
\end{lemma}
\paragraph*{Spence's Job Market Signaling Model} One worker has productivity $\theta\in\{\theta_L,\theta_H\}$ with $P(\theta_H)=\lambda$. The worker signal through his education with cost $e>0$. The education doesn't change his productivity. The payoff of the worker is the wage minus the cost:
\begin{equation}
    \begin{aligned}
        u(w,e|\theta)=w-c(e,\theta)
    \end{aligned}
    \nonumber
\end{equation}
where $c(0,\theta)=0,c_e(e,\theta):=\frac{\partial c(e,\theta)}{\partial e}>0, c_\theta(e,\theta):=\frac{\partial c(e,\theta)}{\partial \theta}<0$, and $c_{e\theta}(e,\theta):=\frac{\partial^2 c(e,\theta)}{\partial e\partial \theta}<0$ (Single-Crossing Property, the difference $c(e,\theta_L)-c(e,\theta_H)$ is increasing in $e$ (i.e., $c_e(e,\theta_L)-c_e(e,\theta_H)>0$), which means if $c(e,\theta_L)$ and $c(e,\theta_H)$ intersect as functions of $e$, they only intersect at one time.)
\begin{enumerate}[]
    \item \underline{Stage 0}: Nature chooses the $\theta\in\{\theta_L,\theta_H\}$ with $P(\theta_H)=\lambda$.
    \item \underline{Stage 1}: The worker learns $\theta$ and chooses $e(\theta)\geq 0$.
    \item \underline{Stage 2}: Firms observe $e(\theta)$. Then, they simultaneously make wage offers $w_1$ and $w_2$.
    \item \underline{Stage 3}: The worker observes $w_1,w_2$ and makes employment decision.
\end{enumerate}
Let $r(\theta_L)$ and $r(\theta_H)=0$. Let $\mu(e)\in[0,1]$ be the probability that in the beginning of stage 2, firms think that the worker is $\theta_H$ type with probability $\mu(e)$ when observing $e$. The corresponding expected productivity (the highest wage) that the firm can pay is
\begin{equation}
    \begin{aligned}
        w(e)=\mu(e)\theta_H+(1-\mu(e))\theta_L
    \end{aligned}
    \nonumber
\end{equation}
In stage 2, both firm will set $w(e)$ (complete competition).

\begin{definition}[Perfect Bayesian Equilibrium]
    %\normalfont
    A PBE is a strategy profile ($e^*(\theta_L)$, $e^*(\theta_H)$, $w^*_1: \mathbb{R}_+ \rightarrow \mathbb{R}$, $w^*_2: \mathbb{R}_+ \rightarrow \mathbb{R}$), and a belief $\mu^*: \mathbb{R} \rightarrow [0,1]$ such that
    \begin{enumerate}
        \item $\forall \theta\in\{\theta_L,\theta_H\}$, the worker strategy optimal given firm strategies.
        \item The belief $\mu^*(e)$ is derived from $\lambda, e^*(\theta_L), e^*(\theta_H)$ via Bayes' rule whenever possibly (on the equilibrium path). Outside the equilibrium path the belief $\mu^*(e)$ is arbitrarily.
        \item Firms offer wages that form a NE of the stage 2 game, where their belief $\mu^*(e)$ about their workers' type. (sequential rationality).
    \end{enumerate}
\end{definition}
We simplify the game by backward induction:
\begin{enumerate}
    \item \underline{Stage 3}: The worker chooses the highest wage off if it is $\geq 0$.
    \item \underline{Stage 2}: After observing $e(\theta)$, firms chooses the wage as the expected productivity in NE,
    \begin{equation}
        \begin{aligned}
            w^*(e)=\mu^*(e)\theta_H+(1-\mu^*(e))\theta_L
        \end{aligned}
        \nonumber
    \end{equation}
    because it is a Bertrend competition.
\end{enumerate}
\paragraph*{Separating Equilibrium}
In separating equilibrium, $e^*(\theta_L)\neq e^*(\theta_H)$.
\begin{lemma}
    In any separating PBE, $w^*(e^*(\theta))=\theta, \forall \theta\in\{\theta_L,\theta_H\}$.
\end{lemma}
\begin{proof}
    By Bayes' rule, after firm observe $e^*(\theta_L)$, $\mu^*(e^*(\theta_L))=0$. Then, $w^*(e^*(\theta_L))=\theta_L$. ($e^*(\theta_H)$ is similar.)
\end{proof}

\begin{lemma}
    In separating PBE, low type always chooses zero education, $\theta^*(\theta_L)=0$.
\end{lemma}
\begin{proof}
    If not, the low type worker always has profitable deviation, $\theta^*(\theta_L)=0$.
\end{proof}

\begin{lemma}
    Define $\underline{e}$ and $\overline{e}$ such that
    \begin{enumerate}
        \item $\theta_L=\theta_H-c(\underline{e},\theta_L)$ (the lowest effort can prevent the low type from mimicking high type) and
        \item $\theta_L=\theta_H-c(\overline{e},\theta_H)$ (the highest effort can prevent the high type from pooling with low type).
    \end{enumerate}
    Then, in all separating PBEs, $e\in \left[\underline{e},\overline{e}\right]$.\\
    Conversely, $\forall \hat{e}\in \left[\underline{e},\overline{e}\right]$, there is a separating PBE where $e^*(\theta_H)=\hat{e}$.
\end{lemma}
These different PBEs are Pareto ranked. High type prefers the PBE with a lower $e$ (the best is the one with $e^*(\theta_H)=\underline{e}$.)

\paragraph*{Pooling PBE}
$e^*(\theta)=e^*,\theta\in\{\theta_L,\theta_H\}$, $\mu^*(e^*)=\lambda$, and $w^*(e^*)=\mathbb{E}[\theta]$.

\begin{lemma}
    Define $e'$ by $\theta_L=\mathbb{E}[\theta]-c(e',\theta_L)$ (the highest effort can prevent the low type from choosing $e=0$ and get $w=\theta_L$.)\\
    Then, for all pooling PBE, $e^*(\theta_L)=e^*(\theta_H)=e^*\in[0,e']$. Conversely, for all $\hat{e}\in [0,e']$, there is a pooling PBE with $e^*=\hat{e}$.
\end{lemma}


\subsection{Cho-Kreps Intuitive Criterion}
\begin{definition}[Equilibrium Dominated Message]
    %\normalfont
    A message is \textbf{equilibrium dominated} for a type if the type must do strictly worse by sending the message than it does in equilibrium (i.e., payoff in eq. is strictly better than maximum payoff from deviating).
\end{definition}

\begin{definition}[Cho-Kreps Intuitive Criterion]
    %\normalfont
    If an information set is off the eq. path and a message is eq. dominated for a type, then beliefs should assign zero probability to the message coming from that type (if possible).
\end{definition}

Fix a PBE $e^*(\theta), \theta\in\{\theta_L,\theta_H\}, \mu^*(\cdot)$ (We know $w_1^*(e)=w_2^*(e)=\mu(e)\theta_H+(1-\mu(e))\theta_L$). Let $u^*(\theta),\theta\in\{\theta_L,\theta_H\}$ be the PBE utility of the type $\theta$ worker.

The criterion requires the (off-path) belief $\mu^*(e):=P(\tilde{\theta}=\theta_H|e)=1-P(\tilde{\theta}=\theta_L|e)$ satisfies $$P(\tilde{\theta}=\theta|e)=0,\forall e,\theta$$ such that
\begin{enumerate}
    \item $u^*(\theta)>\max_{w\in[\underline{\theta},\overline{\theta}]}[w-c(e,\theta)]$
    \item $\exists \theta'$ s.t. $u^*(\theta')\leq \max_{w\in[\underline{\theta},\overline{\theta}]}[w-c(e,\theta')]$ (make sure the sum of beliefs given $e$ is nonzero.)
\end{enumerate}
In this application, the only PBE that survives Intuitive Criterion is the best separating PBE, $e^*(\theta_H)=\underline{e}$ (the lowest effort).



\subsection{Grossman-Perry-Farrell Equilibrium}
For the equilibrium refinement, we can also introduce the Grossman-Perry-Farrell equilibrium based on the perfect sequential equilibrium (grossman1986sequential) and the neologism-proof equilibrium (farrell1993meaning). We formally define the Grossman-Perry-Farrell equilibrium by ruling out the self-signaling sets in Perfect Bayesian Equilibrium (bertomeu2018verifiable,glode2018voluntary).

A binary example is given as follows:
\begin{definition}[Grossman-Perry-Farrell Equilibrium (GPFE)]
	A pure-strategy perfect Bayesian equilibrium $(p^*_L,p^*_H,b^*(\cdot))$ is a ``Grossman-Perry-Farrell equilibrium'' (GPFE) if there does not exist a self-signaling set, which is defined by a set $\chi\subseteq\{L,H\}$ such that there exists a price $p'$ such that
	\begin{equation}
		\begin{aligned}
			\chi=\{j\in\{L,H\}:U(p',\mu_\chi)>U(p_j^*,b^*(p_j^*))\},
		\end{aligned}
		\nonumber
	\end{equation}
	where $\mu_\chi=\frac{q_L\rho_e\mathbf{1}_{L\in\chi}+q_H(1-\rho_e)\mathbf{1}_{H\in\chi}}{\rho_e\mathbf{1}_{L\in\chi}+(1-\rho_e)\mathbf{1}_{H\in\chi}}$ is the average quality of types in $\chi$ based on the relative prior probabilities.
\end{definition}

\begin{note}
    The GPFE is a strong refinement of PBE that may lead to no equilibrium exists.
\end{note}

\chapter{Screening}
\section{Screening Model}
Workers can undertake a contractible/observable task level $t\geq 0$. The utility of a worker is defined by $u(w,t,\theta):=w-c(t,\theta)$, where $c(\cdot,\cdot)$ satisfies the same assumption as in signaling model \ref{sec:signaling}.

The Game follows
\begin{enumerate}[]
    \item \underline{Stage 1}: Two firms simultaneously determine sets of contracts, $(w,t)$.
    \item \underline{Stage 2}: The worker observes all offer contracts and makes employment decision.
    (If indifference, choose lower task contract, favor employment over unemployment. If contracts of firms are indifferent, choose each with probability 1/2.)
\end{enumerate}

The null contract is $(w,t)=(0,0)$. Assume WLOG at stage 1, each firm appears a non-empty set of contracts.

\subsubsection*{Perfect Information}
\begin{proposition}[Perfect Information]
    If firms can observe the worker types, then in SPE firms make zero profit and type $\theta_i$ worker signs $(w^*_i,t^*_i)=(\theta_i,0)$.
\end{proposition}
\begin{proof}
    \begin{claim}
        Firms make zero profits from this contract.
    \end{claim}
    \begin{proof}
        Suppose not,
    \begin{enumerate}[$\circ$]
        \item $w^*_i>\theta_i$ $\Rightarrow$ negative profits, firms benefit from offering null contract.
        \item $w^*_i<\theta_i$ $\Rightarrow$ Let $\Pi$ be the total profits of the firms. Then one of the firms makes profit $\leq \frac{\Pi}{2}$. Then, this firm can benefit from offering $(w^*_i+\Delta,t^*_i)$, where $\Delta \rightarrow 0^+$.
    \end{enumerate}
    \end{proof}
    Then, we prove the firms must choose $(w^*_i,t^*_i)=(\theta_i,0)$. Suppose by the way of contradiction that $t_i^*>0$. Then, one firm can profitably deviate by offering $(w^*_i,0)$.
\end{proof}

\subsubsection*{Asymmetric Information}
\begin{lemma}\label{lemma:zero_profit}
    In any SPE, firms obtain zero profits,
\end{lemma}
\begin{proof}
    Firms must make profits $\geq 0$. Suppose by the way of contradiction that the total profit $\Pi>0$. Let $(w_L,t_L)$ be the contract signed by $\theta_L$ and $(w_H,t_H)$ be the contract signed by $\theta_H$. One firm can profitably deviate by offering $(w_L+\Delta,t_L)$ and $(w_H+\Delta,t_H)$, where $\Delta \in (0,\Pi)$.
\end{proof}

\begin{lemma}
    There is \textbf{no} pooling SPE.
\end{lemma}
\begin{proof}
    Suppose for a contradiction, $\exists$ an SPE where both worker types sign $(w_p=\mathbb{E}[\theta],t_p)$. Suppose one firm offers $(w_p,t_p)$, then another firm can only employ high type workers by offering $(\tilde{w},\tilde{t})$, where $\tilde{w}-c(\tilde{t},\theta_H)>\mathbb{E}[\theta]-c(t_p,\theta_H)$, $\tilde{w}-c(\tilde{t},\theta_L)<\mathbb{E}[\theta]-c(t_p,\theta_L)$, and $\tilde{w}<\theta_H$. (The existence is given by $\frac{\partial^2 c(t,\theta)}{\partial t\partial \theta}<0$.)
\end{proof}

\begin{lemma}
    Let $(w_L,t_L)$ be the contract signed by $\theta_L$ and $(w_H,t_H)$ be the contract signed by $\theta_H$ in separating SPE. Then, $w_L=\theta_L$ and $w_H=\theta_H$.
\end{lemma}
\begin{proof}
    Suppose $w_i>\theta_i,i\in\{L,H\}$, firms benefit from not offering this contract. So, $w_L\leq \theta_L$ and $w_H\leq \theta_H$.
    \begin{enumerate}
        \item \underline{$w_L=\theta_L$:} Suppose $w_L<\theta_L$. Either firm can profitably deviate by setting $(w'_L,t_L)$ such that $w_L<w'_L<\theta_L$. This offer can win all low-type workers and get a positive profit from hiring them. If $w'_L-c(t_L,\theta_H)\geq w_H-c(t_H,\theta_H)$, the offer can also hire high-type workers, which also give positive profit for the firm. Hence, there is a contradiction.
        \item \underline{$w_H=\theta_H$:} Suppose $w_H<\theta_H$, firms get positive profits, which contradicts to the Lemma \ref{lemma:zero_profit}.
    \end{enumerate}
\end{proof}

\begin{lemma}
    $\theta_L$ signs the contract $(\theta_L,0)$ in SPE.
\end{lemma}
\begin{proof}
    Suppose $t_L>0$. One firm can profitably deviate by offering $(\theta_L-\Delta,0)$.
\end{proof}

\begin{proposition}
    In any (pure strategy) SPE, $\theta_L$ signs $(w_L,t_L)=(\theta_L,0)$ and $\theta_H$ signs $(w_H,t_H)=(\theta_H,t_H)$, where $t_H$ solves
    \begin{equation}
        \begin{aligned}
            \theta_H-c(t_H,\theta_L)=\theta_L
        \end{aligned}
        \nonumber
    \end{equation}
\end{proposition}

If $\lambda:=P(\theta_H)$ is high, the pure SPE may not exist (exist $(\tilde{w},\tilde{t})$ can attract both types and make positive profit).

Cross subsidizing deviation by a firm (prices one product above its market value to fund another product), $(\tilde{w},\tilde{t})$ (signed by low type) and $(\tilde{\tilde{w}},\tilde{\tilde{t}})$ (signed by high type), is a profitable deviation if $\lambda$ is large enough.


\chapter{Bargaining}
Bargaining refers to "a process to determine the terms of trade that is not adequately captured by off-the-shelf oligopoly models." (Loertscher and Marx, 2021).

\section{Axiomatic Complete Information Bargaining}
The axiomatic approach abstracts away from the specifics of the bargaining process, focusing instead on identifying ``reasonable" or ``natural" properties that outcomes should satisfy.

\subsection{Bilateral Negotiations}
Let $X$ denote the \textit{set of possible agreements} and $D$ the \textit{disagreement} outcome.
\begin{example}
    Suppose $X=\{(x_1,x_2):x_1+x_2=1, x_i\geq 0\}$ and $D=(0,0)$.
\end{example}

Each player $i$ has preferences represented by a utility function $u_i$ defined over $X\cup\{D\}$. The set of possible payoffs, denoted by $U$, is defined as:
\begin{equation}
    \begin{aligned}
        U&=\{(v_1,v_2):u_1(x)=v_1, u_2(x)=v_2 \textnormal{ for some } x\in X\},\\
        d&=\left(u_1(D), u_2(D)\right).
    \end{aligned}
    \nonumber
\end{equation}
\begin{definition}[Bargaining Problem]
    A \textbf{bargaining problem} is a pair $(U,d)$ where $U\subseteq \mathbb{R}^2$ and $d\in U$. Typically, we assume:
    \begin{enumerate}
        \item $U$ is a convex and compact set.
        \item There exists some $v\in U$ such that $v>d$ (i.e., $v_i>d_i$ for all $i$).
    \end{enumerate}
\end{definition}
We denote the set of all possible bargaining problems by $B$. A bargaining solution is a function $f:B \rightarrow U$.

\begin{definition}[Axioms]
    We study bargaining solutions $f(\cdot)$ that satisfy the following axioms:
    \begin{enumerate}
        \item \textbf{Axiom 1 (Pareto Efficiency)}: A bargaining solution $f(U,d)$ is \textit{Pareto efficient} if there does not exist a $(v_1, v_2)\in U$ such that $v\geq f(U,d)$ and $v_i>f_i(U,d)$ for some $i$.
        \begin{note}
            Intuition: An inefficient outcome is unlikely, as it leaves room for renegotiation.
        \end{note}
        \item \textbf{Axiom 2 (Symmetry)}: A bargaining solution $f$ is \textit{symmetric} if for any symmetric bargaining problem $(U,d)$ (i.e., $(u_1,u_2)\in U$ if and only if $(u_2,u_1)\in U$, and $d_1=d_2$), we have $f_1(U,d)=f_2(U,d)$.
        \begin{note}
            Intuition: If the players are indistinguishable, the agreement should not favor one over the other. (This axiom can be relaxed to account for bargaining power.)
        \end{note}
        \item \textbf{Axiom 3 (Invariance to Linear Transformations)}: A bargaining solution $f$ is \textit{invariant} if for any bargaining problem $(U, d)$ and all $\alpha_i\in (0, \infty)$, $\beta_i\in \mathbb{R}$ ($i=1,2$), if we consider the transformed bargaining problem $(U',d')$ with
        \begin{equation}
            \begin{aligned}
                U'&=\{(\alpha_1u_1+\beta_1,\alpha_2u_2+\beta_2):(u_1,u_2)\in U\},\\
                d'&=\left(\alpha_1d_1+\beta_1,\alpha_2d_2+\beta_2\right),
            \end{aligned}
            \nonumber
        \end{equation}
        then $f_i(U',d')=\alpha_i f_i(U,d) + \beta_i$ for $i=1,2$.
        \begin{note}
            Intuition: Utility functions are merely one cardinal representation of ordinal preferences. They lack intrinsic cardinal meaning, so monotonic (especially linear) transformations should not affect the outcome.
        \end{note}
        \item \textbf{Axiom 4 (Independence of Irrelevant Alternatives)}: A bargaining solution $f$ is \textit{independent} if for any two bargaining problems $(U,d)$ and $(U',d)$ with $U'\subseteq U$ and $f(U,d)\in U'$, we have $f(U',d)=f(U,d)$.
        \begin{note}
            Intuition: Removing options that were not chosen should not change the outcome. This axiom arguably reflects behavioral assumptions and may require additional justification.
        \end{note}
    \end{enumerate}
\end{definition}

\subsection{Nash Bargaining Solution}
\begin{definition}[Nash Bargaining Solution]
    The Nash (1950) bargaining solution $f_N$ is defined as:
    \begin{equation}
        \begin{aligned}
            f_N(U,d)=\argmax_{u\in U, u\geq d}\ (u_1-d_1)(u_2-d_2)
        \end{aligned}
        \nonumber
    \end{equation}
\end{definition}
Given the assumptions on $(U, d)$, the solution to this optimization problem exists (if $U$ is compact and the objective function is continuous) and is unique (if the objective function is strictly quasi-concave).

\begin{theorem}[Nash, 1950]
    $f_N$ is the unique bargaining solution that satisfies the four axioms.
\end{theorem}

The Nash bargaining solution can be generalized to account for unequal bargaining weights:
\begin{equation}
    \begin{aligned}
        f_\beta(U,d)=\argmax_{u\in U, u\geq d}\ (u_1-d_1)^{\beta}(u_2-d_2)^{1-\beta}
    \end{aligned}
    \nonumber
\end{equation}
This solution is straightforward to compute and can be microfounded using an alternating offers bargaining game, as in Rubinstein (1982). See, for example, Binmore et al. (1986). Essentially, it selects a point on the efficient frontier (above $d$).

The Nash bargaining solution assumes no breakdown in negotiations.

\begin{note}
    \textbf{Weakness:} In Nash bargaining, the disagreement point plays a disproportionately significant role. If the negotiation is over a set of options that dominates $d$, it is unclear why the disagreement point should matter (e.g., Binmore et al. (1986)). This characteristic of Nash bargaining often has a substantial impact in empirical applications.
\end{note}

\subsection{Multiple Simultaneous Negotiations (Nash-in-Nash)}
\begin{enumerate}
    \item Consider a scenario with a finite set of agents $\{1,...,N\}$.
    \item Let $\mathcal{G}$ represent the set of feasible pairs, $ij$, that can potentially form agreements to collaborate.
    \item Denote $p_{ij}$ as the transfer from agent $j$ to agent $i$ if they reach an agreement.
    \item The value to agent $i$ when a set $A\subseteq \mathcal{G}$ of agreements is realized is $\pi_i(A)$. The net payoff to $i$ is $\pi_i(A)+p_i$, where $p_i$ is the total payment received by $i$.
    \begin{note}
        Agreements may generate externalities, but payments themselves do not.
    \end{note}
    \item For $B\subset A\subset \mathcal{G}$, define $\Delta\pi_i(A,B)=\pi_i(A)-\pi_i(A\setminus B)$ as the marginal value of adding the set $B$ of agreements, given that $A\setminus B$ is already realized.
\end{enumerate}

\begin{assumption}[Gains from Trade]
    For all $ij\in \mathcal{G}$, $\Delta\pi_i(\mathcal{G},ij)+\Delta\pi_j(\mathcal{G},ij)>0$.
\end{assumption}

\begin{definition}[Nash-in-Nash Bargaining Solution]
    In the Nash-in-Nash solution, under the gains-from-trade assumption, all agreements are reached. The transfer from $j$ to $i$ is given by:
    \begin{equation}
        \begin{aligned}
            p_{ij}^N=\argmax_{p}\left(\Delta\pi_i(\mathcal{G},ij)+p\right)^{b_i}\left(\Delta\pi_j(\mathcal{G},ij)-p\right)^{b_j}=\frac{b_i\Delta\pi_j(\mathcal{G},ij)-b_j\Delta\pi_i(\mathcal{G},ij)}{b_i+b_j}.
        \end{aligned}
        \nonumber
    \end{equation}
    In other words, the outcome between each pair is the bilateral Nash bargaining solution, given the "equilibrium" conjecture that all other agreements are reached. This represents a Nash equilibrium in Nash bargaining.
\end{definition}

\begin{remark}
    \textbf{Weakness:}
    \begin{enumerate}
        \item Inherits the limitations of the Nash bargaining solution.
        \item Assumes binary agreements.
        \item Considers externalities only over agreements, not lump-sum payments.
        \item Relies on ``passive beliefs'': if the negotiation between $i$ and $j$ breaks down, $i$ negotiates with $k$ as if the agreement between $i$ and $j$ were still in place.
        \item Assumes complete information.
    \end{enumerate}
\end{remark}


\section{Incomplete Information Bargaining}

\subsection{Bilateral Trade with Incomplete Information}

Consider a bilateral trade setting with:
\begin{itemize}
    \item A buyer $B$ with value $v\in[\underline{v},\bar{v}]$ for a good
    \item A seller $S$ with value (or production cost) $c\in[\underline{c},\bar{c}]$
    \item Types are independently distributed
\end{itemize}

By the revelation principle, we can focus on incentive compatible direct mechanisms, which consist of:

\begin{definition}[Direct Mechanism]
    A direct mechanism in bilateral trade is characterized by:
    \begin{enumerate}
        \item An allocation rule $q:[\underline{v},\bar{v}]\times[\underline{c},\bar{c}]\rightarrow[0,1]$ determining the probability of trade
        \item Payment rules $p_j:[\underline{v},\bar{v}]\times[\underline{c},\bar{c}]\rightarrow\mathbb{R}$ determining the payment made by agent $j\in\{B,S\}$
    \end{enumerate}
\end{definition}

The payoffs under a direct mechanism are:
\begin{itemize}
    \item Buyer's payoff: $vq(v,c)-p_B(v,c)$
    \item Seller's payoff: $p_S(v,c)-cq(v,c)$
\end{itemize}

For budget balance, we require $p_B(v,c)=p_S(v,c)$ for all $(v,c)$.

\begin{remark}[与双重拍卖的关系]
双边交易是双重拍卖的一个特例:
\begin{itemize}
    \item 双重拍卖中有多个买家和卖家,而双边交易只有一个买家和卖家
    \item 双重拍卖中的价格通常由市场出清决定,而双边交易中的价格由机制设计决定
    \item 双重拍卖中的交易量由供需平衡决定,而双边交易中交易量要么是0要么是1
\end{itemize}
\end{remark}

\begin{theorem}[Myerson-Satterthwaite定理]
在双边交易中,不存在同时满足以下条件的机制:
\begin{itemize}
    \item 个人理性(Individual Rationality)
    \item 预算平衡(Budget Balance)
    \item 激励相容(Incentive Compatibility)
    \item 效率(Efficiency)
\end{itemize}
\end{theorem}

\begin{proof}
假设存在这样的机制,那么当 $v>c$ 时,交易应该发生;当 $v<c$ 时,交易不应该发生。但根据显示原理,这会导致激励相容性被破坏。
\end{proof}

\section{Strategic Delay in Bargaining with Two-Sided Uncertainty (Cramton, 1992)}
A seller with valuation $S$ and a buyer with valuation $B$ are bargaining over the price of an object. The valuation is symmetric and private, that is, $B$ and $S$ are i.i.d. drawn from a distribution $F$ with density $f$ over $[0,1]$.

An outcome of the game is the time and the price, $\left<t,p\right>$. The discount rate is $r$, that is, the payoff to $S$ is $e^{-rt}(p-S)$ and the payoff to $B$ is $e^{-rt}(B-p)$. The discount rate $r$ and the valuation distribution $F$ are common knowledge.

As in Admati and Perry (1987), the players alternate making offers with a minimum time of $t^0=-\frac{1}{r}\log\delta$ between offers. Initially, both traders have the option of making the first offer or terminating negotiations (at time $t\geq-t^0$). If the traders happen to make initial offers at the same time, then a fair coin is flipped to determine which offer stands as the initial offer. After an offer is made, the other trader has three possible responses: (1) a counter-offer, (2) acceptance, or (3) termination.

Suppose that trader $T\in\{S,B\}$ makes the first offer p1 after a delay of $\Delta_1$, and that in round $i$ the offer $p_i$ is made after a delay $\Delta_i$ beyond the minimum time $t_0$ between offers. The history after $n$ rounds is $h^n=\{T,(\Delta_i,p_i)_{i=1,...,n}\}$.

The pure strategy of the seller and the buyer are denoted by $\pi_S$ and $\pi_B$. The profile of strategies is $\pi=\{\pi_S,\pi_B:\forall (S,B)\}$, which result in an outcome $\{t(S,B),p(S,B)\}$ that depends on the traders' valuations $(S,B)$. ``No trade'' is represented by $t=\infty$. Since all actions are publicly observed, $S$'s belief about $B$'s valuation is independent of $S$ after any history $h^n$. The belief after $h^n$ can be denoted by $\mu=\{F_B(\cdot\mid h^n),F_S(\cdot\mid h^n)\}$.





























\bibliography{ref}

\end{document}