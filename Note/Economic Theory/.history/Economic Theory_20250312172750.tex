\documentclass[11pt]{elegantbook}
\definecolor{structurecolor}{RGB}{40,58,129}
\linespread{1.6}
\setlength{\footskip}{20pt}
\setlength{\parindent}{0pt}
\newcommand{\argmax}{\operatornamewithlimits{argmax}}
\newcommand{\argmin}{\operatornamewithlimits{argmin}}
\elegantnewtheorem{proof}{Proof}{}{Proof}
\elegantnewtheorem{claim}{Claim}{prostyle}{Claim}
\DeclareMathOperator{\col}{col}
\title{\textbf{Microeconomic Theory}}
\author{Wenxiao Yang}
\institute{Haas School of Business, University of California Berkeley}
\date{2024}
\setcounter{tocdepth}{2}
\cover{cover.png}
\extrainfo{All models are wrong, but some are useful.}

% modify the color in the middle of titlepage
\definecolor{customcolor}{RGB}{9,119,119}
\colorlet{coverlinecolor}{customcolor}
\usepackage{cprotect}

%\addbibresource[location=local]{reference.bib} % bib

\begin{document}

\maketitle
\frontmatter
\tableofcontents
\mainmatter




\chapter{Correspondence: $\Psi : X \rightarrow 2^Y$ \small{(@ Lec 07 of ECON 204)}}
\begin{definition}[Correspondence]
    %\normalfont
    A \textbf{correspondence} $\Psi : X \rightarrow 2^Y$ from $X$ to $Y$ is a function from $X$ to $2^Y$, that is, $\Psi(x) \subseteq Y$ for every $x \in X$. ($2^Y$ is the set of all subsets of $Y$)
\end{definition}
\begin{example}
Let $u : \mathbb{R}_+^n \rightarrow \mathbb{R}$ be a continuous utility function, $y > 0$ and $p \in \mathbb{R}_{++}^n$, that is, $p_i > 0$ for each $i$. Define $\Psi : \mathbb{R}_{++}^n \times \mathbb{R}_{++} \rightarrow 2^{\mathbb{R}_{+}^n}$ by
\begin{equation}
    \begin{aligned}
        \Psi(p,y)=&\argmax u(x)\\
        \textnormal{s.t. }&x\geq 0\\
        &p\cdot x\leq y
    \end{aligned}
    \nonumber
\end{equation}
$\Psi$ is the demand correspondence associated with the utility function $u$; typically $\Psi(p, y)$ is multi-valued.
\end{example}

\section{Continuity of Correspondences}
\subsection{Upper/Lower Hemicontinuous}
Let $X\subseteq \mathbb{E}^n$, $Y\subseteq \mathbb{E}^m$, and $\Psi: X \rightarrow 2^Y$.
\begin{definition}[Upper Hemicontinuous]
    %\normalfont
    $\Psi$ is \textbf{upper hemicontinuous} (uhc) at $x_0 \in X$ if, for every \underline{open set} $V$ with $\Psi(x_0)\subseteq V$, there is an \underline{open set} $U$ with $x_0 \in U$ s.t.
    $$x\in U \Rightarrow \Psi(x)\subseteq V$$
\end{definition}
Upper hemicontinuity reflects the requirement that $\Psi$ doesn't “jump down/implode in the limit” at $x_0$. \textit{(A set to “jump down” at the limit $x_0$: It should mean the set suddenly gets smaller -- it “implodes in the limit” -- that is, there is a sequence $x_n \rightarrow x_0$ and points $y_n \in \Psi(x_n)$ that are far from every point of $\Psi(x_0)$ as $n \rightarrow \infty$.)}
\begin{definition}[Lower Hemicontinuous]
    %\normalfont
    $\Psi$ is \textbf{lower hemicontinuous} (lhc) at $x_0 \in X$ if, for every \underline{open set} $V$ with $\Psi(x_0)\cap V \neq \emptyset$, there is an \underline{open set} $U$ with $x_0 \in U$ s.t.
    $$x\in U \Rightarrow \Psi(x)\cap V\neq \emptyset$$
\end{definition}
Lower hemicontinuity reflects the requirement that $\Psi$ doesn't “jump up/explode in the limit” at $x_0$. \textit{(A set to “jump up” at the limit $x_0$: It should mean that the set suddenly gets bigger -- it “explodes in the limit” -- that is, there is a sequence $x_n \rightarrow x_0$ and a point $y_0\in\Psi(x_0)$ that is far from every point of $\Psi(x_n)$ as $n \rightarrow \infty$.)}

\begin{definition}[Continuous Correspondence]
    %\normalfont
    $\Psi$ is \textbf{continuous} at $x_0 \in X$ if it is both \textbf{uhc} and \textbf{lhc} at $x_0$.
\end{definition}

\begin{proposition}
    $\Psi$ is upper hemicontinuous (respectively lower hemicontinuous, continuous) if it is uhc (respectively lhc, continuous) at every $x \in X$.
\end{proposition}

\begin{center}\begin{figure}[htbp]
    \centering
    \includegraphics[scale=0.2]{uhc.png}
    \caption{The correspondence $\Psi$ “implodes in the limit” at $x_0$. $\Psi$ is not upper hemicontinuous at $x_0$.}
    \label{}
\end{figure}\end{center}

\begin{center}\begin{figure}[htbp]
    \centering
    \includegraphics[scale=0.2]{lhc.png}
    \caption{The correspondence $\Psi$ “explodes in the limit” at $x_0$. $\Psi$ is not lower hemicontinuous at $x_0$.}
    \label{}
\end{figure}\end{center}

\subsection{Theorem: $\Psi(x)=\{f(x)\}$ is uhc $\Leftrightarrow$ $f$ is continuous}
\begin{theorem}[$\Psi(x)=\{f(x)\}$ is uhc $\Leftrightarrow$ $f$ is continuous]
    Let $X \subseteq \mathbb{E}^n$, $Y \subseteq \mathbb{E}^m$ and $f : X \rightarrow Y$. Let $\Psi : X \rightarrow  2^Y$ be defined by $\Psi(x) = \{f(x)\}$ for all $x \in X$. Then $\Psi$ is \textbf{uhc} \underline{if and only if} $f$ is \textbf{continuous}.
\end{theorem}


\subsection{Berge's Maximum Theorem: the set of maximizers is uhc with non-empty compact values}
\begin{theorem}[Berge's Maximum Theorem]\label{thm:Berge's Maximum Theorem}
    Let $X \subseteq \mathbb{R}^n$ and $Y \subseteq \mathbb{R}^m$. Consider the function $f : X \times Y \rightarrow \mathbb{R}$ and the correspondence $\Gamma : Y \rightarrow 2^X$. Define $v(y) = \max_{x\in\Gamma(y)} f(x, y)$ and the set of maximizers $$\Omega(y) = \argmax_{x\in\Gamma(y)} f(x, y)=\{x:f(x,y)=v(y)\}$$
    Suppose $f$ and $\Gamma$ are continuous, and that $\Gamma$ has non-empty compact values. Then, $v$ is continuous and $\Omega$ is uhc with non-empty compact values.
\end{theorem}




\section{Graph of Correspondence}
An alternative notion of continuity looks instead at properties of the graph of the correspondence.
\begin{definition}[Graph of Correspondence]
    %\normalfont
    The \textbf{graph} of a correspondence $\Psi : X \rightarrow 2^Y$ is the set
    $$\textnormal{graph}\Psi=\{(x,y)\in X\times Y:y\in\Psi(x)\}$$
\end{definition}

\subsection{Closed Graph}
By the definition of continuous function $f:\mathbb{R}^n \rightarrow \mathbb{R}$,  each convergent sequence $\{(x_n, y_n)\}$ in graph $f$ converges to a point $(x, y)$ in graph $f$, that is, graph $f$ is closed.

\begin{definition}[Closed Graph]
    %\normalfont
    Let $X\subseteq \mathbb{E}^n$, $Y\subseteq \mathbb{E}^m$. A correspondence $\Psi: X \rightarrow 2^Y$ has closed graph if its graph is a closed subset of $X \times Y$, that is, if for any sequences $\{x_n\} \subseteq X$ and $\{y_n\} \subseteq Y$ such that $x_n \rightarrow x \in X$, $y_n \rightarrow y \in Y$ and $y_n \in \Psi(x_n)$ for each $n$, then $y \in \Psi(x)$.
\end{definition}
\begin{example}
    Consider the correspondence $\Psi(x)=\left\{\begin{matrix}
        \{\frac{1}{x}\},&\textnormal{ if }x\in(0,1]\\
        \{0\},&\textnormal{ if }x=0
    \end{matrix}\right.$ ("implode in the limit")\\
    Let $V = (-0.1, 0.1)$. Then $\Psi(0) = \{0\} \subseteq V$, but no matter how close $x$ is to $0$, $\Psi(x)=\{\frac{1}{x}\}\nsubseteq V$, so $\Psi$ is not uhc at $0$. However, note that $\Psi$ has closed graph.
\end{example}

\section{Closed-valued, Compact-valued, and Convex-valued Correspondences}
\begin{definition}[Closed-valued, Compact-valued, and Convex-valued Correspondences]
    %\normalfont
    Given a correspondence $\Psi : X \rightarrow 2^Y$,
    \begin{enumerate}
        \item $\Psi$ is \textbf{closed-valued} if $\Psi(x)$ is a closed subset of $Y$ for all $x$;
        \item $\Psi$ is \textbf{compact-valued} if $\Psi(x)$ is compact for all $x$.
        \item $\Psi$ is \textbf{convex-valued} if $\Psi(x)$ is convex for all $x$.
    \end{enumerate}
\end{definition}

\subsection{Closed-valued, uhc and Closed Graph}
For closed-valued correspondences these concepts can be more tightly connected. A closed-valued and upper hemicontinuous correspondence must have closed graph. For a closed-valued correspondence with a compact range, upper hemicontinuity is equivalent to closed graph.

\begin{theorem}[uhc and Closed Graph]
    Let $X\subseteq \mathbb{E}^n$, $Y\subseteq \mathbb{E}^m$, and $\Psi: X \rightarrow 2^Y$.
    \begin{enumerate}
        \item $\Psi$ is \textbf{closed-valued} and \textbf{uhc} $\Rightarrow$ $\Psi$ has \textbf{closed graph}.
        \item $\Psi$ is \textbf{closed-valued} and \textbf{uhc} $\Leftarrow$ $\Psi$ has \textbf{closed graph}. (If $Y$ is \textbf{compact})
    \end{enumerate}
\end{theorem}

\begin{theorem}
    Let $X\subseteq \mathbb{E}^n$, $Y\subseteq \mathbb{E}^m$, and $\Psi: X \rightarrow 2^Y$. If $\Psi$ has \textbf{closed graph} and there is an \textbf{open set} $W$ with $x_0 \in W$ and a \textbf{compact set} $Z$ such that $x \in W \cap X \Rightarrow \Psi(x) \subseteq Z$, then $\Psi$ is \textbf{uhc} at $x_0$.
\end{theorem}



\subsection{Theorem: compact-valued, uhs correspondence of compact set is compact}
\begin{theorem}\label{thm:compact-valued, uhs correspondence of compact set is compact}
    Let $X$ be a compact set and $\Psi : X \rightarrow 2^X$ be a non-empty, compact-valued upper-hemicontinuous correspondence. If $C \subseteq X$ is compact, then $\Psi(C)$ is compact.
\end{theorem}
\begin{proof}
    Given the compact-valued $\Psi$, we can have an open cover of $\Psi(C)$, $\{U_\lambda:\lambda\in\Lambda\}$. So $\forall x\in C$, there exists $U_{l(x)},l(x)\in\Lambda$ such that $U_{l(x)}$ is an open cover of $\Psi(x)$.

    Consider a $c\in C$. Since $\Psi$ is uhs and $\Psi(c)\subseteq U_{l(c)}$, there exists open set $V_c$ s.t. $c\in V_c$ and $\Psi(x)\subseteq U_{l(c)}, \forall x\in V_c\cap C$.

    $\{V_c:c\in C\}$ is an open cover of $C$. Because $C$ is compact, there is a finite subcover $\{V_{c_i}: i=1,...,m\},m\in \mathbb{N}$, where $\{c_i:i=1,...,m\}\subseteq C$.

    Because $\Psi(x)\subseteq U_{l(c_i)}, \forall x\in V_{c_i}\cap C$ and $\{V_{c_i}: i=1,...,m\},m\in \mathbb{N}$ is a open cover for $C$, we can infer $\{U_{l(c_i)}:i=1,...,m\}$ is a finite subcover of $\{U_{l(c)}:c\in C\}$ for $\Psi(C)$. Hence, $\Psi(C)$ is compact.
\end{proof}

\section{Fixed Points for Correspondences \small{(@ Lec 13 of ECON 204)}}
\subsection{Definition}
\begin{definition}[Fixed Points for Correspondences]
    %\normalfont
    Let $X$ be nonempty and $\psi : X \rightarrow 2^X$ be a correspondence. A point $x^* \in X$ is a fixed point of $\psi$ if $x^* \in \psi(x^*)$.
\end{definition}
\begin{note}
    We only need $x^*$ to be in $\psi(x^*)$, not $\{x^*\} = \psi(x^*)$. That is, $\psi$ need not be single-valued at $x^*$. So $x^*$ can be a fixed point of $\psi$ but there may be other elements of $\psi(x^*)$ different from $x^*$.
\end{note}



\subsection{Kakutani's Fixed Point Theorem: uhs, compact, convex values correspondence has a fixed point over compact convex set}
\begin{theorem}[Kakutani's Fixed Point Theorem]\label{thm:Kakutani's Fixed Point Theorem}
    Let $X \subseteq \mathbb{R}^n$ be a non-empty, \textbf{compact}, \textbf{convex} set and $\psi : X \rightarrow 2^X$ be an \textbf{upper hemi-continuous} correspondence with non-empty and \textbf{convex} values. Then $\psi$ has a fixed point in $X$.
\end{theorem}


\subsection{Theorem: $\exists$ compact set $C = \cap_{i=0}^\infty \Psi^i(X)$ s.t. $\Psi(C)=C$}
\begin{theorem}
    Let $(X, d)$ be a compact metric space and let $\Psi(x) : X \rightarrow 2^X$ be a upper-hemicontinuous, compact-valued correspondence, such that $\Psi(x)$ is non-empty for every $x \in X$. There exists a compact non-empty subset $C\subseteq X$, such that $\Psi(C) \equiv \cup_{x\in C}\Psi(x) = C$.
\end{theorem}
\begin{proof}
    Let's construct a sequence $\{C_n\}$ such that $C_0=X$, $C_1=\Psi(C_0)$, ..., $C_n=\Psi(C_{n-1}),...$ We claim that $C=\cap_{i=0}^\infty C_i$ is a non-empty compact set and satisfies $\Psi(C)=C$.
    \begin{enumerate}
        \item Because we can infer $\Psi(X_1)\subseteq \Psi(X_2)$ if $X_1\subseteq X_2$, $X=C_0\supseteq C_1 \Rightarrow C_1=\Psi(C_0)\supseteq C_2=\Psi(C_1)$,...., so $C_0\supseteq C_1\supseteq \cdots C_n\supseteq \cdots$. Hence, $C$ is not empty.
        \item Because $X$ is compact, by the theorem \ref{thm:compact-valued, uhs correspondence of compact set is compact}, we can infer $C_n$ is compact for all $n$. Then, $C_n$ is closed for all $n$, so $C$ is closed. Because $C$ is a closed set of compact set $X$, $C$ is compact.
        \item $C\subseteq C_n,\forall n \Rightarrow \Psi(C)\subseteq \Psi(C_n),\forall n \Rightarrow \Psi(C)\subseteq C$
        \item Assume $C\subseteq \Psi(C)$ doesn't hold, that is $\exists y\in C$ s.t. $y\notin \Psi(C)$. Because $y\in C$ and $C_0\supseteq C_1\supseteq \cdots C_n\supseteq \cdots$, there exists $k\in C_n$ for all $n$ s.t. $y\in\Psi(k)$. $k\in \cap_{i=1}^\infty C_i=C$, so $\Psi(k)\subseteq \Psi(C)$, which contradicts to $y\notin \Psi(C)$. Hence, $C\subseteq \Psi(C)$.
    \end{enumerate}
    All in all the claim "$C=\cap_{i=0}^\infty C_i$ is a non-empty compact set and satisfies $\Psi(C)=C$" is proved.
\end{proof}





\chapter{Preference and Utility Function}
Based on
\begin{enumerate}[$\circ$]
    \item Mas-Colell, Whinston, and Green, Microeconomic Theory, Oxford University Press (1995).
    \item UIUC ECON 530 21Fall, Nolan H. Miller
    \item UC Berkeley ECON 201A 23Fall
    \item UC Berkeley MATH 272 23Fall, Alexander Teytelboym
    \item  Jehle, G., Reny, P.: Advanced Microeconomic Theory. Pearson, 3rd ed. (2011). Ch. 6.
    \item Notes on Social Choice and Welfare, Alejandro Saporiti
    \item Yu, N. N. (2012). A one-shot proof of Arrow's impossibility theorem. \textit{Economic Theory}, 523-525.
\end{enumerate}


\section{Preferences}

\subsection{Preference Relation}
\begin{definition}[Weak, Strict, Indifference]
    %\normalfont
    $\succeq$ referred to as the \textbf{weak preference relation}: "$x$ is at least as good as $y$". (ordinal);\\
    "\textbf{No better than}": $y \preceq x$ if and only if $x \succeq y$.\\
    "\textbf{Strict preference}": $x \succ y$ if and only if $x \succeq y$ and not $y \succeq x$.\\
    "\textbf{Indifference}": $x \sim y$ if and only if $x \succeq y$ and $y \succeq x$.
\end{definition}


\subsection{Basic Assumptions}





\subsection{Rational Preference}
\begin{definition}[Rantional Relation = Preference]
    %\normalfont
    A binary relation $\succeq$ on $X$ is a \textbf{preference relation} if it is a weak order, i.e., \textbf{complete} and \textbf{transitive}.\\
    {Rationality}: $\succeq$ is \textbf{rational} \underline{if and only if} it is \textbf{complete} and \textbf{transitive}.
    \begin{enumerate}[$\circ$]
        \item $\succeq$ is \textbf{complete} iff $\forall x,y\in X$, $x \succeq y$ or $y \succeq x$.
        \item $\succeq$ is \textbf{transitive} iff $\forall x, y, z \in X$, if $x \succeq y$ and $y \succeq z$, then $x \succeq z$.
    \end{enumerate}
\end{definition}
The completeness means
\begin{enumerate}[-]
    \item Any two bundles can be compared
    \item Indifference is allowed
\end{enumerate}
The transitivity
\begin{enumerate}[-]
    \item like transitivity of the real numbers
    \item extends pairwise preferences to longer chains in the logical way.
\end{enumerate}


\section{Utility Function}
\subsection{Utility Function $\Leftrightarrow$ Rational Preference}
\begin{definition}[Unitility Function]
    %\normalfont
    We can say a function $u: X \rightarrow \mathbb{R}$ represents $\succeq$ if $\forall x,y\in X$, $$x\succeq y \Leftrightarrow u(x)\geq u(y)$$
\end{definition}

\begin{proposition}[Rational $\succeq$ $\Rightarrow$ $\exists u(\cdot)$]
    If $\exists$ a function $u: X \rightarrow \mathbb{R}$ represents $\succeq$, then $\succeq$ is rational (i.e., completeness and transitivity)
\end{proposition}
\begin{note}
    The reverse may not true.
\end{note}

\subsection{Convex Preference}
\begin{definition}[Convex $\succeq$]
    %\normalfont
    $\succeq$ is \textbf{convex} if for every $x\in X$ the $\{y\in X: y\succeq x\}$ is convex, i.e., $y_1\succeq x$ and $y_2\succeq x$ $\Rightarrow$ $\alpha y_1+ (1-\alpha) y_2\succeq x$ for all $\alpha\in[0,1]$.
\end{definition}
Convex relations imply \textit{averages are preferred to extremes}.

\begin{definition}[Strictly Convex]
    %\normalfont
    $\succeq$ is \textbf{strictly convex} iff $\forall x, y, z \in X$, if $x \succeq z$ and $y \succeq z$, then $\alpha x+(1-\alpha) y \succ z$ for all $\alpha\in (0,1)$
\end{definition}

\subsection{Convex Preference $\Leftrightarrow$ Quasiconcave Utility Function}
\begin{definition}[Quasi-Concave Function]
    %\normalfont
    A function $u$ is \textbf{quasi-concave} if and only if for all $t\in \mathbb{R}$, $\{x\in X: u(x)\geq t\}$ is convex.
    $$\forall x, y \in X, t \in \mathbb{R}, 0 \leq a \leq 1: u(x) \geq t, u(y) \geq t \Rightarrow u(a x+(1-a) y) \geq t$$
\end{definition}
\begin{proposition}[Concave Function $\Rightarrow$ Quasi-Concave Function]
    Any function that is concave is also quasi-concave.
\end{proposition}

\begin{proposition}[Convex $\succeq$ $\Leftrightarrow$ quasi-concave $u(\cdot)$]
    $\succeq$ is convex, $\Leftrightarrow$ $\exists$ a \underline{quasi-concave} $u(\cdot)$ that represents $\succeq$.
\end{proposition}

\section{Preferences over Nearby Bundles}
\subsection{Monotone Preference}
\begin{definition}[Monotone $\succeq$]
    %\normalfont
    $\succeq$ is \textbf{monotone} if $x,y\in X$ with $x\geq y\Rightarrow x\succeq y$ (and $x> y\Rightarrow x\succ y$).
\end{definition}

\begin{proposition}[Monotone $\succeq$ $\Rightarrow$ monotone $u(\cdot)$]
    If $\succeq$ is monotone, then $\exists$ a monotone $u(\cdot)$ that represents $\succeq$.
\end{proposition}

\begin{note}
    Complete, transitive, and monotone are three assumptions that made by all theories (either EU or non-EU).
\end{note}

\subsection{Strongly monotone}
\begin{definition}[Strongly Monotone $\succeq$]
    %\normalfont
    $\succeq$ is \textbf{strongly monotone} if and only if for any $x=(x_1,...,x_n), y=(y_1,...,y_n)\in X$, if $\forall i: x_{i} \geq y_{i}$ and $\exists j$ such that $x_{j}>y_{j}$, then $x \succ y$.
\end{definition}
(When we compare elements that have more than one dimension, strongly monotone holds if at least one relation is not equal.)
$$
A=(1,1), B=(2,1), C(1,2), D=(2,2)
$$
Strongly monotone can infer that $D \succ B \succ A, D \succ C \succ A$.

\subsection{Local Non-Satiation}
Even weaker assumptions will ensure that the consumer's choice exhausts their budget.
\begin{definition}[Local Nonsatiation]
    %\normalfont
    For any bundle $x$, there is a nearby bundle $y$ in the consumption set such that $y$ is preferred to $x$. That is, for all $x\in X$ and every $\varepsilon>0$,
    $$
    \exists y \in|x-y|<\varepsilon, \text { s.t. } y \succ x
    $$
\end{definition}

We have
\begin{center}
    Strong Monotonicity $\Rightarrow$ Monotonicity $\Rightarrow$ Local Nonsatiation
\end{center}


\section{Common Assumptions of Preference}

\begin{center}\begin{figure}[htbp]
    \centering
    \includegraphics[scale=0.2]{Pref_prop.png}
    \caption{Properties of Preference and Utility Function}
    \label{}
\end{figure}\end{center}


\subsection{Independence of Preference}
The 'standard' model of decisions under risk is based on von Neumann and Morgenstern Expected Utility (EU), which requires the independence assumption.
\begin{definition}[Independence of Preference]
    %\normalfont
    \textbf{Independence}: For any $x,y,z\in X$ and $0<\alpha<1$, if $x\succeq y$ then $\alpha x+(1-\alpha)z \succeq \alpha y+(1-\alpha)z$.
\end{definition}



\subsection{Continuous Preference}
\begin{definition}[Continuous $\succeq$]
    %\normalfont
    $\succeq$ is \textbf{continuous} on $X$ \underline{if and only if} for any sequence $\{x^n,y^n\}_{n=1}^\infty$ with $x^n\succeq y^n$ and we note $x=\lim_{n \rightarrow \infty}x^n$, $y=\lim_{n \rightarrow \infty}y^n$, we have $x\succeq y$ (i.e., the graph $\{(x,y)\mid x\succeq y\subseteq X\times X\}$ is closed).
\end{definition}
\begin{proposition}[Debreu's Theorem, Continuous $\succeq$ $\Rightarrow$ continuous $u(\cdot)$]
    If $\succeq$ is continuous (on $X$, a convex subset of $\mathbb{R}^k$), then $\exists$ a continuous $u(\cdot)$ that represents $\succeq$.
\end{proposition}

\begin{example}[ (Lexicographic preferences (not continuous))]
    Under Lexicographic preference $\succ$, $x \succ y$ \underline{if and only if}
    \begin{enumerate}[$\circ$]
        \item $x_{1}>y_{1}$, or
        \item $x_{1}=y_{1}$, and $x_{2}>y_{2}$, or
        \item $x_{1}=y_{1}$ and $x_{2}=y_{2}$ and $x_{3}>y_{3}$, or
        \item etc.
    \end{enumerate}
    Under Lexicographic preferences, there is no indifference.
    
    We can find the Lexicographic preference violates continuity: $\left(1+\frac{1}{n}, 1\right) \succ(1,2)$ and $\lim \left(1+\frac{1}{n}, 1\right)=(1,1) \prec(1,2)$.
\end{example}

\begin{example}[ (Utility Representation for Lexicographic Preferences)]
    Consider the lexicographic preference $\succeq$ over the restricted domain $X=\left(\mathbb{Q}\cap[0,1]\right)\times [0,1]$. Enumerate the rationals in $[0,1]$ as $\mathbb{Q}\cap[0,1]=\{q^1,q^2,q^3,...\}$ where $q^i\neq q^j$ if $i\neq j$. The utility representation of this preference is
    $$u(x_1,x_2)=\sum_{q^i<q^j}\frac{1}{2^i}+\frac{1}{2^j}x_2, \textnormal{ where }q^j=x_1$$
\end{example}




\subsection{Homothetic Preference}
\begin{definition}[Homotheticity]
    %\normalfont
    $\succeq$ are homothetic if $x\succeq y \Rightarrow \alpha x\succeq \alpha y$ for all $\alpha>0$.
\end{definition}

\begin{proposition}[Homothetic preference $\Leftrightarrow$ homogeneous $u(\cdot)$]
    A continuous $\succeq$ is homothetic $\Leftrightarrow$ $\exists$ a continuous homogeneous $u(\cdot)$ that represents $\succeq$ such that $u(\alpha x)=\alpha u(x)$ for all $x>0$.
\end{proposition}


\subsection{Quasi-linearity}
\begin{definition}[Quasi-Linearity]
    %\normalfont
    $\succeq$ on $X$ is \textbf{quasi-linear} on $x_1$ if $$x\succeq y \Rightarrow (x+\epsilon e_1)\succeq (y+\epsilon e_1)$$ where $e_1=(1,0,...,0)$ and $\epsilon>0$.
\end{definition}

\begin{theorem}[Quasi-Linearity $\Leftrightarrow$ $u(x)=x_1+v(x_{-1})$]
    A continuous $\succeq$ on $(-\infty,\infty)\times \mathbb{R}_+^{K-1}$ is quasi-linear in $x_1$ $\Leftrightarrow$ $\exists$ a $u(\cdot)$ that represents $\succeq$ such that $$u(x)=x_1+v(x_{-1})$$
    where $v(\cdot)$ satisfies $(v(x_{-1}),0,...,0)\sim(0,x_{-1})$.
\end{theorem}

\subsection{Separability}
\begin{definition}[Separability]
    %\normalfont
    $\succeq$ satisfies \textbf{separability} if for any $x_i$
    \begin{equation}
        \begin{aligned}
            (x_i, x_{-i})\succeq (x'_i, x_{-i}) \Leftrightarrow (x_i, x'_{-i})\succeq (x'_i, x'_{-i})
        \end{aligned}
        \nonumber
    \end{equation}
\end{definition}

\begin{theorem}[Separability $\Rightarrow$ Additive $u(\cdot)$]
    $\succeq$ with \textbf{separability} admits additive $u$-representation
    \begin{equation}
        \begin{aligned}
            u(x)=v_1(x_1)+\cdots v_K(x_K)
        \end{aligned}
        \nonumber
    \end{equation}
\end{theorem}

\begin{note}
    Strong assumption, usually ignored in practice.
\end{note}


\subsection{Differentiable Preference}
Consider a vector of values $v(x)\in \mathbb{R}^K_+$ for the $K$ commodities and a feasible the direction $x+\varepsilon d\in X$ from $x$ for small enough $\varepsilon>0$.

$d$ is considered \underline{improvement} if and only if $$d\cdot v(x)>0$$

Given $v(x): X \rightarrow \mathbb{R}^K_+$, let $$D_v(x)=\{d:d\cdot v(x)>0\}$$
be the set of directions that are improvements relative to $x$.

$d\in \mathbb{R}^k$ is an \underline{improvement direction} at $x$ if there is $\lambda^*>0$ such that $\lambda d$ is an improvement $$x+\lambda d\succ x$$
for any $\lambda\leq \lambda^*$. Let $D_{\succeq}(x)$ be the set of all improvement directions at $x$.

Any improvement is an improvement direction if
\begin{enumerate}[-]
    \item $\succeq$ are strictly convex.
    \item $\succeq$ are convex, strongly monotonic, and continuous.
\end{enumerate}

\begin{definition}[Differentiable Preference]
    %\normalfont
    $\succeq$ is \textbf{differentiable} if there exists a function $v(x): X \rightarrow \mathbb{R}_+^K$ such that $$D_{\succeq}(x)=D_v(x),\ \forall x\in X$$
\end{definition}

\begin{example}
    $\succeq$ represented by
    \begin{enumerate}[(1).]
        \item $\alpha x_1+\beta x_2$ for $\alpha,\beta>0$ are differentiable: $v(x)=(\alpha,\beta)$.
        \item $\min\{\alpha x_1,\beta x_2\}$ are differentiable where $\alpha x_1\neq \beta x_2$: $v(x)=\left\{\begin{matrix}
            (1,0)& \textnormal{ if }\alpha x_1<\beta x_2\\
            (0,1)& \textnormal{ otherwise}
        \end{matrix}\right.$
    \end{enumerate}
\end{example}

\begin{proposition}[Sufficient condition for differentiable $\succeq$]
    Any (monotonic and convex) $\succeq$ can be represented by a (strongly monotonic and quasi-concave) and differentiable $u$ is differentiable.
\end{proposition}


\chapter{Choice Theory}
\section{Choice}
Let $\mathcal{B}=2^X$ (all subsets of $X$) and $B\in \mathcal{B}$ be the all potential alternatives that can be chosen.

The choice of an agent can be represented by $C(B)\subseteq B, \forall B\in \mathcal{B}$.

\begin{definition}[Choice Correspondence (More than one choice)]
    %\normalfont
    A choice correspondence $C$ assigns a non-empty \underline{subset} for every non-empty set $A$
    $$\emptyset\neq C(A)\subseteq A$$
\end{definition}

\begin{definition}[Induced Choice Rule]
    %\normalfont
    Given a binary relation $\succeq$, the \textbf{induced choice rule} $C_\succeq$ is defined by
    $C(A)=C_\succeq(A)=\{x\in A:x\succeq y, \forall y\in A\}, \forall A\subseteq X$.\\
    A choice function $c$ can be \textbf{rationalizable} if there is a preference relation $\succeq$ on $X$ such that $c=c_\succeq$.
\end{definition}

\begin{definition}[Revealed Preference]
    %\normalfont
    Given a choice rule $\succeq$, its \textbf{revealed preference relation $\succeq_C$} is defined by $x \succeq_C y$ if there exists some $A$ such that $x, y \in A$ and $x \in C(A)$.
\end{definition}

\begin{proposition}
    If $C$ is rationalized by $\succeq$, then $\succeq=\succeq_C$.
\end{proposition}

\begin{definition}[Rubinstein's Condition $\alpha$]
    %\normalfont
    A choice function $c$ satisfies \textbf{condition $\alpha$} if for any two problems $A,B$, if $A\subseteq B$ and $c(B)\in A$, then $c(A)=c(B)$.
\end{definition}
%Condition $\alpha$ is sufficient for $C$ is being formulated by maximizing a preference $\succeq$.

\subsection{Choice Function}
\begin{definition}[Choice Function]
    %\normalfont
    A \textbf{choice function} $c$ such that $c(A)\in A$ which specifies a unique element for each nonempty subset $A\subseteq X$ (no indifferent preferences).
\end{definition}
\begin{proposition}[Rubinstein's Condition $\alpha$ $\Rightarrow$ Rationalizable Choice Function $c$]
    \begin{enumerate}[(1).]
        \item  Let $c$ be a choice function defined on a domain containing at least all subsets of $X$ of \underline{size of at most $3$}. If $c$ satisfies condition $\alpha$, then there is a preference $\succeq$ on $X$ such that $c=c_\succeq$.
        \item Let $c$ be a choice function with a domain $D$ satisfying that if \underline{$A, B \in D$, then $A \cup B \in D$}. If $c$ satisfies condition $\alpha$, then there is a preference relation $\succeq$ on $X$ such that $c = c_\succeq$.
    \end{enumerate}
\end{proposition}

\subsection{Choice Correspondence}
\begin{definition}[Sen's $\alpha$ or Independence of Irrelevant Alternatives]
    %\normalfont
    If $a\in A\subseteq B$, then $a\in C(B) \Rightarrow a\in C(A)$.
\end{definition}

\begin{definition}[Sen's $\beta$]
    %\normalfont
    If $a,b\in A\subseteq B$, then $a,b\in C(A)$ and $b\in C(B) \Rightarrow a\in C(B)$.
\end{definition}
$\alpha$ and $\beta$ are equivalent to WARP.
\begin{definition}[Weak Axiom of Revealed Preference (WARP)]\label{WARP}
    %\normalfont
    Given a choice structure $(C,\mathcal{B})$ satisfies \textbf{WARP}. If $\exists B\in \mathcal{B}$ with $x,y\in B$, such that $x\in C(B)$. Then, $\forall B'\in \mathcal{B}$ with $x,y\in B'$, $y\in C(B') \Rightarrow x\in C(B')$.\\
    Or we can say, $$x,y\in B\cap B', x\in C(B), \textnormal{ and }y\in C(B') \Rightarrow x\in C(B')$$
\end{definition}

\begin{proposition}[Rational $\Rightarrow$ WARP]
    Given $\succeq$ is rational, then $(C^*_{\succeq},\mathcal{B})$ satisfies WARP.\\
    ($C^*_{\succeq}$ is the choice rule that picks the maximal alternatives by $\succeq$)
\end{proposition}

\begin{proposition}[Sen's Condition $\alpha,\beta$ $\Rightarrow$ Rationalizable Choice Correspondence $C$]
    Let $C$ be a choice correspondence defined on a domain containing at least all subsets of $X$ of \underline{size of at most $3$}. If $C$ satisfies condition $\alpha$ and $\beta$, then there is a preference $\succeq$ on $X$ such that $C=C_\succeq$.
\end{proposition}

\section{Revealed Preference}
Given choice data $(p^t,x^t)$, we say $u$-fucntion \underline{rationalizes} the observed behavior $(p^t,x^t)$ if for all $t=1,...,T$, $p^t x^t\geq p^t x \Rightarrow u(x^t)\geq u(x)$, that is, $u(\cdot)$ achieves its maximum value on the budget set at the chosen bundles.

If ``locally non-satiated'' $u$-function, $p^t x^t> p^t x \Rightarrow u(x^t)> u(x)$.

\begin{definition}[Revealed Preferred]
    %\normalfont
    We say $x^t$ is
    \begin{enumerate}[-]
        \item $x^t R^D x$: \underline{\textbf{directly} revealed preferred} to $x$, if $p^t x^t \geq p^t x$; ($x$ is available under $p^t$)
        \item $x^t P^D x$: \underline{\textbf{strictly directly} revealed preferred} to $x$, if $p^t x^t > p^t x$;
        \item $x^t R x$: \underline{\textbf{indirectly} revealed preferred} to $x$, if $\exists$ a sequence $\{x_k\}_{k=1}^K$ with $x_1=x^t$ and $x_K=x$ such that $x_k R^D x_{k+1}$ for all $k=1,...,K-1$, i.e., $p^t x^t=p^t x_1\geq p^t x_2\geq ... \geq p^t x_K=p^t x$.
    \end{enumerate}
\end{definition}

\begin{definition}[Generalized Axiom of Revealed Preference (GARP)]
    %\normalfont
    Consider two observations $(p^t,x^t)$ and $(p^s,x^s)$, GARP is satisfied if
    \begin{equation}
        \begin{aligned}
            x^t R x^s &\Rightarrow \textnormal{ not } x^s P^D x^t\\
            \textnormal{i.e., }x^t R x^s &\Rightarrow p^s x^t\geq p^s x^s
        \end{aligned}
        \nonumber
    \end{equation}
\end{definition}

GARP is a generalization of various other revealed preference tests
\begin{definition}
    %\normalfont
    \underline{Weak Axiom of Revealed Preference (WARP)}:
    \begin{equation}
        \begin{aligned}
            x^t R^D x^s,x^t\neq x^s&\Rightarrow\textnormal{ not } x^s P^D x^t\\
            \textnormal{i.e., }p^t x^t \geq p^t x^s,x^t\neq x^s&\Rightarrow p^s x^t\geq p^s x^s
        \end{aligned}
        \nonumber
    \end{equation}
    \underline{Strong Axiom of Revealed Preference (SARP)}:
    \begin{equation}
        \begin{aligned}
            x^t R x^s,x^t\neq x^s&\Rightarrow\textnormal{ not } x^s R x^t
        \end{aligned}
        \nonumber
    \end{equation}
\end{definition}

\begin{theorem}[Afriat's Theorem]
    The following conditions are equivalent:
    \begin{enumerate}
        \item The data satisfies GARP;
        \item There exists a non-satiated $u$-function that rationalizes the data;
        \item There exists a concave, monotonic, continuous, non-satiated $u$-function that rationalizes the data.
        \item There exist positive numbers $(u^t,\lambda^t)$ for $t=1,...,T$ that satisfy the so-called Afriat inequalities: $$u^s\leq u^t+\lambda^t p^t(x^s-x^t),\ \forall t,s$$
    \end{enumerate}
\end{theorem}



\section{Choice under Uncertainty}
We want to model an uncertain prospect corresponding forms of function $u$.

The literature contains (basically) three sets of answers to these questions, differing in whether uncertainty is objective or subjective.

\begin{enumerate}[$\circ$]
    \item Objective uncertainty: von Neumann-Morgenstern (vNM).
    \item Subjective uncertainty: Savage.
    \item Horse lottery-roulette wheel theory: Anscombe and Aumann (A-A)
\end{enumerate}

\subsection{von Neumann-Morgenstern (vNM)}
The set of prizes is defined by $X$ and the set of probability measures (or distributions) over $X$ is denoted by $P$.
\underline{A compound lottery:} If $p,q\in P$ and $\alpha\in[0,1]$, then there is an element $\alpha p+(1-\alpha)q\in P$ which is defined by taking the convex combinations of the probabilities of each prize separately, or
$$(\alpha p+(1-\alpha)q)(x)=\alpha p(x)+(1-\alpha)q(x)$$
$(\alpha p+(1-\alpha)q)$ represents a compound lottery.

\begin{definition}[Three Axioms]
    %\normalfont
    \textbf{\underline{Three Axioms}}
    \begin{enumerate}
        \item[(A1)] $\succ$ is a preference relation (asymmetric and negatively transitive);
        \item[(A2)] For all $p,q,r\in P$ and $\alpha\in[0,1]$, $p\succ q \Rightarrow \alpha p +(1-\alpha)r\succ \alpha q+(1-\alpha)r$.
        \item[(A3)] For all $p,q,r\in P$ such that $p\succ q\succ r$, $\exists \alpha,\beta\in (0,1)$ such that
        \begin{equation}
            \begin{aligned}
                \alpha p +(1-\alpha)r\succ q\succ \beta p +(1-\beta)r
            \end{aligned}
            \nonumber
        \end{equation}
    \end{enumerate}
\end{definition}
\begin{theorem}[vNM]
    $\succ$ on $P$ satisfies axioms (A1)-(A3) if and only if there exists a function $u:X \rightarrow \mathbb{R}$ such that
    \begin{equation}
        \begin{aligned}
            p\succ q \Leftrightarrow \sum_x p(x) u(x)>\sum_x q(x) u(x)
        \end{aligned}
        \tag{*}
        \label{*}
    \end{equation}
    Moreover, $u$ is unique up to a \underline{positive affine transformation}: there is another $u'$ represents $\succ$ in the sense of (\ref{*}) if and only if there exists $c>0$ and $d$ such that
    \begin{equation}
        \begin{aligned}
            u'(\cdot)=cu(\cdot)+d
        \end{aligned}
        \nonumber
    \end{equation}
\end{theorem}
\begin{remark}
    \begin{enumerate}[$\circ$]
        \item If $u$ represents $\succ$ then so will $v(\cdot)=f(u(\cdot))$ for any \textbf{strictly increasing} $f$.
        \item $k(p)=\sum_x p(x)u(x)$ gives an ordinal representation of $\succ$.
    \end{enumerate}
\end{remark}

\begin{lemma}[Four Lemmas obtained by the three axioms]
    If $\succ$ satisfies (A1) to (A3), then
    \begin{enumerate}[(L1).]
        \item If $p\succ q$ and $0\leq\alpha<\beta\leq 1$, then $$\beta p+(1-\beta)q\succ \alpha p+(1-\alpha)q$$
        \item If $p\succeq q\succeq r$ and $p\succ r$ $\Rightarrow$ there exists a unique $\alpha^*\in[0,1]$ such that $$q\sim \alpha^*p+(1-\alpha^*)r$$
        \item If $p\sim q$ and $\alpha\in[0,1]$ $\Rightarrow$ for all $r\in P$, $$\alpha p+(1-\alpha)r\succ \alpha q+(1-\alpha)r$$
        \item For any $x\in X$, let $\delta_x$ be the probability distribution degenerate at $x$, that is $\delta_x(x')=\left\{\begin{matrix}
            1,&\textnormal{ if }x'=x\\
            0,&\textnormal{ if }x'\neq x
        \end{matrix}\right.$ For all $p\in P$, we have $x_1,x_2\in X$ such that
        \begin{equation}
            \begin{aligned}
                \delta_{x_1}\succeq p\succeq \delta_{x_2}
            \end{aligned}
            \nonumber
        \end{equation}
    \end{enumerate}
\end{lemma}

\subsection{Savage (1954)}
Consider the situation that what the decision maker chooses depends critically on his/her subjectively assesses as the odds of the outcomes.

\underline{The basics of the Savage formulation:}
\begin{enumerate}[$\circ$]
    \item a set of $X$ of prizes/consequences;
    \item a set $S$ of the nature (states of the world).
\end{enumerate}
Each $s\in S$ is a compilation of all characteristics/factors about which the DM is uncertain and which are relevant to the consequences that will
result from her/his choice. The set $S$ is an exhaustive list of mutually exclusive states — only one
$s\in S$ will be the realized state.

We denote the choice space by $H$, as the set of all functions from $S$ to $X$ ($H=X^S$).

Savage seeks to find a subjective taste (the utility function) $u(\cdot)$ and a subjective belief (the probability measure) $\pi$ such that
\begin{equation}
    \begin{aligned}
        h\succ h' \Leftrightarrow \sum_{s\in S}\pi(s)u(h(s))>\sum_{s\in S}\pi(s)u(h'(s))
    \end{aligned}
    \nonumber
\end{equation}
Note that, it contains an assumption that $u(\cdot)$ is a function about $x$ which doesn't depend on the state of the world when it receives $x$.



\section{Social Choice}
Notations:
\begin{enumerate}
    \item We consider \underline{finite} set of alternatives $X$ and \underline{finite} set of agents $I$.
    \item We use $\mathcal{B}$ to denotes the set of all preference relations.
    \item We use $\mathcal{R}\subseteq \mathcal{B}$ to denotes the set of all rational preference relations.
    \item We use $\succeq\in \mathcal{R}$ to represents individual rational preference relation.
\end{enumerate}

\subsection{Social Welfare Function and Properties}
\begin{definition}[Social Welfare Function (SWF)]
    %\normalfont
    A \textbf{social welfare function} (SWF) is a mapping $$f: \mathcal{A}\subseteq \mathcal{R}^I\rightarrow \mathcal{B}$$
    $\trianglerighteq=f(\succeq_1,...,\succeq_I)$ is interpreted as the \textbf{social preference relation}. It doesn't need to be rational (i.e., complete and transitive).
\end{definition}

\begin{definition}[SWF's Properties]\label{SWF_properties}
    %\normalfont
    A social welfare function $f: \mathcal{A}\rightarrow \mathcal{B}$
    \begin{enumerate}[$\circ$]
        \item has \textbf{unrestricted domain} (UD) if $\mathcal{A}=\mathcal{R}^n$;
        \item is \textbf{transitive} (T) if $f(\succeq_1,...,\succeq_I)$ is transitive for all $(\succeq_1,...,\succeq_I)\in \mathcal{A}$;
        \item is \textbf{nondictatorial} (ND) if there is no agent $i\in I$ such that $\forall \{x,y\}\subseteq X$ $x\succeq_i y \Rightarrow x\trianglerighteq y$. (That is there is no distinguished voter who can choose the winner).
        \item is \textbf{weakly Paretian} (PA) if, $\forall \{x,y\}\subseteq X$ and any preference profile $(\succeq_1,...,\succeq_I)\in \mathcal{A}$, we have $x\succeq_i y,\forall i\in I \Rightarrow x\trianglerighteq y$.
        \item is \textbf{independent of irrelevant alternatives} (IIA) if, $\forall \{x,y\}\subseteq X$, and any $\succeq$ and $\succeq'$ with $\succeq_i\mid_{x,y}=\succeq'_i\mid_{x,y}, \forall i\in I$, if $x\trianglerighteq y$ then $x\trianglerighteq' y$.
    \end{enumerate}
\end{definition}


\subsection{Arrow's Theorem}
\begin{theorem}[Arrow's impossibility theorem]
    Suppose $|X|\geq 3$, $\mathcal{A}=\mathcal{R}^I$ (UD). Then if a SWF $f$ satisfies T, PA, and IIA, then it fails to be ND.
\end{theorem}
\begin{proof}
    %\normalfont
    Yu, N. N. (2012). A one-shot proof of Arrow's impossibility theorem. \textit{Economic Theory}, 523-525.
\end{proof}
Any dictatorship satisfies UD, PA, and IIA.


\chapter{Demand Theory}
\section{Utility Maximization Problem (UMP)}
Budget set is given by $B=\{x\in X\subseteq \mathbb{R}_+^K: p\cdot x\leq w\}$, where $w$ is the DM's wealth and $p$ is the vector of prices. Without losing generality, we can assume $w=1$.

The DM's problem is finding the $\succeq$-optimal bundle $x\in B(p)$. With the corresponding utility function $u(x)$, we can consider a consumer's problem
\begin{equation}
    \begin{aligned}
        \max_{x\in X} u(x)\\
        s.t.\ p\cdot x\leq w
    \end{aligned}
    \tag{UMP}
    \label{UMP}
\end{equation}
The set $\succeq$-optimal bundle is represented by $x(p,w)$.

\subsection{Marshallian Demand: Existence and Properties}

\begin{proposition}[Continuous Preference$\Rightarrow$ Solution $x(p,w)$ Existence]
    If $\succeq$ ($u(\cdot)$) is continuous, then all such problems have a solution $x(p,w)$.
\end{proposition}
\begin{proof}
    By the Weierstrass Extreme Value Theorem.
\end{proof}

\begin{proposition}[Convex Preference$\Rightarrow$ Convex $x(p,w)$]
    If $\succeq$ is convex ($u(\cdot)$ is quasi-concave), then $x(p,w)$ is convex.
\end{proposition}
\begin{proof}
    Suppose $x,x'\in X$. The optimal utility $u^*=u(x)=u(x')$. For any $\alpha\in[0,1]$, let $x''=\alpha x+(1-\alpha)x'$. Because $\succeq$ is convex, we have $u(\cdot)$ is quasi-concave, that is $u(x'')\geq u^*$. $x''$ is also feasible. So, $x''\in x(p,w)$.
\end{proof}

\begin{proposition}[Strictly Convex Preference$\Rightarrow$ Singleton $x(p,w)$]
    If $\succeq$ is strictly convex ($u(\cdot)$ is strictly quasi-concave), then $x(p,w)$ is (at most) a singleton.
\end{proposition}

\begin{proposition}[Differentiable Preference$\Rightarrow$ Margianl Utility equals to Price]
    If $\succeq$ is differentiable, $x^*\in x(p,w)$, and the vector of marginal values at $x^*$ (as defined above) is denoted by $v(x^*)=(v_1(x^*),...,v_K(x^*))$, where $v_k(x^*)$ is usually taken by $\frac{\partial u}{\partial x_k}(x^*)$ in "classic" problem. Then, we have
    \begin{equation}
        \begin{aligned}
            \frac{v_k(x^*)}{v_j(x^*)}=\frac{p_k}{p_j} \textnormal{ for any }x^*_k,x^*_j>0
        \end{aligned}
        \nonumber
    \end{equation}
    and for any $k$ with $x_k^*>0$ (consumed commodity)
    \begin{equation}
        \begin{aligned}
            \frac{v_k(x^*)}{p_k}\geq \frac{v_j(x^*)}{p_j}\textnormal{ for any }j\neq k
        \end{aligned}
        \tag{*}
        \label{*}
    \end{equation}
\end{proposition}

\begin{corollary}[Sufficient Conditions for Optimality]
    If $\succeq$ is strongly monotonic, convex, continuous, and differentiable and if $p\cdot x^*=w$ and \eqref{*} is satisfied then $x^*\in x(p,w)$
\end{corollary}

\begin{definition}[Rationalize]
    %\normalfont
    $\succeq$ \textbf{fully rationalize} the demand function $x$ if for any $(p,w)$, the bundle $x(p,w)$ is the \underline{unique} $\succeq$-maximal bundle within $B$.\\
    A monotonic $\succeq$ \textbf{rationalize} the demand function $x$ if for any $(p,w)$, the bundle $x(p,w)$ is a $\succeq$-maximal bundle within $B$.
\end{definition}

The \underline{unique} solution is called Marshallian (Uncompensated) Demand.

\begin{proposition}[Properties of Marshallian Demand]
    \begin{enumerate}[(i).]
        \item \textbf{Walras' Law}: If $\succeq$ is local nonsatiation, $\forall x^*\in x(p,w): p\cdot x^*=w$.
        \item \textbf{Homogeneity of degree zero in $( p, w )$:} $x (\alpha p, \alpha w )\equiv x ( p, w ),\ \forall \alpha > 0$.
        \item \textbf{Continuous in prices and in wealth} if the $\succeq$ is continuous.
    \end{enumerate}
\end{proposition}


\begin{proposition}[Weak Axiom of Revealed Preference of Marshallian Demand]
    If demand is single valued then WARP(\ref{WARP}) is equivalent to $$p\cdot y'\leq w \textnormal{ and }y\neq y' \Rightarrow p'\cdot y>w$$
    where $y\equiv x(p,w)$ and $y'\equiv x(p',w')$.
    ($y'$ is feasible under $(p,w)$ but $y=x(p,w)$, which means $y$ is better and it can't be feasible under $(p',w')$.)
\end{proposition}

\subsection{Lagrangian Approach: $\frac{\partial u\left(x^{*}\right)}{\partial x_{i}}=\lambda^{*} p_{i}$ and $\lambda^*\left(x_i(p,w)+\sum_{j=1}^{K} p_{j} \frac{\partial x_{j}}{\partial p_{i}}\right)=0$}
The Lagrangian of the problem is $$L(x,\lambda)=u(x)-\lambda(p \cdot x-w)$$
By the KKT necessary conditions, we have
\begin{equation}
    \begin{aligned}
        \frac{\partial L}{\partial x_{i}}(x^*,\lambda^*)=\frac{\partial u\left(x^{*}\right)}{\partial x_{i}}-\lambda^{*} p_{i}=0,\ \forall i=1,...,K\\
        \lambda^*\geq 0 \textnormal{ and } \lambda^*(p\cdot x^*-w)=0
    \end{aligned}
    \nonumber
\end{equation}
Based on that, we have
\begin{lemma}
    \begin{enumerate}[(i).]
        \item $\frac{\partial u\left(x^{*}\right)}{\partial x_{i}}=\lambda^{*} p_{i}$;
        \item $\lambda^*\left(x(p,w)+p\cdot\frac{\partial x(p,w)}{\partial p}\right)=0$ i.e., $\lambda^*\left(x_i(p,w)+\sum_{j=1}^{K} p_{j} \frac{\partial x_{j}}{\partial p_{i}}\right)=0$.
    \end{enumerate}
\end{lemma}

\subsection{Envelope Theorem $\Rightarrow \lambda^{*}=\frac{\partial u(x(p, w))}{\partial w}$}
\begin{theorem}[Envelope Theorem]
    Consider the constrained maximization problem,
    \begin{equation}
        \begin{aligned}
            \max_{x\in \mathbb{R}^n}\quad & f(x;\theta)\\
            \textnormal{s.t. }& g(x;\theta)\leq 0
        \end{aligned}
        \nonumber
    \end{equation}
    where $x\in \mathbb{R}^n$ is the choice variable and $\theta\in \mathbb{R}^m$ is some parameter. Let $f,g$ be continuously differentiable real-valued functions.
    \begin{enumerate}[$\bullet$]
        \item Let the value function of the problem be $V(\theta)\triangleq f(x^*(\theta),\theta)$.
        \item The Lagrangian for this problem is $$L(x,\lambda;\theta)=f(x; \theta)-\lambda g(x; \theta)$$
        \item Let $x^{*}$ and $\lambda^{*}$ denote the optimized values of the variables.
        \item[](By KKT necessary conditions, we have $\frac{\partial f}{\partial x}(x^*;\theta)=\lambda^* \frac{\partial g}{\partial x}(x^*;\theta)$ and $\lambda^* g(x^*;\theta)=0$)
    \end{enumerate}
    Then the following is true for any $\bar{\theta}\in \mathbb{R}^m$
    \begin{equation}
        \begin{aligned}
            \frac{\partial V}{\partial \theta_i}(\bar{\theta})=\frac{\partial L}{\partial \theta_i}(x^{*}, \lambda^{*};\bar{\theta})=\frac{\partial f}{\partial \theta_i}(x^{*};\bar{\theta})-\lambda^*\frac{\partial g}{\partial \theta_i}(x^{*};\bar{\theta})
        \end{aligned}
        \nonumber
    \end{equation}
\end{theorem}
\begin{proof}
    The proof of the envelope theorem is a straightforward calculation.\\
    Firstly, by KKT necessary conditions, we have $\frac{\partial f}{\partial x}(x^*;\bar{\theta})=\lambda^* \frac{\partial g}{\partial x}(x^*;\bar{\theta})$ and $\lambda^* g(x^*;\bar{\theta})=0 \Rightarrow \lambda^*\left[\frac{\partial g}{\partial x}(x^*;\bar{\theta})\frac{\partial x^*(\bar{\theta})}{\partial \theta_i}+\frac{\partial g}{\partial \theta_i}(x^*;\bar{\theta})\right]=0$. Then we have
    \begin{equation}
        \begin{aligned}
            \frac{\partial V}{\partial \theta_i}(\bar{\theta})&=\frac{\partial f}{\partial \theta_i}(x^*;\bar{\theta})+\frac{\partial f}{\partial x}(x^*;\bar{\theta})\frac{\partial x^*(\bar{\theta})}{\partial \theta_i}\\
            &\left(\textnormal{by }\frac{\partial f}{\partial x}(x^*;\bar{\theta})=\lambda^* \frac{\partial g}{\partial x}(x^*;\bar{\theta})\right)\\
            &=\frac{\partial f}{\partial \theta_i}(x^*;\bar{\theta})+\lambda^* \frac{\partial g}{\partial x}(x^*;\bar{\theta})\frac{\partial x^*(\bar{\theta})}{\partial \theta_i}\\
            &\left(\textnormal{by }\lambda^*\left[\frac{\partial g}{\partial x}(x^*;\bar{\theta})\frac{\partial x^*(\bar{\theta})}{\partial \theta_i}+\frac{\partial g}{\partial \theta_i}(x^*;\bar{\theta})\right]=0\right)\\
            &=\frac{\partial f}{\partial \theta_i}(x^{*};\bar{\theta})-\lambda^*\frac{\partial g}{\partial \theta_i}(x^{*};\bar{\theta})
        \end{aligned}
        \nonumber
    \end{equation}
\end{proof}

\begin{corollary}\label{corollary:marginal_wealth}
    $\lambda^{*}=\frac{\partial u(x(p, w))}{\partial w}$.
\end{corollary}
\begin{proof}
    By the envelope theorem, we have $\frac{\partial u(x(p, w))}{\partial w}=\left.\frac{\partial L}{\partial w}\right|_{x^{*}, \lambda^{*}}=\lambda^{*}$.
\end{proof}

\subsection{Indirect Utility Function $v(p, w) \equiv u(x(p, w))$}
\begin{proposition}[Properties of Indirect Utility Function]
    \begin{enumerate}
        \item $v(p,w)$ is homogeneous of degree zero in $(p, w)$;
        \item $v(p,w)$ is strictly increasing in $w$ and non-increasing in $p_{i}$;
        \item $v(p,w)$ is quasi-convex, that is the set $\{p:v(p,w)\leq u\}$ is convex for all $u\in \mathbb{R}$.
        \item $\lambda^{*}=\frac{\partial v(p, w)}{\partial w}$ (Corollary \ref{corollary:marginal_wealth}).
    \end{enumerate}
\end{proposition}

\subsection{Roy's Identity $x^*_i=-\frac{\frac{\partial v}{\partial p_{i}}}{\frac{\partial v}{\partial w}}$: recover $x(p, w)$ from $v(p, w)$}
\begin{proposition}[Roy's Identity]
    $x^*_i(p,w)=-\frac{\frac{\partial v(p,w)}{\partial p_{i}}}{\frac{\partial v(p,w)}{\partial w}}$.
\end{proposition}
\begin{proof}
    By the definition,
    \begin{equation}
        \begin{aligned}
            v(p, w) & \equiv u(x(p, w))&\\
            \frac{\partial v}{\partial p_{i}} & \equiv \sum_{j=1}^{K} \frac{\partial u}{\partial x_{j}} \frac{\partial x_{j}}{\partial p_{i}}&\\
            &=\sum_{j=1}^{K} \lambda^{*} p_{j} \frac{\partial x_{j}}{\partial p_{i}}& \textnormal{ by }\frac{\partial u\left(x^{*}\right)}{\partial x_{i}}=\lambda^{*} p_{i}\\
            &=-\lambda^{*} x_{i}^{*}&\textnormal{ by }\lambda^*\left(x_i(p,w)+\sum_{j=1}^{K} p_{j} \frac{\partial x_{j}}{\partial p_{i}}\right)=0\\
            x^*_i&=-\frac{\frac{\partial v}{\partial p_{i}}}{\frac{\partial v}{\partial w}}&\textnormal{ by }\lambda^{*}=\frac{\partial v(p, w)}{\partial w}
        \end{aligned}
        \nonumber
    \end{equation}
\end{proof}

\section{Expenditure Minimization Problem (EMP)}
Consider the duality
\begin{equation}
    \begin{aligned}
        \min_{x\in X}\quad & p\cdot x\\
        s.t.\quad & u(x)\geq u
    \end{aligned}
    \tag{EMP}
    \label{EMP}
\end{equation}
The optimal solutions are represented by $h(p,u)$. With uniqueness, we call it \textit{Hicksian (compensated) demand}.

\subsection{Hicksian Demand $h(p,u)$: Properties}
\begin{proposition}[Properties of Hicksian Demand]
    \begin{enumerate}[(i).]
        \item $h(p,u)$ is homogeneous of degree zero in $p$: $$h(tp,u)=h(p,u),\ \forall t\in \mathbb{R}_+$$
        \item $u ( x )$ is strictly quasi-concave $\Rightarrow$ $h ( p, u )$ is unique;
        \item For $u>u(0)$ and $u(\cdot)$ is locally non-satiated, \underline{constraint is active}: for all $x^*\in h(p,u)$, $$u(x^*)=u$$
    \end{enumerate}
\end{proposition}

\begin{lemma}[$\sum_{j=1}^{K} \frac{\partial u}{\partial x_j} \frac{\partial h_{j}}{\partial p_{i}}=0$]
    If $u ( x )$ is strictly quasi-concave, the Hicksian demand satisfies $\sum_{j=1}^{K} \frac{\partial u}{\partial x_j} \frac{\partial h_{j}}{\partial p_{i}}=0$
\end{lemma}
\begin{proof}
    $u(h(p,u))\equiv u \Rightarrow \sum_{j=1}^{K} \frac{\partial u}{\partial x_j}\frac{\partial h_{j}}{\partial p_{i}}=0$.
\end{proof}

\subsection{Expenditure Function $e(p,u)\equiv p\cdot h(p,u)$}
Given the Hicksian demand $h(p,u)$, we can define the expenditure function as $e(p,u)\equiv p\cdot h(p,u)$.
\begin{proposition}[Properties of Expenditure Function]\label{expen}
    \begin{enumerate}[(i).]
        \item $e ( p, u )$ is homogeneous of degree $1$ in $p$: $$e(tp,u)=tp\cdot h(tp,u)=tp\cdot h(p,u)=t e(p,u)$$
        \item $e ( p, u )$ is strictly increasing in $u$, non-decreasing in $p_i$;
        \item $e ( p, u )$ is \textbf{concave} in $p$;
        \item $e ( p, u )$ is continuous in $p$ for all $p>>0$;
        \item For all $x^*\in h(p,u)$, $x^*\in x(p,e(p,u))$;
        \item For $w>0$, $e(p,v(p,w))\equiv w$;
        \item $e(p,u)$'s derivative property: $$\frac{\partial e(p, u)}{\partial p_{i}} \equiv h_{i}(p, u)$$
    \end{enumerate}
\end{proposition}
\begin{proof}[Proof for concavity]
    Suppose the price of good 1 increases from $p_1^0$ to $p_1^1$: $p^0 \rightarrow	p^1$. Set $p^a=ap^0+(1-a)p^1,\ 0\leq a\leq 1$. So, $p^0\leq p^a\leq p^1$ and
    \begin{equation}
        \begin{aligned}
            e(p^a,u)&=p^a\cdot h(p^a,u)\\
            &=(ap^0+(1-a)p^1)\cdot h(p^a,u)\\
            &=a[p^0\cdot h(p^a,u)]+(1-a)[p^1\cdot h(p^a,u)]\\
            &h(p^a,u)\text{ is feasible in both EMP, but not optimal solutions}\\
            &\geq a[p^0\cdot h(p^0,u)]+(1-a)[p^1\cdot h(p^1,u)]\\
            &= ae(p^0,u)+(1-a)e(p^1,u)
        \end{aligned}
        \nonumber
    \end{equation}
\end{proof}

\begin{proof}[Proof for Derivative]
    \begin{enumerate}
        \item \underline{Direct proof:}
        $$
        \begin{aligned}
        e(p, u) & \equiv p \cdot h(p, u)& \\
        \frac{\partial e}{\partial p_{i}} & \equiv \sum_{j=1}^{K} p_{j} \frac{\partial h_{j}}{\partial p_{i}}+h_{i}& \\
        & \equiv \sum_{j=1}^{K} \frac{1}{\lambda^{*}} \frac{\partial u\left(x^{*}\right)}{\partial x_{j}} \frac{\partial h_{j}}{\partial p_{i}}+h_{i}& \textnormal{ by }\frac{\partial u\left(x^{*}\right)}{\partial x_{i}}=\lambda^{*} p_{i}\\
        &=h_i&\textnormal{ by }\sum_{j=1}^{K} \frac{\partial u}{\partial x_j} \frac{\partial h_{j}}{\partial p_{i}}=0
        \end{aligned}
        $$
        \item \underline{Envelope Theorem Proof:}
        \begin{equation}
            \begin{aligned}
                L(x,\lambda;(p,u))&=p\cdot x-\lambda (u(x)-u)\\
                \frac{\partial e(p,u)}{\partial p_i}&=\frac{\partial L(x,\lambda;(p,u))}{\partial p_i}\bigg|_{x^*=h(p,u)}=x_i|_{x^*=h(p,u)}=h_i(p,u)
            \end{aligned}
            \nonumber
        \end{equation}
    \end{enumerate}
\end{proof}

\subsection{Law of Compensated Demand: $\frac{\partial h_i}{\partial p_i}\leq 0$}
\begin{corollary}[Law of Compensated Demand]
    Hicksian demand is downward sloping in its own price, $$\frac{\partial h_i}{\partial p_i}\leq 0$$
\end{corollary}
\begin{proof}
    By the concavity of $e(p,u)$ (\ref{expen}), we can conclude $\nabla^2 e(p,u)\preceq 0$ (negative semi-definite). Then, we know its diagonal elements are non-positive $\frac{\partial e^2}{\partial^2 p_i}=\frac{\partial h_i}{\partial p_i}\leq 0$.
\end{proof}

\subsection{Shifts in Hicksian Demand: $\frac{\partial h_{i}}{\partial u}  \equiv \frac{\partial x_{i}}{\partial w} \frac{\partial e}{\partial u}$, same direction as $\frac{\partial x_{i}}{\partial w}$}
How does Hicksian demand curve shift when utility changes?
$$
\begin{aligned}
h_{i}(p, u) & \equiv x_{i}(p, e(p, u)) \\
\frac{\partial h_{i}}{\partial u} & \equiv \frac{\partial x_{i}}{\partial w} \frac{\partial e}{\partial u}
\end{aligned}
$$
We know $\frac{\partial e}{\partial u}>0$, so the direction of Hicksian demand shift is the same as $\frac{\partial x_{i}}{\partial w}$.\\
- Normal good: increasing utility shifts $h_{i}$ to the right.\\
- Inferior good: increasing utility shifts $h_{i}$ to the left.

\section{UMP and EMP}
\subsection{Slutsky Equation: substitution effect and income effect}
Slutsky: how change of $p_j$ (price in good $j$) affects $x_i$ (the demand of product $i$).
\begin{proposition}[Slutsky Equation]
    \begin{equation}
        \begin{aligned}
            \frac{\partial x_i(p,w)}{\partial p_j}=\underbrace{\frac{\partial h_i(p,u)}{\partial p_j}}_{\textnormal{substitution effect}}\underbrace{-\frac{\partial x_i(p,w)}{\partial w} x_j(p,w)}_{\textnormal{income effect}}
        \end{aligned}
        \nonumber
    \end{equation}
\end{proposition}
\begin{proof}
    \begin{equation}
        \begin{aligned}
            h_i(p,u)&\equiv x_i(p,e(p,u))\\
            \frac{\partial h_{i}}{\partial p_{j}} & \equiv \frac{\partial x_{i}}{\partial p_{j}}+\frac{\partial x_{i}}{\partial w} \frac{\partial e}{\partial p_{j}} \\
            & \equiv \frac{\partial x_{i}}{\partial p_{j}}+\frac{\partial x_{i}}{\partial w} h_{j}(p, u) \\
            & \equiv \frac{\partial x_{i}}{\partial p_{j}}+\frac{\partial x_{i}}{\partial w} x_{j}(p, e(p,u))
        \end{aligned}
        \nonumber
    \end{equation}
\end{proof}
\begin{enumerate}[$\circ$]
    \item \textbf{Substitution effect}: $\frac{\partial h_{i}}{\partial p_{j}}$, the change of relative prices change with constant utility will change the $x_i$.
    \item \textbf{Income (Wealth) effect}: $-\frac{\partial x_{i}}{\partial w} x_{j}(p, w)$, the change of price can be seen as change of wealth, which will also impact the $x_i$.
\end{enumerate}

\subsection{Relationship Between UMP and EMP}
\begin{figure}[htbp]
    \centering
    \includegraphics[scale=0.6]{UMP-EMP.png}
    \caption{Relationship Between UMP and EMP}
    \label{}
\end{figure}





















\chapter{General Equilibrium}
\section{Exchange Economy}
\begin{enumerate}
    \item There are $L$ perfectly divisible commodities indexed by $l=1,...,L$ over $\mathbb{R}^L$.
    \item There are $m$ agents, indexed by $i=1,...,m$. $N=\{1,...,m\}$.
    \item Each agent has a preference relation $\succeq_i$ on $\mathbb{R}_+^L$ represented by a utility function $u_i: \mathbb{R}_+^L \rightarrow \mathbb{R}$.
    \item Each agent has a vector of \underline{initial endowments} $w_i\in \mathbb{R}_+^L$.
    \item The \underline{aggregate endowment} is $w=\sum_{i=1}^m w_i$.
\end{enumerate}

\begin{example}[ (Endowment Box Economy)]
    The endowment box economy has $2$ goods ($L=2$) and 2 agents ($m=2$). The commodity space is $\mathbb{R}^2$.\\
    Each agent's consumption set is $\mathbb{R}_+^2=\{x\in \mathbb{R}^2:x\geq 0\}$.\\
    Each agent $i=a,b$ has preference relation $\succ_i$ over $\mathbb{R}_+^2$ represented by a utility function $u_i: \mathbb{R}_+^2 \rightarrow \mathbb{R}$.\\
    Each agent has a vector of initial endowments $w_i=(w_{i1},w_{i2})\in \mathbb{R}^2$.
\end{example}

\begin{definition}[Allocation]
    %\normalfont
    An \textbf{allocation} in an exchange economy is an assignment of goods to agents $x=(x_1,...,x_m)\in \mathbb{R}_+^{L\times m}$ such that $\sum_{i=1}^mx_i=w$.
\end{definition}


\subsection{Pareto Optimal/Efficient}
\begin{definition}[Pareto Optimal]
    %\normalfont
    An allocation $x$ is \textbf{Pareto optimal/efficient} if there doesn't exist an allocation $y$ s.t. $u_i(y_i)\geq u_i(x_i)$ ($y_i\succeq_i x_i$) for each $i$ and $u_j(y_j)> u_j(x_j)$ ($y_j\succ_j x_j$) for some $j$.
\end{definition}

Consider the following social planner's problem:
\begin{enumerate}[$\circ$]
    \item Fix an agent $j$ and $\{\bar{u}_i\}_{i\neq j}$,
    \begin{equation}
        \begin{aligned}
            \max_{(x_1,...,x_n)\in \mathbb{R}^{L\times m}_+} u_j(x_j)\\
            \textnormal{s.t. } u_i(x_i)\geq \bar{u}_i, i\neq j\\
            \sum_{i=1}^mx_i=w\\
            x_i\geq 0, \forall i
        \end{aligned}
        \tag{P}
        \label{Planner P}
    \end{equation}
\end{enumerate}

\begin{proposition}[P.O. $\Leftrightarrow$ Solutions of Problem (\ref{Planner P})]
    Suppose each agent's utility function is continuous and strongly monotone. Then, an allocation $x^*$ in an exchange economy is Pareto-Optimal \underline{iff} it is a solution of Problem (\ref{Planner P}) for some choice of $\{\bar{u}_i\}_{i\neq j}$.
\end{proposition}
\begin{proof}
    \begin{enumerate}
        \item ``$\Leftarrow$'': Suppose $x^*$ is a solution to Problem (\ref{Planner P}) for $\{\bar{u}_i\}_{i\neq j}$. Suppose by the way of contradiction that $x^*$ is not Pareto-Optimal. Then there is another allocation $\hat{x}$ such that
        \begin{enumerate}[(i).]
            \item Either: $u_j(\hat{x}_j)> u_j(x^*_j)$ and $u_i(\hat{x}_i)\geq u_i(x^*_i)$ for all $i\neq j$.
            \item Or: $u_j(\hat{x}_j)\geq u_j(x^*_j)$, $u_k(\hat{x}_k)\geq u_k(x^*_k)$ for some $k\neq j$, and $u_i(\hat{x}_i)\geq u_i(x^*_i)$ for all $i\neq j,k$.
        \end{enumerate}
        Suppose (i) holds: Since $\hat{x}$ is an allocation, $\sum_{i=1}^m \hat{x}_i = m$ and $\hat{x}_i\geq 0, \forall i$. By assumption and $x^*$ is solution of Problem (\ref{Planner P}), $u_i(\hat{x}_i)\geq u_i(x^*_i)\geq \bar{u}_i$ for all $i\neq j$. So, $\hat{x}$ satisfies the constraints of Problem (\ref{Planner P}). Because $u_j(\hat{x}_j)> u_j(x^*_j)$, $x^*$ is not the solution to Problem (\ref{Planner P}). Contradiction!\\
        Suppose (ii) holds: Prove by constructing another allocation $\tilde{x}$ as follows: By continuity, $\exists \epsilon>0$ sufficiently small s.t. $u_k((1-\epsilon)\hat{x}_k)\geq u_k(x^*_k)$. Set $\tilde{x}_k=(1-\epsilon)\hat{x}_k$, $\tilde{x}_j=\hat{x}_j+\epsilon \hat{x}_k$, and $\tilde{x}_i=\hat{x}_i$ for all $i\neq j,k$. Then, $\sum_{i=1}^m \tilde{x}_i=\sum_{i=1}^m \hat{x}_i=w$, $u_i(\tilde{x}_i)\geq u_i(x^*_i)\geq \bar{u}_i$ for all $i\neq j$ and $u_j(\tilde{x}_j)> u_j(x^*_j)\geq \bar{u}_j$ (by strong monotonicity). Hence, $x^*$ is not the solution to Problem (\ref{Planner P}). Contradiction!
        \item ``$\Rightarrow$'': Suppose $x^*$ is Pareto-Optimal. Set $\bar{u}_i=u_i(x^*)$ for all $i\neq j$.
        \begin{claim}
            $x^*$ solves Problem (\ref{Planner P}) given $\{\bar{u}_i\}_{i\neq j}$.
        \end{claim}
        Firstly, $x^*$ is feasible for Problem (\ref{Planner P}). Then, suppose by the way of contradiction that there is another allocation $x'$ such that $\sum_{i=1}^m x'_i=w$, $u_i(x'_i)\geq \bar{u}_i=u_i(x^*_i)$ for all $i\neq j$, and $u_j(x'_j)>u_j(x^*_j)$. Hence, $x^*$ is not Pareto-Optimal, which is a contradiction!
    \end{enumerate}
\end{proof}

\begin{proposition}
    From the first-order condition (FOC) of Problem (\ref{Planner P}), a \underline{necessary} condition for \underline{interior} Pareto-Optimal allocations when each $u_i$ is also differentiable is
    \begin{enumerate}[$\circ$]
        \item $Du_j(x^*_j)=\lambda_i Du_i(x^*_i)$ for some $\lambda_i>0$ and $\forall i\neq j$ where $x^*_i>>0, \forall i$.
    \end{enumerate}
\end{proposition}


\subsection{Individually Rational, Block, Core}
Are all Pareto-Optimal allocations equally likely are reasonable?

How the initial endowment allocation affects the Pareto-Optimal allocation?

One agent should block any proposed trades leading to allocations that generate lower utility.

\begin{definition}[Individually Rational]
    %\normalfont
    A \underline{bundle} $x_i$ is \textbf{individually rational} (IR) for agent $i$ if $x_i\succeq_i w_i$.\\
    An \underline{allocation} $x=(x_1,...,x_m)$ is \textbf{individually rational} (IR) if $x_i\succeq_i w_i$ for all $i=1,...,m$.
\end{definition}

Let $N:=\{1,...,m\}$ be the set of agents. A \textbf{coalition} is a nonempty subset $S\subseteq N$.
\begin{example}
    With two agents $\{a,b\}$, there are 3 possible coalitions: $\{a\},\{b\},\{a,b\}$.
\end{example}

\begin{definition}[Block]
    %\normalfont
    A coalition $S$ can \textbf{block} an allocation $x=(x_1,...,x_m)$ if there exists bundles $y_i\in \mathbb{R}^L_+$ for all $i\in S$ s.t.
    \begin{enumerate}[1.]
        \item $\sum_{i\in S}y_s=\sum_{i\in S}w_i$
        \item $y_i\succeq_i x_i$ for all $i\in S$;
        \item $y_j\succ_j x_j$ for some $j\in S$.
    \end{enumerate}
\end{definition}

\begin{definition}[Core]
    %\normalfont
    The \textbf{core} is the \underline{set of allocations} that cannot be blocked by any coalition.
\end{definition}

\begin{note}(Core and P.O.)
    \begin{enumerate}[$\circ$]
        \item Every allocation in the core is Pareto-Optimal (directly by definition).
        \item Not every Pareto-Optimal allocation is in the core.
        \item For the \underline{two agent} case, the core is the set of individually rational Pareto-Optimal allocations.
    \end{enumerate}
\end{note}




\subsection{Competitive Equilibrium}
\begin{assumption}
    %\normalfont
    Suppose there are markets for all available goods and all agents are \underline{price-takers} in these markets.
\end{assumption}
Given a vector of price $p\in \mathbb{R}^L$, agent $i$ chooses $x_i^*$ to solve the following problem:
\begin{equation}
    \begin{aligned}
        \max_{x_i\in \mathbb{R}^L_+}\ & u_i(x_i)\\
        s.t.\ & p\cdot x_i\leq p\cdot w_i
    \end{aligned}
    \nonumber
\end{equation}



\begin{definition}[Competitive Equilibrium]
    %\normalfont
    Given endowment $w=(w_i)_{i\in N}$. A \textbf{competitive (Walrasian) equilibrium} in an exchange economy is a pair $p^*\in \mathbb{R}^L$ (price vector over $L$ goods) and an allocation $x^*=(x^*_i)_{i\in N}$ such that:
    \begin{enumerate}[(i).]
        \item $x_i^*\in \argmax u_i(x)$ s.t. $p^*\cdot x_i\leq p^*\cdot w_i, \forall i\in N$.
        \item $\sum_{i\in N}x_i^*=w$.
    \end{enumerate}
    We call $x^*=(x^*_i)_{i\in N}$ the \underline{competitive equilibrium (Walrasian) allocation}\\ and $p^*$ the \underline{competitive equilibrium (Walrasian) price vector}.
\end{definition}

Demand notations:
\begin{definition}[Excess Demand]
    %\normalfont
    Let $x_i(p):=x_i(p,p\cdot w_i)$ denote agent $i$'s \textbf{demand} given the price vector $p\in \mathbb{R}^L$ and income $p\cdot w_i$.\\
    Agent $i's$ \textbf{individual excess demand at} $p$ is $x_i(p)-w_i$.\\
    The \textbf{aggregate excess demand} at $p$ is $\sum_{i\in N}x_i(p)-w$.
\end{definition}
\begin{note}(Excess Demand and Competitive Equilibrium)
    \begin{enumerate}[$\circ$]
        \item $p^*$ is a competitive equilibrium price vector \underline{if and only if} $0\in \sum_{i\in N}x_i(p)-w$
        \item $(x^*,p^*)$ is a competitive equilibrium \underline{if and only if} $x^*$ satisfies $x_i^*\in x_i(p^*), \forall i$ and $\sum_{i\in N}x^*_i-w=0$.
    \end{enumerate}
\end{note}











\subsection{First Welfare Theorem: CE $\Rightarrow$ P.O.}
Given non-satiated preference, every CE is P.O. (P.E.).
\begin{theorem}[First-order (fundamental) Welfare Theorem: CE $\Rightarrow$ P.O.]
    If each agent's preference relation is locally non-satiated, then every \underline{competitive equilibrium} allocation is \underline{Pareto optimal (Pareto efficient)}.
\end{theorem}
\begin{proof}
    Let $x^*=(x_1^*,...,x_n^*)$ be a CE allocation with corresponding CE price vector $p^*$.\\
    Suppose by way of contradiction that $x^*$ is \underline{not} P.O. allocation. Then, there is another allocation $\hat{x}=(\hat{x}_1,...\hat{x}_n)$ such that $\hat{x}_j\succ_j x^*_j$ for some $j\in N$ and $\hat{x}_i\succeq_i x^*_i$ for all $i\neq j$.\\
    By the definition of CE, $\hat{x}_j$ should not be affordable for $j$, i.e., $p^*\hat{x}_j>p^*w_j$. By the definitions of non-satiated and CE, $p^*\hat{x}_i\geq p^*w_i$ for all $i\neq j$. (If $p^*\hat{x}_i<p^*w_i$, $\exists \tilde{x}_i$ s.t. $p^*\tilde{x}_i\leq p^*w_i$ and $\tilde{x}_i\succ_i\hat{x}_i\succeq_i x_i^*$, which contradicts to the definition of CE.)\\
    Add up all inequalities, we get
    \begin{equation}
        \begin{aligned}
            p^*\cdot\left(\sum_{i=1}^n\hat{x}_i\right)> p^*\cdot\left(\sum_{i=1}^n w_i\right)=p^*\cdot w
        \end{aligned}
        \nonumber
    \end{equation}
    which contradicts to the definition of a feasible allocation that $\sum_{i}^n \hat{x}_i=w$. Hence, $x^*$ is P.O.
\end{proof}
\begin{note}
    This requires \textbf{only} \underline{local non-satiation} of preferences. In particular, does not require convexity of preferences.
\end{note}

\subsection{CE $\Rightarrow$ IR; CE $\subseteq$ P.O. $\cap$ Core}
\begin{note}[CE $\Rightarrow$ IR]
    At any prices $p$, an agent can always afford their initial endowment, so by revealed preference, every CE allocation is individually rational.
\end{note}
\begin{corollary}[CE Allocation is in Core]
    If each agent's preference relation is locally non-satiated, then every CE allocation is in the core.
\end{corollary}
\begin{proof}
    In exercise.
\end{proof}
\begin{note}
    Not every P.O. allocation is in the core. But every CE allocation is a P.O. allocation in the core.
    \begin{equation}
        \begin{aligned}
            \textnormal{CE}\subseteq \textnormal{P.O.}\cap \textnormal{Core}
        \end{aligned}
        \nonumber
    \end{equation}
\end{note}


\subsection{Equilibrium with Transfers}
What scope does planner have for redistribution using only decentralized market mechanism?

Not every P.O. allocation is ``equitable.'' To implement a more "equitable" allocation. Some possible mechanisms:
\begin{enumerate}[$\circ$]
    \item will need taxes or transfers (should be budget-balancing, i.e., no money leaves the economy).
    \item taxes/transfers should be lump-sum.
\end{enumerate}

\begin{definition}[``Supportable'' as a Price Equilibrium with Transfers]
%\normalfont
    An \underline{allocation $x^*$} is \textbf{supportable} as a \textbf{price equilibrium with transfers} if there exists a price vector $p^*\in \mathbb{R}^L$ and lump-sum budget-balancing transfers $\{T_i:i=1,...,m\}$ so that $\sum_{i=1}^m T_i=0$, such that $\forall i$:
    \begin{equation}
        \begin{aligned}
            x_i^*\in \arg\max_{x\in \mathbb{R}^L_+ \textnormal{ s.t. } p^*\cdot x_i\leq p^*\cdot w_i + T_i}u_i(x_i)
        \end{aligned}
        \nonumber
    \end{equation}
\end{definition}


\subsection{Second Welfare Theorem: sufficient condition for P.O. be supported as a price equilibrium with transfers}
\begin{remark}
    Is every P.O. allocation supportable as a price equilibrium with transfers? \textbf{No.} (e.g. a non-convex indifference curve (preference relation): for a P.O. allocation, there exists an allocation such that gives a bundle with lower cost but equal utility for an agent.)
\end{remark}
Formally, a sufficient condition can be given:
\begin{theorem}[Second Welfare Theorem]\label{SWT_sup}
    If each consumers' preference relation is convex, continuous, and strongly monotone, then every interior P.O. allocation in an exchange economy can be supported as a price equilibrium with transfers.
\end{theorem}
To give the proof of the theorem, we need firstly give some definitions and results.
\begin{definition}[Supported by a price; Supported]
    %\normalfont
    An allocation $\bar{x}=\left(\bar{x}_1,...,\bar{x}_m\right)$ in an exchange economy is \textbf{supported by a non-zero price vector $p\in \mathbb{R}^L$} if $$\forall i: x_i\succeq_i\bar{x}_i \Rightarrow p\cdot x_i\geq p\cdot\bar{x}_i$$
    If an allocation $\bar{x}=\left(\bar{x}_1,...,\bar{x}_m\right)$ is \textbf{supported}, then a \underline{common} price $p$ supports each agent's ``better-than set'' at $\bar{x}_i$ ($\{x_i\in \mathbb{R}_L^+: x_i\succeq_i\bar{x}_i\}$): $\forall x_i\in \{x_i\in \mathbb{R}_L^+: x_i\succeq_i\bar{x}_i\}: p\cdot x_i\geq p\cdot \bar{x}_i$.
\end{definition}
\begin{note}
    An allocation is supported as a price equilibrium with transfers $\Leftrightarrow$ the allocation that is supported (i.e., all agents' bundles are supported by a common price).
\end{note}

Recall:
\begin{theorem}[Separating Hyperplane Theorem]\label{SHT}
    Let $A,B\subseteq \mathbb{R}^n$ be non-empty disjoint, convex sets. Then $\exists p\in \mathbb{R}^n,p\neq 0$, s.t. $$p\cdot a\leq p\cdot b,\ \forall a\in A, \forall b\in B$$
\end{theorem}

\begin{proof}[Second Welfare Theorem \ref{SWT_sup}]
    Let $x^*=\left(x_1^*,...,x_m^*\right)$ be an interior P.O. allocation, so $x_i^*>>0, \forall i$. Let
    \begin{equation}
        \begin{aligned}
            P_i:=\left\{x_i\in \mathbb{R}_+^L:u_i(x_i)> u_i(x_i^*)\right\}, \ \forall i
        \end{aligned}
        \nonumber
    \end{equation}
    Properties about $P_i$:
    \begin{enumerate}[(1).]
        \item By strong monotonicity, $P_i\neq \emptyset$ (interior allocation) for all $i$.
        \item By convexity, $P_i$ is convex for all $i$.
    \end{enumerate}
    Let
    \begin{equation}
        \begin{aligned}
            P:&=P_1+\cdots+P_m\\
            &=\left\{z\in \mathbb{R}_L^+: z=\sum_{i=1}^m x_i \textnormal{ for some }(x_1,...,x_m)\in \mathbb{R}_+^{L\times m} \textnormal{ s.t. }u_i(x_i)> u_i(x_i^*), \forall i\right\}
        \end{aligned}
        \nonumber
    \end{equation}
    Properties about $P$:
    \begin{enumerate}[(1).]
        \item By construction, $P\neq \emptyset$.
        \item $P$ is convex: because the sum of convex sets is convex.
        \item $w\notin P$: this follows from the Pareto optimality of $x^*=\left(x_1^*,...,x_m^*\right)$.
    \end{enumerate}
    As $\{w\}$ is convex and $\{w\}\cap P=\emptyset$, by the Separating Hyperplane Theorem \ref{SHT}, $\exists p\in \mathbb{R}^L, p\neq 0$ s.t.  $p\cdot z\geq p\cdot w$ for all $z\in P$.
    \begin{enumerate}[$\circ$]
        \item
            \subitem Fix $j$ and suppose $u_j(x_j)>u_j(x_j^*)$ for some $x_j\in \mathbb{R}_+^L$. By continuity, $\exists \epsilon\in (0,1)$ sufficient small s.t. $u_j((1-\epsilon)x_j)>u_j(x_j^*)$. Let $y_j:=(1-\epsilon)x_j$. For $i\neq j$: set $y:=x_i^*+\frac{\epsilon}{m-1}x_j$. By strong monotonicity, $u_i(y_i)>u_i(x_i^*), \forall i\neq j$. So, $\sum_{i=1}^m y_i\in P$ by the definition of $P$.\\
            Then,
            \begin{equation}
                \begin{aligned}
                    p\cdot \left(\sum_{i=1}^m y_i\right)&\geq p\cdot w\\
                    \textnormal{By }\sum_{i=1}^m x_i^*=w,\quad p\cdot \left(\sum_{i=1}^m y_i-\sum_{i=1}^m x_i^*\right)&\geq 0\\
                    p\cdot \left(x_j-x_j^*\right)&\geq 0\\
                    p\cdot x_j&\geq p\cdot x_j^*
                \end{aligned}
                \nonumber
            \end{equation}
            That is, with $p$, $u_j(x_j)>u_j(x_j^*) \Rightarrow p\cdot x_j\geq p\cdot x_j^*$.
            \subitem
            By strong monotonicity, $u_j(x_j^*+(0,0,..,0,1,0,..,0))>u_j(x_j^*)$, hence, $p\cdot (x_j^*+(0,0,..,0,1,0,..,0))\geq p\cdot x_j^* \Rightarrow p\cdot (0,0,..,0,1,0,..,0)\geq 0$. That is, $p_i\geq 0,\forall i$. By definition, $p\neq 0$, $p>0$.\\
            By assumption $x_j^*>>0$, $p\cdot x_j^*>0$. Now suppose $\exists x_j\in \mathbb{R}_+^L$ s.t. $u_j(x_j)>u_j(x_j^*)$ and $p\cdot x_j=p\cdot x_j^*$. By continuity, $\exists\delta\in (0,1)$ s.t. $u_j(\delta x_j)>u_j(x_j^*)$. By what we show above, $u_j(x_j)>u_j(x_j^*) \Rightarrow p\cdot x_j\geq p\cdot x_j^*$. We have $p\cdot x_j>\delta p\cdot x_j=p\cdot (\delta x_j)\geq p\cdot x_j^*>0$. There is a contradiction. Hence, we prove that
            \begin{equation}
                \begin{aligned}
                    u_j(x_j)>u_j(x_j^*) \Rightarrow p\cdot x_j> p\cdot x_j^*
                \end{aligned}
                \nonumber
            \end{equation}
        \item Let the transfers be $T_i:=p\cdot x_i^*-p\cdot w_i, \forall i$ such that $\sum_{i}T_i=p\cdot (\sum_{i}x_i^*-\sum_{i}w_i)=0$.
    \end{enumerate}
    All in all,
    \begin{equation}
        \begin{aligned}
            x_i^*\in \arg\max_{x\in \mathbb{R}^L_+ \textnormal{ s.t. } p^*\cdot x_i\leq p^*\cdot w_i + T_i}u_i(x_i)
        \end{aligned}
        \nonumber
    \end{equation}
    $x^*$ is a price equilibrium with transfers $\{T_i\}_i$ and the price vector $p$.
\end{proof}

\subsection{Second Welfare Theorem: P.O. with Endowments Used $\Rightarrow$ CE}
\begin{theorem}[Second Welfare Theorem (corollary)]
    Suppose that interior $x^*$ is Pareto efficient and consumers receive endowment worth $p\cdot w^i=p\cdot {x^i}^*$ for all $i=1,...,m$. Then, if a competitive equilibrium exists for such $w$, then $x^*$ is a competitive equilibrium allocation.
\end{theorem}
\begin{proof}
    By the Second Welfare Theorem \ref{SWT_sup}, interior P.O. allocation $x^*$ can be supported by transfers $\{T_i\}_{i=1}^m$. Then, $p\cdot x_i\leq p\cdot w_i + T_i$. Because $p\cdot w^i=p\cdot {x^i}^*$, $T_i=0,\forall i$. So, $x^*$ is exactly a competitive equilibrium allocation.
\end{proof}

\subsection{Walras' Law in Competitive Equilibrium}
Recall that ``$p$ is a competitive equilibrium price vector'' $\Leftrightarrow$ ``$0\in \sum_{i=1}^m x_i(p)-w$.''

\begin{note}
Only relative prices matter, as the Marshallian demand has homogeneity of degree zero: $x(\lambda p)=x(p)$.
\end{note}

Hence, if $p^*$ is a competitive equilibrium price vector, so is $\lambda p^*, \forall \lambda$, which correspond to the same competitive equilibrium.

\begin{remark}
    \textbf{Are markets independent?} No.
\end{remark}

If $\succeq_i$ is locally non-satiated for all $i$, then $\forall i$: $p\cdot x_i(p)=p\cdot w_i, \forall p$. Adding over agents: $p\cdot \sum_{i=1}^mx_i(p)=p\cdot w, \forall p$.

This is \textbf{Walras' Law} in aggregate level: $p\cdot \left[\sum_{i=1}^mx_i(p)-w\right]=0, \forall p$.

\begin{remark}
    If Walras' Law holds and there exists $p^*>>0$ such that all markets but one clear, then the $p^*$ must clear the last market too.\\
    Let $Z(p)=\sum_{i=1}^mx_i(p)-w$. By Walras' Law, $p\cdot Z(p)=0, \forall p$. Suppose that exists $p^*>>0$ such that $Z_l(p^*)=0,l=1,...,L-1$. Then,
    \begin{equation}
        \begin{aligned}
            0=p^*\cdot Z(p^*)=\sum_{l=1}^Lp_l^*\cdot Z_l(p^*)=p^*_LZ_L(p^*)\\
            p_L^*>0 \Rightarrow Z_L(p^*)=0
        \end{aligned}
        \nonumber
    \end{equation}
\end{remark}

\section{Private Ownership Production Economy}
\begin{enumerate}
    \item There are $L$ perfectly divisible goods. The commodity space is $\mathbb{R}^L$.
    \item There are $m$ consumers. Each consumer $i=1,...,m$ has a preference relation $\succeq_i$ represented by a utility function $u_i: \mathbb{R}_+^L \rightarrow \mathbb{R}$, an initial endowment $w_i\in \mathbb{R}_+^L$, and owns shares $\{\theta_{ij}:j=1,...,J\}$ in the $J$ firms, where $\theta_{ij}\geq 0, \forall i,j$ and $\sum_{i=}^m\theta_{ij}=1, \forall j$.
    \item There are $J$ firms. Each firm $j=1,...,J$ has a production set $Y_j\subseteq \mathbb{R}^L$ that is nonempty, (representing the constraints of production).
    \begin{note}
        Standard sign convention regarding net output vectors: $y$ represents net output;\\
        $y_k\leq 0$ $\Rightarrow$ good $k$ is a net input in $y$;\\
        $y_k\geq 0$ $\Rightarrow$ good $k$ is a net output in $y$.
    \end{note}
    \item The set of allocation is
    \begin{equation}
        \begin{aligned}
            \mathcal{A}:=\left\{(x,y)=(\underbrace{x_1,...,x_m}_\textnormal{consumption},\underbrace{y_1,...,y_J}_\textnormal{production})\in \mathbb{R}^{L\times m}\times \mathbb{R}^{L\times J}: \sum_{i=1}^mx_i=\sum_{j=1}^J y_j+\sum_{i=1}^mw_i,y_j\in Y_j, \forall j\right\}
        \end{aligned}
        \tag{A}
        \label{A}
    \end{equation}
\end{enumerate}
Given $p\in \mathbb{R}^L$, firm $j$'s problem is to choose production plan $y_j^*$ s.t. $y_j^*\in y_j(p)=\arg\max p\cdot y_j$
\begin{equation}
    \begin{aligned}
        y_j^*\in y_j(p)=\arg\max_{y_j\in Y_j}\quad &p\cdot y_j
    \end{aligned}
    \tag{ystar}
    \label{eq:y_star}
\end{equation}

Given $p\in \mathbb{R}^L$ and production plans in $\{y_j(p),j=1,...,J\}$, consumer $i$'s problem is to choose $x_i^*$ s.t.
\begin{equation}
    \begin{aligned}
        x_i^*\in x_i(p)=\arg\max_{x_i\in \mathbb{R}^L_+}\quad& u_i(x_i)\\
        s.t.\quad& p\cdot x_i\leq p\cdot w_i+\sum_{j=1}^J\theta_{ij}p\cdot y_j(p)
    \end{aligned}
    \tag{xstar}
    \label{eq:x_star}
\end{equation}

\subsection{Competitive Equilibrium}
\begin{definition}[Competitive Equilibrium]
    %\normalfont
    An allocation $(x^*,y^*)$ and a price vector $p^*\in \mathbb{R}^L$ are a \textit{competitive equilibrium} in a \underline{private ownership production economy} if
    \begin{enumerate}[(i).]
        \item $x_i^*\in x_i(p^*)$ (given by \eqref{eq:x_star}) for all agent $i$. That is,
        \begin{equation}
            \begin{aligned}
                x_i^*\in x_i(p^*)=\arg\max_{x_i\in \mathbb{R}^L_+}\quad& u_i(x_i)\\
        s.t.\quad& p^*\cdot x_i\leq p^*\cdot w_i+\sum_{j=1}^J\theta_{ij}p^*\cdot y^*_j(p^*)
            \end{aligned}
            \nonumber
        \end{equation}
        \item $y_j^*\in y_j(p^*)$ (given by \eqref{eq:y_star}) for all firm $j$. That is,
        \begin{equation}
            \begin{aligned}
                y_j^*\in y_j(p^*)=\arg\max_{y_j\in Y_j}\quad &p^*\cdot y_j
            \end{aligned}
            \nonumber
        \end{equation}
        \item Market clearing: $(x^*,y^*)\in \mathcal{A}$ (given by \eqref{A}). That is $$\sum_{i=1}^m x_i^*=\sum_{j=1}^J y_j^*+\sum_{i=1}^m w_i$$
    \end{enumerate}
\end{definition}

\begin{example}[ (Representative Agent Model)]
    There is a single consumer ($m=1$) and a single firm ($J=1$).\\
    For example, suppose there are $L=2$ goods: time (label/leisure) $x_l$ and consumption $x_c$.\\
    Suppose the firm's production set is defined by a production function $f: \mathbb{R}_+ \rightarrow \mathbb{R}$, so $$Y:=\{(-y_l,y_c)\in \mathbb{R}^2:y_l\geq 0,y_c\leq f(y_l)\}$$
    The set of feasible consumption bundles is
    \begin{equation}
        \begin{aligned}
            \Hat{Y}:=(\underbrace{Y+\{\omega\}}_{\{y+w:y\in Y\}})\cap \mathbb{R}_+^L
        \end{aligned}
        \nonumber
    \end{equation}
\end{example}

\subsection{Pareto Optimal}
\begin{definition}[Pareto Optimal]
    %\normalfont
    An allocation $(x^*,y^*)$ in a private ownership production economy is \textbf{Pareto optimal} if there is no other allocation $(x,y)$ s.t. $x_i\succeq_i x_i^*, \forall i$ and $x_h\succ_h x_h^*$ for some $h$.
\end{definition}

\subsection{First-Welfare Theorem (production)}
\begin{theorem}[First-Welfare Theorem]
    If each consumer's preference relation is locally non-satiated, then every competitive equilibrium allocation in a private ownership production economy is Pareto optimal.
\end{theorem}
\begin{proof}
    Let $(x^*,y^*)$ be a competitive equilibrium allocation with corresponding equilibrium price vector $p^*$. Suppose by the way of contradiction that $(x,y)$ is not Pareto optimal. That is, $\exists$ an allocation $(x,y)$ s.t. $x_i\succeq_i x_i^*, \forall i$ and $x_h\succ_h x_h^*$ for some $h$. Then, by the \eqref{eq:x_star},
    \begin{equation}
        \begin{aligned}
            p^*\cdot x_h>p^*\cdot w_h+\sum_{j=1}^J\theta_{hj}p^*\cdot y_j^*
        \end{aligned}
        \nonumber
    \end{equation}
    and by local non-satiation,
    \begin{equation}
        \begin{aligned}
            p^*\cdot x_i\geq p^*\cdot w_i+\sum_{j=1}^J\theta_{ij}p^*\cdot y_j^*
        \end{aligned}
        \nonumber
    \end{equation}
    Adding together
    \begin{equation}
        \begin{aligned}
            \sum_{i=1}^m p^*\cdot x_i&>\sum_{i=1}^mp^*\cdot w_i+\sum_{j=1}^Jp^*\cdot y^*_j\\
            \Rightarrow \sum_{i=1}^m p^*\cdot x_i-\sum_{i=1}^mp^*\cdot w_i&=p^*\cdot\left[\sum_{i=1}^mx_i-\sum_{i=1}^mw_i\right]>\sum_{j=1}^Jp^*\cdot y^*_j
        \end{aligned}
        \nonumber
    \end{equation}
    As $\sum_{i=1}^m x_i=\sum_{j=1}^J y_j+\sum_{i=1}^m w_i$, we have $\sum_{i=1}^m x_i-\sum_{i=1}^m w_i=\sum_{j=1}^J y_j$,
    \begin{equation}
        \begin{aligned}
            \sum_{j=1}^Jp^*\cdot y_j=p^*\cdot\left[\sum_{i=1}^mx_i-\sum_{i=1}^mw_i\right]>\sum_{j=1}^Jp^*\cdot y^*_j
        \end{aligned}
        \nonumber
    \end{equation}
    There is a contradiction, since this implies there is a firm $j$ and $y_j\in Y_j$ s.t. $p^*\cdot y_j>p^*\cdot y^*_j$, which contradicts to the assumption that $y_j^*$ maximizes profits for form $j$ at $p^*$ (\eqref{eq:y_star}).
\end{proof}

\subsection{Equilibrium with Transfers}
\begin{definition}[``Supportable'' as a Price Equilibrium with Transfers]
%\normalfont
    An \underline{allocation $(x^*,y^*)$} in a private ownership production economy is \textbf{supportable} as a \textbf{price equilibrium with transfers} if there exists a price vector $p^*\in \mathbb{R}^L$ and lump-sum budget-balancing transfers $\{T_i:i=1,...,m\}$ so that $\sum_{i=1}^m T_i=0$, such that:
    \begin{enumerate}
        \item $\forall i$,
        \begin{equation}
            \begin{aligned}
                x_i^*\in &\arg\max_{x\in \mathbb{R}^L_+}u_i(x_i)\\
                &\textnormal{ s.t. } p^*\cdot x_i\leq p^*\cdot w_i +\sum_{j=1}^J\theta_{ij}p^*\cdot y^*_j + T_i
            \end{aligned}
            \nonumber
        \end{equation}
        \item $\forall j$,
        \begin{equation}
            \begin{aligned}
                y_j^*\in \arg\max_{y_j\in Y_j}\ p^*\cdot y_j
            \end{aligned}
            \nonumber
        \end{equation}
        \item Feasibility: $$\sum_{i=1}^m x_i^*=\sum_{j=1}^J y_j^*+\sum_{i=1}^m w_i$$
    \end{enumerate}
\end{definition}

\subsection{Second Welfare Theorem (production)}
Production economy is a more general form of exchange economy. It is the same that not every P.O. allocation can be supported as a price signal with transfers.

\begin{theorem}[Second Welfare Theorem (production)]
    If each consumer's preference relation is continuous, strongly monotone, and convex, and each firm's production set is convex, then every interior P.O. allocation in a private ownership production economy can be supported as a price equilibrium with transfers.
\end{theorem}
\begin{proof}
    Let $(x^*,y^*)$ be an interior P.O. allocation, so $x_i^*>>0, \forall i$. The same as exchange economy, for each agent $i$, let $P_i:=\{x_i\in \mathbb{R}_+^L: u_i(x_i)>u_i(x_i^*)\}$ and let
    \begin{equation}
        \begin{aligned}
            P:&=P_1+\cdots+P_m\\
            &=\left\{z\in \mathbb{R}_L^+: z=\sum_{i=1}^m x_i \textnormal{ for some }(x_1,...,x_m)\in \mathbb{R}_+^{L\times m} \textnormal{ s.t. }u_i(x_i)> u_i(x_i^*), \forall i\right\}
        \end{aligned}
        \nonumber
    \end{equation}
    Then $P$ is non-empty and convex.

    In production side, let
    \begin{equation}
        \begin{aligned}
            Y:=\sum_{j=1}^JY_j=\left\{y\in \mathbb{R}^L:y=\sum_{j=1}^Jy_j \textnormal{ for some }(y_1,...,y_J) \textnormal{ s.t. }y_j\in Y_j, \forall j\right\}
        \end{aligned}
        \nonumber
    \end{equation}
    Then $Y+\{w\}$ is non-empty and convex.
    \begin{claim}
        $P\cap (Y+\{w\})=\emptyset$. This follows from the assumption that $(x^*,y^*)$ is P.O. (There is no allocation gives higher utilities while satisfies constraints).
    \end{claim}
    By the Separating Hyperplane Theorem \ref{SHT}, $\exists p\in \mathbb{R}^L, p\neq 0$, s.t.
    \begin{equation}
        \begin{aligned}
            p\cdot z\geq p\cdot(y+w), \forall z\in P \textnormal{ and } y\in Y
        \end{aligned}
        \tag{p:SHT}
        \label{SHTp}
    \end{equation}

    For each $i$, set $$T_i:=p\cdot x_i^*-p\cdot w_i-\sum_{j=1}^J\theta_{ij}p\cdot y_j^*$$
    Then,
    \begin{equation}
        \begin{aligned}
            p\cdot x_i^*=p\cdot w_i+\sum_{j=1}^J\theta_{ij}p\cdot y_j^*+T_i
        \end{aligned}
        \nonumber
    \end{equation}
    and
    \begin{equation}
        \begin{aligned}
            \sum_{1=1}^m T_i&=\sum_{i=1}^m \left(p\cdot x_i^*-p\cdot w_i-\sum_{j=1}^J\theta_{ij}p\cdot y_j^*\right)\\
            &=p\cdot \left(\sum_{i=1}^m x_i^*-\sum_{i=1}^m w_i-\sum_{j=1}^J y_j^*\right)\\
            &=p\cdot 0=0
        \end{aligned}
        \nonumber
    \end{equation}
    So, $\{T_i:i=1,...,m\}$ are budget balancing.
    \begin{claim}\label{C1}
        \begin{equation}
            \begin{aligned}
                p\cdot z\geq p\cdot (\sum_{i=1}^m x^*)=p\cdot (\sum_{j=1}^J y_j^*+\sum_{i=1}^m w_i)\geq p\cdot(y+w), \forall z\in P \textnormal{ and } y\in Y
            \end{aligned}
            \nonumber
        \end{equation}
    \end{claim}
    To prove this, first note the feasibility, $\sum_{j=1}^J y_j^*+\sum_{i=1}^m w_i\in Y+\{w\}$ and $\sum_{i=1}^m x_i^*=\sum_{j=1}^J y_j^*+\sum_{i=1}^m w_i$. By the \ref{SHTp},
    \begin{equation}
        \begin{aligned}
            p\cdot z\geq p\cdot\left(\sum_{j=1}^J y_j^*+\sum_{i=1}^m w_i\right)=p\cdot \left(\sum_{i=1}^mx_i^*\right), \forall z\in P
        \end{aligned}
        \nonumber
    \end{equation}
    Now using the strong monotonicity, for each $i$, we can choose a sequence $\{x_i^n\}\subseteq P_i$ s.t. $x_i^n \rightarrow x_i^*$ (e.g. $x_i^n=\left(1+\frac{1}{n}\right)x_i^*$). Let $z^n=\sum_{i=1}^m x_i^n$ for all $n$. Then, $z^n \in P$ for all $n$ and $z^n \rightarrow \sum_{i=1}^m x_i^*$.\\
    Let $y\in Y$ be arbitrary. By the \ref{SHTp},
    \begin{equation}
        \begin{aligned}
            &p\cdot z^n\geq p\cdot (y+w), \forall n\\
            \Rightarrow &\lim_{n \rightarrow \infty} p\cdot z^n=p\cdot\left(\sum_{i=1}^m x^*\right)\geq p\cdot (y+w)\\
            \Rightarrow & p\cdot\left(\sum_{j=1}^J y_j^*+\sum_{i=1}^m w_i\right)=p\cdot \left(\sum_{i=1}^mx_i^*\right)\geq p\cdot (y+w)
        \end{aligned}
        \nonumber
    \end{equation}
    That is, claim \ref{C1} is proved.
    \begin{claim}\label{C2}
        $\forall j$: $p\cdot y_j^*\geq p\cdot y_j, \forall y_j\in Y_j$
    \end{claim}
    To show this, we fix $k$ and $y_k\in Y_k$, such that $y_k+\sum_{j\neq k}y_j^*\in Y$. By claim \ref{C1},
    \begin{equation}
        \begin{aligned}
            &p\cdot (\sum_{j=1}^J y_j^*+w)\geq p\cdot \left(y_k+\sum_{j\neq k}y_j^*+w\right)\\
            \Rightarrow &p\cdot y_k^*\geq p\cdot y_k
        \end{aligned}
        \nonumber
    \end{equation}
    Hence, claim \ref{C2} is proved.
    \begin{claim}\label{C3}
        $\forall i$: $u_i(x_i)> u_i(x_i^*) \Rightarrow p\cdot x_i>p\cdot x_i^*$.
    \end{claim}
    Note that in the proof for the SWT in exchange economy, it is sufficient to show $\forall i$: $u_i(x_i)> u_i(x_i^*) \Rightarrow p\cdot x_i\geq p\cdot x_i^*$. Fix $h$ and let $x_h\in P_h$. So, $u_h(x_h)>u_h(x_h^*)$. By the continuity and strong monotonicity of preference, we have $x_h+\sum_{i\neq h}x_i^*\in P$ (we can increase each $x_i^*$ a little and reduce $x_h$). Hence, by \ref{C1},
    \begin{equation}
        \begin{aligned}
            &p\cdot (x_h+\sum_{i\neq h}x_i^*)\geq p\cdot \left(\sum_{i=1}^m x_i^*\right)\\
            \Rightarrow &p\cdot x_h\geq p\cdot x_h^*
        \end{aligned}
        \nonumber
    \end{equation}
    Hence, claim \ref{C3} is proved.

    All in all, SWT is proved.
\end{proof}


\section{Existence of Competitive Equilibrium}
\subsection{Excess Demand in Exchange Economies}
\begin{assumption}\label{C}
    Suppose
    \begin{itemize}
        \item each consumer's preference relation is continuous, strongly monotone, and strictly convex, and
        \item $\sum_{i}w_i>>0$.
    \end{itemize}
\end{assumption}
Based on this assumption \ref{C}, we have
\begin{enumerate}[$\circ$]
    \item Each agent's demand function $x_i:\mathbb{R}^L_{++} \rightarrow \mathbb{R}^L_+$ is well-defined, continuous, homogeneous of degree $0$, and satisfies Walras' Law (for individual).
    \item Excess demand function $Z:\mathbb{R}_{++} \rightarrow \mathbb{R}^L$ given by $$Z(p)=\sum_{i=1}^m x_i(p)-\sum_{i=1}^m w_i$$ is
    \begin{definition}[Condition (1) to (4)]
        %\normalfont
        Given the Assumption \ref{C}, following conditions are satisfied:
        \begin{enumerate}[(1).]
            \item Continuous;
            \item Homogeneous of degree $0$;
            \item Satisfies Walras' Law: $p\cdot Z(p)=0, \forall p$;
            \item Bounded below: $\exists s>0$ s.t. $Z_l(p)\geq -s, \forall p, \forall l=1,...,L$.
        \end{enumerate}
    \end{definition}
\end{enumerate}

\subsection{Excess Demand in Production Economies}
We use the same assumption \ref{C} for consumers' preferences, and we add a assumption on the production side.
\begin{assumption}\label{F}
    Suppose each firm's production set $Y_j$ is
    \begin{itemize}
        \item closed,
        \item strictly convex ($y,y'\in Y_j \Rightarrow \alpha y+(1-\alpha)y'\in \textnormal{int}Y_j, \forall \alpha\in (0,1)$),
        \item bounded above ($\exists \bar{y}_j\in \mathbb{R}^L$ s.t. $y\leq \bar{y}_j, \forall y_j\in Y_j$), and
        \item $0\in Y_j$.
    \end{itemize}
\end{assumption}
Based on this assumption \ref{F}, we have
\begin{enumerate}[$\circ$]
    \item Each firm's supply function $y_j: \mathbb{R}_{++}^L \rightarrow \mathbb{R}^L$ is well-defined, continuous, and homogeneous of degree $0$.
\end{enumerate}
Based on assumption \ref{C} and \ref{F}, we have
\begin{enumerate}[$\circ$]
    \item Excess demand function $Z:\mathbb{R}_{++} \rightarrow \mathbb{R}^L$ given by $$Z(p)=\sum_{i=1}^m x_i(p)-\sum_{j=1}^Jy_j(p)-\sum_{i=1}^m w_i$$ is
    \begin{definition}[Condition (1) to (4)]
        %\normalfont
        Given the Assumption \ref{F}, following conditions are satisfied:
        \begin{enumerate}[(1).]
            \item Continuous;
            \item Homogeneous of degree $0$;
            \item Satisfies Walras' Law: $p\cdot Z(p)=0, \forall p$;
            \item Bounded below: $\exists s>0$ s.t. $Z_l(p)\geq -s, \forall p, \forall l=1,...,L$.
        \end{enumerate}
    \end{definition}
    \item If $Z(p^*)=0$, then $p^*$ is a competitive equilibrium price vector, with corresponding equilibrium allocation $\left(x_1(p^*),...,x_m(p^*),y_1(p^*),...,y_J(p^*)\right)$.
\end{enumerate}

\subsection{Boundary Condition}
Since $Z$ is homogeneous of degree $0$, we can normalize prices, set
\begin{equation}
    \begin{aligned}
        \Delta:=\left\{p\in \mathbb{R}^L_+:\sum_{l=1}^Lp_l=1\right\}
    \end{aligned}
    \nonumber
\end{equation}
We give other notations:
\begin{equation}
    \begin{aligned}
        \partial\Delta:&=\left\{p\in \Delta: p_l=0 \textnormal{ for some }l\right\}\\
        \textnormal{int}\Delta:&=\left\{p\in \Delta: p_l>0, \forall l\right\}=\Delta\cap \mathbb{R}^L_{++}\\
    \end{aligned}
    \nonumber
\end{equation}

Consider an exchange economy. Let $p\in \partial \Delta$. Let $w_i>>0$. If $\succeq_i$ is strongly monotone on $\mathbb{R}^L_{++}$, then demand of agent $i$ is undefined at $p_i$ (infinity for the zero price good).

So, we add a condition for excess demand $Z$:
\begin{definition}[Condition (5)]
    %\normalfont
    \begin{enumerate}
        \item[(5).] If $p^n\in \textnormal{int}\Delta, \forall n$ and $p^n \rightarrow p$, where $p\in \partial \Delta$, then $\max_l\{Z_l(p^n)\}\rightarrow +\infty$.
    \end{enumerate}
\end{definition}
\begin{note}
    \begin{enumerate}[-]
        \item Condition (5) holds in an exchange economy with assumption \ref{C};
        \item Condition (5) holds in a production economy with assumption \ref{C} and \ref{F};
        \item Condition (5) is \underline{not} true in general, and the condition (5) does \underline{not} imply $p^n_l \rightarrow 0 \Rightarrow Z_l(p^n) \rightarrow +\infty$ (relative prices matter!)
        \item By Walras' Law and lower bound on $Z$, then the converse holds:
        \begin{equation}
            \begin{aligned}
                Z_l(p^n)\rightarrow +\infty \Rightarrow p^n_l \rightarrow 0
            \end{aligned}
            \nonumber
        \end{equation}
    \end{enumerate}
\end{note}

So, if condition (1) to (5) hold and $p^n \rightarrow p$ where $p^n >>0$ and $p_l>0$, then $\{Z_l(p^n)\}$ is bounded.

\subsection{Existence of Competitive Equilibrium}
\begin{theorem}[Condition (1) to (5) $\Rightarrow$ $\exists$ a competitive equilibrium]
    Let $Z: \mathbb{R}^L_{++} \rightarrow \mathbb{R}^L$ be a function s.t. condition (1) to (5) are satisfied, that is
    \begin{enumerate}[(1).]
        \item Continuous;
        \item Homogeneous of degree $0$;
        \item Satisfies Walras' Law: $p\cdot Z(p)=0, \forall p$;
        \item Bounded below: $\exists s>0$ s.t. $Z_l(p)\geq -s, \forall p, \forall l=1,...,L$.
        \item If $p^n\in \textnormal{int}\Delta, \forall n$ and $p^n \rightarrow p$, where $p\in \partial \Delta$, then $\max_l\{Z_l(p^n)\}\rightarrow +\infty$.
    \end{enumerate}
    \begin{note}
        (Fulfilled when Assumption \ref{C} and \ref{F} are satisfied).
    \end{note}
    Then, $\exists \bar{p}\in \mathbb{R}^L_{++}$ s.t. $Z(\bar{p})=0$.
\end{theorem}
\begin{proof}
    First, restrict attention to $p\in\Delta=\left\{p\in \mathbb{R}^L_+:\sum_{l=1}^Lp_l=1\right\}$.
    \begin{enumerate}[$\circ$]
        \item For $p\in \textnormal{int}\Delta$, define a subset of good $\Lambda(p)\subseteq \{1,...,L\}$ by
        \begin{equation}
            \begin{aligned}
                \Lambda(p):=\left\{l\in \{1,...,L\}: Z_l(p)=\max_k Z_k(p)\right\}
            \end{aligned}
            \nonumber
        \end{equation}
        \item For $p\in \Delta\backslash\textnormal{int}\Delta=\partial \Delta$, let
        \begin{equation}
            \begin{aligned}
                \Lambda(p):=\left\{l\in \{1,...,L\}: p_l=0\right\}
            \end{aligned}
            \nonumber
        \end{equation}
    \end{enumerate}
    Then, note $\Lambda(p)\neq\emptyset, \forall p\in\Delta$.\\
    Define the correspondence $\varphi: \Delta \rightarrow 2^\Delta$ that maps a price vector $p$ to a set of prices by
    \begin{equation}
        \begin{aligned}
            \varphi(p):=\left\{q\in \Delta:q_l=0, \forall l\notin \Lambda(p)\right\}
        \end{aligned}
        \nonumber
    \end{equation}
    As $\Lambda(p)\neq\emptyset, \forall p\in\Delta$, we have $\varphi(p)\neq \emptyset$ for all $p$ and
    \begin{equation}
        \begin{aligned}
            \varphi(p):=\left\{\begin{matrix}
                \left\{q\in \Delta:q\in\arg\max_{\tilde{q}\in\Delta}\tilde{q}\cdot Z(p) \right\}& \textnormal{ if }p\in \textnormal{int}\Delta\\
                \{q\in\Delta:q\cdot p=0\}&\textnormal{ if }p\in \partial\Delta
            \end{matrix}\right.
        \end{aligned}
        \nonumber
    \end{equation}
    \begin{note}
        \begin{enumerate}
            \item $\varphi(p)\subseteq \partial \Delta \Leftrightarrow \Lambda(p)\neq\{1,...,L\}$
            \item $p\in\partial \Delta \Rightarrow p$ is not a fixed point of $\varphi$.
            \item $p$ is a fixed point of $\varphi$ $\Leftrightarrow$ $p\in \textnormal{int}\Delta$ and $\Lambda(p)=\{1,...,L\}$ $\Leftrightarrow$ $p\in \textnormal{int}\Delta$ and $\exists m\in \mathbb{R}$ s.t. $Z_l(p)=m, \forall l=\{1,...,L\}$.
            \item $p$ is a fixed point of $\varphi$ $\Leftrightarrow$ $p\in \textnormal{int}\Delta$ and $Z(p)=0$. (By Walras' Law: $0=p\cdot Z(p)=m \sum_{l}p_l=m$.)
            \item $Z(p)\neq 0$ $\Rightarrow$ $\Lambda(p)\neq\{1,...,L\}$ $\Rightarrow$ $\varphi(p)\subseteq \partial \Delta$.
        \end{enumerate}
    \end{note}
    Now it suffices to show $\varphi$ has a fixed point. Note that $\forall p\in \Delta$, $\varphi(p)$ is non-empty, convex, and compact, and $\Delta$ is non-empty, convex, and compact.
    \begin{claim}
        $\varphi$ has closed graph.
    \end{claim}
    Let $p^n \rightarrow p\in\Delta$ and $q^n \rightarrow q\in\Delta$ where $(p^n,q^n)\in$ graph $\varphi$ $\forall n$. We want to show $(p,q)\in$ graph $\varphi$ (i.e., $q\in \varphi(p)$):
    \begin{enumerate}[$\circ$]
        \item \underline{Case 1}: Suppose $p\in \textnormal{int}\Delta$. Since $p>>0$, assume without losing generality, $p^n>>0$ $\forall n$. Suppose $l\notin \Lambda(p)$, we must show $q_l=0$. Since $p>>0$, $l\notin \Lambda(p) \Rightarrow Z_l(p)<\max_{k}Z_k(p)$. Since $Z$ is continuous, $\exists N$, such that $\forall n\geq N$, $Z_l(p^n)<\max_k Z_k(p^n)$ $\Rightarrow l\notin \Lambda(p^n), \forall n\geq N$ $\Rightarrow q^n_l=0, \forall n\geq N$ $\Rightarrow q_l=\lim_{n \rightarrow \infty}q^n_l=0$. So, $q\in \varphi(p)$.
        \item \underline{Case 2}: Suppose $p\in \partial\Delta$. Without losing generality, we write $p=(0,...,0,p_{r+1},...,p_L)$, where $p_l>0$ for all $l=r+1,...,L$. So, $\Lambda(p)=\{1,...,r\}$ and $\varphi(p)=\{\tilde{q}\in\Delta:\tilde{q}_l=0,l=r+1,...,L\}$.
        \begin{enumerate}[-]
            \item \underline{Case 2A}: Suppose $\{p^n\}$ has a subsequence in $\textnormal{int}\Delta$. Without losing generality, let $\{p^n\}$ denote this subsequence. Since $p^n \rightarrow p\in\partial \Delta$, $\max_k Z_k(p^n) \rightarrow +\infty$. Also, by Walras' Law and lower bound in $Z$, $\{Z_l(p^n)\}$ is bounded for $l=r+1,...,L$.\\
            Since $p^n\in \textnormal{int}\Delta, \forall n$, $\exists N_2$ s.t. $\forall n\geq N_2$, $\Lambda(p^n)\subseteq \{1,...,r\}$. Since $q^n\in \varphi(p^n), \forall n$ and $q^n_l=0,\forall l=r+1,...,L, \forall n\geq N_2$, we have $q_l=\lim_n q_l^n=0$. Hence, $q\in \varphi(p)$.
            \item \underline{Case 2B}: No subsequence of $\{p^n\}$ lies in $\textnormal{int}\Delta$. Without losing generality, take $\{p^n\}\subseteq \partial \Delta$. Now, because $p_l>0$ for $l=r+1,...,L$, $\exists N_3$ s.t. $\forall n\geq N_3$, $p^n_l>0$ for $l=r+1,...,L$. Then, $\Lambda(p^n)\subseteq \{1,...,r\}, \forall n\geq N_3$.\\
            By the same argument above (in Case 2A), we have $q^n\in\varphi(p^n),\forall n$ $\Rightarrow$ $q^n_l=0,\forall l=r+1,...,L, \forall n\geq N_3$ $\Rightarrow$ $q_l=\lim_{n \rightarrow \infty}q_l^n=0, \forall l=r+1,...,L$. Hence, $q\in\varphi(p)$.
        \end{enumerate}
    \end{enumerate}
    All in all, $\varphi$ has closed graph. By Kakutani's Fixed Point Theorem, $\varphi$ has a fixed point. By above augment, $\bar{p}\in \textnormal{int}\Delta$ and $Z(\bar{p})=0$.
\end{proof}


\begin{corollary}
    If an exchange economy satisfies assumption \ref{C}, then it has a competitive equilibrium. If a private ownership production economy satisfies assumption \ref{C} and \ref{F}, then it also has a competitive equilibrium.
\end{corollary}


\section{Uniqueness of Equilibrium}
When is the equilibrium unique?

One condition:
\begin{definition}[Strong Weak Axiom]
    %\normalfont
    The function $Z: \mathbb{R}_{++}^L \rightarrow \mathbb{R}^L$ satisfies the \underline{strong weak axiom} if for any $\bar{p}\in \mathbb{R}_{++}^L$ s.t. $Z(\bar{p})=0$ and any $p\in \mathbb{R}^L_{++}$ s.t. $p\neq \alpha\bar{p}, \forall \alpha>0$, $$\bar{p}\cdot Z(p)>0$$
\end{definition}

\begin{theorem}[Uniqueness of Equilibrium]
    If $Z: \mathbb{R}_{++}^L \rightarrow \mathbb{R}^L$ satisfies \underline{condition (1)-(5)} and \underline{strong weak axiom}, then there is a unique $p^*\in \textnormal{int}\Delta$ s.t. $Z(p^*)=0$.
\end{theorem}
\begin{proof}
    Since $Z$ satisfies condition (1)-(5), $\exists p^*\in \textnormal{int}\Delta$ s.t. $Z(p^*)=0$. By the strong weak axiom, if $p\in \textnormal{int}\Delta$ and $p\neq p^*$, then $p^*\cdot Z(p)>0$ $\Rightarrow$ $Z(p)\neq 0$. So, there is a unique $p^*\in \textnormal{int}\Delta$ s.t. $Z(p^*)=0$.
\end{proof}
\begin{example}
    \begin{enumerate}
        \item In an exchange economy with a representative consumer with strictly quasi-concave, strongly monotone, and $C^1$ (first-order continuously differentiable) utility function and $\omega>>0$, the excess demand function satisfies the strong weak axiom.
        \item If $Z$ satisfies \underline{gross substitutes} (for each $l$, $Z_l$ is increasing in $p_k,\forall k\neq l$), then $Z$ satisfies the strong weak axiom.
    \end{enumerate}
\end{example}


\section{Market Demand and Observable Implications}
Given an outcome, can we say it is obtained from an economy?

Restrict to exchange economies for simplicity.

\begin{theorem}[Sonnenschein-Mantel-Debreu (SMD) Theorem]
    Let $Z: \mathbb{R}_{++}^L \rightarrow \mathbb{R}^L$ be a function that is continuous and satisfies Walras' Law ($p\cdot Z(p)=0, \forall p$). Then $\forall \epsilon>0$ there is an exchange economy with $L$ consumers having continuous, strictly convex, strongly monotone preferences, and endowments $\{w_i:i=1,...,L\}\subseteq \mathbb{R}_+^L$, s.t., the excess demand function for this economy is equivalent to $Z$ on $\Delta^\epsilon=\{p\in\Delta: p_l\geq \epsilon, \forall l\}$.
\end{theorem}


\begin{theorem}[Mas-Colell Theorem]
    Let $E\subseteq \textnormal{int}\Delta$ be compact. Then there exists an exchange economy with $L$ consumers for which $E$ is the set of competitive equilibrium prices.
\end{theorem}

\begin{example}
    Suppose $L=2$, $Z: \mathbb{R}_{++}^2 \rightarrow \mathbb{R}^2$ satisfies condition (1)-(5). Choose good $2$ as numéraire, and set $p=(p_1,1)$. Then,
    \begin{enumerate}
        \item If $p_1$ is close to $0$, then $Z_1(p)>0$ by the boundary condition (5) and lower bound condition (4).
        \item If $p_1$ is large, then $Z_2(p)>0$ by the boundary condition (5) and lower bound condition (4). So, $Z_1(p)=-\frac{1}{p_1}Z_2(p)<0$ by Walras' Law (condition (3)).
        \item $\exists p_1^*$ s.t. $Z_1(p^*)=0$ by continuous $Z$ (condition (1)), and $p^*$ is an equilibrium price vector by Walras' Law (condition (3)).
    \end{enumerate}
\end{example}


\section{Comparative Statics and Local Uniqueness}
Restrict to exchange economies for simplicity.

Let $\vec{w}=(w_1,...,w_m)\in \mathbb{R}^{L\times m}_+$ denote profile of initial endowments.\\
Let $\mathcal{E}(\vec{w})$ denote the economy with fixed preference relations $\{\succeq_i:i=1,...,m\}$ and endowment profile $\vec{w}$.

Let $Z: \mathbb{R}_{++}^L\times \mathbb{R}_{++}^{L\times m} \rightarrow \mathbb{R}^L$ denote excess demand as a function of $(p,\vec{w})$, so
\begin{equation}
    \begin{aligned}
        Z(p,\vec{w}):= \sum_{i=1}^m x_i(p,p\cdot w_i)-\sum_{i=1}^m w_i
    \end{aligned}
    \nonumber
\end{equation}

Let
\begin{equation}
    \begin{aligned}
        Z_{-L}(p,\vec{w}):= \left(Z_1(p,\vec{w}),...,Z_{L-1}(p,\vec{w})\right)
    \end{aligned}
    \nonumber
\end{equation}
Normalize $p_L=1$. Given $\vec{w}$, equilibrium in $\mathcal{E}(\vec{w})$ corresponds to $p$ s.t. $Z_{-L}(p,\vec{w})=0$.

Assume $Z(\cdot,\vec{w})$ is $C^1$ and satisfies conditions (1)-(5), $ \forall \vec{w}$.

\begin{definition}[Regular Equilibrium]
    %\normalfont
    Given $\vec{w}$, an equilibrium price vector $p$ is a \textbf{regular equilibrium} if $D_p Z_{-L}(p,\vec{w})$ is non-singular (has full rank $L-1$).
\end{definition}

\begin{definition}[Regular/Critical Economy]
    %\normalfont
    If every equilibrium in the economy $\mathcal{E}(\vec{w})$ is regular, then $\mathcal{E}(\vec{w})$ is a \textbf{regular economy}. An economy that is not regular is a \textbf{critical economy}.
\end{definition}


\begin{proposition}
    \begin{enumerate}
        \item Regular equilibria are \underline{locally unique}.\\ ($\exists$ open set $V$ with $p^*\in V$ s.t. if $p\in V$, then $Z_{-L}(p,\vec{w})=0 \Leftrightarrow p=p^*$)
        \item A regular economy has \underline{finitely} many equilibria.
        \item In a regular economy, local equilibrium \underline{comparative statics} are determinate.
        \item $\mathcal{E}(\vec{w})$ is a regular economy if \underline{$0$ is a regular value} of $Z(\cdot,\vec{w})$. ($D_p Z(0,\vec{w})$ is non-singular).
    \end{enumerate}
\end{proposition}

\begin{theorem}[$C^1$ Demand Functions $\Rightarrow$ Regular Economy]
    Suppose each agent $i=1,...,m$ has a $C^1$ demand function: $x_i:\mathbb{R}_{++}^L\times \mathbb{R}_{++}^{L} \rightarrow \mathbb{R}_+^L$. Then almost all economy are regular. That is,
    \begin{equation}
        \begin{aligned}
            \left\{\vec{w}\in \mathbb{R}_{++}^{L\times m}: \mathcal{E}(\vec{w}) \textnormal{ is a critical economy}\right\}
        \end{aligned}
        \nonumber
    \end{equation}
    has Lebesgue measure zero in $\mathbb{R}^{L\times m}$.
\end{theorem}
\begin{proof}
    To prove $DZ_{-L}$ has full rank, it is sufficient to show the sub-matrix has rank $L-1$.
    
    Denote the derivative matrix of $DZ_{-L}(p,\vec{\omega})$ by $A$. The $(j,k)$ item of the matrix is
    \begin{equation}
        \begin{aligned}
            \frac{\partial x_{jk}}{\partial p_k}+\frac{\partial x_{jk}}{\partial w}\omega_k
        \end{aligned}
        \nonumber
    \end{equation}
    Compute the derivatives of $Z_{-L}$ with respect to the initial endowment of consumer $1$ ($\omega_1$). Denote the derivative matrix by $A$. The $(j,k)$ item of the matrix is
    \begin{equation}
        \begin{aligned}
            A_{jk}=\left\{\begin{matrix}
                \frac{\partial x_{1j}}{\partial w}(p,p\cdot\omega_1)p_k-1,&k=j\\
                \frac{\partial x_{1j}}{\partial w}(p,p\cdot\omega_1)p_k,&k\neq j
            \end{matrix}\right.
            ,j=1,...,L-1, k=1,...,L
        \end{aligned}
        \nonumber
    \end{equation}

    Minus $k=1,...,L-1$ column by $\frac{p_k}{p_L}$ times $L^\textnormal{th}$ column. We can get
    \begin{equation}
        \begin{aligned}
            \begin{bmatrix}
                    &A_{1L}\\
                -I_{L-1\times L-1}	&\vdots\\
                &A_{L-1,L}
            \end{bmatrix}
        \end{aligned}
        \nonumber
    \end{equation}
    which has rank $L-1$. Hence, $A$ has rank $L-1$.

    Hence, $0$ is a regular value of $Z_{-L}$. So, the result follows from the Transversality Theorem.
\end{proof}


\section{General Equilibrium with Uncertainty}
Set-up:
\begin{enumerate}
    \item There are $L$ (physical) goods.
    \item There are two time periods, $t=0,t=1$.
    \item At date $t=0$, the state of nature is determinate.
    \item At date $t=1$, there are $S$ ($s\in\{1,...,S\}$) possible states of nature, and all uncertainty resolves at $t=1$.
    \item Hence, the commodity space is $\mathbb{R}^{L\times(S+1)}$.
    \item Each consumer has preference relation $\succeq_i$ over $\mathbb{R}_+^{L\times(S+1)}$ represented by utility function $u_i$ and initial endowment under different states $w_i=(w_{i0},w_{i1},...,w_{iS})\in \mathbb{R}_+^{L\times(S+1)}$, where $w_{i0}$ is the endowment at $t=0$ and $w_{is}$ is the endowment at $t=1$ with state $s\in\{1,...,S\}$.
\end{enumerate}

\subsection{Basic Settings: (Complete) Contingent Commodities, Arrow-Debreu Equilibrium}
\begin{definition}[Contingent Commodity]
    %\normalfont
    A unit of the \textbf{contingent commodity} (or \textbf{Arrow security}) $l_s$ is a claim to receive a unit of good $l$ \underline{if and only if} state $s$ occurs at date $t=1$
\end{definition}
Suppose at date $t=0$, there are markets for ``date $t=0$ consumption'' and ``a complete set of Arrow securities''.

Given price vector $p\in \mathbb{R}^{L\times (S+1)}$, agent $i$'s budget set is
\begin{equation}
    \begin{aligned}
        B_i(p)=\left\{x\in \mathbb{R}_+^{L\times(S+1)}: p\cdot x\leq p\cdot w_i\right\}
    \end{aligned}
    \nonumber
\end{equation}
\begin{definition}[Arrow-Debreu Equilibrium]
    %\normalfont
    A competitive equilibrium in this model is an \textbf{Arrow-Debreu equilibrium}.
\end{definition}


\subsection{General: Asset Markets and Radner Equilibrium}
There is a market of spots. Assets are using for the trading across different stages.

\begin{note}
    \underline{Simplify:} assume all assets payoff in units of good $1$.
\end{note}

An asset is defined by its return vector $r\in\mathbb{R}^{S}$.

\begin{example}
    \begin{enumerate}
        \item Arrow securities $l_{\bar{s}}$ (for good $1$): $r_s=\left\{\begin{matrix}
            1,&s=\bar{s}\\
            0,&s\neq \bar{s}
        \end{matrix}\right.$
        \item Riskless bond: $r_s=1,\forall s$.
        \item Another asset: $r_s=\left\{\begin{matrix}
            1,& s \textnormal{ is even}\\
            -1,& s \textnormal{ is odd}
        \end{matrix}\right.$. Hence, $r=\left(-1,1,...,(-1)^{|S|}\right)$.
    \end{enumerate}
\end{example}

Suppose there are $K$ assets traded at date $t=0$, indexed by $k=1,...K$. Summarize their payoffs in an $S\times K$ matrix.
\begin{equation}
    \begin{aligned}
        R:=\begin{bmatrix}
            r_{11}&\cdots&r_{K1}\\
            \vdots&\vdots&\vdots\\
            r_{1S}&\cdots&r_{KS}\\
        \end{bmatrix}
    \end{aligned}
    \nonumber
\end{equation}
where $r_{ks}$ is the asset $k$'s return in state $s$.

Assume assets are in zero total supply. Let $z_i\in \mathbb{R}^k$ denote agent $i$'s portfolio. So,
\begin{equation}
    \begin{aligned}
        z_{ik}=\textnormal{\# units of asset $k$ bought/sold by agent $i$}
    \end{aligned}
    \nonumber
\end{equation}

Let $q=(q_1,...,q_K)\in \mathbb{R}^K$ denote the vector of asset prices.

Given an asset payoff structure $R$, the payoff from the portfolio $z_i$ is $Rz_i\in \mathbb{R}^S$.

Let $p_s\in \mathbb{R}^L$ denote price vector expected at date $0$ to hold in the spot market at date $t=1$ if state $s$ occurs.

\begin{definition}[Radner Equilibrium]
    %\normalfont
    A consumption allocation $(x_1^*,...,x_m^*)$, portfolio profile $(z_1^*,...,z_m^*)$, spot price vectors $(p_0^*,p_1^*,...,p_S^*)$, and asset price vector $q^*$ are a \textbf{Radner equilibrium} if
    \begin{enumerate}
        \item For every agent $i$: $(x_i^*,z_i^*)$ solves
        \begin{equation}
            \begin{aligned}
                \max_{(x_i,z_i)\in \mathbb{R}^{L\times (S+1)}_+\times \mathbb{R}^K} &u_i(x_i)\\
                \textnormal{s.t. } &
                \left.\begin{matrix}
                p_0^*\cdot x_{i0}+q^*\cdot z_i\leq p_0^*\cdot w_{i0}\\
                p_s^*\cdot x_{is}\leq p_s^*\cdot w_{is}+p_{1s}^*(Rz_i)_s, \forall s
                \end{matrix}\right\}\triangleq B_i(p^*,q^*,R)
            \end{aligned}
            \nonumber
        \end{equation}
        (reminds that we assume all assets are payoff in good $1$).
        \item $\sum_{i=1}^mz_i^*=0$, $\sum_{i=1}^m x_i^*=\sum_{i=1}^m w_i$.
    \end{enumerate}
\end{definition}

\begin{note}
    \underline{Normalize:} $p_{1s}=1, \forall s=1,...,S$.
\end{note}
Then consumer $i$'s budget set is
\begin{equation}
    \begin{aligned}
        B_i(p,q,R):=\bigg\{x_i\in \mathbb{R}^{L\times (S+1)}_+:\exists z\in \mathbb{R}^K\textnormal{ s.t. }p_0\cdot x_{i0}+q\cdot z\leq p_0\cdot w_{i0}
        \textnormal{ and }p_s\cdot(x_{is}-w_{is})\leq (Rz_i)_s, \forall s\bigg\}
    \end{aligned}
    \nonumber
\end{equation}


\begin{definition}[Complete Asset Structure $R$]
    %\normalfont
    The asset structure with return matrix $R$ is \textbf{complete} if $\textnormal{rank}R=S$. And the asset structure is \textbf{incomplete} if $\textnormal{rank}R<S$.
\end{definition}

\begin{theorem}[Arrow-Debreu Equilibrium $\Leftrightarrow$ Radner Equilibrium]
    Suppose preferences are \underline{strongly monotone} and the asset structure is \underline{complete}.
    \begin{enumerate}
        \item If $(x^*,p^*)$ is an Arrow-Debreu equilibrium, then there exists a portfolio profile $z^*$ and asset price vector $q^*$ such that $(x^*,p^*,q^*,z^*)$ is a Radner equilibrium.
        \item If $(x^*,p^*,q^*,z^*)$ is a Radner equilibrium, then there exists a vector $\mu\in \mathbb{R}_{++}^S$ (i.e., common beliefs) such that $(x^*,(p_0^*,\mu_1p_1^*,...,\mu_Sp_S^*))$ is an Arrow-Debreu equilibrium.
    \end{enumerate}
\end{theorem}



\begin{definition}[Arbitrage-free]
%\normalfont
    For asset structure with return matrix $R$, the asset price vector $q\in \mathbb{R}^K$ is \textbf{arbitrage-free} if $\nexists z\in \mathbb{R}^K$ s.t. $q\cdot z\leq 0$ and $R z\geq 0$ with strict inequality for one.
\end{definition}

\begin{proposition}[Strongly Monotone $\Rightarrow$ Arbitrage-free]
    If preferences are strongly monotone, then in any Radner equilibrium, asset prices must be arbitrage-free.
\end{proposition}

We can also infer the components of assets from arbitrage-free prices.
\begin{theorem}[Reconstruct Assets from Arbitrage-free Prices]
    Consider an asset structure with return matrix $R$. If
    \begin{itemize}
        \item $q\in\mathbb{R}^K$ is arbitrage-free, \textbf{or}
        \item $q\in\mathbb{R}^K$ is a Radner equilibrium asset prices with $r_k\geq 0, r_k\neq 0, \forall k$.
    \end{itemize}
    then there exists $\mu\in \mathbb{R}_{++}^S$ such that $q=\mu^T R$, that is, s.t. $$q_k=\sum_s \mu_s r_{ks}, \forall k$$
\end{theorem}









\chapter{Game Theory}
Based on
\begin{enumerate}[$\circ$]
    \item "Kreps, D. M., \& Sobel, J. (1994). Signalling. \textit{Handbook of game theory with economic applications}, 2, 849-867."
    \item Mas-Colell, Whinston, and Green, Microeconomic Theory, Oxford University Press (1995).
    \item UIUC ECON 530 21Fall, Nolan H. Miller
    \item UC Berkeley ECON 201A 23Fall, 201B 24Spring
    \item UC Berkeley MATH 272 23Fall, Alexander Teytelboym
    \item  Jehle, G., Reny, P.: Advanced Microeconomic Theory. Pearson, 3rd ed. (2011). Ch. 6.
    \item MIT 14.16 Strategy and Information, Mihai Manea
\end{enumerate}



\section{Basic Game Theory}
\subsection{Action and Domination Theorem}
Let $A$ be the finite set of possible actions and $\Omega$ be the finite set of possible states. A function can map the action and state to a value, $u(a,\omega)$. It can be represented by $\vec{u}(a)=\{u(a,\omega)\}_{\omega\in\Omega}$. It is common in game theory to assume the utility function is given or known.

A \textbf{mixed action} is a probability distribution over $A$, $\sigma\in\Delta(A)$.

A \textbf{belief} of the agent is a probability distribution over $\Omega$, $\mu\in\Delta(\Omega)$.

\begin{definition}[Optimal and Justifiable Mixed Action]
    %\normalfont
    A mixed action $\sigma\in\Delta(A)$ is \textbf{optimal} given $\mu\in\Delta(\Omega)$ if $$\mathbb{E}_\mu u(\sigma,\tilde{\omega})\geq \mathbb{E}_\mu u(\sigma',\tilde{\omega}),\ \forall \sigma'\in \Delta(A)$$
    A mixed action $\sigma\in\Delta(A)$ is \textbf{justifiable} if it is optimal for some belief $\mu\in\Delta(\Omega)$.
\end{definition}

\begin{definition}[Dominant and Dominated Action]
    %\normalfont
    A mixed action $\sigma\in\Delta(A)$ is \textbf{dominant} if $$u(\sigma,\omega)>u(\sigma',\omega),\ \forall \omega\in \Omega, \sigma'\in \Delta(A),\sigma\neq\sigma'$$
    A mixed action $\sigma\in\Delta(A)$ is \textbf{dominated} if $$u(\sigma,\omega)<u(\sigma',\omega),\ \forall \omega\in \Omega, \text{ and for some } \sigma'\in \Delta(A)$$
    In this case we say $\sigma'$ dominates $\sigma$.
\end{definition}

\begin{theorem}[Domination Theorem: Justifiable $=$ Not Dominated]
    A mixed action is justifiable \underline{if and only if} it is not dominated.
\end{theorem}
\begin{proof}
    $\Rightarrow$ is easily proved by the definition. We focus on proving $\Leftarrow$:
    
    Let $\mathcal{U}=\{\vec{u}(\sigma):\sigma\in\Delta(A)\}$ and $\sigma^*$ be an undominated mixed action. Then, we have $\mathcal{U}\cap(\{\vec{u}(\sigma^*)\}+\mathbb{R}_{++}^\Omega)=\emptyset$. Because $\mathcal{U}$ and $\{\vec{u}(\sigma^*)\}+\mathbb{R}_{++}^\Omega$ are disjoint, convex, and nonempty, we can use the Separating Hyperplane Theorem \ref{SHT}: $\exists p\in \mathbb{R}^\Omega,p\neq 0$ such that $p\cdot a\leq p\cdot b, \forall a\in\mathcal{U}, b\in (\{\vec{u}(\sigma^*)\}+\mathbb{R}_{++}^\Omega)$.

    \begin{claim}
        $p\cdot \vec{u}(\sigma)\leq p\cdot \vec{u}(\sigma^*), \forall \sigma\in\Delta(A)$.
    \end{claim}
    \begin{proof}
        For any positive number $m$, $\vec{u}(\sigma^*)+(\frac{1}{m},....,\frac{1}{m})\in \{\vec{u}(\sigma')\}+\mathbb{R}_{++}^\Omega$. So, for any $\sigma\in\Delta(A)$, $p\cdot \vec{u}(\sigma)\leq p\cdot\left(\vec{u}(\sigma^*)+(\frac{1}{m},....,\frac{1}{m})\right)$. By taking limit, $p\cdot \vec{u}(\sigma^*)=\lim_{m \rightarrow \infty}p\cdot\left(\vec{u}(\sigma^*)+(\frac{1}{m},....,\frac{1}{m})\right)\geq p\cdot \vec{u}(\sigma)$.
    \end{proof}
    \begin{claim}
        $p>0$.
    \end{claim}
    \begin{proof}
        Prove by the contradiction. Suppose $p_\omega<0$ for some $\omega\in\Omega$. Let $z=(\epsilon,...,\epsilon)+M\mathbb{1}_\omega, M>0,\epsilon>0$. So, $\vec{u}(\sigma^*)+z\in (\{\vec{u}(\sigma^*)\}+\mathbb{R}_{++}^\Omega)$. We have $p\cdot\vec{u}(\sigma^*)\leq p\cdot (\vec{u}(\sigma^*)+z)$ by the result of SHT. There is a contradiction since $p_\omega<0$. So, we have $p\geq 0$. Because $p\neq 0$, $p>0$ is proved.
    \end{proof}
    Finally, we normalize $p$ to $\mu=\frac{1}{\sum_{\omega}p_\omega}p$. Then, $\sigma^*$ is optimal for the belief $\mu$, which means $\sigma^*$ is justifiable.
\end{proof}










\subsection{Extensive Game}
\begin{definition}[History]
    %\normalfont
    The sequences of actions are called \textbf{histories}. $h'=(\underbrace{a_1,...,a_n}_{h: \text{prefix of }h'},a_{n+1},...)\in H$. We call $h'$ is the \textbf{continuation} of $h$. $h$ is a \textbf{terminal} of $H$ if there is no continuation of $h$ in $H$. ($\emptyset\in H$.)
\end{definition}

\begin{definition}[Extensive form Perfect Information Game]
    %\normalfont
    Am extensive form game with prefect information is defined as $G=\{N,A,H,Z,P,O,o,\succ_{n\in N}\}$, where $N$ is the set of players, $A$ is the set of actions, $H$ is the set of all histories, $Z$ is the set of all histories that are terminals, $P:H/Z \rightarrow N$ is a mapping from a non-terminal histories to a player (who moves after a non-terminal history), $O$ is the set of outcomes, and $o$ is a function from $Z$ to $O$.\\
    A PIG is \underline{finite horizon} if there is a bound on the length of its histories.
\end{definition}


We denote $A(h)$ as the actions available to player $P(h)$ after a history $h$.

Let $H_i=\{h\in H/Z:i=P(h)\}$ be the set of histories that player $i$ moves after.

\begin{definition}[Strategy]
    %\normalfont
    A \textbf{strategy} is defined as a function $s_i:H_i \rightarrow A$ for which $s_i(h)\in A(h),\forall h\in H_i$. Let $S_i$ be the set of all strategies available to the player $i$. A \textbf{strategy profile} is a collection of strategy $s=(s_i)_{i\in N}$.
\end{definition}



\begin{definition}[Subgame]
%\normalfont
    A \textbf{subgame} of a PIG $G=\{N,A,H,Z,P,O,o,\succ_{n\in N}\}$ is a game (a PIG) that starts after a given finite history $h\in H$. Formally, the subgame $G(h)$ associated with $h=(h_1,...,h_n)\in H$ is $G(h)=\{N,A,H_h,Z,P_h,O,o_h,\succ_{n\in N}\}$, where
    \begin{equation}
        \begin{aligned}
            H_h=\{(a_1,a_2,...):(h_1,...,h_n,a_1,a_2,...)\in H\}\\
            o_h(h')=o(hh'), P_h(h')=P(hh')
        \end{aligned}
        \nonumber
    \end{equation}
    A strategy $s$ of $G$ defines a strategy $s_h$ of $G(h)$ by $s_h(h')=s(hh')$.
\end{definition}

\begin{definition}[Subgame Perfect Equilibrium (SPNE)]
    %\normalfont
    A \textbf{subgame perfect equilibrium (SPNE)} of $G$ is a strategy profile $s^*$ such that for every subgame $G(h)$ it holds that $h^{\prime} \mapsto s_i^*\left(h h^{\prime}\right)$ is an optimal strategy in $G(h)$, given beliefs that the rest of the players behave according to $s_{-i}^*$ (or its restriction to $G(h)$).
\end{definition}

\begin{definition}[Profitable Deviation]
    %\normalfont
    Let $s$ be a strategy profile. We say that $s_i^{\prime}$ is a \textbf{profitable deviation} from $s$ for player $i$ at history $h$ if $s_i^{\prime}$ is a strategy for $G$ such that
    $$
    o_h\left(s_i^{\prime}, s_{-i}\right) \succ_i o_h(s)
    $$
\end{definition}
Note that a strategy profile has no profitable deviations iff it's a SPNE.

\begin{theorem}[The one-deviation principle]
    Let $G=\left(N, A, H, O, o, P,\left\{\preceq_i\right\}_{i \in N}\right)$ be a finite horizon, extensive form game with perfect information. Let $s$ be a strategy profile that is \underline{not} a subgame perfect equilibrium. There exists some history $h$ and a profitable deviation $\bar{s}_i$ for player $i=P(h)$ in $G(h)$ such that $\bar{s}_i(k)=s_i(k)$ for all $k \neq h$.
\end{theorem}
\begin{enumerate}[$\circ$]
    \item Let $G=\left(N, A, H, O, o, P,\left\{\preceq_i\right\}_{i \in N}\right)$ be a PIG.
    \item $A(\emptyset)$ is the set of allowed initial actions for player $i=P(\emptyset)$. For each $b \in A(\emptyset)$, let $s^{G(b)}$ be some strategy profile for the subgame $G(b)$.
    \item Given some $a \in A(\emptyset)$, we denote by $s^a$ the strategy profile for $G$ in which player $i=P(\emptyset)$ chooses the initial action $a$, and for each action $b \in A(\emptyset)$ the subgame $G(b)$ is played according to $s^{G(b)}$.
    \item So $s_i^a(\emptyset)=a$ and for every player $j, b \in A(\emptyset)$ and $b h \in H \backslash Z$, $s_j^a(b h)=s_j^{G(b)}(h)$.
\end{enumerate}
\begin{lemma}[Backward Induction]
    Let $G=\left(N, A, H, Z, O, o, P,\left\{\preceq_i\right\}_{i \in N}\right)$ be a finite PIG. Assume that for each $b \in A(\emptyset)$ the subgame $G(b)$ has a subgame perfect equilibrium $s^{G(b)}$. Let $i=P(\emptyset)$ and let $a$ be the $\succ_i$-maximizer over $A(\emptyset)$ of $o_a\left(s^{G(a)}\right)$. Then $s^a$ is a subgame perfect equilibrium of $G$.
\end{lemma}


\subsection{Strategic Form Game}
\begin{definition}[Normal Form Game]
    %\normalfont
    A game in \textbf{normal form} is denoted by $$G =\left(\underbrace{N}_{\textnormal{players}},\underbrace{\{S_i\}_{i\in N}}_{\textnormal{Strategy Set}},\underbrace{\{u_i(\cdot)\}_{i\in N}}_{\textnormal{VNM utility}}\right)$$

    $u_i:\prod_{i\in I}S_i \rightarrow \mathbb{R}$ is the utility function that maps all players' strategies to a player's utilities.

    A \underline{finite} game is a normal-form game in which the set of players $N$ is a finite set, and the set of strategy profiles $S$ is finite.
\end{definition}

\begin{definition}[Mixed/Pure Strategy]
%\normalfont
A mixed strategy  for player $i$ is a probability distribution $\sigma_i\in\Delta(S_i)$.\\
Elements of $S_i$ are called pure strategies.
\end{definition}

\begin{definition}[Dominant/Dominated Strategy]
    %\normalfont
    A strategy $\sigma_i\in \Delta(S_i)$ is a \textbf{dominant strategy} for $i$ in $G$, if we have $u_i(\sigma_i,\sigma_{-i})> u_i(\sigma'_i,\sigma_{-i}), \forall \sigma'_i\neq \sigma_i, \sigma_{-i}\in\times_{j\neq i}\Delta(S_j)$.\\
    A strategy $\sigma_i\in \Delta(S_i)$ is a \textbf{dominated strategy} for $i$ in $G$, if $\exists \sigma'_i\neq \sigma_i$, $u_i(\sigma_i,\sigma_{-i})<u_i(\sigma'_i,\sigma_{-i}), \forall \sigma_{-i}\in\times_{j\neq i}\Delta(S_j)$.\\
    A strategy $\sigma_i\in \Delta(S_i)$ is a \textbf{weakly dominated strategy} for $i$ in $G$, if $\exists \sigma'_i\neq \sigma_i$, $u_i(\sigma_i,\sigma_{-i})\leq u_i(\sigma'_i,\sigma_{-i}), \forall \sigma_{-i}\in\times_{j\neq i}\Delta(S_j)$ and there is a $\sigma_{-i}\in\times_{j\neq i}\Delta(S_j)$, $u_i(\sigma_i,\sigma_{-i})< u_i(\sigma'_i,\sigma_{-i})$
\end{definition}

\begin{lemma}
    1. A dominant strategy is always pure.\\
    2. A strategy $\sigma'_i$ dominates $\sigma_i$ iff $u_i(\sigma'_i,s_{-i})> u_i(\sigma_i,s_{-i})$, for all pure strategy profiles $s_{-i}\in S_{-i}$.
\end{lemma}


\begin{definition}[Belief, Best Response]
    %\normalfont
    A \textbf{belief} for player $i$ is a probability distribution $\mu\in\Delta(S_{-i})$.\\
    A strategy $\sigma_i \in \Delta(S_i)$ is the \textbf{best response} to beliefs $\mu$ if it solves the problem of $\max_{\sigma_i\in\Delta(S_i)}u_i(\sigma_i,s_{-i})$.\\
    Denote the set of all best responses to $\mu$ by $\beta_i(\mu)$.
\end{definition}
\begin{lemma}[Mixed Strategy is BR iff its Pure Strategies are Indifferent]
    A mixed strategy $\sigma_i$ is in $\beta_i(\mu)$ iff every pure strategy in the support of $\sigma_i$ is in $\beta_i(\mu)$. In particular, every strategy in the support of $\sigma_i$ yields the same payoff to $i$.
\end{lemma}

\begin{theorem}[Domination Theorem rephrased]
    In a finite game, a strategy is dominated iff there is no belief to which it is a best response.
\end{theorem}


\begin{definition}[Algorithm: Iterated Elimination of Dominated Strategies (IEDS)]
    %\normalfont
    Let $\left(N,\left(S_i\right),\left(u_i\right)\right)$ be a finite game; $N=[n]$.
    \begin{enumerate}[$\bullet$]
        \item We define (inductively) $n$ sequences of sets of mixed strategies.
        \item Let $D_i^0=\Delta\left(S_i\right)$.
        \item Given $D_1^{k-1}, \ldots, D_n^{k-1}$, let
        $$
        D_i^k=\left\{\sigma_i: \nexists \bar{\sigma}_i: u_i\left(\sigma_i, \sigma_{-i}\right)<u_i\left(\bar{\sigma}_i, \sigma_{-i}\right) \forall \sigma_{-i} \in \times_{j \neq i} D_j^{k-1}\right\} .
        $$
        \item Note that $\left\{D_i^k\right\}$ is a decreasing sequence of sets.
        \item Let $D_i = \cap_{k=0}^\infty D_i^k$.
        \item The set $D = \times_{i=1}^n D_i$ be the set of strategies that survive the iterated elimination of dominated strategies.
    \end{enumerate}
    A game is called \textbf{dominance-solvable} if $D$ is a singleton.
\end{definition}


\begin{definition}[Rationalizable Strategies]
%\normalfont
    \begin{enumerate}[$\bullet$]
        \item $R_i^0=\Delta\left(S_i\right)$.
        \item Given $R_1^{k-1}, \ldots, R_n^{k-1}$, Let
        $$
        \begin{aligned}
        & Z_i^k=\left\{s_i \in S_i: \sigma_i\left(s_i\right)>0 \text { for some } \sigma_i \in R_i^{k-1}\right\} \\
        & R_i^k=\left\{\sigma_i \in \Delta\left(S_i\right): \exists \mu \in \Delta\left(\times_{j \neq i} Z_j^k\right) \text { s.t. } \sigma_i \in \beta_i(\mu)\right\}
        \end{aligned}
        $$
    \end{enumerate}
    Note: $\left\{R_i^k\right\}_{k=0}^{\infty}$ is a decreasing sequence of sets.\\
    Let $R_i=\cap_{k=0}^{\infty} R_i^k$.\\
    The \textbf{rationalizable strategies} are the elements of $R=\times_{i=1}^n R_i$.
\end{definition}

\begin{lemma}
    In a finite game, $R$ is always non-empty and contains a pure strategy profile.
\end{lemma}

\begin{proposition}
    $\sigma_i \in \Delta(S_i)$ is \textbf{rationalizable} iff there are sets $Z_1,..., Z_n, Z_j \subseteq S_j$ such that
    \begin{enumerate}
        \item $\sigma_i \in \beta_i(\mu_i)$ for some $\mu_i \in \Delta(\times_{h\neq i}Z_h)$.
        \item for every $s_j \in Z_j$ there is $\mu_j \in \Delta(\times_{h\neq j}Z_h)$ such that $s_j\in \beta_j(\mu_j)$.
    \end{enumerate}
\end{proposition}

\begin{corollary}[Rationalizable = IEDS]
    Rationalizable strategies are exactly the strategies survive the iterated elimination of dominated strategies, $$R=D$$
\end{corollary}
















\subsection{Nash Equilibrium and Existence}
\begin{definition}[Nash Equilibrium]
    %\normalfont
    A strategy profile $\Sigma=(\sigma_1,...,\sigma_I)$ is a \textbf{Nash} equilibrium of the game $G$ if for every $i\in I$, we have: $u_i(\sigma^*_i,\sigma^*_{-i})\geq u_i(\sigma'_i,\sigma^*_{-i}), \forall \sigma'_i\in \Delta(S_i)$ (no profitable deviation). In other words,
    \begin{enumerate}
        \item $\sigma_i$ is the \underline{best response} to beliefs $\mu_i\in \Delta (S_{-i})$
        \item $\mu_i=\sigma_{-i}$ (correct beliefs).
    \end{enumerate}
\end{definition}
\begin{enumerate}
    \item In rationalizable strategies, beliefs can be incorrect.
    \item In a Nash equilibrium, beliefs are correct. Any strategy in a Nash equilibrium is rationalizable.
\end{enumerate}

\begin{definition}[Best Response Correspondence]
    %\normalfont
    In a Nash equilibrium the player $i$'s best response correspondence $\beta_i:\Delta(S_{-i})\rightarrow 2^{\Delta(S_i)}$ is defined as $\beta_i(\sigma_{-i})=\arg\max_{\sigma_i\in\Delta(S_i)}u_i(\sigma_i,\sigma_{-i})$. Let $\beta(\sigma)=\times_{i\in I}\beta_i(\sigma_{-i})$. Then $\sigma$ is a Nash equilibrium iff $\beta(\sigma)=\sigma$. $\beta$ is called the \textbf{best response correspondence} of the game.
\end{definition}

\begin{theorem}[Existence of Nash Equilibrium]
    A Nash equilibrium exists in a finite game $\Gamma$, if for all $i\in I$,
    \begin{enumerate}[(i).]
        \item $S_i$ is non-empty, convex, compact, subset of $\mathbb{R}^m$ (i.e., for some finite dimensions of real numbers).
        \item $u_i(s_i,...,s_I)$ is continuous in $(s_i,...,s_I)$ and quasi-concave in any $s_i$.
    \end{enumerate}
\end{theorem}
\begin{proof}
    We prove a lemma for the best response correspondence $\beta_i(s_{-i})=\argmax_{s_i\in S_i}u(s_i,s_{-i})$ firstly.
    \begin{lemma}
        Suppose $\{S_i\}_{i\in I}$ are non-empty. Suppose that $S_i$ is compact and convex and $u_i$ is continuous in $(s_i,...,s_I)$ and quasi-concave in any $s_i$, then best response correspondence $\beta_i(s_{-i})$ is non-empty, convex-valued and uhc.
    \end{lemma}
    \begin{proof}
        This lemma is proved by Berge's Maximum Theorem (Theroem \ref{thm:Berge's Maximum Theorem}).
    \end{proof}
    Consider the best response correspondence of the game $\beta$ with $\beta(s_i,...,s_I)=\{\beta_1(s_{-1}),...,\beta_I(s_{-I})\}$.

    As we proved $\beta$ is non-empty, convex-valued and uhc from $S$ to $S$ where $S$ is non-empty, compact, and convex. By the Kakutani's Fixed Point Theorem (Theorem \ref{thm:Kakutani's Fixed Point Theorem}), we have $\beta$ has a fixed point $s\in S$, which should be the Nash equilibrium.
\end{proof}

\subsection{Bayesian Game}
\begin{definition}[Bayesian Game]
    %\normalfont
    A \textbf{Bayesian game} is defined by $$\Gamma=(I, \Omega, \{A_i\}_{i\in I}, \{u_i(\cdot)\}_{i\in I},\{\Theta_i\}_{i\in I}, \{F_i\}_{i\in I})$$
    where $\Omega$ is the state space, $u_i:A\times \Omega$ is $i$'s payoff function, and $F_i\in\Delta\left(\Omega\times\Theta_i\right)$ is the (prior) distribution of the player $i$'s type.
\end{definition}

\begin{definition}[Normal-form Bayesian game]
    %\normalfont
    Assume a finite game. The \textbf{normal-form game} can be represented by $$\left(I,(S_i,U_i)_{i\in I}\right)$$ defined by letting $S_i$ be the set of strategies based on types $s_i:\Theta_i \rightarrow A_i$ and
    \begin{equation}
        \begin{aligned}
            U_i(s)=\sum_{\omega\in\Omega}\sum_{(\theta_i)_{i\in I}\in \Theta}p(\omega,\theta_1,...,\theta_I)u_i(s_1(\theta_1),..., s_I(\theta_I),\omega)
        \end{aligned}
        \nonumber
    \end{equation}
    for all $s\in S$.

    A \textbf{Bayesian Nash equilibrium} (BNE) of a Bayesian game is a strategy profile $(s_1,...,s_n)$ that is a Nash equilibrium of the derived normal-form game.
\end{definition}

\begin{definition}[Best Response, Interim Payoff]
    %\normalfont
    $s_i$ is a BR to $s_{-i}$ iff for all $\theta_i$, $s_i(\theta_i)$ maximizes the \textbf{interim payoff} of player $i$. The interim payoff is defined by the expected payoff given the type $\theta_i$ of player $i$ by playing action $a_i$.
    \begin{equation}
        \begin{aligned}
            \mathbb{E}_{\omega\in\Omega,\tilde{\theta}_{-i}\in\Theta_{-i}}[u_i(a_i,s_{-i}(\tilde{\theta}_{-i}),\omega)|\theta_i]
        \end{aligned}
        \nonumber
    \end{equation}
\end{definition}

\subsection{Zero-sum Game}
Let $G=(\{1,2\},(S_1,S_2),(u_1,u_2))$. Suppose there is a constant $c$ so that $u_1(s)+u_2(s)=c, \forall s\in S$. Then $G$ is equivalent to a zero-sum game.

\begin{definition}[Saddle Point]
    %\normalfont
    Let $X,Y$ be sets and $f:X\times Y \rightarrow \mathbb{R}$ a real function. $(x^*,y^*)\in X\times Y$ is a \textbf{saddle point} of $f$ if $x^*\in\argmax_{x\in X}f(x,y^*)$ and $y^*\in\argmin_{y\in Y}f(x^*,y)$. That is,
    \begin{equation}
        \begin{aligned}
            f(x,y^*)\leq f(x^*,y^*)\leq f(x^*,y),\forall x\in X,y\in Y
        \end{aligned}
        \nonumber
    \end{equation}
\end{definition}

Consider a zero-sum game of two players. The strategy of the player 1 is max-min strategy, which is given by
\begin{equation}
    \begin{aligned}
        \max_{\sigma_1}\min_{\sigma_2}u(\sigma_1,\sigma_2)
    \end{aligned}
    \nonumber
\end{equation}
and the strategy of the player 2 is min-max strategy, which is given by
\begin{equation}
    \begin{aligned}
        \min_{\sigma_2}\max_{\sigma_1}u(\sigma_1,\sigma_2)
    \end{aligned}
    \nonumber
\end{equation}
\begin{proposition}[min-max$\geq$max-min]
    Min-max strategy is always better than max-min strategy. That is,
    \begin{equation}
        \begin{aligned}
            \max_{\sigma_1}\min_{\sigma_2}u(\sigma_1,\sigma_2)\leq \min_{\sigma_2}\max_{\sigma_1}u(\sigma_1,\sigma_2)
        \end{aligned}
        \nonumber
    \end{equation}
\end{proposition}
\begin{proof}
    As $u(\sigma_1',\sigma_2)\leq \max_{\sigma_1}u(\sigma_1,\sigma_2)$ for all $\sigma_1'$, we have
    \begin{equation}
        \begin{aligned}
            \min_{\sigma_2}u(\sigma_1',\sigma_2)&\leq \min_{\sigma_2}\max_{\sigma_1}u(\sigma_1,\sigma_2),\forall \sigma_1'\\
            \Rightarrow \max_{\sigma_1}\min_{\sigma_2}u(\sigma_1,\sigma_2)&\leq \min_{\sigma_2}\max_{\sigma_1}u(\sigma_1,\sigma_2)
        \end{aligned}
        \nonumber
    \end{equation}
\end{proof}
We shall prove that these are in fact equal in the zero-sum game.

\begin{definition}[Value of a zero-sum game]
    %\normalfont
    A \textbf{value} for a zero-sum game $G$ is a number $v \in \mathbb{R}$ for which there exists a strategy profile $\left(\bar{\sigma}_1, \bar{\sigma}_2\right)$ such that
    $$
    \begin{matrix}u\left(\bar{\sigma}_1, \sigma_2\right) \geq v & \text { for all } \sigma_2 \\ u\left(\sigma_1, \bar{\sigma}_2\right) \leq v & \text { for all } \sigma_1\end{matrix}
    $$
\end{definition}
Note: $v=u\left(\bar{\sigma}_1, \bar{\sigma}_2\right)$.
The value is unique (if its exists), and represents a guaranteed payoff for the players.
(Uniqueness: Suppose there are two values, $v$ and $v^{\prime}>v$, achieved by profiles $\sigma$ and $\sigma^{\prime}$. Then when 2 plays $\sigma_2$ we have that $u\left(\sigma_1^{\prime}, \sigma_2\right) \geq v^{\prime}$ because $\sigma_1^{\prime}$ guarantees $v^{\prime}$. And when 1 plays $\sigma_1^{\prime}$ we have $v \geq u\left(\sigma_1^{\prime}, \sigma_2\right)$ because $\sigma_2$ guarantees $v$. So $v^{\prime}>v$ leads to a contradiction.)

\begin{proposition}
    The following statements are equivalent
    \begin{enumerate}
        \item $\left(\bar{\sigma}_1, \bar{\sigma}_2\right)$ is a Nash equilibrium.
        \item $v=u\left(\bar{\sigma}_1, \bar{\sigma}_2\right)$.
    \end{enumerate}
\end{proposition}


\begin{corollary}
    If $\left(\sigma_1^*, \sigma_2^*\right)$ and $\left(\bar{\sigma}_1, \bar{\sigma}_2\right)$ are Nash equilibria of $G$, then so are the profiles $\left(\bar{\sigma}_1, \sigma_2^*\right)$ and $\left(\sigma_1^*, \bar{\sigma}_2\right)$.
\end{corollary}

\begin{theorem}[Minimax Theorem]
    Let $G$ be a zero-sum game. There is a strategy profile $(\sigma_1^*, \sigma_2^*)$ s.t.
    \begin{equation}
        \begin{aligned}
            \max_{\sigma_1}\min_{\sigma_2}u(\sigma_1,\sigma_2)=v= \min_{\sigma_2}\max_{\sigma_1}u(\sigma_1,\sigma_2)
        \end{aligned}
        \nonumber
    \end{equation}
    where $v=u(\sigma_1^*, \sigma_2^*)$ is the value of the game and $(\sigma_1^*, \sigma_2^*)$ is a Nash equilibrium.
\end{theorem}
\begin{proof}
    Let $n_i=|S_i|$. Let $\vec{u}(\sigma_2):=\{u(s_1,\sigma_2):s_1\in S_1\}$. Let $\mathbb{C}=\{\vec{u}(\sigma_2):\sigma_2\in \Delta(S_2)\}$. We can find $\mathbb{C}$ is convex and compact.

    Let $m(x):=\max\{x_i:i=1,...,n_1\}$. Then, the player 2's min max payoff is given by $v:=\inf\{m(x):x\in \mathbb{C}\}$. By compactness, exists strategy $\sigma_2^*$ such that $\vec{u}(\sigma_2^*)=(v,v,...,v)$. That is, $u(s_1,\sigma_2^*)=v,\forall s_1\in S_1$.

    Let $\mathbb{A}:=\{z\in \mathbb{R}^{n_1}:z<<(v,v,...,v)\}=(v,v,...,v)-\mathbb{R}^{n_1}_{++}$. As $\mathbb{C}$ and $\mathbb{A}$ are disjoint. By SHT, we can find a $p\neq 0$ s.t. $p\cdot \mathbb{A}\leq p\cdot \mathbb{C}$. $p>0$ since $\mathbb{A}$ can have arbitrary small elements in any dimension. Then, we normalize $p$ to be in $\Delta(S_1)$, denote it by $\sigma^*_1$.

    By limitation, $v=\sigma^*_1\cdot(v,v,...,v)=\lim_{\epsilon \rightarrow 0^+}\sigma^*_1\cdot (v-\epsilon,v-\epsilon,...,v-\epsilon)\leq \sigma^*_1\cdot\vec{u}(\sigma_2),\forall \sigma_2\in \Delta(S_2)$.

    Hence, $u(s_1,\sigma_2^*)\leq m(\vec{u}(\sigma_2^*))=v=u(\sigma_1^*,\sigma_2^*)\leq u(\sigma_1^*,\sigma_2)$
\end{proof}

\subsection{Correlated equilibrium}
Suppose there is a mediator that give advices to each player based on a distribution $p \in \Delta(S)$. A player doesn't know other players' advices but knows the distribution $p$.
\begin{definition}[Correlated Equilibrium]
    %\normalfont
    A \textbf{correlated equilibrium} of $G$ is any probability distribution $p \in \Delta(S)$ such that, for all $i$ and $s_i,s'_i \in S_i$, the player $i$ can't get a higher expected profit than by following the advice,
    \begin{equation}
        \begin{aligned}
            \sum_{s_{-i}\in S_{-i}}p(s_i,s_{-i})u_i(s_i,s_{-i})\geq\sum_{s_{-i}\in S_{-i}}p(s_i,s_{-i})u_i(s'_i,s_{-i})
        \end{aligned}
        \nonumber
    \end{equation}
    where LHS is the expected profit of player $i$ when he receives an advice $s_i$ from the mediator.
\end{definition}

Let $G$ be a finite game,
\begin{proposition}
    If we identify a Nash equilibrium $\sigma$ of $G$ with a probability distribution on $\Delta(S)$, then any Nash equilibrium of $G$ is also a correlated equilibrium.
\end{proposition}

\begin{proposition}
    The set of correlated equilibria is a non-empty, convex, compact subset of $\Delta(S)$.
\end{proposition}

For $p\in\Delta(S)$, the \textbf{marginal distribution} on $S_i$ is given by
\begin{equation}
    \begin{aligned}
        p_i(s_i)=\sum_{s_{-i}\in S_{-i}}p(s_i,s_{-i})
    \end{aligned}
    \nonumber
\end{equation}
\begin{proposition}
    A correlated equilibrium that is the independent mixture of its marginal distributions is a Nash equilibrium.
\end{proposition}

\begin{proposition}[Correlated Equilibrium Strategy $\Rightarrow$ Rationalizable]
    Let $G = (S_i, u_i)_{i=1}^n$ be a finite normal-form game, and $\rho$ a correlated equilibrium of $G$. Suppose that the profile $(s_i, s_{-i})$ receives strictly positive probability in $\rho: \rho(s_i, s_{-i}) > 0$. $s_i$ is rationalizable.
\end{proposition}
\begin{proof}
    Let $Z_i$ be the set of pure strategies of player $i$ that receive strictly positive probability in $\rho_i$. For each $s_i \in Z_i$ we have that $\sum_{s_{-i}\in S_{-i}} \rho(s_i, s_{-i}) \geq \sum_{s_{-i}\in S_{-i}} \rho(s'_i, s_{-i}),\forall s'_i\in S_i$, so for any $s'_i\in S_i$:
    \begin{equation}
        \begin{aligned}
            \mathbb{E}_{\mu_i}u_i(s_i,s_{-i})&=\sum_{s_{-i}\in S_{-i}}\frac{\rho(s_i, s_{-i})}{\sum_{\tilde{s}_{-i}}\in S_{-i}\rho(s_i,\tilde{s}_{-i})}u_i(s_i,s_{-i})\\
            &\geq \sum_{s_{-i}\in S_{-i}}\frac{\rho(s_i, s_{-i})}{\sum_{\tilde{s}_{-i}}\in S_{-i}\rho(s_i,\tilde{s}_{-i})}u_i(s'_i,s_{-i})=\mathbb{E}_{\mu_i}u_i(s'_i,s_{-i})
        \end{aligned}
        \nonumber
    \end{equation}
    where $\mu_{i}\in\Delta(S_{-i})$ are the beliefs over $S_{-i}$ obtained by $\rho$ conditioning on $s_i$. This means that $s_i$ is a best response to beliefs $\mu_i$. Since $i$ and $s_i\in Z_i$ are arbitrary, we are done.
\end{proof}


\subsection{Quantal Response Equilibrium}
Imagine choosing $a \in A$.
\begin{definition}[Quantal Response (softmax)]
    %\normalfont
    A \textbf{quantal response} is a function $\gamma: \mathbb{R}^A \rightarrow \Delta(A)$ mapping from a vector of utility values $v$ to a probability distribution over actions, which satisfies that
    \begin{enumerate}[$\circ$]
        \item $\gamma(v) \gg 0$ for all $v$ (interior);
        \item $\gamma$ is continuous,
        \item $\gamma$ is monotonic $\left(v_h<v_j \rightarrow \gamma_h(v)<\gamma_j(v)\right)$
        \item $\gamma$ is responsive $\left(v_j<v_j^{\prime} \rightarrow \gamma_j(v)<\gamma_j\left(v_j^{\prime}, v_{-j}\right)\right)$
    \end{enumerate}
\end{definition}
Interpret $\gamma(v)$ as the probability of choosing each of $a \in A$ alternatives when $v$ is the vector of utility values of the alternatives in $A$.

Interpretation: mistakes or random utility.

One common quantal response function is the logistic function:
\begin{equation}
    \begin{aligned}
        \gamma_j(v)=\frac{e^{\lambda v_j}}{\sum_{h\in I} e^{\lambda v_h}}
    \end{aligned}
    \nonumber
\end{equation}
for $\lambda>0$, where $\lambda$ captures how close the quantal response is to choosing according to the largest values of $v$.

Fix a normal-form game $G=\{N,\{S_i,i\in N\},\{u_i,i\in N\}\}$. Let $\gamma_i: \mathbb{R}^{S_i} \rightarrow \Delta(S_i)$ be a quantal response for player $i$.
\begin{definition}[Quantal Response Equilibrium]
    %\normalfont
    A \textbf{quantal response equilibrium} of $G$ is a strategy profile $\sigma^*$ such that $$\sigma^*_i=\gamma_i\left(\vec{u}_i(\sigma^*_{-i})\right)$$
    where $\vec{u}_i(\sigma^*_{-i})=(u_i(s_i,\sigma^*_{-i}))_{s_i\in S_{i}}$.
\end{definition}

\begin{proposition}
    Every finite normal-form game, with any profile of quantal responses, has a quantal response equilibrium.
\end{proposition}


\underline{Observation}: $\lambda$ measures the distance to Nash.
\begin{proposition}
    Let $\{\lambda^k\}$ be a sequence such that $\lim_{k \rightarrow \infty}\lambda^k=\infty$ and $\sigma^*(\lambda^k)$ be a QRE when the logistic quantal responses take parameter value $\lambda^k$. If $\{\sigma^*(\lambda^k)\}$ is a convergent sequence, it converges to a Nash equilibrium.
\end{proposition}

\section{Knowledge and Common Knowledge}
\subsection{Knowledge and Information}
\begin{enumerate}
    \item Let $\Omega$ be a (finite) set of possible states of the world. Information is provided by a subset of $\Omega$. The smaller a subset is, the more information it provides.
    \item Subsets $E\subseteq\Omega$ are \textbf{events}. %\textcolor{red}{An agent ``knows $E$'' if $E$ obtains at all the states that the agent believes are possible.}
    \item
    \begin{definition}[Information Function]
        %\normalfont
        We define the function $P : \Omega \rightarrow 2^\Omega$ an \textbf{information function}. \textcolor{red}{$P(\omega)$ is the set of states that the agent considers possible when the actual state is $\omega$}.
    \end{definition}
    When the state is $\omega$ the decision-maker knows only that the state is in the set $P(\omega)$. Means that, if $\omega' \in P(\omega)$, then when the state is $\omega$, information doesn't allow one to distinguish between $\omega$ and $\omega'$.
    \item
    %Given our interpretation of an information function, a decision-maker for whom $P(\omega) \subseteq E$ knows, in the state $\omega$, that some state in the event $E$ has occurred. The set $K(E)$ is the set of all states in which the decision-maker knows $E$.
    \begin{definition}[Knowledge]
        %\normalfont
        \textbf{Knowledge} is modeled through a function $K:2^\Omega \rightarrow 2^\Omega$.
        $K(E)$ (which we write as $KE$) \textcolor{red}{is the set of states at which the agent knows that the event $E$ has occurred}.
        That is, given an information function $P: \Omega \rightarrow 2^\Omega$, the \textbf{knowledge} $K : 2^\Omega \rightarrow 2^\Omega$ is defined as $$KE=\{\omega\in\Omega:P(\omega)\subseteq E\}$$
        We can say \textcolor{red}{$KE$ is the set of states that ``the agent knows $E$.''}
    \end{definition}
    \item Given a state $\omega$, $\{E \subseteq \Omega : KE \ni \omega\}=\{E \subseteq \Omega : P(\omega)\subseteq E\}$ is the set of all events that the agent ``knows'' (all events that the agent believes are possible). The most accurate information provided by it is $\cap\{E \subseteq \Omega : KE \ni \omega\}$. Hence, $$P(\omega) = \cap\{E \subseteq \Omega : KE \ni \omega\}$$
\end{enumerate}
These two equations provide the back and fourth relationship between knowledge and information. However, we don't give any restrictions for the settings of the knowledge or the information function, so they can be any forms.

\subsection{Partitional Information Function}
\begin{definition}[P1\&P2 Conditions for Information Function]
    %\normalfont
    Usually, we assume following two conditions of a information function:
    \begin{enumerate}[P1.]
        \item $\omega\in P(\omega)$ for every $\omega\in\Omega$. (Reality will not be excluded from agent's information structure.)
        \item If $\omega'\in P(\omega)$ then $P(\omega')=P(\omega)$.
    \end{enumerate}
\end{definition}

\begin{definition}[S5 Conditions for Knowledge]
    %\normalfont
    There are 5 axioms of knowledge that can restrict the form of knowledge.
    \begin{enumerate}
        \item $K \Omega=\Omega$\\ Alice knows that some state of the world has occurred.
        \item $K A \cap K B=K(A \cap B)$\\ Alice knows $A$ and knows $B$ iff she knows $A$ and $B$.
        \item $K A \subseteq A$ (Axiom of knowledge)\\ If Alice knows $A$, then $A$ has indeed occurred (some state in $A$ is true).
        \item $K K A=K A$ (Axiom of positive introspection)\\ If Alice knows $A$ then she knows that she knows $A$.
        \item $(K A)^c=K\left((K A)^c\right)$ (Axiom of negative introspection)\\ If Alice doesn't know $A$ then she knows that she doesn't know $A$.
    \end{enumerate}
    They are not independent.
\end{definition}

\begin{definition}[Partitional]
    %\normalfont
    Information function $P$ (and associated knowledge operator $K$) is \textbf{partitional}
    if $\{P(\omega) : \omega \in \Omega\}$ constitute a partition of $\Omega$.
\end{definition}
When $P$ is partitional we abuse notation and denote the partition by $P$ as well.

\begin{theorem}[S5 $\Leftrightarrow$ P1\&P2 $\Leftrightarrow$ Partitional]
    For a $P$ (and associated $K$), following conditions are equivalent:
    \begin{enumerate}[$\circ$]
        \item $P$/$K$ is partitional;
        \item $P$ satisfies P1 and P2.
        \item $K$ satisfies S5.
    \end{enumerate}
\end{theorem}

\begin{example}[ (Partitional)]
    $\Omega=\{\omega_1,\omega_2,\omega_3,\omega_4\}$.
    \begin{enumerate}
        \item Information Structure: $P(\omega_1)=P(\omega_2)=\{\omega_1,\omega_2\}$, $P(\omega_3)=\{\omega_3\}$, and $P(\omega_4)=\{\omega_4\}$.
        \item Knowledge Function: $K(\{\omega_3,\omega_4\})=\{\omega_3,\omega_4\}$, $K(\{\omega_1,\omega_3\})=\{\omega_3\}$.
    \end{enumerate}
\end{example}

\begin{example}[ (Non-Partitional)]
    $\Omega=\{\textnormal{bark, don't bark}\}$. $P(\omega)=\left\{\begin{matrix}
        \{\textnormal{bark}\} & \textnormal{ if }\omega=\textnormal{bark}\\
        \{\textnormal{bark, don't bark}\}& \textnormal{ if }\omega=\textnormal{don't bark}
    \end{matrix}\right.$ Then, $K(\{\textnormal{bark}\})=\{\textnormal{bark}\}$ and $K(\{\textnormal{don't bark}\})=\emptyset$.  Axiom 5 is violated.
\end{example}

\subsection{Self-evident and Algebra}
\begin{definition}[Self-evident]
    %\normalfont
    An event $A \in 2^\Omega$ is \textbf{self-evident} if $$KA = A$$ That is, the agent ``knows A'' if and only if $A$ happens.
\end{definition}

\begin{proposition}[Self-evident $\Leftrightarrow$ Unions of Elements of Partition]
    ``$A$ is self-evident'' \underline{if and only if} it is the union (1 or more) of elements of the partition $P$.
\end{proposition}
\begin{proof}
    $A$ is self-evident iff $A=KA=\{\omega:P(\omega)\subseteq A\}$.
\end{proof}

\begin{definition}[Algebra]
    %\normalfont
    A collection of events $\Sigma$ is an \textbf{algebra} if it satisfies:
    \begin{enumerate}
        \item $\Omega\in\Sigma$;
        \item If $A\in\Sigma$ then $A^c\in\Sigma$;
        \item If $A,B\in\Sigma$ then $A\cup B\in\Sigma$.
    \end{enumerate}
\end{definition}
Let $\Sigma$ be \textbf{the collection of self-evident events}.
\begin{corollary}[Set of Self-Evident Events is an Algebra]
    Then $\Sigma$ is an algebra (in fact, $\Sigma = \Sigma_P$, the algebra generated by the partition $P$).
\end{corollary}

\begin{example}
    $\Omega=\{\omega_1,\omega_2,\omega_3\}$
    \begin{enumerate}
        \item $P_1=\{\{\omega_1,\omega_2\},\{\omega_3\}\}$, $\Sigma_1=\{\emptyset,\{\omega_1,\omega_2\},\{\omega_3\},\Omega\}$.
        \item $P_2=\{\{\omega_1\},\{\omega_2\},\{\omega_3\}\}$, $\Sigma_2=\{\emptyset,\{\omega_1\},\{\omega_2\},\{\omega_3\},\{\omega_1,\omega_2\},\{\omega_1,\omega_3\},\{\omega_2,\omega_3\},\Omega\}$.
        \item $P_3=\{\{\omega_1\},\{\omega_2,\omega_3\}\}$, $\Sigma_3=\{\emptyset,\{\omega_1\},\{\omega_2,\omega_3\},\Omega\}$.
    \end{enumerate}
\end{example}

\begin{corollary}[$KA\in\Sigma$]
    For any $A$ (despite whether it is in $\Sigma$) and $K$ is partitional, we have $$KA = \cup\{S \in \Sigma : S \subseteq A\} \in \Sigma$$
    So $KA$ is always self-evident. And, since any algebra is closed under unions, it follows that $KA$ is the largest element of $\Sigma$ that is contained in $A$.
\end{corollary}
\begin{proof}
    We prove by two directions:
    \begin{enumerate}[(1).]
        \item \underline{$\omega\in KA \Rightarrow \omega\in\cup\{S \in \Sigma : S \subseteq A\}$}: Given $\omega\in KA$, we have $P(\omega)\subseteq A$. Since $K$ is partitional, $P(\omega)\in\Sigma$. Therefore, $P(\omega)\in\{S \in \Sigma : S \subseteq A\}$. Hence, $\omega\in \cup\{S \in \Sigma : S \subseteq A\}$.
        \item \underline{$\omega\in\cup\{S \in \Sigma : S \subseteq A\} \Rightarrow \omega\in KA$}: Given $\omega\in\cup\{S \in \Sigma : S \subseteq A\}$, there exists $S\in\Sigma$ such that $S \subseteq A$ and $\omega\in S$. Since $\Sigma$ is the collection of self-evident events, we have $KS=S$. Hence, $\omega\in KS\subseteq KA$.
    \end{enumerate}
\end{proof}
Then, we can recover $P(\omega)$ from $\Sigma$ by
\begin{equation}
    \begin{aligned}
        P(\omega)=\cap\{S\in\Sigma:\omega\in S\}\in\Sigma
    \end{aligned}
    \nonumber
\end{equation}

All in all, a knowledge space can be defined by $P$, $K$, or $\Sigma$.

\subsection{Common Knowledge}
Consider a finite set of N agents, each with a (partitional) knowledge function $K_i:2^\Omega \rightarrow 2^\Omega, i\in N$.

\begin{definition}[Refinement and Coarsening of Sub-algebras]
    %\normalfont
    Let $\Sigma, \Pi$ be two sub-algebras of some algebra. We say that $\Sigma$ is a \textbf{refinement} of $\Sigma$ and $\Pi$ is a \textbf{coarsening} of $\Sigma$ if $\Pi \subseteq \Sigma$.
\end{definition}
\begin{example}
    Consider two sub-algebras of $\{\emptyset,\{\omega_1\},\{\omega_2\},\{\omega_3\},\{\omega_1,\omega_2\},\{\omega_1,\omega_3\},\{\omega_2,\omega_3\},\Omega\}$, $$\Sigma=\{\emptyset,\{\omega_1\},\{\omega_2\},\{\omega_3\},\{\omega_1,\omega_2\},\{\omega_1,\omega_3\},\{\omega_2,\omega_3\},\Omega\}\text{ and }\Pi=\{\emptyset,\Omega\}$$
    where $\Pi\subseteq \Sigma$. The $\Pi$ provides a less inaccurate information structure than $\Sigma$.
\end{example}

\begin{definition}[Meet and Join of Sub-algebras]
    %\normalfont
    The \textbf{meet} of two algebras $\Sigma_1, \Sigma_2 \subseteq \Sigma$ is the finest sub-algebra of $\Sigma$ that is a coarsening of each $\Sigma_i$.\\
    Their \textbf{join} is the coarsest sub-algebra of $\Sigma$ that is a refinement of each $\Sigma_i$.
\end{definition}

\begin{definition}[Common Knowledge]
    %\normalfont
    An event $A$ is said to be \textbf{common knowledge} at $\omega\in\Omega$ if for any sequence $i_1,i_2,...,i_k\in N$ it holds that $$\omega\in K_{i_1}K_{i_2}\cdots K_{i_k}A$$
\end{definition}
Let $\Sigma_C=\cap_i\Sigma_i$ be the meet of the player's algebras.
\begin{proposition}
    The following are equivalent:
    \begin{enumerate}
        \item $C\in\Sigma_C$.
        \item $K_iC=C,\forall i\in N$.
    \end{enumerate}
\end{proposition}
\begin{definition}[$K_C$]
    %\normalfont
    Recall that $K_i A=\cup\{S\in\Sigma_i:S\subseteq A\}$. Analogously, we define a knowledge operator from $\Sigma_C$:
    \begin{equation}
        \begin{aligned}
            K_CA=\cup\{S\in\Sigma_C:S\subseteq A\}
        \end{aligned}
        \nonumber
    \end{equation}
\end{definition}
Since $K_C A\in\Sigma_i$ for any $i$, we have $K_i K_C A=K_C A$. Hence, $K_C A$ is common knowledge at $\omega\in K_C A$. As $\Sigma_C$ is also the set of $\{K_C A: A\in 2^\Omega\}$, we can conclude that
\begin{proposition}[$E\in\Sigma_C$$\Leftrightarrow$$E$ is common knowledge at $\omega\in E$]
    For any $E\in\Sigma_C$, $E$ is common knowledge at $\omega\in E$. Conversely, if $E$ is common knowledge at any $\omega \in E$, then $E\in\Sigma_C$ ($E = K_iE$ for any $i$).
\end{proposition}

\begin{example}
    Consider the following game, which is a simplification of the popular board game “Clue.” There are two decks of cards, with four cards each. Deck 1 has numbered cards, with numbers 1,2,3 and 4. Deck 2 has cards with the suites: $\heartsuit$, $\diamondsuit$, $\spadesuit$ and $\clubsuit$. One card from each deck is set aside; turned face down and not seen by any of the players. Of the remaining cards, Player 1 gets all the odd-numbered cards from Deck 1 while Player 2 gets the even cards. From the remainder of Deck 2, Player 1 gets the red cards ($\heartsuit$ and $\diamondsuit$) while 2 gets the black cards ($\spadesuit$ and $\clubsuit$). For example, if the cards set aside are 4 and $\diamondsuit$, then P1 gets 1, 3 and $\heartsuit$, while P2 gets 2, $\clubsuit$ and $\spadesuit$. Players don't see how many cards the other player received.

    A state of the world is a pair of cards that was set aside (and the point of the game is to find out what these are).

    Show that the event
    $$E = \{(3, \clubsuit),(3, \spadesuit),(1, \clubsuit),(1, \spadesuit)\}$$ is common knowledge when the cards set aside are the number 3 and $\clubsuit$.
    \paragraph*{Answer}
    The state is a pair $(n, s)$, with $n$ being a number and $s$ a suit. At any state in $E$, $n$ is odd and $s$ is a black suit. So for any such state the players' information functions take values:
    $P_1(n, s) = \{(n, \clubsuit),(n, \spadesuit)\}$ and $P_2(n, s)= \{(1, s),(3, s)\}$.

    Thus, for any $(n, s) \in E, P_i(n, s) \subseteq E$. This means that $E$ is the union of elements of the partition $P_i$, and thus $E$ is self-evident for each player. So $E \in \Sigma_1 \cap \Sigma_2$ and it is common knowledge at any $(n, s) \in E$.
\end{example}


\subsection{Application: agreeing to disagree}
\begin{definition}[Belief Space]
    %\normalfont
    Suppose
    \begin{enumerate}
        \item a finite set $N$ of players,
        \item a state-space $\Omega$,
        \item each agent endowed with a partitional knowledge $K_i$; resulting in an algebra $\Sigma_i$ on $\Omega$,
        \item each agent endowed with a prior belief $\mu_i\in\Delta(\Omega)$
    \end{enumerate}
    The tuple $\left(N,\Omega,\{\mu_i\}_{i\in N},\{\Sigma_i\}_{i\in N}\right)$ is a \textbf{belief space}.
\end{definition}

Fix $\mu\in\Delta(\Omega)$, the expectation of a random variable $X:\Omega \rightarrow \mathbb{R}$ according to $\mu$ is denoted by $\mathbb{E}_\mu X$.

Suppose that $\mu(B)\in(0,1)$, a conditional probability can be written as $\mu(A|B)=\frac{\mu(A\cap B)}{\mu(B)},\forall A\subseteq \Omega$. Conditional expectation $\mathbb{E}[X|B]=\mathbb{E}_{\mu(\cdot|B)}X$.

Given an algebra $\Sigma_i$ (which represent a knowledge space), we define the expectation of $X$ conditioned on player $i$'s information at $\omega$ is
\begin{equation}
    \begin{aligned}
        \mathbb{E}[X|\Sigma_i](\omega)=\mathbb{E}[X|P_i(\omega)]=\frac{\sum_{\omega'\in P_i(\omega)}\mu(\omega')X(\omega')}{\mu(P_i(\omega))}
    \end{aligned}
    \nonumber
\end{equation}
(where we assume $\mu(P_i(\omega))>0$.)

An event that $\mathbb{E}[X|\Sigma_i]=q$ is the event that, given $i$'s information ($\Sigma_i$), her conditional expectation for the random variable $X$ is $q$: $\{\omega\in\Omega:\mathbb{E}[X|\Sigma_i](\omega)=q\}$.

\begin{proposition}
    The \textbf{Law of Iterated (Total) Expectations}: let $S$ be an element of an algebra $\Pi$, and let $X:\Omega \rightarrow \mathbb{R}$ be a r.v. Then,
    \begin{equation}
        \begin{aligned}
            \mathbb{E}[X|S]=\mathbb{E}[\mathbb{E}[X|\Pi]|S]
        \end{aligned}
        \nonumber
    \end{equation}
\end{proposition}

Consider a belief space $\left(N,\Omega,\{\mu_i\}_{i\in N},\{\Sigma_i\}_{i\in N}\right)$ and a \textbf{common prior} $\mu=\mu_i,\forall i\in N$.

\begin{theorem}[Aumann's Agreement Theorem, 1976]
    Let $X:\Omega \rightarrow \mathbb{R}$ be a random variable. Suppose that, \underline{at $\omega_0$}, it is common knowledge of posteriors that $\mathbb{E}[X|\Sigma_i]=q_i$ for $i=1,...,n$ and some $(q_1,q_2,...,q_n)\in \mathbb{R}^n$. Then $q_1=q_2=\cdots =q_n$.
\end{theorem}
\begin{proof}
    For $i\in N$, let $A_i$ be the event that $\mathbb{E}[X|\Sigma_i]=q_i$. By the common knowledge hypothesis there is a $C\in \Sigma_C=\cap_i\Sigma_i$ such that $\omega_0\in C\subset \cap_i A_i$. Hence, $\mathbb{E}[X|\Sigma_i](\omega)=q_i$ for all $\omega\in C$. Thus, for all $i$, $$\mathbb{E}[X|C]=\mathbb{E}[\mathbb{E}[X|\Sigma_i]|C]=\mathbb{E}[q_i|C]=q_i,\forall i\in N$$
\end{proof}

\begin{corollary}
    If two players have common priors over a finite space, and it is common knowledge that their posteriors for some event $S$ are $q_1$ and $q_2$, then $q_1 = q_2$.
\end{corollary}


\section{Extensive-Form Games}
In this section, we want to study extensive-form games with incomplete information.

We have discussed the extensive form game with perfect information (PIG), which is denoted by $G=\{N,A,H,Z,P,O,o,\succ_{n\in N}\}$. $P$ is a function mapping from $H/Z$ (non-terminal history) to $N$ (one of the players). Now we shall allow two or more players to make simultaneous moves, and the $P$ is a function from $H/Z$ to $2^N/\{\emptyset\}$, the power set of the set of players, so that after history $h \in H$ the set of players who play simultaneously is $P(h)$.

Without losing generality, we can represent an extensive-form game as $\Gamma=\left(N,A,H,P,\{u_i\}_{i\in N}\right)$.

For each player $i \in N$, let $$H_i=\{h\in H/Z:i\in P(h)\}$$ be the set of histories at which $i$ moves.

A \textbf{strategy} for player $i$ is a function $s_i : H_i \rightarrow A$ for which $$s_i(h)\in A_i(h),\ \forall h\in H_i$$
Let $S_i$ be the set of all strategies available for player $i$. A strategy profile is a collection of all players' strategies.

Thus, we obtain a normal-form game $(N,(S_i, u_i)_{i\in N})$ defined from $\Gamma$.

\subsection{Mixed and Behavioral Strategies}
\begin{figure}[htbp]
    \centering
    \includegraphics[scale=0.2]{BS.png}
    \caption{Example}
    \label{}
\end{figure}
In this example, a mixed strategy of the P1 can be represented by a probability distribution over $\{OU,OD,IU,ID\}$. To replicate mixed strategy, we can use \textbf{behavioral strategy} that an agent's (probabilistic) choice at each node is independent of his/her choices at other nodes.
\begin{example}
    \begin{enumerate}
        \item Mixed Strategy: $OU$ with probability $\frac{1}{2}$ and $ID$ with probability $\frac{1}{2}$.
        \item Behavioral Strategy: play $O$ and $I$ with prob $\frac{1}{2}$ each at $h = \emptyset$;
        and then $D$ with prob $1$ at history $h = I$.
    \end{enumerate}
\end{example}

\begin{definition}[Behavioral Strategy]
    %\normalfont
    A \textbf{behavioral strategy} for player $i$ is a function that maps histories $h \in H_i$ into probability distributions over $A_i(h)$,
    $$\sigma_i(h)\in\Delta(A_i(h))$$
\end{definition}

A profile of behavioral strategies, $\sigma=\{\sigma_i:i\in N\}$, includes a probability distribution over $Z$. So does a profile of mixed strategies. Then $u_i(\sigma)$ is the expected payoff.

\begin{theorem}[Outcome Equivalent]
    Behavioral and mixed strategies are ``outcome equivalent'' in games of \textbf{perfect recall}.
\end{theorem}
Without perfect recall, behavioral and mixed strategies are not equivalent.


\subsection{Subgame Perfect Equilibrium}
\begin{definition}
    %\normalfont
    A strategy profile $\sigma^*$ is a \textbf{Nash equilibrium} if, for all $i$,
    \begin{enumerate}
        \item $\sigma_i^*$ is the best response to beliefs $\mu_i$,
        \item $\mu_i$ coincides with $\sigma_{-i}^*$.
    \end{enumerate}
\end{definition}
When $s$ is a pure-strategy Nash eq. $h(s)$ is the equilibrium path of $s$.

\begin{definition}
%\normalfont
A subgame of an extensive-form game $G=\{N,A,H,Z,P,O,o,\succ_{n\in N}\}$ is a game that starts after a given finite history $h \in H$.

Formally, the subgame $\Gamma(h)$ associated with $h=(h_1,...,h_n)\in H$ is $\Gamma(h)=\left(N,A,H_h,P_h,\{u_{i,h}\}_{i\in N}\right)$, where $H_h=\{(a_1,a_2,...):(h_1,...,h_n,a_1,a_2,...)\in H\}$. $P_h(h')=P(hh')$ for all non-terminal $h'\in H_h$, and $u_{i,h}=u_i(hh')$ for any terminal $h'\in H$.
\end{definition}
A behavioral strategy $\sigma_i$ of $i$ in $\Gamma$ defines a behavioral strategy $\sigma_i(\cdot|h)$ of $\Gamma(h)$ by $\sigma_i(h'|h)=\sigma_i(hh')$.



\begin{definition}[Subgame Perfect Equilibrium]
    %\normalfont
    A \textbf{subgame perfect equilibrium}[SPNE/SPE] of $\Gamma$ is a strategy profile $\sigma^*$ such that for every subgame $\Gamma(h)$ it holds that $\sigma^*$ (more precisely, its restriction to $H_h$) is a Nash equilibrium of $\Gamma(h)$.
\end{definition}
Over time $t=1,...,T$, the set of actions is $A^T$.
\begin{lemma}
    $\sigma^*$ is a SPNE \underline{iff} its restriction $\sigma^*(\cdot|h)$ is SPNE of $\Gamma(h)$, for all subgames $\Gamma(h)$.
\end{lemma}

\subsection{Infinite Horizon Games}
\begin{definition}[Finite Horizon]
    %\normalfont
    An extensive-form game has a \textbf{finite horizon} if there is a bound on the length of any history in $Z$.
\end{definition}
In these games, SPNE can be found through ``generalized backwards induction.''

\begin{definition}[Continuous At Infinity Utility Function]
    %\normalfont
    A utility function $v:A^T \rightarrow \mathbb{R}$ is \textbf{continuous at infinity} if
    \begin{equation}
        \begin{aligned}
            \lim_{t \rightarrow \infty}\sup\{|v(\hat{h})-v(h)|:h,\hat{h}\in A^T\}=0
        \end{aligned}
        \nonumber
    \end{equation}
    That is, eventually, a negligible effect on utility.
\end{definition}

\begin{definition}[Discounted Utility]
%\normalfont
We focus on additively separable models with a sub-utility $u:A \rightarrow \mathbb{R}$ that is the same for each $t$. We model a \textbf{discounted utility} by assuming a discount factor $\delta\in[0,1)$:
\begin{equation}
    \begin{aligned}
        v(a_1,...)=(1-\delta)\sum_{t=1}^\infty \delta^{t-1}u(a_t)
    \end{aligned}
    \nonumber
\end{equation}
\end{definition}
Can think of $\delta$ as the probability that the game will go on for another period, and $1-\delta$ the prob that the game will end.

“Ending” events are independent: A geometric distribution.

The probability that the game ends at time $t$ is $(1 - \delta)\delta^{t-1}$ and $\sum(1-\delta)\delta^t u(a_t)$ becomes expected utility.

\begin{proposition}
    $v(a_1,...)=(1-\delta)\sum_{t=1}^\infty \delta^{t-1}u(a_t)$ is continuous at infinity.
\end{proposition}


\subsection{Repeated Games}
Let $G_0=\left(N,\{A_i:i\in N\},\{u_i:i\in N\}\right)$ be a normal-form game. We let $A=\Pi_{i\in N}A_i$. We only consider games in which $A$ is compact and each $u_i$ is continuous.

Let $T$ be the number of periods (or repetitions) of the repeated game. $T$ can be either finite or infinite. $G_0$ is the stage game.

\begin{definition}[Repeated Game]
    %\normalfont
    A $T$-repeated, $\delta$-discounted, game of $G_0$ is an extensive form game $\Gamma=\left(N,A,H,P,\{u_i\}_{i\in N},\delta\right)$, where
    \begin{enumerate}
        \item $P(h)=N$ for all $h\in H/Z$.
        \item The set of histories $H$ has terminal histories $Z=A^T$; recall that $Z$
        uniquely determines $H$ as the set of all prefixes of $Z$.
        \item $v_i:Z \rightarrow \mathbb{R}$ is a payoff function of the form
        \begin{equation}
            \begin{aligned}
                v((a_t)_{t=1}^T)=\left\{\begin{matrix}
                    (1-\delta)\sum_{t=1}^\infty \delta^{t-1}u(a_t),&\delta\in(0,1)\\
                    \sum_{t=1}^T u_i(a_t),&\delta=1
                \end{matrix}\right.
            \end{aligned}
            \nonumber
        \end{equation}
    \end{enumerate}
\end{definition}
Given a strategy profile $s$, we obtain a terminal history $z \in Z$, and (abusing notation) write $v_i(s)$ for $v_i(z)$.

We call $(v_1(s),...,v_n(s))$ the \textbf{payoff profile} associated with $s$.

%The set $\mathcal{F}=\{(u_1(v),...,u_n(v)):v\in\Delta(A)\}$ is the set of \textbf{feasible payoffs}, which is a convex hull of $\{(u_1(a),...,u_n(a)):a\in A\}$.

\begin{definition}[(Profitable) One-Stage Deviation]
    %\normalfont
    A \textbf{one-stage deviation} from $\sigma$ by $i$ is a strategy $\sigma'_i$ that coincides with $\sigma_i$ everywhere except at a single history.\\
    A \textbf{profitable one-stage deviation} from $\sigma$ by $i$ is a strategy $\sigma'_i$ that coincides with $\sigma_i$ at a single history $h$, and for which $v_i(\sigma'_i,\sigma_{-i}|h)>v_i(\sigma_i,\sigma_{-i}|h)$.
\end{definition}

\begin{proposition}[SPNE$\Leftrightarrow$No Profitable One-Stage Deviation]
    Let $\Gamma$ be a discounted, infinitely repeated game. Then a strategy-profile is a SPNE of $\Gamma$ \underline{iff} no player has a profitable one-stage deviation.
\end{proposition}

\begin{corollary}[NE$\Rightarrow$SPNE]
    If $s$ is a pure-strategy Nash equilibrium of $G_0$, then the strategy profile in which each player chooses $s_i$ after any history, is a SPNE of infinitely-repeated discounted $G_0$.
\end{corollary}

\begin{proposition}[Nash Reversion Folk Theorem]
    Let $\sigma$ be a Nash equilibrium of the finite stage game $G_0$. Let $v$ be a feasible payoff profile s.t. $u_i(\sigma) < v_i$ for all players $i$. Then there is $\underline{\delta}$ s.t. if $\delta\geq \underline{\delta}$ then there exists a SPNE of infinitely repeated $G_0$ with discount factor $\delta$ in which players' equilibrium payoffs is the profile $v$.
\end{proposition}
No Nash equilibrium that has higher payoffs for all players than Nash equilibrium can be obtained with relatively high discount factor in an infinite game.


\section{Refinement}
\subsection{Trembling-hand Perfect Equilibrium}
\begin{definition}[$\epsilon$-Constrained Equilibrium]
    An \textbf{$\epsilon$-constrained equilibrium} of a strategic form game is a mixed strategy profile $\sigma^\epsilon$ such that there exists $\bar{\epsilon}: \bigcup_{i \in I} S_i \rightarrow(0, \epsilon)$ such that for each player $i$,
    \begin{equation}
        \begin{aligned}
            \sigma^\epsilon_i\in\argmax_{\sigma_i \textnormal{ s.t. }\sigma_i(s_i)\geq \bar{\epsilon}(s_i)} u_i(\sigma_i,\sigma^\epsilon_{-i})
        \end{aligned}
        \nonumber
    \end{equation}
\end{definition}


\begin{definition}[Trembling-hand Perfect Equilibrium]
    A strategy profile $\sigma$ is a \textbf{trembling-hand perfect equilibrium} of a strategic form game if it is a limit of $\epsilon$-constrained equilibria as $\epsilon \rightarrow 0$.
\end{definition}

\begin{note}
    Trembling-hand perfect equilibrium implies a SPE, while a SPE is not always a trembling-hand perfect equilibrium.
\end{note}

This doesn't imply backward induction. (We do not rationalize the ``trembling-hand''.)


\subsection{Forward Induction: Burning Money (Ben-Polath and Dekel, 1992)}
Consider a game as following:
\begin{center}
    \begin{tabular}{ccc}
        \hline
            &$A_2$ &$B_2$\\
        \hline
            $A_1$& 9,6 & 0,4\\
            $B_1$& 4,0 & 6,9\\
        \hline
    \end{tabular}
\end{center}
Except the actions, the Player 1 can choose to Burn or Not Burn (i.e., lose $2.5$ units of utility), and the Player 2 can see the Player 1's action.

There are four possible strategies of Player 1:
\begin{enumerate}[(S1).]
    \item Burn and $A_1$
    \item Burn and $B_1$
    \item Not Burn and $A_1$
    \item Not Burn and $B_1$
\end{enumerate}
The potential payoffs of playing (S2) are $1.5$ and $3.5$, which is dominated by playing (S4). Then, if the Player 1 chooses Burn, he must play (S1). Thus, the Player 2 plays $A_2$, which gives $(6.5,6)$. Therefore, the Player 1 chooses Burn (i.e., (S1)) can dominate (S4). Now, we only remain two possible strategies of Player 1, (S1) and (S3). (S1) is dominated by (S3), so (S3) is the optimal strategy of Player 1.

\section{Signaling Game}
\subsection{Canonical Game}
\begin{definition}[Canonical Game]
    %\normalfont
    \begin{enumerate}
        \item There are two players: $\mathbf{S}$ (sender) and $\mathbf{R}$ (receiver).
        \item $\mathbf{S}$ holds more information than $\mathbf{R}$: the value of some random variable $t$ with support $\mathcal{T}$. (We say that $t$ is the \textbf{type} of $\mathbf{S}$)
        \item Prior belief of $\mathbf{R}$ concerning $t$ are given by a probability distribution $\rho$ over $\mathcal{T}$ (common knowledge)
        \item $\mathbf{S}$ sends a \textbf{signal $s\in \mathcal{S}$} to $\mathbf{R}$ drawn from a signal set $\mathcal{S}$.
        \item $\mathbf{R}$ receives this signal, and then takes an \textbf{action} $a\in \mathcal{A}$ drawn from a set $\mathcal{A}$ (which could depend on the signal $s$ that is sent).
        \item $\mathbf{S}$'s payoff is given by a function $u: \mathcal{T}\times \mathcal{S} \times \mathcal{A} \rightarrow \mathbb{R}$ and $\mathbf{R}$'s payoff is given by a function $v: \mathcal{T}\times \mathcal{S} \times \mathcal{A} \rightarrow \mathbb{R}$.
    \end{enumerate}
\end{definition}

\subsection{Nash Equilibrium}
\begin{definition}[Strategy]
    %\normalfont
    A \textbf{behavior strategy} for $\mathbf{S}$ is given by a function $\sigma: \mathcal{T}\times\mathcal{S} \rightarrow [0,1]$ such that $\sum_s \sigma(t,s)$ for each $t$.\\
    A \textbf{behavior strategy} for $\mathbf{R}$ is given by a function $\alpha: \mathcal{S}\times\mathcal{A} \rightarrow [0,1]$ such that $\sum_a \alpha(s,a)$ for each $t$.
\end{definition}

\begin{definition}[Nash Equilibrium]
    %\normalfont
    Behavior strategies $\alpha$ and $\sigma$ form a \textbf{Nash equilibrium} if and only if
    \begin{enumerate}
        \item For all $t\in \mathcal{T}$,
        \begin{center}
            $\sigma(t,s)>0$ implies $\sum_a \alpha(s,a)u(t,s,a) = \max_{s'\in \mathcal{S}}\left(\sum_a \alpha(s',a)u(t,s',a)\right)$
        \end{center}
        \item For each $s\in \mathcal{S}$ such that $\sum_{t}\sigma(t,s)\rho(t)>0$,
        \begin{center}
            $\alpha(s,a)>0$ implies $\sum_{t}\mu(t;s)v(t,s,a) = \max_{a'}\sum_{t}\mu(t;s)v(t,s,a')$
        \end{center}
        where $\mu(t;s)$ is the $\mathbb{R}$'s posterior belief about $t$ given $s$, $\mu(t;s)=\frac{\sigma(t,s)\rho(t)}{\sum_{t'}\sigma(t',s)\rho(t')}$ if $\sum_t\sigma(t,s)\rho(t)>0$ and $\mu(t;s)=0$ otherwise.
    \end{enumerate}
\end{definition}

\begin{definition}[Separating \& Pooling Equilibrium]
    %\normalfont
    An equilibrium $(\sigma,\alpha)$ is called a \textbf{separating} equilibrium if each type $t$ sends different signals; i.e., the set $\mathcal{S}$ can be partitioned into (disjoint) sets $\{\mathcal{S}_t; t\in \mathcal{S}\}$ such that $\sigma(t, \mathcal{S}_t) = 1$. An equilibrium $(\sigma,\alpha)$ is called a \textbf{pooling} equilibrium if there is a single signal $s^*$ that is sent by all types; i.e., $\sigma(t, s^*) = 1$ for all $t\in \mathcal{T}$.
\end{definition}


\subsection{Single-crossing}

\subsubsection{Situation over real line}
Consider the situation that $\mathcal{T},\mathcal{S},\mathcal{A}\subseteq \mathbb{R}$ and $\geq$ is the usual "greater than or equal to" relationship.

\begin{enumerate}
    \item We let $\Delta \mathcal{A}$ denote the set of probability distributions on $\mathcal{A}$.
    \item For each $s\in \mathcal{S}$ and $\mathcal{T}'\subseteq \mathcal{T}$, we let $\Delta\mathcal{A}(s,T')$ be the set of mixed strategies that are the best responses by $\mathbf{R}$ to $s\in \mathcal{S}$ for some probability distribution with support $\mathcal{T}'$.
    \item For $\alpha\in \Delta\mathcal{A}$, we write $u(t,s,\alpha)\triangleq \sum_{a\in \mathcal{A}}u(t,s,a)\alpha(a)$.
\end{enumerate}

\begin{definition}[Single-crossing]
    %\normalfont
    The data of the game are said to satisfy the \textbf{single-crossing property} if the following holds: If $t\in \mathcal{T}$, $(s,\alpha)\in \mathcal{S}\times \Delta\mathcal{A}$ and $(s',\alpha')\in \mathcal{S}\times \Delta\mathcal{A}$ are such that $\alpha\in \Delta\mathcal{A}(s,\mathcal{T})$, $\alpha'\in \Delta\mathcal{A}(s',\mathcal{T})$, $s>s'$ and $u(t,s,\alpha)\geq u(t,s',\alpha')$, then for all $t'\in T$ such that $t'>t$, $u(t',s,\alpha)\geq u(t',s',\alpha')$.
\end{definition}

\section{Adverse Selection}
Consider a labor market that has many identical firms. In competitive equilibrium, firms' profits are $0$. Firms are price (wage) takers, risk-neutral, and CRS. There are continuum of workers with productivity levels $\theta\in\left[\underline{\theta},\overline{\theta}\right]$ (Assume workers work if it is indifferent for them between employment and non-employment).
\begin{enumerate}
    \item $\theta\sim F$, $F(\cdot)$ is a c.d.f. over $\left[\underline{\theta},\overline{\theta}\right]$.
    \item $N$ is the total mass of workers.
    \item Type $\theta$ worker has a reservation utility $r(\theta)$.
\end{enumerate}

\begin{enumerate}[$\circ$]
    \item Suppose the competitive equilibrium wages are $\theta=w^*(\theta)$.
    \item An allocation is denoted by $I:\left[\underline{\theta},\overline{\theta}\right] \rightarrow \{0,1\}$, where $I(\theta)=0$ denotes $\theta$ is unemployed and $I(\theta)=1$ denotes $\theta$ is employed.
    \item Aggregate welfare = sum of utilities of all participants
    \begin{equation}
        \begin{aligned}
            =N\int_{\underline{\theta}}^{\overline{\theta}} \left[I(\theta)\times\theta+[1-I(\theta)]r(\theta)\right]dF(\theta)
        \end{aligned}
        \nonumber
    \end{equation}
    Then we have the optimal allocation satisfies
    \begin{equation}
        \begin{aligned}
            I^*(\theta)\left\{\begin{matrix}
                =1,&\theta>r(\theta)\\
                \in\{0,1\}&\theta=r(\theta)\\
                =0,&\theta<r(\theta)
            \end{matrix}\right.
        \end{aligned}
        \nonumber
    \end{equation}
\end{enumerate}
In the asymmetric information case,
\begin{definition}
%\normalfont
$w$ is CE wage if $w=\mathbb{E}[\theta|r(\theta)\leq w]$.
\end{definition}

\subsection{Adverse Selection}
\begin{assumption}
    \begin{enumerate}[({A}1).]
        \item $r$ is strictly increasing in $\theta$.
        \item $F(\cdot)$ has a strictly positive density, $F(\theta)>0, \forall \theta\in \left[\underline{\theta},\overline{\theta}\right]$.
        \item $r(\theta)\leq\theta$ (outside option is worse than productivity, i.e., full employment is optimal).
    \end{enumerate}
\end{assumption}

\begin{lemma}
    Under A1-A3, $\Phi(w):=\mathbb{E}[\theta|r(\theta)\leq w]$ is well-defined, continuous, and non-decreasing.
\end{lemma}

Hence, there exists underemployment, which makes $1^{st}$ welfare theorem fails. There may exist multiple CEs, where the one with the highest wage Pareto dominates others.

\begin{example}
    Suppose $\theta\in[0,2]$, $F(\theta)=\frac{\theta}{2}$, $f(\theta)=\frac{1}{2}$, $r(\theta)=\alpha\theta,\alpha\in(0,1)$.
    \begin{equation}
        \begin{aligned}
            \mathbb{E}[\theta|r(\theta)\leq w]=\mathbb{E}\left[\theta|\theta \leq \frac{w}{\alpha}\right]=\left\{\begin{matrix}
                1,&w\geq 2\alpha\\
                \frac{1}{F\left(\frac{w}{\alpha}\right)}\int_0^{\frac{w}{\alpha}}\theta f(\theta)d\theta=\frac{w}{2\alpha},&w\leq 2\alpha
            \end{matrix}\right.
        \end{aligned}
        \nonumber
    \end{equation}
    CEs are given by $\mathbb{E}[\theta|r(\theta)\leq w]=w$. $w^*=0$ is always CE and $w^*=1$ is CE if $\alpha\leq\frac{1}{2}$.
\end{example}

\subsection{Game Theoretical Approach}
\begin{enumerate}
    \item Suppose there are two firms setting wages simultaneously.
    \item Workers observe the wages in stage 1 and make an employment decision.
\end{enumerate}
Let $W^*$ be the set of CE wages and $w^*:=\max W^*$.
\begin{lemma}\label{lemma:ad_l2}
    $\forall w'\in\left(w^*,\overline{\theta}\right]$: $\mathbb{E}[\theta|r(\theta)\leq w']<w'$.
\end{lemma}
\begin{proof}
    Suppose by the contradiction that $\exists w'\in \left(w^*,\overline{\theta}\right]$ s.t. $\mathbb{E}[\theta|r(\theta)\leq w']\geq w'$. Since $\mathbb{E}[\theta|r(\theta)\leq \overline{\theta}]<\overline{\theta}$, there must exist a $w''\in [w',\overline{\theta})$ s.t. $\mathbb{E}[\theta|r(\theta)\leq w'']=w''$ by intermediate value theorem, which contradicts to the definition of $w^*$.
\end{proof}

\begin{proposition}
    \begin{enumerate}[(i).]
        \item If $w^*>r(\underline{\theta})$ and $\exists \epsilon>0$ s.t. $\mathbb{E}[\theta|r(\theta)\leq w']>w',\forall w'\in \left(w^*-\epsilon,w^*\right)$. Then, there is a unique SPE where both firms set wage $=w^*$.
        \item If $w^*=r(\underline{\theta})$ (complete market shutdown at $w^*$), there are multiple SPE that all give the same outcome as complete market shutdown where both firms set wage $=w^*$.
    \end{enumerate}
\end{proposition}
\begin{proof}
    \begin{lemma}\label{lemma:p1}
        In all SPE, firms make zero profits.
    \end{lemma}
    \begin{proof}
        Suppose not, i.e., at least one firm makes strictly positive profits. Then, the total profits of firms $1\&2$, $$\Pi=M(\bar{w})\left[\mathbb{E}[\theta|r(\theta)\leq\bar{w}]-\bar{w}\right]>0$$
        where $\bar{w}$ is the max wage set by the two firms and $M(\bar{w})$ is the mass of workers willing to work at $\bar{w}$. At least one firm, $i$, makes profit $\leq\frac{\Pi}{2}$. Then, $i$'s profits from setting $\bar{w}+\delta$, with $\delta \rightarrow 0^+$, is higher:
        \begin{equation}
            \begin{aligned}
                &M(\bar{w}+\delta)\left[\mathbb{E}[\theta|r(\theta)\leq\bar{w}+\delta]-\bar{w}-+\delta\right]\\
                \geq &M(\bar{w})\left[\mathbb{E}[\theta|r(\theta)\leq\bar{w}+\delta]-\bar{w}-+\delta\right] \rightarrow \Pi \textnormal{ as }\delta \rightarrow 0
            \end{aligned}
            \nonumber
        \end{equation}
        Hence, the $i$ has incentive to deviate.
    \end{proof}
    \begin{lemma}\label{lemma:p2}
        In all SPE, firm $i$ sets $w_i\leq w^*, i\in\{1,2\}$.
    \end{lemma}
    \begin{proof}
        Directly given by Lemma \ref{lemma:ad_l2} and Lemma \ref{lemma:p1}.
    \end{proof}
    \begin{enumerate}[(i):]
        \item In SPE, no firm $i$ sets $w_i<w^*$: suppose $w_i<w^*$ and let $j\neq i$, take any $w'_j$ s.t. $w'_j\in\left(w_i,w^*\right)$ and $w'_j>w^*-\epsilon$. Then, $j$ gets profit: $M(w'_j)\left[\mathbb{E}[\theta|r(\theta)\leq w'_j]-w'_j\right]>0$ (by Case (i)'s conditions).
        \item By Lemma \ref{lemma:p2}, both firms set $w_i\leq w^*=r(\underline{\theta})$. Check that $\{(w_1,w_2):w_1,w_2\leq w^*\}$ is SPE wage profiles.
    \end{enumerate}
\end{proof}

\subsection{Planner Intervention}
Planner can't observe the true type $\theta$.

The planner's tools:
\begin{enumerate}
    \item Take over the firms.
    \item $w_e$, employment wage.
    \item $w_u$, unemployment wage.
\end{enumerate}
s.t. budget balanced.

\begin{definition}[Constrained Efficient]
    %\normalfont
    A CE $w$ is \textbf{constrained efficient} if it cannot be Pareto improved upon by an intervention by the planner.
\end{definition}

\begin{proposition}[$w^*:=\max W^*$ is constrained efficient]
    Let $W^*$ be the set of CE wages. $w^*:=\max W^*$ is constrained efficient.
\end{proposition}
\begin{proof}
    Note that both firms are making zero profits by the Lemma \ref{lemma:p1}. Any CE wage $w\neq w^*$ can be Pareto improved by $\{w_e=w^*,w_u=0\}$. Then, we prove $w^*$ can't be Pareto improved.
    \begin{enumerate}
        \item Case 1: if $w^*$ gives full-employment in CE, then $w^*$ is Pareto efficient.
        \item Case 1: suppose $w^*$ doesn't give full-employment in CE.
        
        Consider taking an intervention $w_e\&w_u$. Then, $\{\theta:r(\theta)+w_u\leq w_e\}=[\underline{\theta},\hat{\theta}]$ for some $\hat{\theta}\in[\underline{\theta},\overline{\theta}]$ such that
        \begin{equation}
            \begin{aligned}
                r(\hat{\theta})+w_u=w_e
            \end{aligned}
            \label{con:1}
        \end{equation}
        The budget balanced gives
        \begin{equation}
            \begin{aligned}
                w_e F(\hat{\theta})+w_u (1-F(\hat{\theta}))=\int_{\underline{\theta}}^{\hat{\theta}}\theta d F(\theta)
            \end{aligned}
            \label{con:2}
        \end{equation}
        Plug \eqref{con:1} into \eqref{con:2}:
        \begin{equation}
            \begin{aligned}
                \left\{\begin{matrix}
                    &w_u(\hat{\theta})=\int_{\underline{\theta}}^{\hat{\theta}}\theta d F(\theta)-r(\hat{\theta})F(\hat{\theta})=F(\hat{\theta})\left(\mathbb{E}[\theta|\theta\leq\hat{\theta}]-r(\hat{\theta})\right)\\
                    &w_e(\hat{\theta})=\int_{\underline{\theta}}^{\hat{\theta}}\theta d F(\theta)+r(\hat{\theta})(1-F(\hat{\theta}))
                \end{matrix}\right.
            \end{aligned}
            \nonumber
        \end{equation}
        Let $\theta^*$ be s.t. $r(\theta^*)=w^*$. Because $w^*$ is a CE price, $\mathbb{E}[\theta|\theta\leq\theta^*]=r(\theta^*)=w^*$. So, CE with $w^*$ can be implemented by $w_u(\theta^*)=0$ and $w_e(\theta^*)=w^*$.
        \begin{enumerate}
            \item If $\hat{\theta}<\theta^*$. $\underline{\theta}$ is worse off under the intervention since $w_e(\hat{\theta})<w^*$.
            \item If $\hat{\theta}>\theta^*$. $\overline{\theta}$ is worse off under the intervention since $w_u(\hat{\theta})=F(\hat{\theta})\left(\mathbb{E}[\theta|\theta\leq\hat{\theta}]-r(\hat{\theta})\right)<0$ by the Lemma \ref{lemma:ad_l2}
        \end{enumerate}
    \end{enumerate}
\end{proof}


\subsection{Signaling}\label{sec:signaling}
Suppose the worker $\theta\in[\underline{\theta},\overline{\theta}]$ can properly and costlessly reveal his type to the firms. Then,
\begin{lemma}
    All workers revel their types.
\end{lemma}
\paragraph*{Spence's Job Market Signaling Model} One worker has productivity $\theta\in\{\theta_L,\theta_H\}$ with $P(\theta_H)=\lambda$. The worker signal through his education with cost $e>0$. The education doesn't change his productivity. The payoff of the worker is the wage minus the cost:
\begin{equation}
    \begin{aligned}
        u(w,e|\theta)=w-c(e,\theta)
    \end{aligned}
    \nonumber
\end{equation}
where $c(0,\theta)=0,c_e(e,\theta):=\frac{\partial c(e,\theta)}{\partial e}>0, c_\theta(e,\theta):=\frac{\partial c(e,\theta)}{\partial \theta}<0$, and $c_{e\theta}(e,\theta):=\frac{\partial^2 c(e,\theta)}{\partial e\partial \theta}<0$ (Single-Crossing Property, the difference $c(e,\theta_L)-c(e,\theta_H)$ is increasing in $e$ (i.e., $c_e(e,\theta_L)-c_e(e,\theta_H)>0$), which means if $c(e,\theta_L)$ and $c(e,\theta_H)$ intersect as functions of $e$, they only intersect at one time.)
\begin{enumerate}[]
    \item \underline{Stage 0}: Nature chooses the $\theta\in\{\theta_L,\theta_H\}$ with $P(\theta_H)=\lambda$.
    \item \underline{Stage 1}: The worker learns $\theta$ and chooses $e(\theta)\geq 0$.
    \item \underline{Stage 2}: Firms observe $e(\theta)$. Then, they simultaneously make wage offers $w_1$ and $w_2$.
    \item \underline{Stage 3}: The worker observes $w_1,w_2$ and makes employment decision.
\end{enumerate}
Let $r(\theta_L)$ and $r(\theta_H)=0$. Let $\mu(e)\in[0,1]$ be the probability that in the beginning of stage 2, firms think that the worker is $\theta_H$ type with probability $\mu(e)$ when observing $e$. The corresponding expected productivity (the highest wage) that the firm can pay is
\begin{equation}
    \begin{aligned}
        w(e)=\mu(e)\theta_H+(1-\mu(e))\theta_L
    \end{aligned}
    \nonumber
\end{equation}
In stage 2, both firm will set $w(e)$ (complete competition).

\begin{definition}[Perfect Bayesian Equilibrium]
    %\normalfont
    A PBE is a strategy profile ($e^*(\theta_L)$, $e^*(\theta_H)$, $w^*_1: \mathbb{R}_+ \rightarrow \mathbb{R}$, $w^*_2: \mathbb{R}_+ \rightarrow \mathbb{R}$), and a belief $\mu^*: \mathbb{R} \rightarrow [0,1]$ such that
    \begin{enumerate}
        \item $\forall \theta\in\{\theta_L,\theta_H\}$, the worker strategy optimal given firm strategies.
        \item The belief $\mu^*(e)$ is derived from $\lambda, e^*(\theta_L), e^*(\theta_H)$ via Bayes' rule whenever possibly (on the equilibrium path). Outside the equilibrium path the belief $\mu^*(e)$ is arbitrarily.
        \item Firms offer wages that form a NE of the stage 2 game, where their belief $\mu^*(e)$ about their workers' type. (sequential rationality).
    \end{enumerate}
\end{definition}
We simplify the game by backward induction:
\begin{enumerate}
    \item \underline{Stage 3}: The worker chooses the highest wage off if it is $\geq 0$.
    \item \underline{Stage 2}: After observing $e(\theta)$, firms chooses the wage as the expected productivity in NE,
    \begin{equation}
        \begin{aligned}
            w^*(e)=\mu^*(e)\theta_H+(1-\mu^*(e))\theta_L
        \end{aligned}
        \nonumber
    \end{equation}
    because it is a Bertrend competition.
\end{enumerate}
\paragraph*{Separating Equilibrium}
In separating equilibrium, $e^*(\theta_L)\neq e^*(\theta_H)$.
\begin{lemma}
    In any separating PBE, $w^*(e^*(\theta))=\theta, \forall \theta\in\{\theta_L,\theta_H\}$.
\end{lemma}
\begin{proof}
    By Bayes' rule, after firm observe $e^*(\theta_L)$, $\mu^*(e^*(\theta_L))=0$. Then, $w^*(e^*(\theta_L))=\theta_L$. ($e^*(\theta_H)$ is similar.)
\end{proof}

\begin{lemma}
    In separating PBE, low type always chooses zero education, $\theta^*(\theta_L)=0$.
\end{lemma}
\begin{proof}
    If not, the low type worker always has profitable deviation, $\theta^*(\theta_L)=0$.
\end{proof}

\begin{lemma}
    Define $\underline{e}$ and $\overline{e}$ such that
    \begin{enumerate}
        \item $\theta_L=\theta_H-c(\underline{e},\theta_L)$ (the lowest effort can prevent the low type from mimicking high type) and
        \item $\theta_L=\theta_H-c(\overline{e},\theta_H)$ (the highest effort can prevent the high type from pooling with low type).
    \end{enumerate}
    Then, in all separating PBEs, $e\in \left[\underline{e},\overline{e}\right]$.\\
    Conversely, $\forall \hat{e}\in \left[\underline{e},\overline{e}\right]$, there is a separating PBE where $e^*(\theta_H)=\hat{e}$.
\end{lemma}
These different PBEs are Pareto ranked. High type prefers the PBE with a lower $e$ (the best is the one with $e^*(\theta_H)=\underline{e}$.)

\paragraph*{Pooling PBE}
$e^*(\theta)=e^*,\theta\in\{\theta_L,\theta_H\}$, $\mu^*(e^*)=\lambda$, and $w^*(e^*)=\mathbb{E}[\theta]$.

\begin{lemma}
    Define $e'$ by $\theta_L=\mathbb{E}[\theta]-c(e',\theta_L)$ (the highest effort can prevent the low type from choosing $e=0$ and get $w=\theta_L$.)\\
    Then, for all pooling PBE, $e^*(\theta_L)=e^*(\theta_H)=e^*\in[0,e']$. Conversely, for all $\hat{e}\in [0,e']$, there is a pooling PBE with $e^*=\hat{e}$.
\end{lemma}


\subsection{Cho-Kreps Intuitive Criterion}
\begin{definition}[Equilibrium Dominated Message]
    %\normalfont
    A message is \textbf{equilibrium dominated} for a type if the type must do strictly worse by sending the message than it does in equilibrium (i.e., payoff in eq. is strictly better than maximum payoff from deviating).
\end{definition}

\begin{definition}[Cho-Kreps Intuitive Criterion]
    %\normalfont
    If an information set is off the eq. path and a message is eq. dominated for a type, then beliefs should assign zero probability to the message coming from that type (if possible).
\end{definition}

Fix a PBE $e^*(\theta), \theta\in\{\theta_L,\theta_H\}, \mu^*(\cdot)$ (We know $w_1^*(e)=w_2^*(e)=\mu(e)\theta_H+(1-\mu(e))\theta_L$). Let $u^*(\theta),\theta\in\{\theta_L,\theta_H\}$ be the PBE utility of the type $\theta$ worker.

The criterion requires the (off-path) belief $\mu^*(e):=P(\tilde{\theta}=\theta_H|e)=1-P(\tilde{\theta}=\theta_L|e)$ satisfies $$P(\tilde{\theta}=\theta|e)=0,\forall e,\theta$$ such that
\begin{enumerate}
    \item $u^*(\theta)>\max_{w\in[\underline{\theta},\overline{\theta}]}[w-c(e,\theta)]$
    \item $\exists \theta'$ s.t. $u^*(\theta')\leq \max_{w\in[\underline{\theta},\overline{\theta}]}[w-c(e,\theta')]$ (make sure the sum of beliefs given $e$ is nonzero.)
\end{enumerate}
In this application, the only PBE that survives Intuitive Criterion is the best separating PBE, $e^*(\theta_H)=\underline{e}$ (the lowest effort).



\subsection{Grossman-Perry-Farrell Equilibrium}
For the equilibrium refinement, we can also introduce the Grossman-Perry-Farrell equilibrium based on the perfect sequential equilibrium (grossman1986sequential) and the neologism-proof equilibrium (farrell1993meaning). We formally define the Grossman-Perry-Farrell equilibrium by ruling out the self-signaling sets in Perfect Bayesian Equilibrium (bertomeu2018verifiable,glode2018voluntary).

\begin{definition}[Grossman-Perry-Farrell Equilibrium (GPFE)]
	A pure-strategy perfect Bayesian equilibrium $(p^*_L,p^*_H,b^*(\cdot))$ is a ``Grossman-Perry-Farrell equilibrium'' (GPFE) if there does not exist a self-signaling set, which is defined by a set $\chi\subseteq\{L,H\}$ such that there exists a price $p'$ such that
	\begin{equation}
		\begin{aligned}
			\chi=\{&(j)\in\{L,H\}:U(p',\mu_\chi)>U(p_j^*,b^*(p_j^*))\},
		\end{aligned}
		\nonumber
	\end{equation}
	where $\mu_\chi=\frac{q_L(\rho_e\mathbf{1}_{(A_L,0)\in\chi}+(1-\rho_e)\mathbf{1}_{(A_L,c)\in\chi})+q_H\mathbf{1}_{(A_H,c)\in\chi}}{\rho_e\mathbf{1}_{(A_L,0)\in\chi}+(1-\rho_e)\mathbf{1}_{(A_L,c)\in\chi}+\mathbf{1}_{(A_H,c)\in\chi}}$ is the average quality of types in $\chi$ based on the relative prior probabilities.
\end{definition}

\begin{note}
    The GPFE may lead to no equilibrium exists.
\end{note}

\subsection{Screening Model}
Workers can undertake a contractible/observable task level $t\geq 0$. The utility of a worker is defined by $u(w,t,\theta):=w-c(t,\theta)$, where $c(\cdot,\cdot)$ satisfies the same assumption as in signaling model \ref{sec:signaling}.

The Game follows
\begin{enumerate}[]
    \item \underline{Stage 1}: Two firms simultaneously determine sets of contracts, $(w,t)$.
    \item \underline{Stage 2}: The worker observes all offer contracts and makes employment decision.
    (If indifference, choose lower task contract, favor employment over unemployment. If contracts of firms are indifferent, choose each with probability 1/2.)
\end{enumerate}

The null contract is $(w,t)=(0,0)$. Assume WLOG at stage 1, each firm appears a non-empty set of contracts.

\subsubsection*{Perfect Information}
\begin{proposition}[Perfect Information]
    If firms can observe the worker types, then in SPE firms make zero profit and type $\theta_i$ worker signs $(w^*_i,t^*_i)=(\theta_i,0)$.
\end{proposition}
\begin{proof}
    \begin{claim}
        Firms make zero profits from this contract.
    \end{claim}
    \begin{proof}
        Suppose not,
    \begin{enumerate}[$\circ$]
        \item $w^*_i>\theta_i$ $\Rightarrow$ negative profits, firms benefit from offering null contract.
        \item $w^*_i<\theta_i$ $\Rightarrow$ Let $\Pi$ be the total profits of the firms. Then one of the firms makes profit $\leq \frac{\Pi}{2}$. Then, this firm can benefit from offering $(w^*_i+\Delta,t^*_i)$, where $\Delta \rightarrow 0^+$.
    \end{enumerate}
    \end{proof}
    Then, we prove the firms must choose $(w^*_i,t^*_i)=(\theta_i,0)$. Suppose by the way of contradiction that $t_i^*>0$. Then, one firm can profitably deviate by offering $(w^*_i,0)$.
\end{proof}

\subsubsection*{Asymmetric Information}
\begin{lemma}\label{lemma:zero_profit}
    In any SPE, firms obtain zero profits,
\end{lemma}
\begin{proof}
    Firms must make profits $\geq 0$. Suppose by the way of contradiction that the total profit $\Pi>0$. Let $(w_L,t_L)$ be the contract signed by $\theta_L$ and $(w_H,t_H)$ be the contract signed by $\theta_H$. One firm can profitably deviate by offering $(w_L+\Delta,t_L)$ and $(w_H+\Delta,t_H)$, where $\Delta \in (0,\Pi)$.
\end{proof}

\begin{lemma}
    There is \textbf{no} pooling SPE.
\end{lemma}
\begin{proof}
    Suppose for a contradiction, $\exists$ an SPE where both worker types sign $(w_p=\mathbb{E}[\theta],t_p)$. Suppose one firm offers $(w_p,t_p)$, then another firm can only employ high type workers by offering $(\tilde{w},\tilde{t})$, where $\tilde{w}-c(\tilde{t},\theta_H)>\mathbb{E}[\theta]-c(t_p,\theta_H)$, $\tilde{w}-c(\tilde{t},\theta_L)<\mathbb{E}[\theta]-c(t_p,\theta_L)$, and $\tilde{w}<\theta_H$. (The existence is given by $\frac{\partial^2 c(t,\theta)}{\partial t\partial \theta}<0$.)
\end{proof}

\begin{lemma}
    Let $(w_L,t_L)$ be the contract signed by $\theta_L$ and $(w_H,t_H)$ be the contract signed by $\theta_H$ in separating SPE. Then, $w_L=\theta_L$ and $w_H=\theta_H$.
\end{lemma}
\begin{proof}
    Suppose $w_i>\theta_i,i\in\{L,H\}$, firms benefit from not offering this contract. So, $w_L\leq \theta_L$ and $w_H\leq \theta_H$.
    \begin{enumerate}
        \item \underline{$w_L=\theta_L$:} Suppose $w_L<\theta_L$. Either firm can profitably deviate by setting $(w'_L,t_L)$ such that $w_L<w'_L<\theta_L$. This offer can win all low-type workers and get a positive profit from hiring them. If $w'_L-c(t_L,\theta_H)\geq w_H-c(t_H,\theta_H)$, the offer can also hire high-type workers, which also give positive profit for the firm. Hence, there is a contradiction.
        \item \underline{$w_H=\theta_H$:} Suppose $w_H<\theta_H$, firms get positive profits, which contradicts to the Lemma \ref{lemma:zero_profit}.
    \end{enumerate}
\end{proof}

\begin{lemma}
    $\theta_L$ signs the contract $(\theta_L,0)$ in SPE.
\end{lemma}
\begin{proof}
    Suppose $t_L>0$. One firm can profitably deviate by offering $(\theta_L-\Delta,0)$.
\end{proof}

\begin{proposition}
    In any (pure strategy) SPE, $\theta_L$ signs $(w_L,t_L)=(\theta_L,0)$ and $\theta_H$ signs $(w_H,t_H)=(\theta_H,t_H)$, where $t_H$ solves
    \begin{equation}
        \begin{aligned}
            \theta_H-c(t_H,\theta_L)=\theta_L
        \end{aligned}
        \nonumber
    \end{equation}
\end{proposition}

If $\lambda:=P(\theta_H)$ is high, the pure SPE may not exist (exist $(\tilde{w},\tilde{t})$ can attract both types and make positive profit).

Cross subsidizing deviation by a firm (prices one product above its market value to fund another product), $(\tilde{w},\tilde{t})$ (signed by low type) and $(\tilde{\tilde{w}},\tilde{\tilde{t}})$ (signed by high type), is a profitable deviation if $\lambda$ is large enough.


\section{Bargaining}
The axiomatic approach abstracts away from the details of the bargaining process. Determine directly ``reasonable” or ``natural” properties that outcomes should satisfy.

We use $X$ to denote \textit{set of possible agreements} and $D$ to denote the \textit{disagreement} outcome.
\begin{example}
    $X=\{(x_1,x_2):x_1+x_2=1,x_i\geq 0\}$, $D=(0,0)$.
\end{example}

We assume that each player $i$ has preferences, represented by a utility function $u_i$ over $X\cup\{D\}$. We denote the set of possible payoffs by set $U$ defined by
\begin{equation}
    \begin{aligned}
        U&=\{(v_1,v_2):u_1(x)=v_1,u_2(x)=v_2 \textnormal{ for some }x\in X\}\\
        d&=\left(u_1(D),u_2(D)\right)
    \end{aligned}
    \nonumber
\end{equation}
\begin{definition}[Bargaining Problem]
    A \textbf{bargaining problem} is a pair $(U,d)$ where $U\subset \mathbb{R}^2$ and $d\in U$:
    \begin{enumerate}
        \item $U$ is a convex and compact set.
        \item There exists some $v\in U$ such that $v>d$ (i.e., $v_i>d_i$ for all $i$).
    \end{enumerate}
\end{definition}
We denote the set of all possible bargaining problems by $\mathcal{B}$. A bargaining solution is a function $f: \mathcal{B} \rightarrow U$.

We will study bargaining solutions $f(\cdot)$ that satisfy a list of reasonable axioms.

\begin{definition}[Axioms]
    \textbf{Axiom 1 (Pareto Efficient)}: A bargaining solution $f(U,d)$ is \textit{Pareto efficient} if there does not exist a $(v_1, v_2)\in U$ such that $v\geq f(U,d)$ and $v_i>f_i(U,d)$ for some $i$.\\
    \textbf{Axiom 2 (Symmetry)}: A bargaining solution $f$ is \textit{symmetric} if for any symmetric bargaining problem $(U,d)$ ($(u_1,u_2)\in U$ if and only if $(u_2,u_1)\in U$ and $d_1=d_2$), we have $f_1(U,d)=f_2(U,d)$.\\
    \textbf{Axiom 3 (Invariance to Linear Transformations)}: A bargaining solution $f$ is \textit{invariant} if for any bargaining problem $(U, d)$ and all $\alpha_i\in (0, \infty)$, $\beta_i\in \mathbb{R}$ ($i=1,2$), if we consider the bargaining problem $(U',d')$ with
    \begin{equation}
        \begin{aligned}
            U'&=\{(\alpha_1u_1+\beta_1,\alpha_2u_2+\beta_2):(u_1,u_2)\in U\}\\
            d'&=\left(\alpha_1d_1+\beta_1,\alpha_2d_2+\beta_2\right)
        \end{aligned}
        \nonumber
    \end{equation}
    then $f_i(U',d')=\alpha_i f_i(U,d) + \beta_i$ for $i=1,2$.\\
    \textbf{Axiom 4 (Independence of Irrelevant Alternatives)}: A bargaining solution $f$ is \textit{independent} if for any two bargaining problems $(U,d)$ and $(U',d)$ with $U'\subseteq U$ and $f(U,d)\in U'$, we have $f(U',d)=f(U,d)$.
\end{definition}

\subsection{Nash Bargaining Solution}
\begin{definition}[Nash Bargaining Solution]
    The Nash (1950) bargaining solution $f^N$ is defined by
    \begin{equation}
        \begin{aligned}
            \{f^N(U,d)\}=\argmax_{u\in U,u\geq d}(u_1-d_1)(u_2-d_2)
        \end{aligned}
        \nonumber
    \end{equation}
\end{definition}
Given the assumptions on $(U, d)$, the solution to the optimization problem exists and is unique.

\begin{theorem}
    $f^N$ is the unique bargaining solution that satisfies the four axioms.
\end{theorem}


\subsection{Strategic Delay in Bargaining with Two-Sided Uncertainty (Cramton, 1992)}
A seller with valuation $S$ and a buyer with valuation $B$ are bargaining over the price of an object. The valuation is symmetric and private, that is, $B$ and $S$ are i.i.d. drawn from a distribution $F$ with density $f$ over $[0,1]$.

An outcome of the game is the time and the price, $\left<t,p\right>$. The discount rate is $r$, that is, the payoff to $S$ is $e^{-rt}(p-S)$ and the payoff to $B$ is $e^{-rt}(B-p)$. The discount rate $r$ and the valuation distribution $F$ are common knowledge.

As in Admati and Perry (1987), the players alternate making offers with a minimum time of $t^0=-\frac{1}{r}\log\delta$ between offers. Initially, both traders have the option of making the first offer or terminating negotiations (at time $t\geq-t^0$). If the traders happen to make initial offers at the same time, then a fair coin is flipped to determine which offer stands as the initial offer. After an offer is made, the other trader has three possible responses: (1) a counter-offer, (2) acceptance, or (3) termination.

Suppose that trader $T\in\{S,B\}$ makes the first offer p1 after a delay of $\Delta_1$, and that in round $i$ the offer $p_i$ is made after a delay $\Delta_i$ beyond the minimum time $t_0$ between offers. The history after $n$ rounds is $h^n=\{T,(\Delta_i,p_i)_{i=1,...,n}\}$.

The pure strategy of the seller and the buyer are denoted by $\pi_S$ and $\pi_B$. The profile of strategies is $\pi=\{\pi_S,\pi_B:\forall (S,B)\}$, which result in an outcome $\{t(S,B),p(S,B)\}$ that depends on the traders' valuations $(S,B)$. ``No trade'' is represented by $t=\infty$. Since all actions are publicly observed, $S$'s belief about $B$'s valuation is independent of $S$ after any history $h^n$. The belief after $h^n$ can be denoted by $\mu=\{F_B(\cdot\mid h^n),F_S(\cdot\mid h^n)\}$.












\chapter{Mechanism Design}

\section{Mechanism Design}
Design incentives for agents to reveal their types or achieve particular society outcomes.

Given a ``direct'' mechanism,
\begin{enumerate}
    \item the set of agents $I$ with utility function $u_i(x;\theta_i),i\in I$,
    \item alternatives (outcomes for the society) $X$,
    \item  types (of agents) $\Theta=(\Theta_1,...,\Theta_I)$ with prior probability $\phi$ over $\Theta$,
    \item and a \textbf{social choice function} (SCF) $f:\Theta\rightarrow X$.
\end{enumerate}

\begin{definition}[Mechanism $\Gamma=(S,g)$]
    %\normalfont
    A \textbf{mechanism} is represented as $$\Gamma=\left(S, g\right)$$
    where $S\triangleq(S_1,...,S_I)(S_1,...,S_I)$ represents the set of strategies, $S_i$ represents the strategy set of agent $i$, and $g:S\triangleq(S_1,...,S_I) \rightarrow X$ is the outcome function that determines the social outcome.
\end{definition}

A \textbf{Bayesian game induced by} $\Gamma$ is $(I,S,\Theta,\phi,\tilde{u})$, where the payoffs functions are
\begin{equation}
    \begin{aligned}
        \tilde{u}_i(s;\theta_i)=u_i(g(s);\theta_i)
    \end{aligned}
    \nonumber
\end{equation}
for all $i\in I, s\in S$, and $\theta_i\in\Theta_i$.


\subsection{Implement in Dominant Strategies}
\begin{definition}[$\Gamma$ Implements $f$]
    %\normalfont
    A mechanism $\Gamma$ (indirectly) \textbf{implements} a social choice function (SCF) $f$ if there exists an ``equilibrium'' $s^*(\cdot)=\left(s_1^*(\cdot),...,s_I^*(\cdot)\right)$ of the Bayesian game induced by $\Gamma$ such that $$g(s_1^*(\theta_1),...,s_I^*(\theta_I))=f(\theta_1,...,\theta_I)$$ for all $(\theta_1,...,\theta_I)\in \Theta$. Here the ``equilibrium'' is a dominant strategy equilibrium or BNE.
\end{definition}
That is, the equilibrium in a game induced by $\Gamma$ gives the same outcome as the outcome of $f$ given by revealing agents' true types.

\begin{definition}[Direct Mechanism]
    %\normalfont
    A mechanism is \textbf{direct mechanism} if agents directly report their types (types are observable). $S_i=\Theta_i$ for all $i\in I$ and $g(\theta)=f(\theta)$ for all $\theta=(\theta_1,...,\theta_I)\in\Theta$. So, a direct mechanism can be represented by $\Gamma=(\Theta,f(\cdot))$.
\end{definition}
In indirect mechanism agents don't report their types directly. Types can be observed only indirectly through signals or behavior.

A strategy is weakly dominant if for all $\theta_i\in\Theta_i$ and all $s_{-i}(\cdot)\in S_{-i}$, we have $u_i(g(s_i(\theta_i),s_{-i}),\theta_i)\geq u_i(g(s'_i,s_{-i}),\theta_i)$ for all $s'_i\neq s_i$.


\begin{definition}[Dominant Strategy Equilibrium]
    %\normalfont
    Strategy profile $s^*=(s_1^*(\cdot),...,s_I^*(\cdot))$ is a \textbf{dominant strategy (D-S) equilibrium} of $\Gamma=(S,g(\cdot))$ if for all $i\in I$ and $\theta_i\in \Theta$, we have, for all $s'_i\in S_i$ and $s_{-i}\in S_{-i}$:
    \begin{equation}
        \begin{aligned}
            u_i(g(s_i^*(\theta_i),s_{-i}),\theta_i)&\geq u_i(g(s'_i,s_{-i}),\theta_i)
        \end{aligned}
        \nonumber
    \end{equation}
    equivalently, in the Bayesian game induced by $\Gamma$,
    \begin{equation}
        \begin{aligned}
            \tilde{u}_i(s_i^*(\theta_i),s_{-i},\theta_i)\geq \tilde{u}_i(s'_i,s_{-i},\theta_i)
        \end{aligned}
        \nonumber
    \end{equation}
\end{definition}

\begin{definition}[Implement in dominant strategies]
    %\normalfont
    $\Gamma$ \textbf{implements} $f$ in \textbf{dominant strategies} if $\exists$ a dominant strategy (D-S) equilibrium $s^*$ of $\Gamma$ such that $g(s^*(\theta))=f(\theta)$.
\end{definition}
``$\Gamma$ implements $f$ in dominant strategies'' means that the results of the direct mechanism, $(\Theta,f(\cdot))$, are equivalent to the results of a D-S equilibrium of another (indirect) mechanism $\Gamma$. That is, $\Gamma$ can be used as $(\Theta,f(\cdot))$ ``equivalently.''
\begin{center}\begin{figure}[htbp]
    \centering
    \begin{tikzpicture}[domain=0:3.25]
        \node at (0,0) {Types: $\Theta$};
        \draw[->](1,0)--(7,0) node[right] {Alternatives: $X$};
        \draw[dashed,->](0.5,-0.5)--(3.5,-3);
        \node at (4,-3) {$s^*(\theta)$};
        \draw[dashed,->](4.5,-3)--(7.5,-0.5);
        \draw[->](0,-0.5)--(3.5,-3.5);
        \node at (4,-3.5) {$s^*=\theta$};
        \draw[->](4.6,-3.5)--(8.1,-0.5);
        \node at (0.5,-1.8) {Desirable};
        \node at (0.5,-2.2) {Situation};
        \node at (2.5,-0.9) {Indirect};
        \node at (2.5,-1.3) {Mechanism};
        \node at (2.5,-1.7) {$\Gamma$};
        \node at (6.5,-2.5) {$f(\theta)$};
        \node at (5.5,-1.5) {$g(s^*(\theta))$};
    \end{tikzpicture}
    \caption{How Mechanism Design works}
    \label{}
\end{figure}\end{center}




\subsection{Dominant-Strategy-Incentive-Compatible (DSIC)/Strategy-Proof}
\begin{definition}[Strategy-Proof, DSIC]
    %\normalfont
    $f$ is \textbf{strategy-proof} (also called dominant-strategy-incentive-compatible, \textbf{DSIC}) if $$s^*_i(\theta_i)=\theta_i,\quad \forall \theta_i\in\Theta_i,i\in I$$ is a dominant strategy (D-S) equilibrium of the direct mechanism $\Gamma=(\Theta,f(\cdot))$.
\end{definition}

\begin{theorem}[Revelation Principle]
    If $\exists$ a mechanism $\Gamma=(S,g(\cdot))$ that implements $f$ in dominant strategies (i.e., $\exists$ a D-S equilibrium $s^*$ of $\Gamma$ such that $g(s^*(\theta))=f(\theta)$). Then $f$ is strategy-proof (DSIC).
    \begin{note}
        Based on the Revelation Principle, if a ``indirect'' mechanism has a D-S equilibrium $s^*$, then there exists a ``direct'' DSIC mechanism $f$ with $f(\theta)=g(s^*(\theta))$.
    \end{note}
\end{theorem}
\begin{proof}
    Given $\Gamma$ implements $f$ in dominant strategies, there is a  D-S equilibrium $s^*=\left(s_1^*(\cdot),...,s_I^*(\cdot)\right)$ such that $g(s^*(\theta))=f(\theta)$.\\
    By the definition of D-S equilibrium,
    \begin{equation}
        \begin{aligned}
            u_i(g(s_i^*(\theta_i),s_{-i}),\theta_i)&\geq u_i(g(s'_i,s_{-i}),\theta_i)
        \end{aligned}
        \nonumber
    \end{equation}
    By substituting $g(s^*(\theta))=f(\theta)$, we have
    \begin{equation}
        \begin{aligned}
            u_i(f(\theta_i,\theta_{-i}),\theta_i)&\geq u_i(f(\theta'_i,\theta_{-i}),\theta_i),\quad \forall \theta'_i\in\Theta_i
        \end{aligned}
        \nonumber
    \end{equation}
    which gives that $f$ is DSIC.
\end{proof}



\subsection{Bayesian-Incentive-Compatible (BIC)}
\begin{definition}[BIC]
    %\normalfont
    $f$ is Bayesian-incentive-compatible (B.I.C.) if $$s^*_i(\theta_i)=\theta_i,\quad \forall \theta_i\in\Theta_i,i\in I$$ is a BNE of the Bayesian game induced by the direct mechanism $\Gamma=(\Theta,f(\cdot))$.
\end{definition}
BIC is a weaker condition than DSIC, because a BNE must also be a D-S equilibrium.

\begin{theorem}[Revelation Principle (BIC)]\label{theorem:revelation principle BIC}
    If $\exists$ a mechanism $\Gamma=(S,g(\cdot))$ that implements $f$ in BNE (i.e., $\exists$ a BNE $s^*$ of $\Gamma$ such that $g(s^*(\theta))=f(\theta)$). Then $f$ is BIC.
    \begin{note}
        Based on the Revelation Principle, if a ``indirect'' mechanism has a BNE $s^*$, then there exists a ``direct'' BIC mechanism $f$ with $f(\theta)=g(s^*(\theta))$.
    \end{note}
\end{theorem}


\subsection{Negative Results: dictatorial SCF $f$}
\begin{theorem}[Gibbard-Satterthwaite Theorem]
    Suppose that $|X|\geq 3$ and a social choice function $f$ is surjective (i.e., $\forall x\in X$ $\exists (\theta_1,...,\theta_I)\in\Theta$ s.t. $f(\theta_1,...,\theta_I)=x$). Then, $f$ is strategy-proof (DSIC) $\Leftrightarrow$ $f$ is dictatorial (\ref{SWF_properties}, i.e., $\exists i^*\in\{1,...,I\}$ such that $f(\theta)\in \argmax_{x\in X}u_{i^*}(x;\theta_{i^*})$ for all $\theta\in \Theta$).
\end{theorem}
Note: By the revelation principle, under the conditions of the Theorem, there is no mechanism that implements a non-dictatorial SCF $f$ in dominant strategies.

\begin{comment}
Some lemmas can help to prove the theorem.
\begin{lemma}
    If $f$ is strategy-proof (DSIC) and $f(\succeq)=x$ and $x\succeq_i y \Rightarrow x\succeq'_i y$ for all $i\in I$ and all $x\neq y\in X$, then $f(\succeq')=x$.
\end{lemma}

\begin{lemma}[Pareto Effeciency]
    If $f$ is strategy-proof (DSIC) and $x\succ_i y$ for all $i\in I$, then $f(\succeq')\neq y$.
\end{lemma}

\begin{example}
    Define $\succeq=\begin{pmatrix}
        x&y\\
        y&x\\
        z&z
    \end{pmatrix}$ and $\succeq'=\begin{pmatrix}
        x&y\\
        y&z\\
        z&x
    \end{pmatrix}$, each column 1/2 represents player 1/2's preferences.
\end{example}
\end{comment}


\section{Quasi-linear Model}
Consider $x=(k,\underbrace{t_1,...,t_I}_{t})\in X=K\times \mathbb{R}^I$, in our example, $K$ represents a set of choices for projects and $\mathbb{R}^I$ represents the set of transfers for all agents.

Each agent has a quasi-linear function that represents her utility:
\begin{equation}
    \begin{aligned}
        u_i(k,t,\theta_i)=v(k,\theta_i)+t_i
    \end{aligned}
    \nonumber
\end{equation}
where $v: K\times\Theta_i \rightarrow \mathbb{R}$ represents the utility without transfers.

Let $p(\cdot)=\left(k(\cdot),t(\cdot)\right)$ represents the ``project-choice rule'' $k: \Theta \rightarrow K$ and the ``transfer rule'' $t: \Theta \rightarrow \mathbb{R}^I$.
\begin{definition}[ex-post efficient]
    %\normalfont
    $k(\cdot):\Theta \rightarrow K$ is \textbf{ex-post efficient} if $\nexists \left(\theta\in\Theta, k'\in K, t=(t_1,...,t_I)\in \mathbb{R}^I\right)$ such that
    \begin{enumerate}[(1).]
        \item $\sum_{i=1}^I t_i=0$
        \item $v_i(k',\theta_i)+t_i> v_i(k(\theta),\theta_i)$, $\forall i\in I$
    \end{enumerate}
    i.e., we can't get a higher total social welfare. (Because of the transfers, a higher social welfare can make everyone better off.)
\end{definition}

\begin{proposition}[ex-post efficient $\Leftrightarrow$ maximizing the sum of utilities]
    $\forall$ project-choice rule $k(\cdot)$, $k(\cdot)$ is ex-post efficient \underline{if and only if} $k(\cdot)$ maximizes the sum of utilities, i.e., $\forall \theta\in\Theta$ and $\forall k'\in K$,
    \begin{equation}
        \begin{aligned}
            \sum_{i=1}^I v_i(k(\theta),\theta_i)\geq \sum_{i=1}^I v_i(k',\theta_i)
        \end{aligned}
        \nonumber
    \end{equation}
\end{proposition}
\begin{proof}
    ``$\Leftarrow$'': Suppose by the way of contradiction that there exists $\left(\theta, k', t\right)$ such that $\sum_{i=1}^I t_i=0$ and $v_i(k',\theta_i)+t_i> v_i(k(\theta),\theta_i)$, $\forall i\in I$. Sum together, there is a contradiction.\\
    ``$\Rightarrow$'': Suppose by the way of contradiction that there exists $(\theta,k')$, $\sum_{i=1}^I v_i(k(\theta),\theta_i)<\sum_{i=1}^I v_i(k',\theta_i)$. Then, we can define a $t$ such that satisfies $\sum_{i=1}^I t_i=0$ and $v_i(k',\theta_i)+t_i> v_i(k(\theta),\theta_i)$, $\forall i\in I$. Let $\Delta=\sum_{i=1}^I v_i(k',\theta_i)-\sum_{i=1}^I v_i(k(\theta),\theta_i)$, then $t_i=v_i(k(\theta),\theta_i)-v_i(k',\theta_i)+\frac{\Delta}{I},\forall i\in I$ is the transfer-choice we want.
\end{proof}

\subsection{Vickrey-Clarke-Groves Mechanism}
\begin{proposition}[VCG Mechanism]
    Suppose $k^*(\cdot)$ is ex-post efficient project choice rule. For each $i\in \{1,...,I\}$, let $h_i:\Theta_{-i} \rightarrow \mathbb{R}$ be an arbitrary function.\\
    Define the transfer rule $t(\cdot)$ as follows
    \begin{equation}
        \begin{aligned}
            t_i(\theta_i,\theta_{-i})=\sum_{j\neq i}v_j(k^*(\theta_i,\theta_{-i}),\theta_j)+h_i(\theta_{-i})
        \end{aligned}
        \nonumber
    \end{equation}
    Then the SCF $f(\cdot)=\left(k^*(\cdot),t(\cdot)\right)$ is DSIC.
\end{proposition}
\begin{proof}
    Take any $i,\theta\in\Theta$ and let $\hat{\theta}_i\in\Theta_i$.
    Reporting truthfully gives higher profits than misreporting $\hat{\theta}_i$:
        \begin{equation}
            \begin{aligned}
                v_i(k^*(\theta_i,\theta_{-i}),\theta_i)+t_i(\theta_i,\theta_{-i})&=v_i(k^*(\theta_i,\theta_{-i}),\theta_i)+\sum_{j\neq i}v_j(k^*(\theta),\theta_j)+h_i(\theta_{-i})\\
                &=\sum_{j=1}^Iv_j(k^*(\theta),\theta_j)+h_i(\theta_{-i})\\
                &\geq \sum_{j=1}^Iv_j(k^*(\hat{\theta}_i,\theta_{-i}),\theta_j)+h_i(\theta_{-i})\\
                &=v_i(k^*(\hat{\theta}_i,\theta_{-i}),\theta_i)+t_i(\hat{\theta}_i,\theta_{-i})
            \end{aligned}
            \nonumber
        \end{equation}
    Hence, VCG mechanism with SCF $f(\cdot)=\left(k^*(\cdot),t(\cdot)\right)$ is DSIC.
\end{proof}

\begin{definition}[Pivotal VCG Mechanism (Special Case)]
    %\normalfont
    Let $h_i(\theta_{-i})=\max_{k\in K}\sum_{j\neq i}v_j(k,\theta_j)$.
    \begin{equation}
        \begin{aligned}
            t_i(\theta_i,\theta_{-i})=\sum_{j\neq i}v_j(k^*(\theta_i,\theta_{-i}),\theta_j)-\max_{k\in K}\sum_{j\neq i}v_j(k,\theta_j)\leq 0
        \end{aligned}
        \nonumber
    \end{equation}
    \begin{enumerate}
        \item $i$ is \textbf{pivotal} if $k^*(\theta)$ doesn't maximize $\max_{k\in K}\sum_{j\neq i}v_j(k,\theta_j)$.
        \item $i$ is \textbf{not pivotal} if $k^*(\theta)$ maximizes $\max_{k\in K}\sum_{j\neq i}v_j(k,\theta_j)$.
    \end{enumerate}
    \begin{note}
        $i$ is \textbf{not pivotal} $\Rightarrow$ $t_i(\theta)=0$.
    \end{note}
\end{definition}

\begin{example}
    Suppose $k\in\{0,1\}, \theta\in \Theta\subset \mathbb{R}^I$, $v_i(k,\theta_i)=k\theta_i$. Since $\sum_{i=1}^Iv_i(k,\theta_i)=k\sum_{i=1}^I\theta_i$, $k^*(\cdot)$ is ex-post efficient: $k^*(\theta)=1 \Leftrightarrow \sum_{i=1}^I\theta_i\geq 0$. The pivotal VCG transfers:
    \begin{equation}
        \begin{aligned}
            t_i(\theta)=\left\{\begin{matrix}
                \sum_{j\neq i}\theta_j-0&\textnormal{ if }\sum_{j=1}^I \theta_j\geq 0>\sum_{j\neq i}\theta_j\\
                0-\sum_{j\neq i}\theta_j&\textnormal{ if }\sum_{j=1}^I \theta_j< 0\leq\sum_{j\neq i}\theta_j\\
                0&\textnormal{ otherwise}
            \end{matrix}\right.
        \end{aligned}
        \nonumber
    \end{equation}
\end{example}

\begin{example}[ (Second Price Auction)]
    One indivisible object to be allocated to one of $1,...,I$. Social decision is deciding who gets the object, $K=\{1,...,I\}$, $k=i$ means ``i receives the object''. $\Theta_i\subseteq \mathbb{R}_+$, $\theta_i\in\Theta_i$ denotes $i$'s valuation for the object $v_i(k,\theta_i)=\left\{\begin{matrix}
        \theta_i,& \textnormal{ if }k=i\\
        0,& \textnormal{ if }k\neq i
    \end{matrix}\right.$.
    Ex-post efficient $k^*(\cdot)$ allocates the object to the individual with the highest valuation. The pivotal VCG transfers:
    \begin{equation}
        \begin{aligned}
            t_i(\cdot)=\left\{\begin{matrix}
                0-\theta^{(2)}\textnormal{(the second highest)},& \textnormal{ if }k^*(\theta)=i \textnormal{ ($i$ is pivotal)}\\
                0=\theta^{(1)}-\theta^{(1)},&\textnormal{ if }k^*(\theta)\neq i
            \end{matrix}\right.
        \end{aligned}
        \nonumber
    \end{equation}
\end{example}


\begin{example}[ (Uniform-Price Auction)]
    $m$-identical indivisible objects ($m<I$). Each agent can consume $0$ or $1$ object. $K=\{M\subset \{1,...,I\}\mid |M|=m\}$, where $k=M$ is the set of agents who receive an object. $v_i(k,\theta_i)=\left\{\begin{matrix}
        \theta_i,&i\in k\\
        0,&i\notin k
    \end{matrix}\right.$. The ex-post efficient $k^*(\cdot)$ allocates the objects to top $m$-valuation agents. The pivotal VCG transfers:
    \begin{equation}
        \begin{aligned}
            t_i(\theta)=\left\{\begin{matrix}
                \left(\sum_{j=1}^m\theta_{(j)}-\theta_i\right)-\left(\sum_{j=1}^{m+1}\theta_{(j)}-\theta_i\right)=-\theta^{(m+1)}&,i\in k^*(\theta)\\
                0&,i\notin k^*(\theta)
            \end{matrix}\right.
        \end{aligned}
        \nonumber
    \end{equation}
\end{example}

\begin{example}[ (Package Auction)]
    $2$ identical indivisible objects to be allocated $I=3$ agents. Each agent can consume $0$, $1$, or $2$ units. $K=\{(1,1,0),(1,0,1),(0,1,1),(2,0,0),(0,2,0),(0,0,2)\}$. $\Theta_i=\{\theta_i=(v_1,v_2)\in \mathbb{R}^2_+\mid v_2\geq v_1\geq 0\}$. Consider an example,
    \begin{center}
        \begin{tabular}{c|c|c|c}
            \hline
                &$\theta_1$ &$\theta_2$&$\theta_3$\\
            \hline
                $v_1$&$3$ &$4$&$1$\\
            \hline
                $v_2$&$4$ &$5$&$6$\\
            \hline
        \end{tabular}
    \end{center}
    The ex-post efficient $k^*(\theta)=(1,1,0)$. Then,
    \begin{equation}
        \begin{aligned}
            t_1(\theta)=4-6=-2, t_2(\theta)=3-6=-3, t_3(\theta)=7-7=0
        \end{aligned}
        \nonumber
    \end{equation}
\end{example}


\subsection{Uniqueness of VCG Mechanism}
\begin{assumption}
    $K$ is a compact subset of a topological space which all singletons are closed (metric spaces $K\subset \mathbb{R}^n$, $K$ compact, or any finite $K$.)
\end{assumption}

Let $V_{usc}$ be the set of upper hemicontinuous functions $v: K \rightarrow \mathbb{R}$. ($v$ is upper hemicontinuous if $\forall \alpha\in \mathbb{R}: \{k\in K\mid v(k)\geq \alpha\}$ is closed.)

\textbf{Facts}: A upper hemicontinuous function attains the maximum over a compact set. Sum of upper hemicontinuous functions is upper hemicontinuous.

\begin{proposition}[Green \& Laffont 1979]\label{prop:Uniq_VCG_mechanism}
    Suppose that $\forall i: \{v_i(\cdot,\theta_i): K \rightarrow \mathbb{R}\mid \theta_i\in\Theta_i\}=V_{usc}$. Then, any ex-post efficient and DSIC direct mechanism is a VCG mechanism.
\end{proposition}
\begin{proof}
    Take any $f(\cdot)=\left(k^*(\cdot),t_i(\cdot)\right)$ such that it is ex-post efficient and DSIC. We prove it is VCG mechanism by showing there is a $h_{i}$ satisfies the definition of VCG mechanism.\\
    Define $\forall i, h_i:\Theta \rightarrow \mathbb{R}$ such that
    \begin{equation}
        \begin{aligned}
            h_i(\theta)=-\sum_{j\neq i}v_j(k^*(\theta_i,\theta_{-i}),\theta_j)+t_i(\theta_i,\theta_{-i})
        \end{aligned}
        \nonumber
    \end{equation}
    \textbf{We want to show $h_i(\theta)$ is independent of $\theta_i$ and is actually $h_i(\theta_{-i})$.}\\
    That is, $\forall \theta_i,\hat{\theta}_i,\theta_{-i}$, we want to show $h_i(\theta_i,\theta_{-i})=h_i(\hat{\theta}_i,\theta_{-i})$.
    \begin{lemma}\label{lemma:k_h}
        If $k^*(\theta_i,\theta_{-i})=k^*(\hat{\theta}_i,\theta_{-i})$, then $h_i(\theta_i,\theta_{-i})=h_i(\hat{\theta}_i,\theta_{-i})$.
    \end{lemma}
    \begin{proof}
        $k^*(\theta_i,\theta_{-i})=k^*(\hat{\theta}_i,\theta_{-i})$ requires
        \begin{equation}
            \begin{aligned}
                v_i(k^*(\theta_i,\theta_{-i}),\theta_i)+t_i(\theta_i,\theta_{-i})&\geq v_i(k^*(\hat{\theta}_i,\theta_{-i}),\theta_i)+t_i(\hat{\theta}_i,\theta_{-i})\\
                v_i(k^*(\hat{\theta}_i,\theta_{-i}),\hat{\theta}_i)+t_i(\hat{\theta}_i,\theta_{-i})&\geq v_i(k^*(\theta_i,\theta_{-i}),\hat{\theta}_i)+t_i(\theta_i,\theta_{-i})
            \end{aligned}
            \nonumber
        \end{equation}
        Since $v_i(k^*(\hat{\theta}_i,\theta_{-i}),\hat{\theta}_i)=v_i(k^*(\theta_i,\theta_{-i}),\hat{\theta}_i)$, we have $t_i(\hat{\theta}_i,\theta_{-i})=t_i(\theta_i,\theta_{-i})$. Hence, $h_i(\theta_i,\theta_{-i})=h_i(\hat{\theta}_i,\theta_{-i})$.
    \end{proof}
    \begin{enumerate}
        \item \underline{Case 1:} ``$k^*(\theta_i,\theta_{-i})=k^*(\hat{\theta}_i,\theta_{-i})$'', $h_i(\theta_i,\theta_{-i})=h_i(\hat{\theta}_i,\theta_{-i})$ is given by Lemma \ref{lemma:k_h}.
        \item \underline{Case 2:} ``$k^*(\theta_i,\theta_{-i})\neq k^*(\hat{\theta}_i,\theta_{-i})$''\\
        Suppose by the way of contradiction $h_i(\theta_i,\theta_{-i})\neq h_i(\hat{\theta}_i,\theta_{-i})$, WLOG, we consider $h_i(\theta_i,\theta_{-i})>h_i(\hat{\theta}_i,\theta_{-i})$. There is an $\epsilon>0$ s.t. $h_i(\theta_i,\theta_{-i})>h_i(\hat{\theta}_i,\theta_{-i})+\epsilon$.\\
        Define $v: K \rightarrow \mathbb{R}$ such that
        \begin{equation}
            \begin{aligned}
                v(k)=\left\{\begin{matrix}
                    -\sum_{j\neq i}v_j(k^*(\theta_i,\theta_{-i}),\theta_j),& \textnormal{ if }k=k^*(\theta_i,\theta_{-i})\\
                    -\sum_{j\neq i}v_j(k^*(\hat{\theta}_i,\theta_{-i}),\theta_j)+\epsilon,& \textnormal{ if }k=k^*(\hat{\theta}_i,\theta_{-i})\\
                    -C,&\textnormal{ otherwise}\
                \end{matrix}\right.
            \end{aligned}
            \nonumber
        \end{equation}
        where $C>\max_{k\in K}\sum_{j\neq i}v_j(k^*(k,\theta_{-i}),\theta_j)$.\\
        Hence, $v$ is upper hemicontinuous, $v\in V_{usc}$.

        By the assumption that $\forall i: \{v_i(\cdot,\theta_i): K \rightarrow \mathbb{R}\mid \theta_i\in\Theta_i\}=V_{usc}$, we know $\exists \theta^\epsilon_i\in\Theta_i$ s.t. $v_i(\cdot,\theta^\epsilon_i)=v(\cdot)$.
        \begin{enumerate}[$\circ$]
            \item Because $k^*(\cdot)$ is ex-post efficient,
            \begin{equation}
                \begin{aligned}
                    v_i(k^*(\theta^\epsilon_i,\theta_{-i}),\theta^\epsilon_i)+\sum_{j\neq i} v_j(k^*(\theta^\epsilon_i,\theta_{-i}),\theta_j)&\geq v_i(k^*(\hat{\theta}_i,\theta_{-i}),\theta^\epsilon_i)+\sum_{j\neq i} v_j(k^*(\hat{\theta}_i,\theta_{-i}),\theta_j)\\
                    &=v(k^*(\hat{\theta}_i,\theta_{-i}))+\sum_{j\neq i} v_j(k^*(\hat{\theta}_i,\theta_{-i}),\theta_j)=\epsilon
                \end{aligned}
                \nonumber
            \end{equation}
            By the definition of $v(\cdot)$, we have
            \begin{equation}
                \begin{aligned}
                    v_i(k^*(\theta^\epsilon_i,\theta_{-i}),\theta^\epsilon_i)+\sum_{j\neq i} v_j(k^*(\theta^\epsilon_i,\theta_{-i}),\theta_j)=v(k^*(\theta^\epsilon_i,\theta_{-i}))+\sum_{j\neq i} v_j(k^*(\theta^\epsilon_i,\theta_{-i}),\theta_j)\leq \epsilon
                \end{aligned}
                \nonumber
            \end{equation}
            Hence, we can conclude $k^*(\theta^\epsilon_i,\theta_{-i})=k^*(\hat{\theta}_i,\theta_{-i})$. Then, by the Lemma \ref{lemma:k_h}, $h_i(\theta^\epsilon_i,\theta_{-i})=h_i(\hat{\theta}_i,\theta_{-i})$.
            \item Because $f(\cdot)=\left(k^*(\cdot),t_i(\cdot)\right)$ is DSIC, the agent with $\theta^\epsilon_i$ gets the highest profit from truthfully reporting
            \begin{equation}
                \begin{aligned}
                    v_i(k^*(\theta^\epsilon_i,\theta_{-i}),\theta^\epsilon_i)+t_i(\theta^\epsilon_i,\theta_{-i})&\geq v_i(k^*(\theta_i,\theta_{-i}),\theta^\epsilon_i)+t_i(\theta_i,\theta_{-i})\\
                    \Leftrightarrow -\sum_{j\neq i}v_j(k^*(\theta^\epsilon_i,\theta_{-i}),\theta_j)+\epsilon+t_i(\theta^\epsilon_i,\theta_{-i})&\geq-\sum_{j\neq i}v_j(k^*(\theta_i,\theta_{-i}),\theta_j)+t_i(\theta_i,\theta_{-i})\\
                    \Leftrightarrow h_i(\theta^\epsilon_i,\theta_{-i})+\epsilon&\geq h_i(\theta_i,\theta_{-i})\\
                    \Leftrightarrow h_i(\hat{\theta}_i,\theta_{-i})+\epsilon&\geq h_i(\theta_i,\theta_{-i})
                \end{aligned}
                \nonumber
            \end{equation}
            There is a contradiction.
        \end{enumerate}
    \end{enumerate}
\end{proof}


\subsection{Budget Balancedness of VCG Mechanism}
\begin{definition}[Budget-Balanced VCG Mechanism]
    %\normalfont
    A VCG mechanism is \textbf{budget-balanced} if $\sum_{i=1}^It_i(\theta)=0$.
\end{definition}

Based on the Proposition \ref{prop:Uniq_VCG_mechanism}, we can show the following corollary.
\begin{corollary}
    Suppose $I\geq 2$, $|K|\geq 2$, and $\forall i: \{v_i(\cdot,\theta_i): K \rightarrow \mathbb{R}\mid \theta_i\in\Theta_i\}=V_{usc}$. Then, there does not exist a budget-
    balanced VCG mechanism.
\end{corollary}


\begin{example}
    $K=\{0,1\},\Theta_i=[-1,1], v_i(k,\theta_i)=k\theta_i$. Take a VCG mechanism $k^*(\cdot)$ ex-post efficient, $h_1:\Theta_2 \rightarrow \mathbb{R}$, $h_2:\Theta_1 \rightarrow \mathbb{R}$.
    \begin{equation}
        \begin{aligned}
            t_1(\theta)+t_2(\theta)&=v_2(k^*(\theta),\theta_2)+h_1(\theta_2)+v_1(k^*(\theta),\theta_1)+h_2(\theta_1)\\
            &=\max\{0,\theta_1+\theta_2\}+h_1(\theta_2)+h_2(\theta_1)
        \end{aligned}
        \nonumber
    \end{equation}
    Suppose by the contradiction that it is a budget-balanced VCG mechanism.
    \begin{equation}
        \begin{aligned}
            \max\{0,\theta_1+\theta_2\}+h_1(\theta_2)+h_2(\theta_1)=0
        \end{aligned}
        \nonumber
    \end{equation}
    We have
    \begin{equation}
        \begin{aligned}
            h_2(1)-h_2(0)=\max\{0,\theta_1\}-\max\{0,\theta_1+1\}
        \end{aligned}
        \nonumber
    \end{equation}
    The LHS is constant and the RHS is a function of $\theta_1$, which gives a contradiction.
\end{example}


\subsection{Expected-externality Mechanism (BIC Mechanism)}
\begin{definition}[EE Mechanism]
%\normalfont
    $(k^*(\cdot),t(\cdot))$ is an Expected-externality (EE/AGV) mechanism if $k^*(\cdot)$ is ex-post efficient and there are functions $h_i:\Theta_i: \mathbb{R}$ for all $i$ s.t.
    \begin{equation}
        \begin{aligned}
            t_i(\theta)=\underbrace{\mathbb{E}_{\tilde{\theta}_{-i}}\left[\sum_{j\neq i}v_j(k^*(\theta_i,\tilde{\theta}_{-i}),\theta_j)\right]}_{\triangleq \xi_i(\theta_i)}+h_i(\theta_{-i})
        \end{aligned}
        \label{EE}
    \end{equation}
    where $\xi_i(\theta_i)\triangleq\mathbb{E}_{\tilde{\theta}_{-i}}\left[\sum_{j\neq i}v_j(k^*(\theta_i,\tilde{\theta}_{-i}))\right]$ is the expected externality $i$ imposes on others, from $i$'s interim perspective when her type is $\theta_i$.
\end{definition}

\begin{proposition}
    EE mechanisms are BIC.
\end{proposition}
\begin{proof}
    Take any $i$ and $\theta_i,\hat{\theta}_i\in\Theta_i$. $i$'s expected payoff from truthfully reproting is
    \begin{equation}
        \begin{aligned}
            &\mathbb{E}_{\tilde{\theta}_{-i}}\left[v_i(k^*(\theta_i,\tilde{\theta}_{-i}),\theta_i)+t_i(\theta_i,\tilde{\theta}_{-i})\right]\\
            \textnormal{(substitute \eqref{EE}) }=&\mathbb{E}_{\tilde{\theta}_{-i}}\left[\sum_{j=1}^I v_j(k^*(\theta_i,\tilde{\theta}_{-i}),\theta_j)\right]+\mathbb{E}_{\tilde{\theta}_{-i}}[h_i(\tilde{\theta}_{-i})]\\
           \textnormal{($k^*$ is ex-post efficient) } \geq& \mathbb{E}_{\tilde{\theta}_{-i}}\left[\sum_{j=1}^I v_j(k^*(\hat{\theta}_i,\tilde{\theta}_{-i}),\theta_j)\right]+\mathbb{E}_{\tilde{\theta}_{-i}}[h_i(\tilde{\theta}_{-i})]\\
           =& \mathbb{E}_{\tilde{\theta}_{-i}}\left[\sum_{j=1}^I v_j(k^*(\hat{\theta}_i,\tilde{\theta}_{-i}),\theta_j)\right]+\mathbb{E}_{\tilde{\theta}_{-i}}[t_i(\hat{\theta}_i,\tilde{\theta}_{-i})-\xi_i(\hat{\theta_i})]\\
           =&\mathbb{E}_{\tilde{\theta}_{-i}}\left[v_i(k^*(\hat{\theta}_i,\tilde{\theta}_{-i}),\theta_i)+t_i(\hat{\theta}_i,\tilde{\theta}_{-i})\right]
        \end{aligned}
        \nonumber
    \end{equation}
\end{proof}


\subsection{Budget-Balanced EE Mechanism}
Budget balancedness requires
\begin{equation}
    \begin{aligned}
        0=\sum_{i=1}^I t_i(\theta)=\sum_{i=1}^I[\xi_i(\theta_i)+h_i(\theta_{-i})] \Leftrightarrow \sum_{i=1}^I h_i(\theta_{-i})=-\sum_{i=1}^I \xi_i(\theta_i)
    \end{aligned}
    \nonumber
\end{equation}
Suppose $h_i(\theta_{-i})$ is in the form of $h_i(\theta_{-i})=c\sum_{j\neq i} \xi_j(\theta_j)$. Then,
\begin{equation}
    \begin{aligned}
        \sum_{i=1}^I h_i(\theta_{-i})=c(I-1)\sum_{i=1}^I \xi_i(\theta_i) \Rightarrow c=-\frac{1}{I-1}
    \end{aligned}
    \nonumber
\end{equation}
\begin{proposition}
    The EE mechanism where $h_i(\theta_{-i})=-\frac{1}{I-1}\sum_{j\neq i} \xi_j(\theta_j)$ is budget-balanced, $$t_i(\theta)=\xi_i(\theta_i)-\frac{1}{I-1}\sum_{j\neq i} \xi_j(\theta_j)$$
\end{proposition}

\begin{corollary}
    $\exists$ a BIC, ex-post efficient and budget-balanced direct mechanism.
\end{corollary}


\begin{example}
    Project choice with $K=\{0,1\}$, $\theta_i\sim U[-1,1]$, $v_j(k,\theta_j)=k\theta_j$. Let $k^*(\cdot)$ be:
    \begin{equation}
        \begin{aligned}
            k^*(\theta)=\left\{\begin{matrix}
                1,&\textnormal{ if }\theta_1+\theta_2\geq 0\\
                0,&\textnormal{ if }\theta_1+\theta_2<0
            \end{matrix}\right.
        \end{aligned}
        \nonumber
    \end{equation}
    Then,
    \begin{equation}
        \begin{aligned}
            \xi_i(\theta_i)&\triangleq\mathbb{E}_{\tilde{\theta}_{-i}}\left[v_{-i}(k^*(\theta_i,\tilde{\theta}_{-i}))\right]\\
            &=\int_{-1}^{-\theta_i}0\times \frac{1}{2} d\tilde{\theta}_{-i}+\int_{-\theta_i}^1 \tilde{\theta}_{-i}\times \frac{1}{2} d\tilde{\theta}_{-i}=\frac{1}{4}(1-\theta_i^2)
        \end{aligned}
        \nonumber
    \end{equation}
    Hence, the budget-balanced EE mechanism is given by
    \begin{equation}
        \begin{aligned}
            t_i(\theta_i)=\xi_i(\theta_i)-\xi_j(\theta_j)=\frac{1}{4}(\theta_j^2-\theta_i^2)
        \end{aligned}
        \nonumber
    \end{equation}
\end{example}

\subsection{Linear Utility Model}
Suppose types are real numbers $\Theta_i=[\underline{\theta}_i,\overline{\theta}_i]$ and $v_i(k,t,\theta_i)=\theta_i\cdot v_i(k)+t_i$, where $v_i:K \rightarrow \mathbb{R}_i$.

Given a direct mechanism $\left(\Theta,k(\cdot),t(\cdot)\right)$, define interim expected values of $v_i(\cdot)$ and $t_i(\cdot)$:
\begin{equation}
    \begin{aligned}
        \bar{v}_i(\theta_i)=\mathbb{E}_{\theta_{-i}}\left[v_i(\theta_i,\theta_{-i})\right]\\
        \bar{t}_i(\theta_i)=\mathbb{E}_{\theta_{-i}}\left[t_i(\theta_i,\theta_{-i})\right]
    \end{aligned}
    \nonumber
\end{equation}
The expected utility of agent $i$ when all agents report truthfully,
\begin{equation}
    \begin{aligned}
        U_i(\theta_i)&=\mathbb{E}_{\theta_{-i}}\left[\theta_iv_i(\theta_i,\theta_{-i})+t_i(\theta_i,\theta_{-i})\right]\\
        &=\theta_i \bar{v}_i(\theta_i)+\bar{t}_i(\theta_i)
    \end{aligned}
    \nonumber
\end{equation}

\begin{proposition}\label{Linear utility model_BIC}
    A direct mechanism $\left(\Theta,k(\cdot),t(\cdot)\right)$ is BIC \underline{iff} $\forall i\in\{1,...,I\}$
    \begin{enumerate}[(1).]
        \item $\bar{v}_i(\theta_i)$ is non-decreasing in $\theta_i$.
        \item $\forall \theta_i\in\Theta_i$, $U_i(\theta_i)=U_i(\underline{\theta}_i)+\int_{\underline{\theta}_i}^{\theta_i}\bar{v}_i(s)ds$
    \end{enumerate}
\end{proposition}
\begin{proof}
    ``$\Rightarrow$'': Given the direct mechanism is BIC. Take any $i$, $\theta_i,\hat{\theta}_i\in\Theta$, agents with $\theta_i,\hat{\theta}_i$ both report truthfully
    \begin{equation}
        \begin{aligned}
            \theta_i \bar{v}_i(\theta_i)+\bar{t}_i(\theta_i)&\geq \theta_i \bar{v}_i(\hat{\theta}_i)+\bar{t}_i(\hat{\theta}_i)\\
            \hat{\theta}_i \bar{v}_i(\hat{\theta}_i)+\bar{t}_i(\hat{\theta}_i)&\geq \hat{\theta}_i \bar{v}_i(\theta_i)+\bar{t}_i(\theta_i)\\
            \Rightarrow [\theta_i-\hat{\theta}_i][\bar{v}_i(\theta_i)-\bar{v}_i(\hat{\theta}_i)]&\geq 0
        \end{aligned}
        \nonumber
    \end{equation}
    Hence, $\bar{v}_i(\theta_i)$ is non-decreasing.\\
    $U_i(\theta_i)=\max_{\hat{\theta}_i\in\Theta_i}\left[\theta_i\bar{v}_i(\hat{\theta}_i)+t_i(\hat{\theta}_i)\right]$ by BIC is maximized at $\hat{\theta}_i=\theta_i$.\\
    By Envelope Theorem,
    \begin{equation}
        \begin{aligned}
            U_i(\theta_i)&=U_i(\underline{\theta}_i)+\int_{\underline{\theta}_i}^{\theta_i}U'_i(s)ds\\
            &=U_i(\underline{\theta}_i)+\int_{\underline{\theta}_i}^{\theta_i}\bar{v}_i(s)ds
        \end{aligned}
        \nonumber
    \end{equation}
    ``$\Leftarrow$'': Take any $i$, $\theta_i,\hat{\theta}_i\in\Theta$. $i$'s expected interim payoff from reporting $\hat{\theta}_i$ instead of $\theta_i$ is
    \begin{equation}
        \begin{aligned}
            &\underbrace{\theta_i \bar{v}_i(\theta_i)+\bar{t}_i(\theta_i)}_{U_i(\theta_i)}-\left[\theta_i \bar{v}_i(\hat{\theta}_i)+\bar{t}_i(\hat{\theta}_i)\right]\\
            =&U_i(\theta_i)-\left[U_i(\hat{\theta}_i)+(\theta-\hat{\theta}_i)\bar{v}_i(\hat{\theta}_i)\right]\\
            =&U_i(\underline{\theta}_i)+\int_{\underline{\theta}_i}^{\theta_i}\bar{v}_i(s)ds-[U_i(\underline{\theta}_i)+\int_{\underline{\theta}_i}^{\hat{\theta}_i}\bar{v}_i(s)ds]-(\theta-\hat{\theta}_i)\bar{v}_i(\hat{\theta}_i)\\
            =&\int_{\theta_i}^{\hat{\theta}_i}(\bar{v}_i(\hat{\theta}_i)-\bar{v}_i(s))ds\geq 0
        \end{aligned}
        \nonumber
    \end{equation}
    So the direct mechanism is BIC.
\end{proof}

\section{Auction}
Based on
\begin{enumerate}[$\circ$]
    \item Klemperer, P. (1998). Auctions with almost common values: The Wallet Game'and its applications. \textit{European Economic Review}, 42(3-5), 757-769.
\end{enumerate}


\subsection{Examples: Auctions with Common-value}
\begin{enumerate}[(1).]
    \item Financial assets;
    \item Oilfields;
    \item A takeover target has a common value if the bidders are financial acquirers (e.g. LBO firms) who will follow similar management strategies if successful;
    \item The Personal Communications Spectrum (PCS) licenses sold by the U.S. Government in the 1995 "Airwaves Auction".
\end{enumerate}

\subsection{First / Second Price Sealed-bid Auction}
\begin{enumerate}[$\circ$]
    \item A seller sells an indivisible object.
    \item There are $N=\{1,...,n\}$ bidders, $i\in N$.
    \item Each bidder has a valuation for the object, $X_i\sim F$, $x_i\in[\underline{x},\overline{x}]$. p.d.f. $f(\cdot)$ is strictly positive and continuous.
    \item Strategy of $i$: $b_i:[\underline{x},\overline{x}] \rightarrow \mathbb{R}$, a \underline{bid function}.
\end{enumerate}

\begin{assumption}
    1. Independence; 2. Symmetry; 3. Private Values; 4. Risk-neutrality.
\end{assumption}

Let $X=(X_1,...,X_n)$. The $k^{th}$-order statistic, $X^k$, is the $k^{th}$ the highest value in $X_1,...,X_n$.

\begin{definition}[Second Price Auction]
    %\normalfont
    Highest bidder wins and pays the second-highest bid. (If more than one bidders bid the highest value, they win with equal probability.)\\
    It can be written as the form of Bayesian game: a bidder $i$'s utility function is
    \begin{equation}
        \begin{aligned}
            u_i(b_1,...,b_n;x_i)=\left\{\begin{matrix}
                \frac{1}{|\{j\in N:b_j=b_i\}|}x_i-b^2,&b_i=b^1\\
                0,&b_i\neq b^1
            \end{matrix}\right.
        \end{aligned}
        \nonumber
    \end{equation}
    where $b^k$ is the $k$-th highest bid.
\end{definition}

\begin{theorem}[Second Price Auction: Bid Truthfully]
    In the second-price sealed-bid auction, it is a (weakly) dominant strategy to bid your valuation, i.e., $\forall i\in N,\forall x_i\in[\underline{x},\overline{x}]$, $b_i(x_i)=x_i$.\\
    That is, $\forall i, \forall b'_i\in \mathbb{R}$,
    \begin{equation}
        \begin{aligned}
            u_i(x_i,b_{-i};x_i)\geq u_i(b'_i,b_{-i};x_i), \forall b_{-i}\in \mathbb{R}^{n-1}
        \end{aligned}
        \nonumber
    \end{equation}
    (Moreover, if $\exists b'_{i}\neq x_i$, then $\exists b_{-i}\in \mathbb{R}^{n-1}$ such that $u_i(x_i,b_{-i};x_i)> u_i(b'_{i},b_{-i};x_i)$.)
\end{theorem}
\begin{proof}
    Player $i$ has value $x_i$ and treats $b^1_{-i}$ as a random variable. The payoff conditional on winning is $$x_i-b^1_{-i}$$ By bidding $b_i=x_i$, $i$ ensures that $i$ wins if $b_i=x_i>b^1_{-i}\Leftrightarrow x_i-b^1_{-i}>0$ and $i$ loses if $b_i=x_i<b^1_{-i}\Leftrightarrow x_i-b^1_{-i}<0$.
\end{proof}

\begin{definition}[First Price Auction]
    %\normalfont
    Highest bidder wins and pays her bid. (If more than one bidder bid the highest value, they win with equal probability.)\\
    It can be written as the form of Bayesian game: a bidder $i$'s utility function is
    \begin{equation}
        \begin{aligned}
            u_i(b_1,...,b_n;x_i)=\left\{\begin{matrix}
                \frac{1}{|\{j\in N:b_j=b_i\}|}x_i-b_i,&b_i=b^1\\
                0,&b_i\neq b^1
            \end{matrix}\right.
        \end{aligned}
        \nonumber
    \end{equation}
    where $b^k$ is the $k$-th highest bid.
\end{definition}


\subsubsection*{Bayesian Nash Equilibrium Analysis of First Price Auction}
Conjecture that $\exists$ a BNE with the following properties:
\begin{enumerate}
    \item Symmetry: $b_1(\cdot)=b_2(\cdot)=\cdots=b_n(\cdot):=b(\cdot)$.
    \item $b(\cdot)$ is differentiable.
    \item $b'(\cdot)>0$.
\end{enumerate}
Take any bidder $i$ with valuation $x_i$. Assume $i$ knows $b(\cdot)$ and knows that the other bidder use the same $b(\cdot)$. Take any $b_i\in \mathbb{R}$ ($b_i:=b(x_i)$). (Not that, by the continuity of $X_i$, it is impossible to tie in this case.)

Then, $i$'s expected payoff from bidding $b_i$ is
\begin{equation}
    \begin{aligned}
        P(b(X_j)\leq b_i,\forall j\neq i)(x_i-b_i)=F^{n-1}(b^{-1}(b_i))(x_i-b_i)
    \end{aligned}
    \nonumber
\end{equation}
The necessary F.O.C. gives that optimal $b_i$ satisfies
\begin{equation}
    \begin{aligned}
        (n-1)f(b^{-1}(b_i))\frac{1}{b'(b^{-1}(b_i))}F^{n-2}(b^{-1}(b_i))(x_i-b_i)-F^{n-1}(b^{-1}(b_i))=0
    \end{aligned}
    \nonumber
\end{equation}
Since $b(\cdot)$ is a symmetric BNE, the optimal $b_i$ must be $b(x_i)$, then $b^{-1}(b_i)=x_i$.
\begin{equation}
    \begin{aligned}
        (n-1)f(x_i)\frac{1}{b'(x_i)}F^{n-2}(x_i)(x_i-b(x_i))-F^{n-1}(x_i)=0
    \end{aligned}
    \nonumber
\end{equation}
Hence,
\begin{equation}
    \begin{aligned}
        \underbrace{(n-1)f(x_i)F^{n-2}(x_i)x_i+F^{n-1}(x_i)}_{\frac{\partial F^{n-1}(x_i)x_i}{\partial x_i}}-F^{n-1}(x_i)=\underbrace{(n-1)f(x_i)F^{n-2}(x_i)b(x_i)+b'(x_i)F^{n-1}(x_i)}_{\frac{\partial F^{n-1}(x_i)b(x_i)}{\partial x_i}}
    \end{aligned}
    \nonumber
\end{equation}
Taking integral at both sides in $[\underline{x},x]$,
\begin{equation}
    \begin{aligned}
        F^{n-1}(x)x-\int_{\underline{x}}^x F^{n-1}(t) dt= F^{n-1}(x)b(x)
    \end{aligned}
    \nonumber
\end{equation}
That is,
\begin{equation}
    \begin{aligned}
        b(x)=x-\frac{1}{F^{n-1}(x)}\int_{\underline{x}}^x F^{n-1}(t) dt
    \end{aligned}
    \label{FPA:symmetric BNE}
\end{equation}
Note $b(\cdot)$ is differentiable and $b'(\cdot)>0$. We can extend $b(\cdot)$ to $[\underline{x},\overline{x}]$ by setting $b(\underline{x})=\lim_{x \rightarrow \underline{x}} b(x)=\underline{x}$.

\begin{proposition}[Symmetric BNE of First Price Auction]
    $b(x)=x-\frac{1}{F^{n-1}(x)}\int_{\underline{x}}^x F^{n-1}(t) dt$ is a symmetric BNE of First Price Auction.
\end{proposition}
\begin{proof}
    Any bid higher than $b(\overline{x})$ is suboptimal, and any bid lower than $b(\underline{x})$ is indifferent to the $b(\underline{x})$.\\
    We prove that, for a bidder with $x_i$, she prefers to bid $b(x_i)$ than $b(y),\forall y$.\\
    Bidding $b(y)$ gives expected payoff
    \begin{equation}
        \begin{aligned}
            F^{n-1}(y)(x_i-b(y))&=F^{n-1}(y)(y-b(y))+F^{n-1}(y)(x_i-y)\\
            (\textnormal{by \eqref{FPA:symmetric BNE}})&=\int_{\underline{x}}^y F^{n-1}(t) dt+F^{n-1}(y)(x_i-y)\\
            &=\int_{\underline{x}}^{x_i} F^{n-1}(t) dt-\underbrace{\int_y^{x_i} [F^{n-1}(t)-F^{n-1}(y)] dt}_{\geq 0, \textnormal{ minimized at }y=x_i}
        \end{aligned}
        \nonumber
    \end{equation}
\end{proof}

\begin{theorem}[Lebrun, 1999]
    Consider the bid function $b(\cdot)$ in \eqref{FPA:symmetric BNE}, the First Price Auction has essentially unique. (Bidders with types $x>\underline{x}$ bid $b(x)$ and bid for type $x=\underline{x}$ is not pinned down any further than the support $b_i(\underline{x})$ must lie in $(-\infty,b(\underline{x})]$.)
\end{theorem}

The equilibrium expected payoffs of a bidder in first-price auction and second price auction with valuation $x$ is
\begin{equation}
    \begin{aligned}
        \int_{\underline{x}}^x F^{n-1}(t)dt
    \end{aligned}
    \nonumber
\end{equation}
The equilibrium expected revenue of the seller in first-price auction and second price auction is
\begin{equation}
    \begin{aligned}
        \overline{x}-\int_{\underline{x}}^{\overline{x}} [nF^{n-1}(t)-(n-1)F^n(t)]dt
    \end{aligned}
    \nonumber
\end{equation}




\section{Revenue Equivalence Theorem and Optimal Auctions}
Consider the Optimal Auctions in an Independent Private Values Setting.
\begin{assumption}\label{IPV:assumption}
    There is one object and $N$ bidders.
    \begin{enumerate}
        \item Bidders are risk-neutral;
        \item Bidders have private valuations;
        \item each bidder $i$'s valuation independently drawn from a strictly increasing c.d.f. $F_i(\theta_i)$ (with p.d.f. $f_i(\theta_i),\theta_i\in \Theta_i$) that is continuous and bounded below;
        \item Seller knows each $F_i$ (use $F$ and $f$ to represent all distributions) and have no value for the object.
    \end{enumerate}
\end{assumption}

\begin{comment}
\begin{definition}[General Auction]
    %\normalfont
    A general auction mechanism: bidders have values $x$ and \textbf{strategies} $\beta: \mathcal{X}\triangleq \prod_i^N \mathcal{X}_i \rightarrow \mathcal{B}$ generate message (bids) based on their values, then there is an \textbf{allocation rule} based on bids $\pi: \mathcal{B} \rightarrow \Delta N$ generates a distribution over all bidders and a \textbf{payment rule} $\mu: \mathcal{B} \rightarrow \mathbb{R}^N$ generates payment for all bidders.
\end{definition}

\begin{definition}[Direct Mechansim]
    %\normalfont
    Consider a situation that bidders follow \textit{revelation principle} that provide their true values. Then the outcome can directly base on the true values.\\
    Then a \textbf{direct mechanism} can be represented as $(Q,T)$, where $Q: \mathcal{X} \rightarrow \Delta N$ is the allocation rule and $T: \mathcal{X} \rightarrow \mathbb{R}^N$ is the payment rule, such that
    \begin{equation}
        \begin{aligned}
            Q(x)=\pi(\beta(x)),\quad T(x)=\mu(\beta(x))
        \end{aligned}
        \nonumber
    \end{equation}
\end{definition}

\begin{proposition}[Revelation Principle]
    Take any equilibrium of any auction mechanism $(\mathcal{B},\pi,\mu)$. There is a distinct direct mechanism $(Q,T)$ that produces the same outcome.
\end{proposition}


Consider a direct mechanism $(Q,T)$, an agent $i$ reports $v_i$ while others report their values.
\begin{equation}
    \begin{aligned}
        \textbf{Expected allocation: }&q_i(z_i)=\int_{\mathcal{X}_{-i}}Q_i(z_i,x_{-i})dF_{-i}(x_{-i})\\
        \textbf{Expected payment: }&t_i(z_i)=\int_{\mathcal{X}_{-i}}T_i(z_i,x_{-i})dF_{-i}(x_{-i})
    \end{aligned}
    \nonumber
\end{equation}
where $Q_i, T_i$ are $i^\textnormal{th}$ item of $Q,T$.\\
The bidder wants to maximize
\begin{equation}
    \begin{aligned}
        q_i(z_i) x_i - t_i(z_i)
    \end{aligned}
    \nonumber
\end{equation}
Define the maximum value is
\begin{equation}
    \begin{aligned}
        u_i(x_i)=\max_{z_i\in \mathcal{X}_i}\{q_i(z_i) x_i - t_i(z_i)\}
    \end{aligned}
    \nonumber
\end{equation}

\begin{assumption}
    The condition for direct mechanism being incentive competitive (IC) is:
    \begin{equation}
        \begin{aligned}
            u_i(x_i)\equiv q_i(x_i) x_i - t_i(x_i)\geq q_i(z_i) x_i - t_i(z_i), \forall x_i, z_i\in \mathcal{X}_i
        \end{aligned}
        \tag{Ass 1}
        \label{Ass 1}
    \end{equation}
\end{assumption}


Firstly, we can compute, for any $z_i\in \mathcal{X}_i$
\begin{equation}
    \begin{aligned}
        &q_i(x_i)z_i-t_i(x_i)\\
        =&q_i(x_i)x_i-t_i(x_i)+q_i(x_i)(z_i-x_i)\\
        =&u_i(x_i)+q_i(x_i)(z_i-x_i)
    \end{aligned}
    \nonumber
\end{equation}
Based on the assumption \ref{Ass 1}, we have
\begin{equation}
    \begin{aligned}
        u_i(z_i)\geq q_i(x_i)z_i-t_i(x_i)=u_i(x_i)+q_i(x_i)(z_i-x_i)
    \end{aligned}
    \nonumber
\end{equation}
which shows that $u_i(\cdot)$ is convex.

If $u_i$ is differentiable, $u'_i(x_i)=q_i(x_i)$. Then, we can write the \textbf{Envelope theorem/condition}:
\begin{equation}
    \begin{aligned}
        u_i(x_i)=u_i(0)+\int_0^{x_i} q_i(y_i)dy_i
    \end{aligned}
    \nonumber
\end{equation}
which only depends on the allocation rule.

\begin{theorem}[Revenue Equivalence Theorem]
    If the direct mechanism $(Q,T)$ is incentive competitive (IC), then for all $i,x$, the \textbf{expected payment} is
    \begin{equation}
        \begin{aligned}
            t_i(x_i)=\underbrace{t_i(0)}_{e.g.=0}+q_i(x_i)x_i-\int_0^{x_i} q_i(y_i)dy_i
        \end{aligned}
        \nonumber
    \end{equation}
\end{theorem}
\begin{proof}
    \begin{equation}
        \begin{aligned}
            u_i(x_i)= q_i(x_i) x_i - t_i(x_i)=u_i(0)+\int_0^{x_i} q_i(y_i)dy_i\\
            \Rightarrow t_i(x_i)=q_i(x_i)x_i-u_i(0)-\int_0^{x_i} q_i(y_i)dy_i
        \end{aligned}
        \nonumber
    \end{equation}
    Set $u_i(0)=-t_i(0)$, that is, if $i$'s value is zero, he pays zero.
\end{proof}

\begin{corollary}[Standard Revenue Equivalence Theorem]
    Suppose that values are \underline{i.i.d.} and bidders are \underline{risk-neutral}.\\
    Consider any auction and its \underline{symmetric} and \underline{increasing} equilibrium, in which the expected payment of bidders have $0$ value is $0$. Then the expected revenue to the seller is the \underline{same}.
\end{corollary}
\begin{proof}
    If equilibrium, is symmetric and increasing, then object is \underline{always} allocated to the bidder with the highest value. Set $t_i(0)=0$.
\end{proof}

Standard Revenue Equivalence Theorem is based on \underline{symmetric}, \underline{independent}, and \underline{private} (uncorrelated) values.
\end{comment}

















\underline{Goal:} Find the \textbf{optimal auction} that maximizes the seller's expected revenue subject to individual rationality (IR) and Bayesian incentive compatibility for the buyers.


\begin{note}
    Note that, in symmetric BNE, the bidders' strategies are same $b(\cdot)$, but the bidders' values can draw from different and independent distributions $\{F_i\}_{i=1}^I$
\end{note}


\subsection{IR, BIC, Direct Mechanism}
Given a direct mechanism $\left(\Theta,y(\cdot),t(\cdot)\right)$, define
\begin{equation}
    \begin{aligned}
        &\textnormal{Expected Allocation: } \bar{y}_i(\theta_i)=\mathbb{E}_{\theta_{-i}}\left[y_i(\theta_i,\theta_{-i})\right]\\
        &\textnormal{Expected Payment: } \bar{t}_i(\theta_i)=\mathbb{E}_{\theta_{-i}}\left[t_i(\theta_i,\theta_{-i})\right]
    \end{aligned}
    \nonumber
\end{equation}
The expected utility of agent $i$ when all agents report truthfully,
\begin{equation}
    \begin{aligned}
        U_i(\theta_i)&=\mathbb{E}_{\theta_{-i}}\left[\theta_iy_i(\theta_i,\theta_{-i})+t_i(\theta_i,\theta_{-i})\right]\\
        &=\theta_i \bar{y}_i(\theta_i)+\bar{t}_i(\theta_i)
    \end{aligned}
    \nonumber
\end{equation}

\begin{definition}[Individual Rationality]
    %\normalfont
    A direct mechanism $\left(\Theta,y(\cdot),t(\cdot)\right)$ is \textbf{individual-rational (IR)} if $\forall i,\theta_i\in\Theta_i$, $U_i(\theta_i)\geq 0$.
\end{definition}

\begin{corollary}[Corollary of Proposition \ref{Linear utility model_BIC}]
    A direct mechanism $\left(\Theta,y(\cdot),t(\cdot)\right)$ is \textbf{BIC and IR} iff $\forall i\in\{1,...,I\}$
    \begin{enumerate}[(1).]
        \item $\bar{y}_i(\theta_i)$ is non-decreasing in $\theta_i$.
        \item $\forall \theta_i\in\Theta_i$, $U_i(\theta_i)=U_i(\underline{\theta}_i)+\int_{\underline{\theta}_i}^{\theta_i}\bar{y}_i(s)ds$
        \item $U_i(\underline{\theta}_i)\geq 0$
    \end{enumerate}
\end{corollary}
For a BIC $\&$ IR mechanism, \textit{the expected auction payment} $\bar{t}_i(\theta_i)$ can be represented as
\begin{equation}
    \begin{aligned}
        U_i(\theta_i)&=U_i(\underline{\theta}_i)+\int_{\underline{\theta}_i}^{\theta_i}\bar{y}_i(s)ds\\
        \theta_i \bar{y}_i(\theta_i)+\bar{t}_i(\theta_i)&=U_i(\underline{\theta}_i)+\int_{\underline{\theta}_i}^{\theta_i}\bar{y}_i(s)ds
    \end{aligned}
    \nonumber
\end{equation}
\begin{equation}
    \begin{aligned}
        \bar{t}_i(\theta_i)&=-\theta_i \bar{y}_i(\theta_i)+U_i(\underline{\theta}_i)+\int_{\underline{\theta}_i}^{\theta_i}\bar{y}_i(s)ds
    \end{aligned}
    \label{t_star}
\end{equation}

\subsection{Revenue Equivalence Theorem}
For a BIC $\&$ IR direct mechanism, the \textit{seller's expected revenues} from bidder $i$:
\begin{equation}
    \begin{aligned}
        \mathbb{E}_\theta[-t_i(\theta)]&=-\int_\Theta t_i(\theta)f(\theta) d\theta\\
        &=-\int_{\Theta_i} \underbrace{\left(\int_{\Theta_{-i}}t_i(\theta_i,\theta_{-i})f_{-i}(\theta_{-i})d\theta_{-i}\right)}_{\bar{t}_i(\theta_i)}f_i(\theta_i) d\theta_i\\
        &=-\int_{\underline{\theta}_i}^{\overline{\theta}_i}\left(-\theta_i \bar{y}_i(\theta_i)+U_i(\underline{\theta}_i)+\int_{\underline{\theta}_i}^{\theta_i}\bar{y}_i(s)ds\right)f_i(\theta_i) d\theta_i\\
        &=-\underbrace{\int_{\underline{\theta}_i}^{\overline{\theta}_i}\int_{\underline{\theta}_i}^{\theta_i}\bar{y}_i(s)dsf_i(\theta_i) d\theta_i}_{\triangleq \star}+\int_{\underline{\theta}_i}^{\overline{\theta}_i}\theta_i \bar{y}_i(\theta_i)f_i(\theta_i) d\theta_i-U_i(\underline{\theta}_i)
    \end{aligned}
    \nonumber
\end{equation}
applying integration by parts
\begin{equation}
    \begin{aligned}
        \star&=\int_{\underline{\theta}_i}^{\overline{\theta}_i}\left(\int_{\underline{\theta}_i}^{\theta_i}\bar{y}_i(s)ds\right)d F_i(\theta_i)\\
        &=\left(\int_{\underline{\theta}_i}^{\theta_i}\bar{y}_i(s)ds\right)F_i(\theta_i)\bigg|_{\underline{\theta}_i}^{\overline{\theta}_i}-\int_{\underline{\theta}_i}^{\overline{\theta}_i}F_i(\theta_i)d\left(\int_{\underline{\theta}_i}^{\theta_i}\bar{y}_i(s)ds\right)\\
        &=\int_{\underline{\theta}_i}^{\overline{\theta}_i}\bar{y}_i(s)ds-\int_{\underline{\theta}_i}^{\overline{\theta}_i}\bar{y}_i(\theta_i)F_i(\theta_i)d\theta_i\\
        &=\int_{\underline{\theta}_i}^{\overline{\theta}_i}(1-F_i(s))\bar{y}_i(s) ds
    \end{aligned}
    \nonumber
\end{equation}
Hence,
\begin{equation}
    \begin{aligned}
        \mathbb{E}_\theta[-t_i(\theta)]&=-\int_{\underline{\theta}_i}^{\overline{\theta}_i}(1-F_i(\theta_i))\bar{y}_i(\theta_i) d\theta_i+\int_{\underline{\theta}_i}^{\overline{\theta}_i}\theta_i \bar{y}_i(\theta_i)f_i(\theta_i) d\theta_i-U_i(\underline{\theta}_i)\\
        &=\int_\Theta y_i(\theta)\left[\theta_i-\frac{1-F_i(\theta_i)}{f_i(\theta_i)}\right]f(\theta)d\theta-U_i(\underline{\theta}_i)
    \end{aligned}
    \nonumber
\end{equation}
The \textit{total expected revenue of the seller} is
\begin{equation}
    \begin{aligned}
        \sum_{i=1}^I\mathbb{E}_\theta[-t_i(\theta)]=\int_\Theta \sum_{i=1}^I y_i(\theta)\left[\theta_i-\frac{1-F_i(\theta_i)}{f_i(\theta_i)}\right]f(\theta)d\theta-\sum_{i=1}^IU_i(\underline{\theta}_i)
    \end{aligned}
    \label{eq:revenue}
\end{equation}

\begin{theorem}[Revenue Equivalence Theorem]
    In the setting of Assumption \ref{IPV:assumption}, BIC $\&$ IR direct mechanisms, with the same allocation rule $y(\cdot)$ and the same interim utilities of the lowest types $(U_i(\underline{\theta}_i))_{i=1,...,I}$, generate the same revenues \eqref{eq:revenue} and the same expected payments of all types bidders \eqref{t_star}.
\end{theorem}
\begin{corollary}[Revenue Equivalence Theorem (indirect form)]
    Any two auction formats $A$ and $B$, fix a BNE of $A$ and a BNE of $B$ such that
    \begin{enumerate}
        \item For every $\theta\in\Theta$, these two BNEs allocate the object with same probabilities;
        \item Interim expected payoff of $\underline{\theta}_i$ is the same for both BNEs.
    \end{enumerate}
    Then, $A$ and $B$ generate the same expected revenues \eqref{eq:revenue} and the same expected payments of all types bidders \eqref{t_star}.
\end{corollary}
\begin{proof}
    Based on the Revelation Principle \ref{theorem:revelation principle BIC}, direct auction mechanisms induced by A and B are BIC and IR. By Revenue Equivalence Theorem, the corollary is proved.
\end{proof}

\subsection{Optimal Auctions}
The optimal auction design is given by
\begin{equation}
    \begin{aligned}
        \max_{\textnormal{BIC and IR }y(\cdot)}\int_\Theta \sum_{i=1}^I y_i(\theta)\left[\theta_i-\frac{1-F_i(\theta_i)}{f_i(\theta_i)}\right]f(\theta)d\theta-\sum_{i=1}^IU_i(\underline{\theta}_i)
    \end{aligned}
    \nonumber
\end{equation}

\begin{definition}[Virtual Valuation]
    %\normalfont
    Define bidder $i$'s \textbf{virtual valuation} is $c_i(v_i)=v_i-\frac{1-F_i(v_i)}{f(v_i)}$.
\end{definition}

\begin{assumption}[Regularity Condition]
    Any bidder $i$'s virtual valuation $c_i(v_i)=v_i-\frac{1-F_i(v_i)}{f(v_i)}$ is strictly increasing.
\end{assumption}

\begin{corollary}[Optimal Auction Mechanism]
    Assume regularity. Then the expected revenue maximizing direct auction mechanism $(y(\cdot),t(\cdot))$ can be described as follows
    \begin{enumerate}[(1).]
        \item $y(\cdot): \Theta \rightarrow K$ is defined as follows. For any $\theta\in \Theta$, $\max_{i\in\{1,...,I\}}c_i(\theta_i)<0$, the seller keeps the object ($y_i(\theta)=0,\forall  i$). Otherwise, the object is allocated to the highest virtual valuation bidder.
        \item Define $t(\cdot):\Theta \rightarrow K$,
        \begin{equation}
            \begin{aligned}
                t_i(\theta):=-\theta_i y_i(\theta_i,\theta_{-i})+U_i(\underline{\theta}_i)+\int_{\underline{\theta}_i}^{\theta_i}y_i(s,\theta_{-i})ds
            \end{aligned}
            \nonumber
        \end{equation}
        which satisfies \eqref{t_star}.
    \end{enumerate}
\end{corollary}

\begin{example}
    Suppose $\Theta_i=[0,1],\theta_i\sim U[0,1]$. $c_i(\theta_i)=\theta_i-\frac{1-\theta_i}{1}=2\theta_i-1$, which is strictly increasing in $\theta_i$ (regularity satisfied). Then, the optimal auction mechanism is allocating the object to the highest (virtual) valuation bidder (iff his value $\theta_i\geq\frac{1}{2}$).
\end{example}

\begin{definition}[Bidder-Specific Reserve Price]
    %\normalfont
    Bidder $i$'s bidder-specific reserve price $r_i^*$ is the value for which $c_i(r_i^*)=0$.
\end{definition}

\begin{theorem}[Myerson (1981)]
    The optimal (single-good) auction in terms of a direct mechanism: The good is sold to the agent $i=\arg\max_i\phi_i(\hat{v}_i)$, as long as $v_i\geq r_i^*$. If the good is sold, the winning agent $i$ is charged the smallest valuation that he could have declared while still remaining the winner:
    \begin{equation}
        \begin{aligned}
            \inf\{v_i^*:c_i(v_i^*)\geq 0 \textnormal{ and }\forall j\neq i, c_i(v_i^*)\geq c_j(\hat{v}_j)\}
        \end{aligned}
        \nonumber
    \end{equation}
\end{theorem}



\section{Myerson-Satterthwaite Theorem}
The result says that there is no efficient way for two parties to trade a good when they each have secret and probabilistically varying valuations for it, without the risk of forcing one party to trade at a loss.

Suppose there is an indivisible object hold by a seller (agent 1), and there is a buyer (agent 2) who wants to buy the object. Agent $i$ has a valuation $\theta_i\in[\underline{\theta}_i,\overline{\theta_i}]$ for the object, which draws from the distribution $\Phi(\cdot)$ (with p.d.f. $\phi(\theta_i),\forall \theta_i\in [\underline{\theta}_i,\overline{\theta_i}]$.) To consider the nontrivial situation, we assume $\underline{\theta}_1<\overline{\theta}_2$ and $\underline{\theta}_2<\overline{\theta}_1$.


\begin{definition}[Double Auction]
    %\normalfont
    A \underline{direct \textbf{double auction} mechanism} is a pair $(y(\cdot),t(\cdot))$, where $y(\cdot):\Theta \rightarrow [0,1]$ and $t(\cdot): \Theta \rightarrow \mathbb{R}$. $y(\theta_1,\theta_2)$ denote the probability of trade at $(\theta_1,\theta_2)$ and $t(\theta_1,\theta_2)$ is the buyer's expected payment to the seller at $(\theta_1,\theta_2)$.
\end{definition}

Fix a direct mechanism $(y(\cdot),t(\cdot))$. For an agent with $\theta_i$, the interim expected probability of trade is
\begin{equation}
    \begin{aligned}
        \bar{y}_i(\theta_i)=\mathbb{E}_{\theta_{-i}}\left[y(\theta_i,\theta_{-i})\right]
    \end{aligned}
    \nonumber
\end{equation}
and the interim expected payment is
\begin{equation}
    \begin{aligned}
        \bar{t}_i(\theta_i)=\mathbb{E}_{\theta_{-i}}\left[t(\theta_i,\theta_{-i})\right]
    \end{aligned}
    \nonumber
\end{equation}

Define the interim expected payoffs when both agent report truthfully
\begin{equation}
    \begin{aligned}
        U_1(\theta_1)&=-\theta_1\bar{y}_1(\theta_1)+\bar{t}_1(\theta_1)\\
        U_2(\theta_2)&=\theta_2\bar{y}_2(\theta_2)-\bar{t}_2(\theta_2)
    \end{aligned}
    \nonumber
\end{equation}
Note that the payoff of seller is actually $U_1(\theta_1)+\theta_1$, here we only consider the difference of payoff induced by a trade so we can only focus on $U_1(\theta_1)$ now.
\begin{enumerate}
    \item \underline{BIC requires}
    \begin{equation}
        \begin{aligned}
            U_1(\theta_1)&\geq-\theta_1\bar{y}_1(\hat{\theta}_1)+\bar{t}_1(\hat{\theta}_1),\quad &\forall \theta_1,\hat{\theta}_1\in\Theta_1\\
            U_2(\theta_2)&\geq\theta_2\bar{y}_2(\hat{\theta}_2)-\bar{t}_2(\hat{\theta}_2),\quad &\forall \theta_2,\hat{\theta}_2\in\Theta_2
        \end{aligned}
        \nonumber
    \end{equation}
    \item \underline{IR requires}
    \begin{equation}
        \begin{aligned}
            U_i(\theta_i)\geq 0,\quad \forall i\in\{1,2\},\theta_i\in\Theta_i
        \end{aligned}
        \nonumber
    \end{equation}
    \item \underline{Ex-post efficient requires}
    \begin{equation}
        \begin{aligned}
            y(\theta)=\left\{\begin{matrix}
                1,& \textnormal{ if }\theta_2>\theta_1\\
                0,& \textnormal{ if }\theta_1>\theta_2
            \end{matrix}\right.
        \end{aligned}
        \nonumber
    \end{equation}
    \item \underline{Budget-balanced (BB)} is automatically satisfied.
\end{enumerate}

\begin{lemma}
    Assume $(y(\cdot),t(\cdot))$ is BIC, then:
    \begin{enumerate}
        \item \underline{Buyer:} $\bar{y}_2(\cdot)$ is non-decreasing in $\theta_2$.
        \item \underline{Buyer:} $\forall \theta_2\in\Theta_2$, $U_2(\theta_2)=U_2(\underline{\theta}_2)+\int_{\underline{\theta}_2}^{\theta_2}\bar{y}_2(s)ds$.
        \item \underline{Seller:} $\bar{y}_1(\cdot)$ is non-increasing in $\theta_1$.
        \item \underline{Seller:} $\forall \theta_1\in\Theta_1$, $U_1(\theta_1)=U_1(\overline{\theta}_1)+\int_{\theta_1}^{\overline{\theta}_1}\bar{y}_1(s)ds$.
        \item
        \begin{equation}
            \begin{aligned}
                U_1(\overline{\theta}_i)+U_2(\underline{\theta}_2)=\int_\Theta \left(\left[\theta_2-\frac{1-\Phi_2(\theta_2)}{\phi_2(\theta_2)}\right]-\left[\theta_1+\frac{\Phi_1(\theta_1)}{\phi_1(\theta_1)}\right]\right)y(\theta)\phi(\theta)d\theta
            \end{aligned}
            \nonumber
        \end{equation}
    \end{enumerate}
\end{lemma}
The proof of 1-4 of the lemma is similar to the proof of Proposition \ref{Linear utility model_BIC} and the proof of 5 follows the similar step as derivation of expected revenues of the seller in optimal auctions \eqref{eq:revenue}.

\begin{theorem}[Myerson-Satterthwaite Theorem]
    $\nexists$ a direct mechanism satisfying BB (which must be satisfied), BIC, IR, and ex-post efficient.
\end{theorem}
\begin{proof}
    Suppose by the way of contradiction that there exists a direct $(y(\cdot),t(\cdot))$ satisfying BB, BIC, IR, and ex-post efficient.\\
    By the requirement of IR and above lemma,
    \begin{equation}
        \begin{aligned}
            0&\leq \int_{\underline{\theta}_2}^{\overline{\theta}_2}\int_{\underline{\theta}_1}^{\overline{\theta}_1} \left(\left[\theta_2-\frac{1-\Phi_2(\theta_2)}{\phi_2(\theta_2)}\right]-\left[\theta_1+\frac{\Phi_1(\theta_1)}{\phi_1(\theta_1)}\right]\right)y(\theta_1,\theta_2)\phi_1(\theta_1)\phi_2(\theta_2)d\theta_1d\theta_2
        \end{aligned}
        \nonumber
    \end{equation}
    By the requirement of ex-post efficient that $y(\theta)=\left\{\begin{matrix}
        1,& \textnormal{ if }\theta_2>\theta_1\\
        0,& \textnormal{ if }\theta_1>\theta_2
    \end{matrix}\right.$,
    \begin{equation}
        \begin{aligned}
            0&\leq \int_{\underline{\theta}_2}^{\overline{\theta}_2}\int_{\underline{\theta}_1}^{\min\{\overline{\theta}_1,\theta_2\}} \left(\left[\theta_2-\frac{1-\Phi_2(\theta_2)}{\phi_2(\theta_2)}\right]-\left[\theta_1+\frac{\Phi_1(\theta_1)}{\phi_1(\theta_1)}\right]\right)\phi_1(\theta_1)\phi_2(\theta_2)d\theta_1d\theta_2\\
            &=\int_{\underline{\theta}_2}^{\overline{\theta}_2}\int_{\underline{\theta}_1}^{\min\{\overline{\theta}_1,\theta_2\}} \left(\theta_2\phi_2(\theta_2)-1+\Phi_2(\theta_2)\right)\phi_1(\theta_1)d\theta_1d\theta_2\\
            &\quad -\int_{\underline{\theta}_2}^{\overline{\theta}_2}\int_{\underline{\theta}_1}^{\min\{\overline{\theta}_1,\theta_2\}}\underbrace{\left(\theta_1\phi_1(\theta_1)+\Phi_1(\theta_1)\right)}_{\frac{\partial \theta_1\Phi_1(\theta_1)}{\partial \theta_1}}\phi_2(\theta_2)d\theta_1d\theta_2\\
            &\textnormal{(Note $\Phi_1(\min\{\overline{\theta}_1,\theta_2\})=\Phi_1(\theta_2)$)}\\
            &=\int_{\underline{\theta}_2}^{\overline{\theta}_2}\left(\theta_2\phi_2(\theta_2)-1+\Phi_2(\theta_2)\right)\Phi_1(\theta_2)d\theta_2
            -\int_{\underline{\theta}_2}^{\overline{\theta}_2}\min\{\overline{\theta}_1,\theta_2\}\Phi_1(\theta_2)\phi_2(\theta_2)d\theta_2\\
            &=\int_{\underline{\theta}_2}^{\overline{\theta}_1}\left(-1+\Phi_2(\theta_2)\right)\Phi_1(\theta_2)d\theta_2+\mathbf{1}\{\overline{\theta}_1<\overline{\theta}_2\}\int_{\overline{\theta}_1}^{\overline{\theta}_2}\left(\theta_2\phi_2(\theta_2)-1+\Phi_2(\theta_2)-\overline{\theta}_1\phi_2(\theta_2)\right)d\theta_2\\
            &=\int_{\underline{\theta}_2}^{\overline{\theta}_1}\left(-1+\Phi_2(\theta_2)\right)\Phi_1(\theta_2)d\theta_2>0
        \end{aligned}
        \nonumber
    \end{equation}
    So, there is a contradiction.
\end{proof}



\section{Equilibrium in Auctions with Entry}
\begin{enumerate}[$\circ$]
    \item Levin, D., \& Smith, J. L. (1994). Equilibrium in auctions with entry. The American Economic Review, 585-599.
\end{enumerate}
A single item offered to a group of $N$ potential bidders. There are two stage: each potential  entrant decides whether to enter by a fixed cost $c$, then an auction is conducted among $n$ participants (the number of bidders who enter).
\begin{assumption}
    \begin{enumerate}
        \item The seller and all potential bidders are risk-neutral.
        \item We assume the seller's valuation is $0$ and the possible value's range being $[0,\bar{v}]$.
        \item The domain of possible values for the item ($V$) and the domain of estimates ($x$) are compact: $V\in[0,\bar{v}-\underline{v}]$ and $x\in [0,\bar{x}]$.
        \item Information is symmetric and bidders randomly draw values from the same distribution.
        \item The auction mechanism ($m$) and the number of potential bidders ($N$) are common knowledge, and the number of actual bidders is revealed prior to stage 2.
        \item A unique symmetric Nash equilibrium exists and individual behavior conforms to the symmetric Nash equilibrium.
    \end{enumerate}
\end{assumption}

Given the number of bidders $n$, cost $c$, and mechanism $m$, the \textit{ex-ante} expected gain from entering and bidding according to the symmetric Nash equilibrium of each potential entrant is denoted by $\mathbb{E}[\pi\mid n,m]$.

If $\mathbb{E}[\pi\mid n,m]$ is decreasing in $n$, $\exists$ an unique integer $n^*$ such that $\mathbb{E}[\pi\mid n^*,m]\geq 0>\mathbb{E}[\pi\mid n^*+1,m]$. We focus on the case that $n^*\in (0,N)$.

A symmetric entry equilibrium must yield the same probability of entry for all potential bidders, which is denoted by $q^*\in (0,1)$ and each potential entrant must be indifferent between entering or not:
\begin{equation}
    \begin{aligned}
        \sum_{n=1}^N\left[
            \begin{pmatrix}
            N-1\\
            n-1
        \end{pmatrix}
        (q^*)^{n-1}(1-q^*)^{N-n}\mathbb{E}[\pi\mid n,m]
        \right]=0
    \end{aligned}
    \label{EAE_1}
\end{equation}
where $\begin{pmatrix}N-1\\n-1\end{pmatrix}(q^*)^{n-1}(1-q^*)^{N-n}$ is the probability that exactly $n-1$ rivals also enter. The number of bidders has mean $\bar{n}:=q^*N$ and variance $\bar{n}(1-q^*)$.

We focus on the mechanism that a bidder wins and pays for the item only if his bid is the highest.
\begin{note}
    Different to the original paper, we focus on the ``free entry'' case (i.e., the mechanism can be denoted by the reserve prices $R=\{R_1,...,R_N\}$, where $R_n$ means the reserve price enforced by the seller if $n$ bidders enter).
\end{note}
We let $T$ represent the event that trade occurs (i.e., the highest bid is greater than the reserve price) and let $T_n(R_n):=\textnormal{Pr}[V_{(n)}\geq R_n]$ represent the probability of trade given $n$ bidders enter and the seller's mechanism $R$.

Using symmetry, a bidder's \textit{ex-ante} expected profit, conditional on entering an auction with $n$ bidders, can be written as $\frac{V_n-W_n}{n}-c$, where $V_n:=\mathbb{E}[V_{(n)}\mid V_{(n)}\geq R_n]$ is the expected value of the item to the highest bidder and $W_n$ is the expected payment this bidder makes to the seller, both conditional on trade occurring under the given mechanism.

We use $\Omega$ to denote $\{R,c,N\}$ and $B_i(q,\Omega)$ to denote the $i^{th}$ bidder's expected profit from entering when all $N-1$ rivals are using arbitrary entry probability $q$:
\begin{equation}
    \begin{aligned}
        B_i(q,\Omega)&=\sum_{n=1}^N\left[
            \begin{pmatrix}
            N-1\\
            n-1
        \end{pmatrix}
        (q)^{n-1}(1-q)^{N-n}T_n(R_n)\frac{V_n-W_n}{n}
        \right]-c\\
        &=\frac{1}{Nq}\sum_{n=1}^N\left[
            \begin{pmatrix}
            N\\
            n
        \end{pmatrix}
        (q)^{n}(1-q)^{N-n}T_n(R_n)\left(V_n-W_n\right)
        \right]-c\\
    \end{aligned}
    \nonumber
\end{equation}
If the $i^{th}$ bidder also elects to enter with probability $q$, the expected profit of all $N$ parties must be:
\begin{equation}
    \begin{aligned}
        B(q,\Omega)&=NqB_i(q,\Omega)\\
        &=\sum_{n=1}^N p_n T_n(R_n)(V_n-W_n)-\bar{n}c
    \end{aligned}
    \nonumber
\end{equation}
where $p_n:=\begin{pmatrix}N\\n\end{pmatrix}(q)^{n}(1-q)^{N-n}$ is the binomial probability that exactly $n$ bidders enter in total.

The seller's expected revenue is
\begin{equation}
    \begin{aligned}
        \Pi(q,\Omega)=\sum_{n=1}^N p_n T_n(R_n)W_n
    \end{aligned}
    \nonumber
\end{equation}
and the total social welfare is
\begin{equation}
    \begin{aligned}
        S(q,\Omega)=B(q,\Omega)+\Pi(q,\Omega)=\sum_{n=1}^N p_n T_n(R_n)V_n-\bar{n}c
    \end{aligned}
    \nonumber
\end{equation}

To make it is indifferent between entering and not, $q^*\in (0,1)$, we must have $B_i(q^*,\Omega)=0$, i.e.,
\begin{equation}
    \begin{aligned}
        \frac{1}{Nq^*}\sum_{n=1}^N\left[
            \begin{pmatrix}
            N\\
            n
        \end{pmatrix}
        (q^*)^{n}(1-q^*)^{N-n}T_n(R_n)\left(V_n-W_n\right)
        \right]-c=0
    \end{aligned}
    \label{EAE_2}
\end{equation}
Then, we can define the symmetric entry equilibrium given $\Omega$, $q^*=q(\Omega)$.
\begin{assumption}
    In this paper, we focus on the mechanism that the expected profit of an entrant is negatively correlated with the number of bidders when $R=0$, which is equivalent to $\frac{\partial q^*}{\partial c}<0$ and $\frac{\partial q^*}{\partial (V_i-W_i)}<0$ (Lemma 1 in original paper).
\end{assumption}

Hence, we can find that the total social welfare is equivalent to the expected revenue of the seller.
\begin{equation}
    \begin{aligned}
        &S(q^*,\Omega)=\sum_{n=1}^N p_n T_n(R_n)V_n-\bar{n}c\\
        =&\Pi(q^*,\Omega)=\sum_{n=1}^N p_n T_n(R_n)W_n
    \end{aligned}
    \nonumber
\end{equation}

\begin{proposition}[Revenue Equivalence Holds for Symmetric Entry]
    Any two mechanisms that are revenue-equivalent with fixed $n$ and $R$ remain revenue-equivalent with induced entry.
\end{proposition}

We can write $\tilde{W}_n:=T_n(R_n)W_n$ and $\tilde{V}_n:=T_n(R_n)V_n$. Hence, the \eqref{EAE_2} can be written as
\begin{equation}
    \begin{aligned}
        \frac{1}{Nq^*}\sum_{i=1}^n p_n \left(\tilde{V}_n-\tilde{W}_n\right)=c
    \end{aligned}
    \label{EAE_3}
\end{equation}

\begin{lemma}\label{Lemma:opt_q}
    In independent private value auctions, $\frac{\partial S}{\partial q}(q^*,\Omega)=0$.
\end{lemma}
\begin{proof}
    Given $S(q,\Omega)=\sum_{n=1}^N p_n \tilde{V}_n-\bar{n}c$,
    \begin{equation}
        \begin{aligned}
            \frac{\partial S(q,\Omega)}{\partial q}&=\sum_{n=1}^N \frac{\partial p_n}{\partial q} T_n(R_n)V_n-Nc\\
            &=\sum_{n=1}^N\begin{pmatrix}N\\n\end{pmatrix}\left[n q^{n-1}(1-q)^{N-n}-(N-n)q^n(1-q)^{N-n-1}\right]\tilde{V}_n-Nc\\
            &=\frac{\sum_{n=1}^N p_n \tilde{V}_n (n-qN)}{q(1-q)}-Nc
        \end{aligned}
        \label{EAE_4}
    \end{equation}
    Substituting \eqref{EAE_3}, we get
    \begin{equation}
        \begin{aligned}
            \frac{\partial S(q^*,\Omega)}{\partial q}&=\frac{\sum_{n=1}^N p_n \tilde{V}_n (n-q^*N)}{q^*(1-q^*)}-\frac{1}{q^*}\sum_{i=1}^n p_n \left(\tilde{V}_n-\tilde{W}_n\right)\\
            &=\frac{1}{q^*}\sum_{n=1}^N p_n \left[\frac{\tilde{V}_n (n-q^*N)}{1-q^*}-\left(\tilde{V}_n-\tilde{W}_n\right)\right]\\
        \end{aligned}
        \nonumber
    \end{equation}
    In independent private value auctions, the expected payment is given by
    \begin{equation}
        \begin{aligned}
            \tilde{W}_n&=\int_{0}^{\bar{v}}v n(n-1)f(v)(1-F(v))F(v)^{n-2}dv\\
            &=n\tilde{V}_{n-1}-(n-1)\tilde{V}_{n}
        \end{aligned}
        \nonumber
    \end{equation}
    \begin{equation}
        \begin{aligned}
            \tilde{V}_n-\tilde{W}_n=n\left(\tilde{V}_n-\tilde{V}_{n-1}\right)
        \end{aligned}
        \nonumber
    \end{equation}
    This is the same result as (Milgrom and Weber, 1982 theorem
    0), but we need to note that this result depends on the optimal reservation price being $0$ which is based on the assumption the value of the seller is zero. The intuition is: $\tilde{V}_n-\tilde{V}_{n-1}$ is the difference between the highest value among first $n$ values and the highest value among first $n-1$ values. There is only $\frac{1}{n}$ probability that the $n^{th}$ value is the highest value among the first $n$ values. However, $\tilde{V}_n-\tilde{W}_n$ always gives the difference between the highest value and the second-highest value among first $n$ values.

    Then,
    \begin{equation}
        \begin{aligned}
            \frac{\partial S(q^*,\Omega)}{\partial q}&=\frac{1}{q^*}\sum_{n=1}^N p_n \left[\frac{\tilde{V}_n (n-q^*N)}{1-q^*}-n\left(\tilde{V}_n-\tilde{V}_{n-1}\right)\right]\\
            &=\frac{1}{q^*}\sum_{n=1}^N p_n \left[n\tilde{V}_{n-1}-\frac{\tilde{V}_n q^*(N-n)}{1-q^*}\right]\\
            &=\frac{1}{q^*}\left[\sum_{n=1}^N p_nn\tilde{V}_{n-1}-\sum_{n=1}^{N-1} p_{n+1}(n+1)\tilde{V}_n\right]\\
            &=\frac{1}{q^*}p_1\tilde{V}_{0}=0
        \end{aligned}
        \nonumber
    \end{equation}
\end{proof}


\begin{proposition}[Optimal Reservation Price is the Seller's Value]
    Any mechanism that maximizes the seller's expected revenue also induces socially optimal entry. Such a mechanism has reservation price $R=0$.
\end{proposition}
\begin{proof}
    We want to maximize $\sum_{n=1}^N p_n T_n(R_n)V_n-\bar{n}c$, where $T_n(R_n)V_n$ can be written as
    \begin{equation}
        \begin{aligned}
            T_n(R_n)V_n&=\textnormal{Pr}[V_{(n)}\geq R_n]\mathbb{E}[V_{(n)}\mid V_{(n)}\geq R_n]\\
            &=\textnormal{Pr}[V_{(n)}\geq R_n]
            \int_{R_n}^{\bar{v}}x \frac{f_{V_{(n)}}(x)}{\textnormal{Pr}[V_{(n)}\geq R_n]}dx\\
            &=
            \int_{R_n}^{\bar{v}}x f_{V_{(n)}}(x) dx
        \end{aligned}
        \nonumber
    \end{equation}
    which is maximized at $R_n=0$. Hence, fixing $q$, $S(q,R)$ is maximized at $R=0$.

    Then, by the Lemma \ref{Lemma:opt_q} and the concavity of $S(q,R)$ on $q$, the social welfare is maximized at $R=0$.
\end{proof}
Then, the seller's expected revenue and the total social welfare can be written as
\begin{equation}
    \begin{aligned}
        S(q^*,\Omega)=\sum_{n=1}^N \underbrace{\begin{pmatrix}N\\n\end{pmatrix}(q^*)^{n}(1-q^*)^{N-n}}_{:=p_n} \underbrace{\int_{0}^{\bar{v}}x f_{V_{(n)}}(x) dx}_{:=\tilde{V}_n}-\bar{n}c
    \end{aligned}
    \nonumber
\end{equation}



\begin{proposition}[Effect of Market Thickness]
    In independent private value auctions, the level of social welfare generated by optimal auctions decreases monotonically as $N$ increases beyond $n^*$.
\end{proposition}
\begin{proof}
    Given $N>n^*$, the symmetric equilibrium gives $q^s=q^s(N)<1$. Suppose $N$ drops by $1$ to $N-1$ while each remaining member continuous to use $q^S(N)$. The impact on social welfare is
    \begin{equation}
        \begin{aligned}
            \Delta S&=\sum_{n=1}^Np_n\tilde{V}_n-\sum_{n=1}^{N-1}\phi_n\tilde{V}_n-q^sc\\
            &=\sum_{n=1}^Np_n\tilde{V}_n\frac{n-q^sN}{(1-q^s)N}-q^sc
        \end{aligned}
        \nonumber
    \end{equation}
    where $\phi_n=\begin{pmatrix}N-1\\n\end{pmatrix}(q^s)^{n}(1-q^s)^{N-n-1}=\frac{N-n}{N(1-q^s)}p_n$. By \eqref{EAE_4} and Lemma \ref{Lemma:opt_q}, we have
    \begin{equation}
        \begin{aligned}
            \Delta S=\frac{q^s}{N}\frac{\partial S}{\partial q}=0
        \end{aligned}
        \nonumber
    \end{equation}
    Hence, dropping one potential entrant while holding entry probabilities constant leaves social welfare unchanged. By relaxing the restriction on entry probability, the $q^s$ will increase and then increase the level of social welfare.
\end{proof}

\begin{corollary}
    The expected revenue of any seller who use his optimal mechanism increases monotonically as $N$ decreases beyond $n^*$.
\end{corollary}



















\chapter{Market Design}
Based on
\begin{enumerate}[$\circ$]
    \item Two-Sided Matching: A Study in Game-Theoretic Modeling and Analysis, Roth, Alvin E.\& Sotomayor, Matilda, 1990.
    \item Fleiner, T. (2003). A fixed-point approach to stable matchings and some applications. \textit{Mathematics of Operations research}, 28(1), 103-126.
    \item Hatfield, J. W., \& Kominers, S. D. (2017). Contract design and stability in many-to-many matching. \textit{Games and Economic Behavior}, 101, 78-97.
    \item MIT 14.16 Strategy and Information, Mihai Manea
\end{enumerate}
\section{Matching One-to-One}
Suppose there are doctors ($D$) and hospitals ($H$). For a doctor $d$, define a relation $\succeq_d$ over $H\cup\{d\}$; for a hospital $h$, define a relation $\succeq_h$ over $D\cup\{h\}$. A matching market is defined by $$\left(D,H,\{\succeq_i\}_{i\in D\cup H}\right)$$

\begin{note}
    Given a matching $\mu: D\cup H \rightarrow D\cup H$, we would call $\mu(d)$ be "$d$'s match".
\end{note}

\begin{definition}[Involution]
    %\normalfont
    A matching $\mu: D\cup H \rightarrow D\cup H$ is \textbf{involution} such that $$\mu (d)\neq d \Rightarrow \mu(d)\in H, \forall d\in D$$ and $$\mu (h)\neq h \Rightarrow \mu(h)\in D, \forall h\in H$$
\end{definition}

\begin{definition}[Stable]
    %\normalfont
    A matching $\mu: D\cup H \rightarrow D\cup H$ is \textbf{stable} if it is
    \begin{enumerate}[$\circ$]
        \item Individually Rational: $\nexists$ $i$ for whom $i>\mu(i)$.
        \item (Pairwise) Unblocked: $\nexists$ $(d,h)$ such that $d\succ_h \mu(h)$ and $h\succ_d \mu(d)$.
    \end{enumerate}
\end{definition}

\begin{theorem}[Gale-Shapley, 1962]
    For any matching market, a stable matching $\mu$ exists.
\end{theorem}
\begin{proof}
    We prove it by an algorithm:
    \begin{definition}[Deferred Acceptance Algorithm (DA)]
        %\normalfont
        At each round, every doctor applies for his most preferred hospital among those haven't rejected him. Each hospital chooses its most preferred doctors among its applicants and the one on the previous waitlist, and then rejects others.
    \end{definition}
    Observation: DA terminates $\mu$. We want to prove
    \begin{enumerate}
        \item $\mu$ is IR (obviously);
        \item $\mu$ is unblocked.
        \subitem Suppose there is a block $(d,h)$ such that $d\succ_h \mu(h)$ and $h\succ_d \mu(d)$. That is impossible, because the $d\neq \mu(h)$, the $d$ must be rejected by $h$, which means $h\preceq_d \mu(d)$.
    \end{enumerate}
\end{proof}

\begin{note}
    We call "$h$ is \textbf{achievable} for $d$" if $\mu(d)=h$ for some stable matching $\mu$.
\end{note}


\subsection{Matching Markets: One-to-One}
\begin{definition}[$D$-Optimal Matching]
    %\normalfont
    A matching $\mu: D\cup H \rightarrow D\cup H$ is \textbf{$D$-optimal}, denoted by $\mu^D$, if for any stable $\mu'$ we have that $\mu^D\succeq_D \mu'$ (the best stable matching for all doctors).
\end{definition}

\begin{theorem}[Deferred Acceptance Algorithm $\Rightarrow$ $D$-Optimal Matching]
    Deferred Acceptance Algorithm (with D proposing) terminates in $\mu^D$.
\end{theorem}
\begin{proof}
    %Suppose $d$ proposes to some $h$.
    %\begin{enumerate}
        %\item If $d$ is unacceptable ($d$ is below $\{h\}$) in $h$'s ranking, then $h$ is unachievable anyway.
        %\item Suppose $\exists d'\succ_h d$. If $h$ is achievable for $d'$, we have $h\succ_{d'} h'$
    %\end{enumerate}
    ...Theorem 2.12 (Gale and Shapley)
\end{proof}

\begin{theorem}[Lone-Wolf Theorem]
    The set of matched agent is identical in every stable $\mu$.
\end{theorem}
\begin{proof}
    $|\mu^D(H)|\geq |\mu(H)|\geq |\mu^H(H)|$; by symmetry, $|\mu^H(D)|\geq |\mu(D)|\geq |\mu^D(D)|$. Because $|\mu^D(H)|=|\mu^D(D)|$ and $|\mu^H(H)|=|\mu^H(D)|$ by one-to-one, so everything is equal.
\end{proof}

\subsection{Joint and Meet}
\begin{definition}[Joint and Meet]
    %\normalfont
    \begin{enumerate}
        \item \textbf{Join $\mu \vee_D \mu'$} assign the more preferred match to every $d$ and the less preferred match to every $h$, that is,
        \begin{equation}
            \begin{aligned}
                \mu \vee_D \mu'(d)=\left\{\begin{matrix}
                    \mu(d),&\textnormal{ if }\mu(d)>_d\mu'(d)\\
                    \mu'(d),&\textnormal{ otherwise}
                \end{matrix}\right., \forall d\in D
            \end{aligned}
            \nonumber
        \end{equation}
        \begin{equation}
            \begin{aligned}
                \mu \vee_D \mu'(h)=\left\{\begin{matrix}
                    \mu(h),&\textnormal{ if }\mu(h)<_h\mu'(h)\\
                    \mu'(h),&\textnormal{ otherwise}
                \end{matrix}\right., \forall h\in H
            \end{aligned}
            \nonumber
        \end{equation}
        \item \textbf{Meet $\mu\wedge_D\mu'$} assign the less preferred match to every $d$ and the more preferred match to every $h$, that is,
        \begin{equation}
            \begin{aligned}
                \mu \wedge_D \mu'(d)=\left\{\begin{matrix}
                    \mu(d),&\textnormal{ if }\mu(d)<_d\mu'(d)\\
                    \mu'(d),&\textnormal{ otherwise}
                \end{matrix}\right., \forall d\in D
            \end{aligned}
            \nonumber
        \end{equation}
        \begin{equation}
            \begin{aligned}
                \mu \wedge_D \mu'(h)=\left\{\begin{matrix}
                    \mu(h),&\textnormal{ if }\mu(h)>_h\mu'(h)\\
                    \mu'(h),&\textnormal{ otherwise}
                \end{matrix}\right., \forall h\in H
            \end{aligned}
            \nonumber
        \end{equation}
    \end{enumerate}
\end{definition}

\begin{theorem}[Join and Meet of Stable Matchings are Stable]
    If $\mu$ and $\mu'$ are stable, then $\mu\vee_D\mu'$ and $\mu\wedge_D\mu'$ are stable.
\end{theorem}

\subsection{Strategic Incentives}
\begin{enumerate}[$\circ$]
    \item Type $=$ preference list.
    \item SCF: $f: \Theta \rightarrow \mathcal{M}$, where $\mathcal{M}$ is a set of stable matchings;
    \item Is $f$ strategy-proof?
    \item Does there exist a stable and strategy-proof (direct) mechanism?
\end{enumerate}

\begin{definition}
    %\normalfont
    We say a mechanism $\varphi$ is strategy-proof (SP) if $\varphi(\succ_i,\succ_{-i})\geq \varphi (\succ'_i,\succ_{-i})$ for all $i\in I$ and $\succ'_i$ and $\succ_{-i}$.
\end{definition}

\begin{theorem}[Impossibility theorem (Roth)]
    There is no stable and strategy-proof (SP) mechanism.
\end{theorem}

The mechanism that yields the D-optimal stable matching (in terms of the stated preferences) makes it a dominant strategy for each doctor to state his true preferences. (Similarly, the mechanism that yields the H-optimal stable matching makes it a dominant strategy for every hospital to state its true preferences.)
\begin{theorem}[Dubins and Freedman; Roth]
    The doctor($D$)-optimal stable mechanism is strategy-proof for doctors.
\end{theorem}
\begin{proof}
    %Suppose under truthful $\succ$ (all doctors), a doctor $d$ has $\mu(d)=h$. $d$ changes his report to $\succ'_d$ such that $\mu'(d)=h'\succ_d h$.\\
    %Consider $\succ''_d$ which $\succ'_d$ truncated below $h'$.\\
    %Now, run a doctor-proposing DA (conside $\mu^D$) under $\succ''_d$. $d$ is unmatched.
\end{proof}


\section{Matching Many-to-Many}
Contracts are denoted by $x\in X$, $x_D\in D$, $x_H\in H$. $F\triangleq D\cup H$.

Consider a set of contracts $Y\subseteq X$,
\begin{enumerate}[$\circ$]
    \item $Y_D$ = doctors listed in $Y$;
    \item $Y_d$ = the contract in $Y$ that list the doctor $d$;
    \item $\succ_d$ over set of contracts that name the doctor $d$;
    \item The set of contracts $f\in F$ chooses from $Y$: $C_f(Y)=\max_{\succ_f}\{Z\subseteq X:Z\subseteq Y_f\}\subseteq Y_f$;
    \item The set of contracts doctors choose from $Y$: $C_D(Y)=\cup_{d\in D}C_d(Y)$.
    \item The set of contracts doctors reject from $Y$: $R_D(Y)=Y\backslash C_D(Y)$.
\end{enumerate}
The outcome of matching is $Y\subseteq X$.

\begin{definition}[Stable Contracts]
    %\normalfont
    $A\subseteq X$ is \textbf{stable} if
    \begin{enumerate}[$\circ$]
        \item Individually Rational (IR): for all $f\in F$: $C_f(A)=A_f$;
        \item Unblocked: $\nexists$ non-empty $Z\subseteq X$ such that $Z\cap A=\emptyset $ and for all $f\in F$, $Z_f\subseteq C_f(A\cup Z)$.
    \end{enumerate}
\end{definition}

\begin{example}
    Preferences over doctor $d$: $\{x,y\}>\{x\}>\emptyset>\{y\}$; Preferences over hospital $h$: $\{y\}>\{x,y\}>\{x\}>\emptyset$.\\
    $\{x\}$ $\Rightarrow$ $\{x,y\}$ $\Rightarrow$ $\{y\}$ $\Rightarrow$ $\emptyset$ $\Rightarrow$ $\{x\}$.
\end{example}

\begin{definition}[Substitutability Condition]
    %\normalfont
    Preference of $f$ satisfies the \textbf{substitutability condition} if for all $Y\subseteq X$ and $x,z\in X\backslash Y$:
    $$z\notin C_f(Y\cup\{z\}) \Rightarrow z\notin C_f(Y\cup\{z\}\cup\{x\})$$
    ($Y'\subseteq Y\subseteq X \Rightarrow R_f(Y')\subseteq R_f(X)$, where $R$ is the rejection choice.)
\end{definition}
If $z$ is rejected given a set, then it should also be rejected given a larger set.


\begin{theorem}
    If contracts are substitutes, then $Y\subseteq X$ is stable \underline{if and only if} pairwise stable.
\end{theorem}
\begin{proof}
    Prove $\Leftarrow$:
    (If not pairwise stable $\Rightarrow$ not stable)\\
    Suppose that $Z$ is a block. So, $Z\subseteq C_f(A\cup Z)$ for all $f$ listed in $Z$.\\
    We can pick a $z\in Z$ such that $z\in C_f(A\cup Z)$. By the substitutability condition, $z\in C_f(A\cup \{z\})$. So, it is stable.
\end{proof}

\begin{theorem}
    If contracts are substitutes, then a stable outcome exists.
\end{theorem}

\begin{definition}[Lattice]
    %\normalfont
    On a \textbf{lattice}, $L=(X,<,\wedge,\vee)$ (or we just use $L=(X,<)$), $<$ is a partial order on $X$ in such a way that any two elements $x$ and $y$ of $X$ have a unique greatest lower bound (glb) $x \wedge y$ (meet) and a unique lowest upper bound (lub) $x \vee y$ (join).
\end{definition}



\begin{definition}[Complete Lattice]
    %\normalfont
    A lattice $L=(X,<)$ is \textbf{complete} if there are both a meet (i.e. a greatest lower bound) and a join (i.e. a least upper bound) for any subset $Y\subseteq X$.\\
    These generalized meet and join operations on $Y$ are denoted by $\wedge Y$ and $\vee Y$.
\end{definition}

\begin{definition}[Monotone Function over Lattice]
    %\normalfont
    A function from one lattice to another lattice $f:(X,<) \rightarrow (X',<')$ is \textbf{monotone} if $x\leq y \Rightarrow f(x)\leq' f(y)$ for any $x,y\in X$.
\end{definition}

\begin{theorem}[Tarski 1955]
    Let $L=(X,<)$ be a complete lattice and $f: L \rightarrow L$ be monotone ($\leq$) function on $L$. Then, the set $\{x\in L: f(x)=x\}$ of fixed points is a non-empty, complete lattice with order $\leq$.
\end{theorem}
\begin{proof}
    Fleiner, T. (2003). A fixed-point approach to stable matchings and some applications. \textit{Mathematics of Operations research}, 28(1), 103-126.
\end{proof}


%Given sets of contracts $X^D$ and $X^H$.

%Operator $f(X^D,X^H)$ produces $\left(X\backslash R_H(X^H), X\backslash R_D(X^D)\right)$. Fixed point: $\left\{\begin{matrix}
    %X^D=&X\backslash R_H(X^H)\\
    %X^H=&X\backslash R_D(X^D)
%\end{matrix}\right.$. $X^D\cap X^H=A$ is stable.\\
%(Find the intercection of contracts that are not rejected in both $X^D$ and $X^H$, which is stable.)




%Operator $g(X^D,X^H)$ produces $g_H(X^H)=\{x\in X: x\in C_H(X^H\cup\{x\})\}$ and $g_D(X^D)=\{x\in X: x\in C_D(X^D\cup\{x\})\}$. Fixed point: $\left\{\begin{matrix}
    %X^D=&g_H(X^H)\\
    %X^H=&g_D(X^D)
%\end{matrix}\right.$. $X^D\cap X^H=A$ is stable.\\
%(Find the intercection of contracts that are accepted both $X^D$ and $X^H$, which is stable.)


%Define partial order, $(X^D,X^H)\geq (\bar{X}^D,\bar{X}^H)$ if $X^D\subseteq \bar{X}^D$ and $X^H \supseteq  \bar{X}^H$.

%If $(X^D,X^H)\geq (\bar{X}^D,\bar{X}^H)$, then $g(X^D,X^H)\geq g(\bar{X}^D,\bar{X}^H)$


%Check if $X^D\subseteq \bar{X}^D$, then $g(X^D)\supseteq g(\bar{X}^D)$, by substitutability.


%Prove $X^D\cap X^H=A$ is stable:\\
%Given the claim $C_D(X^D)=A$ (prove later)
%\begin{enumerate}[1).]
    %\item IR: $C_D(X^D)=A$ by $C(A)=A$;
    %\item Unblocked: $z\in X\backslash A$ that blocks. Then $z\notin X^H \Rightarrow z\in C_D(A\cup\{z\})$ and $z\notin C_D(X^D\cup\{z\})$, but $C_D(X^D)=A \Rightarrow z\notin C_D(A\cup\{z\})$.
%\end{enumerate}


If some contracts are not substitute, there are no stable outcomes exist.

\section{Matching Many-to-One}
\underline{Settings}
\begin{enumerate}[$\circ$]
    \item Doctors, $D$; Hospitals, $H$; Contracts $X=D\times H\times \textnormal{terms}$;
    \item Hospitals preference $\succ_h$ over $2^X$;
    \item Doctors preference $\succ_d$ over $X$ (compare one contract with another one contract, not compare over sets of contracts);
    \item Outcome is $Y\subseteq X$ s.t. $|Y_d|\leq 1$ for all $d\in D$ (a doctor signs at most one contract).
\end{enumerate}

What restriction do we need to have a stable matching? Not as strong as substitute.

\begin{corollary}
    Doctor-proposing DA algorithm produces a doctor-optimal stable matching.
\end{corollary}

\begin{example} The preferences of agents are
    \begin{enumerate}[$\circ$]
        \item $d_1: h_1\succ h_2$; $d_2: h_1\succ h_2$; $d_3: h_2\succ h_1$;
        \item $h_1: d_3\succ d_1,d_2\succ d_1\succ d_2$; $h_2: d_1\succ d_2\succ d_3$.
    \end{enumerate}
    There are two stable outcomes
    \begin{enumerate}
        \item $(d_1,h_2)$, $(d_3,h_1)$;
        \item $(d_1,h_1), (d_2,h_1), (d_3,h_2)$.
    \end{enumerate}
    \begin{remark}
        Lone-Wolf Theorem doesn't hold.
    \end{remark}

    Assume the $d_2$'s true preference is $h_2\succ h_1$ and he reveals it, there is only one stable matching: $(d_1,h_2)$, $(d_3,h_1)$. So, the $d_2$ may benefit from lying.\\
    \begin{remark}
        Strategy-proof doesn't hold.
    \end{remark}
\end{example}

\begin{definition}[Law of Aggagate Demand/ Cardianlity Monotomicity (CM)]
    %\normalfont
    For $h$, $Y\subseteq Y'\subseteq X \Rightarrow |C_h(Y)|\leq |C_h(Y')|$
\end{definition}

\begin{theorem}
    Under substitutes and CM, doctor-proposing DA is strategy-proof and LWT holds.
\end{theorem}

\begin{theorem}[Rural Hosptial Theorem]
    Under substitutes / CM, hospitals have same numbers of contracts in every stable outcome.
\end{theorem}

\subsubsection*{Cadets-branch matching}
Can be found in:
\begin{enumerate}[$\circ$]
    \item Jagadeesan, R. (2019). Cadet-branch matching in a Kelso-Crawford economy. \textit{American Economic Journal: Microeconomics}, 11(3), 191-224.
\end{enumerate}


\begin{remark}
    Contracts are not substitutes.
\end{remark}

\begin{definition}[Unilateral Substitute]
    %\normalfont
    Contracts are \textbf{unilateral substitutes} if for all $z,x\in X$ and $Y\subseteq X$ \underline{such that $z_D\notin Y_D$} if $z\notin C_h(Y\cup\{z\}) \Rightarrow z\notin C_h(Y\cup\{z\}\cup\{x\})$
\end{definition}

\begin{remark}
    Preferences of branches satisfying unilateral substitute.
\end{remark}

\begin{remark}
    The outcome of doctor-proposing DA algorithm is doctor-optimal and stable.
\end{remark}

\section{Networks}
Based on
\begin{enumerate}[$\circ$]
    \item Fleiner, T., Jankó, Z., Tamura, A., \& Teytelboym, A. (2015). Trading networks with bilateral contracts. arXiv preprint arXiv:1510.01210.
    \item Fleiner, T., Jankó, Z., Schlotter, I., \& Teytelboym, A. (2023). Complexity of stability in trading networks. \textit{International Journal of Game Theory}, 1-20.
\end{enumerate}

Considering a trading network represented by a directed graph, where nodes are firms $F$ and edges $X$ are contracts (income arrow can be understood as buying products and outcome arrow can be understood as selling products).

The choice function of $f\in F$ is represented by $C^f$, the choice of $f$ over $Y_f\subseteq X_f$ is $C^f(Y_f)\subseteq Y_f$, where $X_f$ is the set of contracts involving $f$.

The choice sets of buyer side (B) and seller side (S) are defined as
\begin{equation}
    \begin{aligned}
        C_B^f(Y|Z)&\triangleq C^f(Y_f^B\cup Z_f^S)\cap X_f^B\\
        C_S^f(Z|Y)&\triangleq C^f(Z_f^S\cup Y_f^B)\cap X_f^S
    \end{aligned}
    \nonumber
\end{equation}
where $Y$ is the contracts from buyer side and $Z$ is the contratcts from seller side.


\begin{definition}[Irrelevance of Rejected Contracts]
    %\normalfont
    Irrelevance of Rejected Contracts (IRC): $C(A)\subseteq B\subseteq A \Rightarrow C(A)=C(B)$
\end{definition}

\begin{definition}[Fully Substitute]
    %\normalfont
    $C^f$ is \textbf{fully substitute} if for $Y'\subseteq Y\subseteq X$ and $Z'\subseteq Z\subseteq X$,
    \begin{equation}
        \begin{aligned}
            R_B^f(Y'|Z)\subseteq R_B^f(Y|Z)\\
            R_S^f(Z'|Y)\subseteq R_S^f(Z|Y)
        \end{aligned}
        \nonumber
    \end{equation}
    and
    \begin{equation}
        \begin{aligned}
            R_B^f(Y|Z)\subseteq R_B^f(Y|Z')\\
            R_S^f(Z|Y)\subseteq R_S^f(Z|Y')
        \end{aligned}
        \nonumber
    \end{equation}
\end{definition}
Define partial order, $(Y,Z)\geq (Y',Z')$ if $Y\subseteq Y'$ and $Z\supseteq  Z'$.


\begin{definition}[Stable Outcome, Hatfield and Kominers (2012)]
    %\normalfont
    An outcome $A\subseteq X$ is stable if it is
    \begin{enumerate}
        \item Individual Rational: $\forall f\in F$, $C^f (A_f)=A_f$;
        \item Unblocked: there is no non-empty set $Z\subseteq X$ s.t. $Z\cap A=\emptyset$ and $\forall f\in F(Z)$, $Z_f\subseteq C^f(A\cup Z)$, where $F(Z)$ is the set of the firms are lined to $Z$.
    \end{enumerate}
\end{definition}

\begin{definition}[Trail]
    %\normalfont
    \textbf{Trail} is the set of distinct edges $T=(X^1,X^2,...,X^M)$ such that the buyer side (the firm who is the buyer in the edge) of $X^i$ is exactly the seller side (the firm who is the seller in the edge) of $X^{i+1}$, which is denoted by $b(X^i)=s(X^{i+1})$, $i=1,...,M-1$.
\end{definition}


\begin{definition}[Trail-stable Outcome]
    %\normalfont
    An outcome $A\subseteq X$ is \textbf{trail-stable} if its is
    \begin{enumerate}
        \item Individual Rational;
        \item There is no locally blocking trail $T=(X^1,X^2,...,X^M)$ such that
        \subitem $X^1\in C^{S(X^1)}(A\cup X^1)$;
        \subitem $\{X^i,X^{i+1}\}\in C^{b(X^{i})}(A\cup X^i\cup X^{i+1})$;
        \subitem $X^M\in C^{b(X^M)}(A\cup X^M)$.
    \end{enumerate}
\end{definition}

\begin{theorem}[Fleiner et al. 2016]
    If $C^f$ is fully substitute and IRC for all $f\in F$, then a trail-stable outcome exists.
\end{theorem}
\begin{proof}
    $Y\subseteq X$ and $Z\subseteq X$,
    \begin{equation}
        \begin{aligned}
            \Phi (Y,Z)=\left(X\backslash R_S(Z|Y), X\backslash R_B(Y|Z)\right)
        \end{aligned}
        \nonumber
    \end{equation}
    where $R_B(Y|Z)=\cup_{f\in F}R_B^f(Y|Z)$.
    \begin{claim}
        If $(Y,Z)$ is a fixed point of $\Phi$, then $A=Y\cap Z$ is trail-stable outcome.
    \end{claim}
    \begin{lemma}
        $C^f$ is fully substitute and IRC, and $(Y,Z)$ such that $Y \cap Z=A$, $C_S(Z|Y)=A$, $C_B(Y|Z)=A$. Then, for a contract $x\in X\backslash A$ and $A\subseteq A'\subseteq X$ if $C_S^{S(x)}(A\cup x|A')$ then $x\in C_S^{S(x)}(Z\cup x|A')$.
    \end{lemma}
    $\Phi$ will be monotone for the partial order $\geq$. As $(Y,Z)\geq (Y',Z')$, then $\Phi(Y,Z)\geq \Phi (Y',Z')$. Using Tarski fixed-point theorem, there is a $(Y,Z)$ fixed point.
    .....


    \textbf{Read} \textnormal{Fleiner, T., Jankó, Z., Tamura, A., \& Teytelboym, A. (2015). Trading networks with bilateral contracts. arXiv preprint arXiv:1510.01210.}
\end{proof}


\begin{proposition}
    $A$ is trail-stable $\Rightarrow$ $\exists$ $(Y,Z)$ such that $Y\cap Z=A$ and $(Y,Z)$ is a fixed point of $\Phi$.
\end{proposition}


\section{Corporate Game Theory}
There is a set of players $N=\{1,...,n\}$. The subset of players $S\subseteq N$ is called \textit{coalition}.

There is a value function about coalition $v: 2^N \rightarrow \mathbb{R}$, which assumes $v(N)\geq \max_{S\subseteq N}v(S)$.

\begin{definition}[Cooperative Game]
    %\normalfont
    A cooperative (or coalitional) game is described by the pair $\left<N,v\right>$.
\end{definition}

\begin{enumerate}
    \item Assume a TU (transferable utility) Economy. $S$ can divide $v(S)$ among its members; $S$ may implement any payoffs $(x_i)_{i\in S}$ with $\sum_{i\in S}x_i=v(S)$ (no externalities). %The efficiency requires $\sum_{i\in N}x_i=v(N)$.
    \begin{definition}[Transferable Utility]
        %\normalfont
        Utility is transferable if one player can losslessly transfer part of its utility to another player.
    \end{definition}
    \item Individual Rational (IR) requires $x_i\geq v(\{i\})$.
    \item $v(S)\geq 0$ is the \textit{worth} of coalition $S$;
    \item \textit{Outcome} is a \textit{partition} $(S_k)_{k=1,...,\bar{k}}$ of $N$ and an \textit{allocation} $(x_i)_{i\in N}$ specifying the division of the worth each $S_k$ among its members:
    \begin{enumerate}
        \item $S_j\cap S_k=\emptyset,\forall j\neq k$ and $\cup_{k=1}^{\bar{k}}S_k=N$;
        \item $\sum_{i\in S_k}x_i=v(S_k),\forall k\in\{1,...,\bar{k}\}$.
    \end{enumerate}
\end{enumerate}

\begin{example}
    \paragraph*{A majority game}
    \begin{enumerate}[-]
        \item Three parties (players 1,2, and 3) can share a unit of total surplus.
        \item Any majority-coalition of 2 or 3 parties-may control the allocation of output.
        \item Output is shared among the members of the winning coalition. $$
        \begin{gathered}
        v(\{1\})=v(\{2\})=v(\{3\})=0 \\
        v(\{1,2\})=v(\{1,3\})=v(\{2,3\})=v(\{1,2,3\})=1
        \end{gathered}
        $$
    \end{enumerate}
    
    \paragraph*{Firm and workers}
    \begin{enumerate}[-]
        \item A firm, player 0, may hire from the pool of workers $\{1,2, \ldots, n\}$.
        \item Profit from hiring $k$ workers is $f(k)$.
        $$
        v(S)= \begin{cases}f(|S|-1) & \text { if } 0 \in S \\ 0 & \text { otherwise }\end{cases}
        $$
    \end{enumerate}
\end{example}

\subsection{Core}
Suppose that it is efficient for the grand coalition to form:
\begin{equation}
    \begin{aligned}
        v(N)\geq\sum_{k=1}^{\bar{k}}v(S_k) \textnormal{ for every partition }(S_k)_{k=1,...,\bar{k}} \textnormal{ of }N
    \end{aligned}
    \nonumber
\end{equation}
Consider an allocation $(x_i)_{i\in N}$ chosen by the grand coalition. Use notation $x_S=\sum_{i\in S}x_i$. Allocation $(x_i)_{i\in N}$ is \textit{feasible} if $x_N=v(N)$.
\begin{definition}
    A coalition $S$ can \textbf{block} the allocation $(x_i)_{i\in N}$ if $x_S<v(S)$.
\end{definition}
\begin{definition}[Core]
    %\normalfont
    The \textbf{core} is the set of feasible allocations where no coalition of agents can block the grand coalition.
    $$C(v,N)=\left\{x\in \mathbb{R}^n: x_N=v(N), x_S\geq v(S), \forall S\subseteq N\right\}$$
\end{definition}

Which games have nonempty core?
\subsection{Bondareva-Shapley Theorem: Sufficient and Necessary Condition for Nonempty Cores}
\begin{definition}[Balancedness]
    A vector $(\lambda_S\geq 0)_{S\subseteq N}$ is \textbf{balanced} if $\sum_{\{S\subseteq N\mid i\in S\}}\lambda_S=1,\forall i\in N$ (all $S$ contains $i$).\\
    A payoff function $v$ is \textbf{balanced} if
    \begin{equation}
        \begin{aligned}
            \sum_{S\subseteq N}\lambda_S v(S)\leq v(N) \textnormal{ for every balanced }(\lambda_S\geq 0)_{S\subseteq N}
        \end{aligned}
        \nonumber
    \end{equation}
\end{definition}
\begin{note}
    \underline{Interpretation:} each player has a unit of time, which can be distributed among his coalitions. If each member of coalition $S$ is active in $S$ for $\lambda_S$ time, a payoff of $\lambda_Sv(S)$ is generated. A game is balanced if there is no allocation of time across coalitions that yields a total value $> v(N)$.
\end{note}

\begin{theorem}[Bondareva-Shapley Theorem]\label{BST}
    The coalitional game $\left<N,v\right>$ has non-empty core ($C(v,N)\neq \emptyset$) \underline{if and only if} it is balanced.
\end{theorem}
\begin{proof}
    Consider the linear program
    \begin{equation}
        \begin{aligned}
            X:=&\min \sum_{i\in N}x_i\\
            &\textnormal{s.t. }\sum_{i\in S}x_i\geq v(S),\forall S\subseteq N
        \end{aligned}
        \nonumber
    \end{equation}
    $C(v,N)=\left\{x\in \mathbb{R}^n: \sum_{i\in N}x_i=v(N), \sum_{i\in S}x_i\geq v(S), \forall S\subseteq N\right\}\neq \emptyset \Leftrightarrow X\leq v(N)$.\\
    The dual program of the linear program $X$ is
    \begin{equation}
        \begin{aligned}
            Y:=&\max \sum_{S\subseteq N}\lambda_S v(S)\\
            &\textnormal{s.t. } \lambda_S\geq 0,\forall S\subseteq N \textnormal{ and }\sum_{S\ni i}\lambda_S=1,\forall i\in N
        \end{aligned}
        \nonumber
    \end{equation}
    $v$ is balanced $\Leftrightarrow$ $Y\leq v(N)$. The primal linear program has an optimal solution. By the duality theorem of linear programming, $X = Y$. Therefore,
    \begin{equation}
        \begin{aligned}
            C(v,N)\neq \emptyset \Leftrightarrow v \textnormal{ is balanced}
        \end{aligned}
        \nonumber
    \end{equation}
\end{proof}

\subsection{Convex Games Have Nonempty Cores}
\begin{definition}[Convex Game]
    A game $\left<v,N\right>$ is convex if for any pair of coalitions $S$ and $T$, $$v(S\cup T)+v(S\cap T)\geq v(S)+v(T)$$
\end{definition}
Convexity implies that the marginal contribution of a player $i$ to a coalition increases as the coalition expands,
\begin{equation}
    \begin{aligned}
        S\cup T \textnormal{ and }i\notin T \Rightarrow v(T\cup\{i\})-v(T)\geq v(S\cup\{i\})-v(S)
    \end{aligned}
    \nonumber
\end{equation}
which can be induced by the convexity of $v$:
\begin{equation}
    \begin{aligned}
        v\left((S\cup\{i\})\cup T\right)+v\left((S\cup\{i\})\cap T\right)\geq v\left(S\cup\{i\}\right)+v(T)
    \end{aligned}
    \nonumber
\end{equation}

\begin{theorem}
    Every convex game has a non-empty core.
\end{theorem}



















\section{}
Consider $B(N)$ be the solutions to: $\left\{\begin{matrix}
    \sum_{S:i\in S}y_S=1,&\forall i\in N\\
    y_S\geq 0,& \forall S\subseteq N
\end{matrix}\right.$

\begin{lemma}[Farkas' lemma]
    Let $A\in \mathbb{R}^{m\times n}$ and $b\in \mathbb{R}^m$. Then, \textbf{exactly one} of the following statement is true
    \begin{enumerate}[(1).]
        \item There exists $x\in \mathbb{R}^n$ such that $Ax=b$ and $x\geq 0$
        \item There exists $y\in \mathbb{R}^n$ such that $A^Ty\geq 0$ and $b^T y<0$.
    \end{enumerate}
\end{lemma}

\begin{lemma}

\end{lemma}

\begin{proof}
    \begin{lemma}[(Alternative) Farkas' lemma]
        Let $A$ be $m\times n$ matrix, $b\in \mathbb{R}^m$ and $F=\{x\in \mathbb{R}^n: Ax\geq b,x\geq 0\}$. Then, either $Cx=d$ or $\exists z$ such that for $y_S\geq 0$, $C^Tz-A^Ty_S=0$ and such that $d^Tz-b^Ty_S<0$, but not both.
    \end{lemma}
    By using this lemma, we can conclude $\left\{\begin{matrix}
        v(N)z-\sum_S v(S)y_S<0\\
        z-\sum{y_S}=0\\
        y_S\geq 0
    \end{matrix}\right.$ must hold at the same time, (let $z=1$, the last two lines are $B(N)$).

    Hence, $\forall y_S\in B(N)$, we have $v(N)\geq \sum_S v(S)y_S$.
\end{proof}

\subsection{Doubly stochastic matrix and Birkhoff-von Neumann Theorem}

Consider a matching game between sellers and buyers: $v(\{i,j\})=v_{ij}$, $v(\{i\})=0$ for buyer $i$ and $v(\{j\})\geq 0$ for seller $j$.

\underline{Core:}
\begin{equation}
    \begin{aligned}
        \max_{\alpha}\quad &\sum_i\sum_j v_{ij}\alpha_{ij}\\
        \textnormal{s.t.}\quad&\sum_{i}\alpha_{ij}=1,\forall j\\
        &\sum_{j}\alpha_{ij}=1,\forall i\\
        &\alpha_{ij}\geq 0
    \end{aligned}
    \nonumber
\end{equation}

\begin{definition}[Doubly Stochastic Matrix]
    %\normalfont
    A \textbf{doubly stochastic matrix} is a square matrix $X=(x_{ij})$ of non-negative real numbers, each of whose rows and columns sums to $1$.\\
    The class of $n\times n$ doubly stochastic matrices is a convex polytope (convex set in euclidean space) known as the \textbf{Birkhoff polytope}.
\end{definition}

\begin{theorem}[Birkhoff-von Neumann Theorem]
    A matrix is doubly stochastic if and only if it is a convex combination of permutation matrices.
\end{theorem}
By this theorem, we can set efficient "integer" assignment.

Can the efficient allocation be competitive equilibrium (CE)?
\begin{theorem}
    The core of assignment of game is non-empty.
\end{theorem}
\begin{proof}
    The duality of core can be written as
    \begin{equation}
        \begin{aligned}
            \min \quad&  \sum_{j}u_j^S + \sum_{i}u_i^B\\
            \textnormal{s.t.}\quad& u_j^S+u_i^B\geq v_{ij}, \forall i,j
        \end{aligned}
        \nonumber
    \end{equation}
    By strong duality, the minimum value should be equal to $V(N)$.

    Hence, $\sum_{j\in T}u_j^S + \sum_{i\in T}u_i^B\geq V(T)$ for a subset $T\subseteq N$. That is, the core is non-empty.
\end{proof}

\begin{corollary}
    For an assignment game, outcome is in the core \underline{if and only if} the outcome is CE outcome.
\end{corollary}



\section{Constrained Demand Theory}
\subsection{Substitutes Valuation}
There are buyers $i\in N$ and goods $j\in J$ with quantities $S\in \mathbb{Z}^J$ sold by a seller.

A buyer's utility is $v(x)-p\cdot x$, where $v(0)=0$, $p\in \mathbb{R}^J$, and $x\in \{0,1\}^J$. The buyer's demand is represented by $D(p)\argmax_{x}\left\{v(x)-p\cdot x\right\}$.

The competitive equilibrium $\left(p^*,(x^{*i})_{i\in N}\right)$ here are
\begin{enumerate}
    \item $x^{*i}\in D^i(p^*)$ for every $i\in N$ and
    \item $\sum_{i}x^{*i}\leq S_i$, where the equality holds for $p_i>0$.
\end{enumerate}

\begin{definition}[Substitutes Valuation]
    %\normalfont
    A valuation $v_i$ is a \textbf{substitutes valuation} if $\forall p: p'=p+\lambda e^j$ ($\lambda>0$), where $D^i(p)=\{x\}$ and $D^i(p')=\{x'\}$, we have that $x'_k\geq x_k$ for all $k\neq j$. (The increase of product $j$'s price increases other product's demand).
\end{definition}

\begin{theorem}[Substitutes Valuation $\Rightarrow$ Competitive Equilibrium Exists]\label{thm:substitute}
    If agents have substitutes valuations, then a competitive equilibrium exists.
\end{theorem}

\begin{theorem}
    If there exists an agent without substitutes valuation, then we can construct \underline{unit-demand preferences} for other agents such that no competitive equilibrium exists.
\end{theorem}


\subsection{Income Effect}
There are buyers $i\in N$ and goods $j\in J$. The endowments (money and goods) of agents are denoted by $w=(w_0,w_I)$.

\underline{Outcome:} The indivisible (bought) goods is represented by $x_I\in\{0,1\}^J$ and the (left) divisible money is represented by $x_0\in(\underline{m},\infty)$. $$w_0=x_0+p_I\cdot x_I$$ must hold, where $p_I$ is the vector of prices of goods.

\underline{Utility Function:} An agent's utility function is defined by $u^i:(\underline{m},\infty)\times \{0,1\}^J \rightarrow (-\infty, +\infty)$ with assumptions of strictly increasing in $x_0$, $\lim_{x_0 \rightarrow \underline{m}}u^i(x_0,x_I)=-\infty$, and $\lim_{x_0 \rightarrow \infty}u^i(x_0,x_I)=+\infty$.

\begin{example}
    Examples of feasible utility functions:
    \begin{enumerate}
        \item $u^i(x)=v(x)-p\cdot x$ with $\underline{m}=-\infty$;
        \item $u^i(x_0,x_I)=\log(x_0)-\log(-V_Q^i(x_I))$ with $V_Q^i:\{0,1\}^J \rightarrow (-\infty,0)$.
    \end{enumerate}
\end{example}

\underline{Demand:}
\begin{enumerate}[$\circ$]
    \item $D_\textnormal{Marshallian}^i(p,w)=\{x^*:x^*\in\arg\max_{x} u^i(x) \textnormal{ s.t. }p\cdot x\leq p\cdot w\}$
    \item $D_\textnormal{Hicksian}^i(p,u)=\{x^*:x^*\in\arg\min_{x} p\cdot x \textnormal{ s.t. }u^i(x)\geq u\}$ which is the dual of $D_\textnormal{Marshallian}^i$.
\end{enumerate}
\begin{definition}[Competitive Equilibrium]
    %\normalfont
    Given $(w^i)_{i\in I}$ s.t. $\sum_{i\in N}w_I^i=y_I$. A \textbf{competitive equilibrium} is a price vector $p_I^*\in \mathbb{R}^J$ and ${x_I^i}^*\in D_\textnormal{Marshallian}(p_I^*,w^i)$ for each $i\in N$ such that $\sum_{i\in N}{x_I^i}^*=y_I$.
\end{definition}

Based on the idea of duality, we can analyze problem based on the dual demand, Hicksian demand.
\begin{definition}[Hicksian Valuation]
    %\normalfont
    Hicksian valuation is defined by $-1$ times "the money that can lead to the utility $u$ with goods $x_I$": $$V_\textnormal{Hicksian}^i(x_I,u)=-(u^i(\cdot,x_I))^{-1}(u)$$
\end{definition}


\begin{proposition}[Using Hicksian Valuation to Represent Hicksian Demand]
    $D_\textnormal{Hicksian}^i(p_I,u)=\arg\max_{x_I}\left\{v_\textnormal{Hicksian}^i(x_I,u)-p_I\cdot x_I\right\}$
\end{proposition}
\begin{proof}
    $D_\textnormal{Hicksian}^i(p_I,u)=\arg\min_{x_I}\{(u^i(\cdot,x_I))^{-1}(u)+p_I\cdot x_I\}=\arg\max_{x_I}\left\{V_\textnormal{Hicksian}^i(x_I,u)-p_I\cdot x_I\right\}$
\end{proof}


\begin{definition}[Hicksian Economy]
    %\normalfont
    Hicksian economy: for a profile $(u^i)_{i\in N}$ is a transferable utility (TU) economy in which each agent's "valuation" is a Hicksian valuation $V_\textnormal{Hicksian}^i$.
\end{definition}
Hicksian Economy works in finding Competitive Equilibrium
\begin{theorem}[Equilibrium Existence Duality(EED)]\label{EED}
    Competitive Equilibrium exists for all feasible endowment profiles \underline{if and only if} Competitive Equilibrium exists in the Hicksian economies for all profiles of utility levels.
\end{theorem}

\begin{center}
    \begin{tabular}{ccc}
        \hline
            Marshallian& Hicksian\\
        \hline
            Housing Market & Assignment Game\\
            Utility is not Quasi-linear & Utility is Quasi-linear\\
            Unit Demand& Unit Demand\\
            Existence in Housing Market& Existence in Assignment Game\\
            $\times$& Lattice structure and Convexity of structure of CE prices\\
            Net-substitutes& $\Rightarrow$ Substitutes\\
        \hline
    \end{tabular}
\end{center}

Like the Theorem \ref{thm:substitute}, we want the Hicksian valuations be "substitutes".
\begin{definition}[Net-Substitutes]
    %\normalfont
    A agent's utility $u^i$ is \underline{net-substitutes} if $\forall u$, $D^i_H(p;u)=\{x\}$ and $D^i_H(p'_j,p_{-j};u)=\{x'\}$, $p'_j>p_j \Rightarrow x'_k\geq x_k$ for all $k\neq j$.
\end{definition}

\begin{theorem}
    Net-Substitutes Valuation $\Rightarrow$ competitive equilibrium exists.
\end{theorem}
\begin{proof}
    Net-substitutes $\Rightarrow$ substitutes holds in Hicksian economy. Hence, CE exists. By \ref{EED}, CE exists in original economy.
\end{proof}

\begin{definition}[Gross-Substitutes]
    %\normalfont
    A agent's utility $u^i$ is \underline{gross-substitutes} if $\forall w$, $D^i_M(p;w)=\{x\}$ and $D^i_M(p'_j,p_{-j};w)=\{x'\}$, $p'_j>p_j \Rightarrow x'_k\geq x_k$ for all $k\neq j$.
\end{definition}

\begin{example}
    In quasi housing market, we consider an example, of holding a house which price increases, the demand of another bad house doesn't change under Hicksian demand, which makes net-substitutes hold. But, the Marshallian demand decreases, which makes gross-substitutes don't hold.
\end{example}

\begin{example}
\textbf{Net, but not gross}:\\
Suppose there is a firm $f$ thinking about workers $s_1,s_2$. $f$ values worker at $\$ 5$ each, and the hiring budget is $\$ 6$;
\begin{enumerate}[$\circ$]
    \item $p_1=2,p_2=4$;
    \item $p_1=3,p_2=4$
\end{enumerate}
Obviously, the gross-substitutes (Marshallian Demand) leads to hiring both under $p_1=2,p_2=4$ and only hiring $s_1$ under $p_1=3,p_2=4$.\\
Let's consider the net-substitutes (Hicksian Demand): As the utility given under $p_1=2,p_2=4$ is $\$ 10$. We can find hiring two workers is still the optimal strategy.
\end{example}


\begin{example}
    \textbf{Net, but no auction:}\\
    Suppose there are two identical firms $f_1,f_2$ and workers $s_1,s'_1,s_2$. The value of workers is $\$ 5$ each, but a firm want at most one of $s_1,s'_1$ and has hiring budget $\$ 6$. A worker has reservation wage of $\$ 1$.\\
    \underline{Equilibrium:} $\$1$ for worker $s_1,s'_1$ and $\$ 5$ for $s_2$; One firm hires one of $s_1,s'_1$ and the other hires $s_2$.
\end{example}


\section{Object Allocation}
Exchange: $i\in N$ agent; Agents have strict preference $\succ_i$ over objects. (We use $\succ$ denote $\{\succ_i\}_{i\in N}$).

\underline{Two settings:}
\begin{enumerate}
    \item Exchange: an agent shows up with exactly one object.
    \item Allocation: One planner owns $N$ objects; agents have $\emptyset$.
\end{enumerate}

A \textbf{mechanism} $\Phi(\succ)$ gives a outcome $\mu$.
We want the final outcome $\mu$ be
\begin{enumerate}
    \item Individual Rationality (IR): for all $i\in N$, $\mu_i\succeq i$ (Exchange) and $\mu_i\succeq \emptyset$ (Allocation).
    \item Pareto Efficient (PE): $\nexists \mu'$ such that $\mu'_i\succeq \mu_i$ for all $i\in N$, strict for at least one.
    \item Strategy-Proof (SP): $\Phi$ induces a game. We want that, in this game, truth-telling is a weakly dominant strategy for all agent $i\in N$.
\end{enumerate}

\subsection{Allocation}
(Random) Serial Dictatorship: Randomly order the agents, ask one by one, and allocate a remaining object. $\Rightarrow$ it satisfies IR, PE, SP, but \underline{unfair}(?).

\subsection{Exchange}
\begin{definition}[Core]
    %\normalfont
    The \textbf{core} is the set of all allocations $\mu$ such that there is no $S\subseteq N$ and $\mu'$ for which:
    \begin{enumerate}[$\circ$]
        \item for $i\in S$, $\mu'_i=j$ for some $j\in S$
        \item $\mu'_i\succeq \mu_i$ for all $i\in S$, at least one strict.
    \end{enumerate}
    Core: IR+PE.
\end{definition}
\begin{theorem}[Core is a Singleton]
    There is a unique element in the core.
\end{theorem}
\begin{proof}
    Run the algorithm: Top Trading Cycles (TTC).
\end{proof}
\begin{definition}[Top Trading Cycles (TTC)]
    %\normalfont
    Agent = node.
    \begin{enumerate}
        \item Step 1: every agent point at her favorite object/agent.
        \subitem (1A): Find cycles.
        \subitem (1B): Allocate object to agent who is pointing at it in cycle.
        \subitem (1C): Remove the cycle.
        \item Step 2: every (remaining) agent point at her favorite object/agent.
        \subitem (2A): Find cycles.
        \subitem (2B): Allocate object to agent who is pointing at it in cycle.
        \subitem (2C): Remove the cycle.
        \item Repeat $\cdots$
    \end{enumerate}
\end{definition}
\begin{proposition}
    TTC produces an allocation that satisfies IR, PE, SP.
\end{proposition}

\begin{theorem}[TTC $\Leftrightarrow$ IR, PE, SP (Ma, 1999)]
    There is at most $1$ IR, PE, SP mechanism (TTC).
\end{theorem}
\begin{proof}
    \begin{definition}
        %\normalfont
        The \textbf{size} of a preference profile $\succ$ is the total number of objects agents find acceptable in $\succ$:
        \begin{equation}
            \begin{aligned}
                S(\succ)=\sum_{i\in N}\# \textnormal{acceptable objects in }\succ_i
            \end{aligned}
            \nonumber
        \end{equation}
    \end{definition}
    Consider two $\Phi$ and $\Psi$ that disagree for some $\succ$, the $\succ$ is defined to be \underline{bad}.\\
    We define the \underline{minimal bad profile} as a bad profile of minimal size.
    Consider the two outcomes given by these mechanisms:
    \begin{center}
        \begin{tabular}{ccc}
            \hline
                $\Phi(\succ)$&\textnormal{same} & $A(\Phi)$\\
            \hline
                $\Psi(\succ)$&\textnormal{same} & $A(\Psi)$\\
            \hline
        \end{tabular}
    \end{center}
    the sum of different parts are $A\triangleq A(\Phi)+A(\Psi)$.
    \begin{lemma}
        If $\Phi$ and $\Psi$ are SP, and $\succ$ is a minimal bad profile, then each agent in $A$ has exactly two acceptable objects.
    \end{lemma}
    \begin{proof}
        Suppose there exists $i\in A$ such that she has $>2$ acceptable objects.\\
        Without losing generality, we consider $\Phi_i(\succ)\succ_i\Psi_i(\succ)$.\\
        Remove all objects from his preference list except $\Phi_i(\succ)$ and endowment of $i$ (call it $\{i\}$). The new preference profile is denoted by $\succ'_i$.\\
        Since $\Phi$ is SP, $\Phi_i(\succ')=\Phi_i(\succ)$; since $\Psi$ is SP, $\Psi_i(\succ')\prec_i\Phi_i(\succ)$.\\
        So, we have $\succ'$ is a bad profile and $S(\succ')<S(\succ)$, a contradiction.
    \end{proof}
\end{proof}


\section{School Choice}
\underline{Model:}
\begin{enumerate}
    \item There is a set of school $S$; a school is denoted by $s\in S$; Quota for each $s$ is $q_s$;
    \item $I$ is the set of all students; A student is denoted by $i\in I$; Student $i$ has preference $\succ_i$.
    \item School places = objects.
    \item Each school has a priority order over students $\pi_s$.
    \item Matching $\mu: I \rightarrow S$ such that $\forall s\in S: \# \mu^{-1}(s)\leq q_s$.
    \item Matching violates priority if $\exists s\in S$ such that
    \begin{enumerate}[(i).]
        \item $s\succ_i\mu(i)$ and
        \item either ``Wastefulness: $\# \mu^{-1}(s)< q_s$'' or ``Justified Envy: $i\pi_s j$ for some $j\in \mu^{-1}(s)$''
    \end{enumerate}
    $\approx$ existence of a blocking pair.\\
    \underline{A matching is \textbf{stable}} if there are no priority violates.\\
    (As we don't consider the preference of $j$ in (ii), it is not true stable $\Rightarrow$ (Pareto) efficient.)
\end{enumerate}

\begin{example}
    Boston (Immediate Acceptance)
    \begin{enumerate}[(1).]
        \item Step 1: students apply for favorite schools; school accepts applicants up to capacity and reject rest permanently.
        \item Step k: students apply for favorite schools among those with capacity and hasn't already rejected them; schools accept applicants up to capacity $q_s$ and reject rest permanently.
    \end{enumerate}
\end{example}

\begin{proposition}
DA gives a matching that satisfies \underline{stability and SP} (not PE).
\end{proposition}

Run TTC:
\begin{definition}[Top Trading Cycles (TTC)]
    %\normalfont
    Schools and Students (agents) = nodes.
    \begin{enumerate}
        \item Step 1: every agent point at her favorite object/agent.
        \subitem (1A): Find cycles.
        \subitem (1B): Allocate object (school) to agent (student) who is pointing at it in cycle. (Usually based on the students' preference.)
        \subitem (1C): Remove the cycle.
        \item Step 2: every (remaining) agent point at her favorite object/agent.
        \subitem (2A): Find cycles.
        \subitem (2B): Allocate object to agent who is pointing at it in cycle.
        \subitem (2C): Remove the cycle.
        \item Repeat $\cdots$
    \end{enumerate}
\end{definition}

\begin{proposition}
    TTC produces an allocation that satisfies \underline{PE and SP} (not stable).
\end{proposition}
Hence, we need to make a trade-off between priority violation and efficiency.

\begin{theorem}[Keslen]
    For all $S, \{q_s\}_{s\in S}$, there exists $I,\succ_i,\{\pi_s\}_{s\in S}$ s.t. in the SOSM, every student gets either their last choice or second-last choice.
\end{theorem}


\begin{theorem}[Abdulkadiroğlu, Pathak, Roth, AER]
    There is no (PE+)SP mechanism that Pareto-dominates SOSM.
\end{theorem}

\begin{theorem}
    There is no PE+SP mechanism that selects a PE+stable matching whenever it exists.
\end{theorem}

\begin{definition}[Kesten/Tang+Yu Algorithm]
    %\normalfont
    Suppose the number of student is not larger than the total capacity $\# I\leq \sum_s q_s$.
    \begin{enumerate}[(i).]
        \item Step 0: Run DA, set SOSM $\mu_0$. Find under-demanded schools = a school that doesn't reject any students.\\
        Assign $\mu^{-1}(s)$ permanently. Call these schools/students ``settled''. Remove all settled schools and students.
        \item Step k: Rerun DA on everyone unsettled.
    \end{enumerate}
\end{definition}

\begin{definition}[Priority-Neutral(PN), Reny 2022]
    %\normalfont
    $\mu$ is \textbf{priority-neutral}(PN) iff $\exists$ no matching $u$ that can make any student whose priority is violated at $\mu$ better off \underline{unless} $u$ violates the priority of some student and make them worse off.\\
    We call $\mu$ is \textbf{priority-efficient} if it is PN and PE.
\end{definition}

\begin{theorem}[Reny 2022]
    \begin{enumerate}
        \item $\exists$ a unique Priority-efficient matching;
        \item Priority efficient $\Leftrightarrow$ SO priority neutral matching;
        \item It can be found by the \underline{CUTE Algorithm};
        \item $\mu$ is priority efficient $\Leftrightarrow$ no matching $u$ can make \underline{any student better off} unless $u$ \underline{unless} $u$ violates the priority of some student and make them worse off.
    \end{enumerate}
\end{theorem}


\section{School Choice with Reserves}
Consider a school choice model, students can be divided into majority ($M$) and minority ($m$), $I=I^M\cup I^m$. Quotas of schools are represented by $q_s=(q,q^M), s\in S$, where $q^M$ is the quota for majority.

\begin{definition}[Stability]
    %\normalfont
    A matching is stable if, for all $s\in S$ such that $s\succ_i \mu(i)$,
    \begin{enumerate}
        \item Either: ``No Wastefulness: $|\mu^{-1}(s)| = q_s$'' and ``No Justified Envy: $i'\pi_s i$ for all $i'\in \mu^{-1}(s)$''
        \item Or: $i\in I^M$, ``$|\mu^{-1}(s)\cap I^M| = q_s^M$'' and ``$i'\pi_s i$ for all $i'\in \mu^{-1}(s)\cap I^M$''
    \end{enumerate}
\end{definition}

\begin{definition}[Stronger Quota]
    %\normalfont
    A $\tilde{setting}$ (with $\tilde{q}_s$) has \textbf{stronger quota} than setting (with $q_s$) if $\tilde{q}_s=q_s$ but $q_s^M\geq \tilde{q}_s^M$.
\end{definition}

\begin{definition}[Good Mechanism]
    %\normalfont
    Mechanism $\Phi$ is \textbf{good}, if whenever a $\tilde{setting}$ has stronger quotas than its setting, it doesn't make all \underline{minority} students worse off.
\end{definition}


\begin{theorem}[Kojima 2012]
    There is no stable good mechanism.
\end{theorem}

\subsection{Minority Reserves (slot-specific priority)}
Suppose $r_s^m$ is reserved for minority only. That is $q_s=q_s^M+r_s^m$.
\begin{definition}[Minority Reserves]
    %\normalfont
     School has minority reserve $r_s^m$ whenever $\#$ of admitted minority students is less than $r_s^m$, then any minority students is admitted ahead of majority students.
\end{definition}
\begin{definition}[No Blocking Pair]
    %\normalfont
    \textbf{No blocking pair} if $s\succ_i \mu(i)$, then $|\mu(s)|=q_s$ and,
    \begin{enumerate}
        \item Either: $i\in I^m$ and ``$i'\pi_s i$ for all $i'\in \mu^{-1}(s)$''
        \item Or: $i\in I^M$, ``$|\mu^{-1}(s)\cap I^m| > r_s^m$'' and ``$i'\pi_s i$ for all $i'\in \mu^{-1}(s)$''
        \item Or: $i\in I^M$, ``$|\mu^{-1}(s)\cap I^m| \leq r_s^m$'' and ``$i'\pi_s i$ for all $i'\in \mu^{-1}(s)\cap I^M$''
    \end{enumerate}
\end{definition}

\begin{theorem}[Smart Reserves]
    Suppose $\mu$ is a stable matching without affirmative action. Let $r_s^m$ be such that $$r_s^m\geq |\mu^{-1}(s)\cap I^m|, \forall s\in S$$
    Then, either $\mu$ is stable w.r.t. $r^m$ or $\exists$ stable matching under $r^m$ that Pareto-dominates $\mu$ for $I^m$.
\end{theorem}


\section{Random Assignment}
Suppose there are agents $i\in I$ and objects $j\in J$, where $|I|=|J|$. Agents have preferences $\succ_i$ over objects, and objects have priorities $\rhd_j$ over agents.

An allocation is represented by a matrix that each row and each column has sum to $1$ probability.

There are two mechanism can be used:
\begin{enumerate}[(i).]
    \item RSD (Random: draw a priority order $\rhd$ uniformly.)
    \item TTC with uniform random endowment.
\end{enumerate}
\begin{theorem}
    These two mechanisms are equivalent (bijection).
\end{theorem}


RSD is not Pareto-efficiently.

\begin{proposition}
    For a row of an allocation matrix ($\tilde{\mu}$) for agent $i$, $\tilde{\mu}_i\succ_i\tilde{\mu}'_i$
    \begin{enumerate}[$\circ$]
        \item \underline{if and only if} $\tilde{\mu}_i\succ_{FOSD}\tilde{\mu}'_i$ (first-order stochastic dominance).
        \item \underline{if and only if} $\mathbb{E}U(\tilde{\mu}_i)\geq\mathbb{E}U(\tilde{\mu}'_i)$ under expected utility.
    \end{enumerate}
\end{proposition}

\begin{definition}
    %\normalfont
    $\tilde{\mu}$ is \textbf{ordinally efficient (sd-efficient)} if there is no $\tilde{\mu}'$ which is $\succ_{FOSD}$ by all agents. (\textit{ex-ante efficient} with respect to cardinal utility)\\
    $\tilde{\mu}$ is \textbf{ex-post efficient} if those are only Pareto efficient outcome in the support.
\end{definition}

\begin{definition}
    %\normalfont
    $\tilde{\mu}$ is \textbf{ordinally envy-free} if $\tilde{\mu}_i\succ_{FOSD}\tilde{\mu}_j, \forall i,j$.
\end{definition}
RSD is not envy-free.

There exists ordinally efficient and envy-free mechanism.
\begin{definition}[Probabilitistic Serial Algorithm]
    %\normalfont
    Based on the preference of agents:
    \begin{enumerate}
        \item Give each agent his most preferred object with the same proportion such that the sum of each object is at most 1.
        \item Repeat by using remaining objects.
    \end{enumerate}
    \begin{example}
        Preference: A: $Obj1\succ Obj3\succ Obj2$; B: $Obj1\succ Obj2 \succ Obj3$; C: $Obj2\succ Obj3\succ Obj1$
        \begin{enumerate}
            \item [$t=\frac{1}{2}$] A: $\frac{1}{2} Obj 1$; B: $\frac{1}{2} Obj 1$; C: $\frac{1}{2} Obj 2$.
            \item [$t=\frac{3}{4}$] A:$ \frac{1}{2} Obj 1+\frac{1}{4} Obj 3$; B: $\frac{1}{2} Obj 1+\frac{1}{4} Obj 2$; C: $\frac{3}{4} Obj 2$.
            \item [$t=1$] :$ \frac{1}{2} Obj 1+\frac{1}{2} Obj 3$; B: $\frac{1}{2} Obj 1+\frac{1}{4} Obj 2+\frac{1}{4} Obj 3$; C: $\frac{3}{4} Obj 2+\frac{1}{4} Obj 3$.
        \end{enumerate}
    \end{example}
\end{definition}
\begin{theorem}[Welfare Theorem]
    Probabilistic Serial Algorithm gives ordinally efficient and envy-free outcome.
\end{theorem}

\begin{definition}[Equal Treatment of Equals (ETE)]
    %\normalfont
    Equal Treatment of Equals: if same preference $\succ_i$ $\Rightarrow$ the same bundle $\tilde{\mu}_i$.
\end{definition}


\begin{proposition}
    For $n=3$, RSD is \textit{ordinally efficient, ETE, Strategy-Proof}. (These three properties are incompatible when $n>3$).
\end{proposition}

\section{Random Assignment in School Choice}
\begin{example}
    \begin{enumerate}[$\circ$]
        \item Preference of Agents: $A: s_2\succ s_3\succ s_1$; $B: s_2\succ s_3\succ s_1$; $C: s_1\succ s_2\succ s_3$.
        \item Priority of Schools: $s_1: A\succ B\succ C$, $s_2: C\succ (A,B)$, $s_3: C\succ B\succ A$.
    \end{enumerate}
    There are two stable outcomes: $\mu: A-s_2, B-s_3, C-s_1$; $\mu': A-s_3, B-s_2, C-s_1$.

    It can't be strategy proof. In $\mu$, $B$ can lie: $s_2\succ s_1\succ s_3$, to make the outcome become $\mu'$. In $\mu'$, $A$ can lie: $s_2\succ s_1\succ s_3$, to make the outcome become $\mu$.
\end{example}

\begin{definition}[Stable Imporovement Cycle (S.I.C.)]
    %\normalfont
    Each student points at schools they prefer and where he doesn't have a lower priority among those students who prefer students to their assignment.
\end{definition}

\begin{theorem}
    If a stable matching is not in the student-optimal stable set, then $\exists$ a S.I.C.
\end{theorem}
\begin{example}
    \begin{enumerate}[$\circ$]
        \item Preference of Agents: $A: s_2\succ s_1\succ s_3$; $B: s_3\succ s_2\succ s_1$; $C: s_2\succ s_3\succ s_1$.
        \item Priority of Schools: $s_1: A\succ (B,C)$, $s_2: B\succ (A,C)$, $s_3: C\succ (A,B)$.
    \end{enumerate}
    DA: $A:s_1, B:s_2, C:s_3$.
    Another allocation: $A:s_1, B:s_3, C:s_2$.

    Consider DA, $A$ wants $s_2$: $C$ also wants $s_2$, which has the same priority as $A$, so $A$ can point at $s_2$. $B$ points at $s_3$. $C$ can also point at $s_2$. So, there is a S.I.C.
\end{example}


\section{Pseduomarket (FF)}
Consider an example that agent $A_1$ wants $a,b$ for $0.9$, $A_2$ wants $a,c$ for $0.9$, $A_3$ wants $b,c$ for $2$. Suppose the budget for each agent is $1$.

Reminds that utility is only meaningful for the agent itself. Here, as the budget is the same, the demand of each agent is the same.

\subsection{Problem of Implementability}
An equilibrium (but can't be implemented): $A_1$ gets $\{\frac{1}{2}:\emptyset; \frac{1}{2}:a+b\}$; $A_2$ gets $\{\frac{1}{2}:\emptyset; \frac{1}{2}:a+c\}$; $A_3$ gets $\{\frac{1}{2}:\emptyset; \frac{1}{2}:b+c\}$.


\begin{center}
    \begin{tabular}{ccc}
        \hline
            Transfer Utility Economy& Pseduomarket\\
        \hline
            Allocation $x_j\in X_j,j=1,...,J$& Lottery $\tilde{x}_j\in \mathcal{L}(X_j)$\\
            Price $p\in \mathbb{R}^I$& Budget $b_j$ and Price $p\in \mathbb{R}^I$\\
            $u_j(x)=v_j(x)-p\cdot x$& $V_j(\tilde{x}_j)=\sum_x v_j(x) \mathbb{P}(\tilde{x}_j=x)$\\
            Demand $D_j(p)=\arg\max_x u_j(x)$& $\tilde{D}_j(p)=\arg\max_{\tilde{x}:p\cdot\tilde{x}\leq b_j} V_j(\tilde{x})$\\
            CE: $(p^*,x^*): x_j^*\in D_j(p^*), \sum_{j}x_j^*\leq S$& RE: $(p^*,\tilde{x}^*): \tilde{x}_j^*\in \tilde{D}_j(p^*), \sum_{j}\tilde{x}_j^*\leq S$\\
            (equality holds for no zero $p^*$)\\
        \hline
    \end{tabular}\\
    $S$ is supply, which equals to $\sum_i \omega_i$ if the economy with endowments.
\end{center}

We want an allocation being implementable that an allocation (a set of lotteries over agents) $\{w_1,...,w_J\}=\mathcal{W}\in \mathcal{L}(\prod_{j}X_j)$ (feasible bundles for each agent).

Define $\bar{w}_j=\mathbb{E}[w_j]$ and $\bar{\mathcal{W}}=\mathbb{E}[\mathcal{W}]$

\begin{definition}[Implementable]
    %\normalfont
    A random equilibrium $(p^*,\tilde{x}^*)$ is \textbf{implementable} if there exists $\mathcal{W}$ over feasible allocations such that $w_j\in D_j(p^*)$ and $\bar{x}^*_j=\bar{w}_j, \forall j=1,...,J$.
\end{definition}
can be implemented by a distribution of allocations. (BvN)

\begin{proposition}
    Random equilibrium always exists.
\end{proposition}

\begin{definition}[Rich]
    %\normalfont
    A set of valuations $\mathcal{V}^j=\{v_j(x):x\in X_j\}$ is \textbf{rich} if whenever $v_j(x)\in \mathcal{V}^j$ then $v_j(x)+a\cdot x\in \mathcal{V}^j$ for all $a\in \mathbb{R}^I$. That is $\exists x'$ such that $v_j(x')=v_j(x)+a\cdot x$.
\end{definition}
Complement may induce unimplementable problem.

Suppose value functions live in $V$ and are \underline{rich}.
\begin{theorem}
    CE exists for all valuations in $V$ $\Leftrightarrow$ RE is implementable for all budgets profiles and all valuations in $V$.
\end{theorem}












\end{document}