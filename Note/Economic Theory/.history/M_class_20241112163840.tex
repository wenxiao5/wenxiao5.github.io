\documentclass[11pt]{elegantbook}
\usepackage{graphicx}
%\usepackage{float}
\definecolor{structurecolor}{RGB}{40,58,129}
\linespread{1.6}
\setlength{\footskip}{20pt}
\setlength{\parindent}{0pt}
\newcommand{\argmax}{\operatornamewithlimits{argmax}}
\newcommand{\argmin}{\operatornamewithlimits{argmin}}
\elegantnewtheorem{proof}{Proof}{}{Proof}
\elegantnewtheorem{claim}{Claim}{prostyle}{Claim}
\DeclareMathOperator{\col}{col}
\title{Miguel Class}
\author{Wenxiao Yang}
\institute{Haas School of Business, University of California Berkeley}
\date{2024}
\setcounter{tocdepth}{2}
\extrainfo{All models are wrong, but some are useful.}

\cover{cover.png}

% modify the color in the middle of titlepage
\definecolor{customcolor}{RGB}{32,178,170}
\colorlet{coverlinecolor}{customcolor}
\usepackage{cprotect}


\bibliographystyle{apalike_three}

\begin{document}
\maketitle

\frontmatter
\tableofcontents

\mainmatter



\chapter{Pricing}
\section{Monopoly}
\subsection{Base Case}
The firm decides its price $p$ to maximize $\Pi(p)=p\cdot D(p)-C(D(p))$, where $D(\cdot)$ is the demand function and $C(\cdot)$ is the cost function.

The monopoly problem is maximizing the profit
\begin{equation}
    \begin{aligned}
        \max_p \Pi(p)=p\cdot D(p)-C(D(p))
    \end{aligned}
    \nonumber
\end{equation}
The F.O.C. (first-order condition) is
\begin{equation}
    \begin{aligned}
        \frac{\partial \Pi(p)}{\partial p}=D(p)+pD'(p)-C'(D(p))D'(p)=0\\
    \end{aligned}
    \nonumber
\end{equation}
and the S.O.C. (second-order condition) is
\begin{equation}
    \begin{aligned}
        \frac{\partial \Pi^2(p)}{\partial p^2}<0
    \end{aligned}
    \nonumber
\end{equation}

The F.O.C. gives that
\begin{equation}
    \begin{aligned}
        (p-C')D'&=-D\\
        p-C'&=-\frac{D}{D'}\\
        \underbrace{\frac{p-C'}{p}}_\text{Lerner Index}&=-\frac{D}{D'p}\\
        &=-\frac{1}{\frac{d D}{d p}\frac{p}{D}}=-\frac{1}{\frac{\frac{d D}{D}}{\frac{d p}{p}}}:=\frac{1}{E}
    \end{aligned}
    \nonumber
\end{equation}
where $\frac{\frac{d D}{D}}{\frac{d p}{p}}<0$ is the elasticity of demand with respect to price. The absolute value of the elasticity is denoted by $E$.

$E$ is supposed to be greater than $1$, otherwise, the optimal price is negative.

In the demand function $D(p)=kp^{-E}$, where the elasticity is constant. Its elasticity is $-E$.


The monopolist gives the production that is lower than social-optimal to maximize the profit (dead weight loss). Rent dissipation can give larger dead weight loss.

\subsection{Multiple Products}
\begin{equation}
    \begin{aligned}
        \max_{p}\sum_{i=1}^N p_i D_i(p)-C(D_1(p),...,D_N(p))
    \end{aligned}
    \nonumber
\end{equation}
\paragraph*{Related Demand and Separable Costs:} $C(D_1(p),...,D_N(p))=C_1(D_1(p))+...+C_N(D_N(p))$. The optimal pricing in this case satisfies
\begin{equation}
    \begin{aligned}
        \frac{p_i-C'_i}{p_i}=\frac{1}{E_{ii}}-\sum_{j\neq i}\frac{(p_j-C'_j)D_j E_{ij}}{R_i E_{ii}}
    \end{aligned}
    \nonumber
\end{equation}
where $E_{ij}=\frac{\partial D_i}{\partial p_j}\frac{p_j}{D_i}$ and $R_i$ is the revenue.

\textbf{Intuition:} In the case of substitutes/complements, we want to increase/decrease the price of products compared to the one product case. (Positive/negative externality by increasing price of substitutes).


\textbf{Similar Intuition:}
Consider a two-period model that the demand at second period depends on the price at first period (assuming $\frac{\partial D_2}{\partial p_1}<0$).
\begin{enumerate}
    \item $q_1=D_1(p_1)$; $C_1(q_1)$
    \item $q_2=D_2(p_2,p_1)$; $C_2(q_2)$
\end{enumerate}
Then, $\frac{p_1-C'_1}{p_1}<\frac{1}{E_1}$ (the negative externality of prices).

\paragraph*{Independent Demands and Related Costs:}
\begin{example}
    Different intensity of demand across periods.
    \begin{enumerate}
        \item Period 1: Low demand. $q_1=D_1(p_1)$.
        \item Period 2: High demand. $q_2=D_2(p_2)$, where $D_1(p)=\lambda D_2(p)$ for some $\lambda<1$.
        \item Marginal cost of Production is $c$ and the Marginal cost of capacity is $\gamma$.
    \end{enumerate}
\end{example}
Intuition: if $\lambda \rightarrow 0$, the marginal cost at period $\rightarrow c+\gamma$ and the marginal cost at period 1 $=c$. Then, we have
\begin{equation}
    \begin{aligned}
        \frac{p_2-(c+\gamma)}{p_2}=\frac{1}{E_2},\ \frac{p_1-c}{p_1}=\frac{1}{E_1}
    \end{aligned}
    \nonumber
\end{equation}
Now, let's consider a not too small $\lambda$. The problem is given as
\begin{equation}
    \begin{aligned}
        \max_{p_1,p_2,k}&\ (p_1-c)D_1(p_1)+(p_2-c)D_2(p_2)-\gamma k\\
        s.t.\ & D_1(p_1)\leq k\\
        & D_2(p_2)\leq k
    \end{aligned}
    \nonumber
\end{equation}
The Lagrangian is given by
\begin{equation}
    \begin{aligned}
        \mathcal{L}=(p_1-c)D_1(p_1)+(p_2-c)D_2(p_2)-\gamma k+\lambda_1(k-D_1(p_1))+\lambda_2(k-D_2(p_2))
    \end{aligned}
    \nonumber
\end{equation}
\begin{equation}
    \begin{aligned}
        \frac{\partial \mathcal{L}}{\partial k}=-\gamma+\lambda_1+\lambda_2=0 \Leftrightarrow \gamma=\lambda_1+\lambda_2
    \end{aligned}
    \nonumber
\end{equation}
Skip the process: $\frac{p_1-(c+\lambda_1)}{p_1}=\frac{1}{E_1}$, $\frac{p_2-(c+\lambda_2)}{p_2}=\frac{1}{E_2}$. Example: If $\lambda_1=0, k>D_1(p_1)$, the second period pays all the capacity cost.

\begin{example}[ (Learning by Doing)]
    Suppose there are two periods $t=1,2$. The demand is $q_t=D_t(p_t)$. The cost in period one is $c_1(q_1)$ and $c_2(q_2,q_1)$ ($\frac{\partial c_2}{\partial q_1}<0$, the more you produce in period one, the lower the cost you are facing in period two).
\end{example}
In continuous form, the cost form is
\begin{equation}
    \begin{aligned}
        C(w(t))
    \end{aligned}
    \nonumber
\end{equation}
where $\dot{w}(t)=\frac{d w}{d t}=q(t)$. We want to maximize
\begin{equation}
    \begin{aligned}
        \max_{q(t),w(t)}\ &\int_0^\infty e^{-\pi t}[q(t)p(q(t))-C(w(t))q(t)]dt\\
        &\textnormal{s.t. }\dot{w}(t)=q(t)
    \end{aligned}
    \nonumber
\end{equation}
By Hamiltonian (skip), average of future marginal costs is
\begin{equation}
    \begin{aligned}
        A(t)=\int_t^\infty C(w(s))\pi e^{-\pi(s-t)}ds
    \end{aligned}
    \nonumber
\end{equation}
\begin{equation}
    \begin{aligned}
        \frac{P(t)-A(t)}{P(t)}=\frac{1}{E(t)}
    \end{aligned}
    \nonumber
\end{equation}

\subsection{Durable Goods}
The demand in one period is substitute to demand in other periods.
\begin{example}
    Two periods $t=1,2$. Three consumers: $v_1=1$ per period, $v_2=2$ per period, and $v_3=3$ per period. The cost of production is zero. The seller chooses $p_1,p_2$.
\end{example}
(Consumer may forward-looking).


Moorthy (1988), Levinthal, D. A., \& Purohit, D. (1989).

$t=1,2$ and zero production cost. The values of consumers $v\sim U[0,1]$ and the discount factor is $\delta<1$. The selling price $p_1,p_2$.

Suppose the consumers bought in first period have $v\geq v_1^*$, which must satisfies
\begin{equation}
    \begin{aligned}
        \delta (v_1^* - p_2)&= v_1^* - p_1 + \delta v_1^*\\
        v_1^*&=p_1-\delta p_2
    \end{aligned}
    \nonumber
\end{equation}
The price in second period should be $p_2=\frac{v_1^*}{2}$. Then, $v_1^*=p_1-\delta \frac{v_1^*}{2} \Rightarrow v_1^*=\frac{2p_1}{2+\delta}$. The price in the first period is given by
\begin{equation}
    \begin{aligned}
        \max_{p_1}\ p_1(1-v_1^*)+\delta \frac{(v_1^*)^2}{4}=p_1\frac{2+\delta-2p_1}{2+\delta}+\delta \frac{p_1^2}{(2+\delta)^2}
    \end{aligned}
    \nonumber
\end{equation}

\paragraph*{Leasing (instead of selling):} The leasing price is $p$ in each period, which is given by $\argmax_p p(1-p)=\frac{1}{2}$.

Leasing may generate more profits than selling for the seller.

\paragraph*{Intuition:}
\begin{enumerate}
    \item Too much flexibility for seller $\Rightarrow$ losses of capital of first period buyers.
    \item Intertemporal price discrimination (first period buyers pay higher price) ``price skimming''.
\end{enumerate}
\begin{theorem}[Coase Conjecture]
    Suppose the seller can change the price faster and faster. What happens to the profits of the seller? The profit goes to zero.
\end{theorem}

Why there is selling in the world?
\begin{enumerate}
    \item Moral hazard of leasing.
    \item Leasing is not anonymous. Reveal reservation price $\Rightarrow$ Price discrimination in leasing. (Even worse than selling.) (Long-term contract + renegotiation = selling).
    \item Commit to sequences of prices.
    \begin{enumerate}
        \item Deposit to third party
        \item Reputation
    \end{enumerate}
    \item Increasing cost
    \item ``Most-favored Nation'' clause.
    \item Consumers are not informed about the production costs.
    \item New consumers coming into the market.
\end{enumerate}

\subsection{Learning Demand}
Firms may not be able to learn the demand function perfectly.

It is relatively easy to learn a quasi-concave profit function. In the case that the profit function is not quasi-concave, the firm may not be able to learn the profit function. (stay at local maximum because of the loss from learning).

Learning the optimal features: By assuming the distribution of marginal increase $\frac{\partial \pi_j}{\partial x_j}$ is symmetric about $0$, we can use Brownian motion to model the continuous profit function.


\section{Short-run Competition}
\subsection{Bertrand Paradox}
Consider two firms with marginal cost $c$:
\begin{equation}
    \begin{aligned}
        \Pi^i(p_i,p_j)=(p_i-c)D_i(p_i,p_j)
    \end{aligned}
    \nonumber
\end{equation}
where
\begin{equation}
    \begin{aligned}
        D_i(p_i,p_j)=\left\{\begin{matrix}
            D_i(p_i),& p_i<p_j,\\
            \frac{1}{2}D_i(p_i),& p_i=p_j,\\
            0,& p_i>p_j
        \end{matrix}\right.
    \end{aligned}
    \nonumber
\end{equation}
The NE is $p_i=p_j=c$. $\Pi^i=\Pi^j=0$.

\subsection{Static Solution to Bertrand Paradox}
\subsubsection*{Capacity Constraints}
Edgeworth: there may exist some constraints of the capacity.
\begin{enumerate}
    \item Firms choose capacity $K_i,K_j$.
    \item Firms choose prices $p_i,p_j$.
\end{enumerate}
Solving by backward induction: That is, firstly solve $p_i^*(K_i,K_j)$ such that
\begin{equation}
    \begin{aligned}
        \max_{p_i}\ & (p_i-c)D_i(p_i,p_j)\\
        \textnormal{s.t. }&D_i(p_i,p_j)\leq K_i
    \end{aligned}
    \nonumber
\end{equation}
and then solve
\begin{equation}
    \begin{aligned}
        \max_{K_i}\ \left(p_i^*(K_i,K_j)-c\right)D_i(p_i^*(K_i,K_j),p_j^*(K_i,K_j))-\gamma K_i
    \end{aligned}
    \nonumber
\end{equation}
where $\gamma$ is the marginal cost of capacity.

Best response in prices: positive correlated, which is called ``strategic complements''.

Best response in quantities (Cournot competition): negative correlated, which is called ``strategic substitutes''. (Quantity competition gives higher profits.)
\begin{example}[ (Simple Example of Couront Competition)]
    $P(q_1,q_2)=1-q_1-q_2$ and $\gamma\in (\frac{3}{4},1)$.
    \begin{equation}
        \begin{aligned}
            \max_{q_1}\ q_1\left(1-q_1-q_2-\gamma\right)\\
            \Rightarrow q_1^*(q_2)=\frac{1-q_2-\gamma}{2}
        \end{aligned}
        \nonumber
    \end{equation}
    Similarly, $q_2^*(q_1)=\frac{1-q_1-\gamma}{2}$. Thus, $q_1^*=q_2^*=\frac{1-\gamma}{3}$.
\end{example}

Similar to the Cournot competition, we can get positive profits with capacity constraints.

\subsubsection*{Differentiation}
\textbf{Idea:} it is easier to change prices than to change products.

\paragraph*{Basic case:} Spatial Competition: There are consumers in $[0,1]$ (uniform distribution). The position chosen by both firms is $\frac{1}{2}$ (the center of the market).

\paragraph*{With price competition:} Transportation cost is $t x^2$, where $x$ is the distance. Suppose the firm $A$ locates at $0$ and the firm $B$ locates at $1$. The profit of consumer $x$ from purchasing $A$ is $v-p_A-tx^2$ and the profit of consumer $x$ from purchasing $B$ is $v-p_B-t(1-x)^2$. The indifferent consumer is
\begin{equation}
    \begin{aligned}
        v-p_A-tx^2=v-p_B-t(1-x)^2\\
        p_A-p_B=t(1-2x)\\
        \Rightarrow x=\frac{1}{2}-\frac{p_A-p_B}{2t}
    \end{aligned}
    \nonumber
\end{equation}
Therefore, the demand of $A$ is $$D_A(p_A,p_B)=\frac{1}{2}-\frac{p_A-p_B}{2t}$$ and the demand of $B$ is $$D_B(p_A,p_B)=\frac{1}{2}+\frac{p_A-p_B}{2t}$$
\begin{note}
    If the transportation cost is $tx$, the demand function is the same as above.
\end{note}
The $p_A^*$ and $p_B^*$ are given by
\begin{equation}
    \begin{aligned}
        \left.\begin{matrix}
            p_A^*(p_B)=\argmax_{p_A}(p_A-c)\left(\frac{1}{2}-\frac{p_A-p_B}{2t}\right)=\frac{c+t+p_B}{2}\\
            p_B^*(p_A)=\argmax_{p_B}(p_B-c)\left(\frac{1}{2}+\frac{p_A-p_B}{2t}\right)=\frac{c+t+p_A}{2}
        \end{matrix}\right\} \Rightarrow p_A^*=p_B^*=c+t
    \end{aligned}
    \nonumber
\end{equation}
Then, the profits are $\Pi_A^*=\Pi_B^*=\frac{t}{2}$.

\paragraph*{Endogenous Differentiation:}
Denote the position of firm $A$ as $a$ and the position of firm $B$ as $1-b$. Then, the indifferent consumer is
\begin{equation}
    \begin{aligned}
        p_A+t(x-a)^2=p_B+t(1-b-x)^2\\
        \Rightarrow x=\frac{p_B-p_A+t\left[(1-b)^2-a^2\right]}{2t(1-b-a)},
    \end{aligned}
    \nonumber
\end{equation}
and the demands are
\begin{equation}
    \begin{aligned}
        D_A(p_A,p_B)=\frac{p_B-p_A+t\left[(1-b)^2-a^2\right]}{2t(1-b-a)},\
        D_B(p_A,p_B)=1-D_A(p_A,p_B)
    \end{aligned}
    \nonumber
\end{equation}
Then, the equilibrium prices given $a$ and $b$ are
\begin{equation}
    \begin{aligned}
        p_A^*=t(1-a-b)\left(1+\frac{a-b}{3}\right)\\
        p_B^*=t(1-a-b)\left(1+\frac{b-a}{3}\right)
    \end{aligned}
    \nonumber
\end{equation}
where $c:=0$. The corresponding profits are
\begin{equation}
    \begin{aligned}
        \Pi_A(a,b)&=p_A^*D_A^*=\left(1+\frac{a-b}{3}\right)\frac{t(1-b-a)(1-b+a)}{2}\\
        \Pi_B(a,b)&=p_B^*D_B^*=\left(1+\frac{b-a}{3}\right)\frac{t(1-b-a)(1-a+b)}{2}
    \end{aligned}
    \nonumber
\end{equation}
\begin{equation}
    \begin{aligned}
        \frac{\partial \Pi_A(a,b)}{\partial a}=\underbrace{(p_A^*-c)\frac{\partial D_A^*}{\partial a}}_\textnormal{direct effect $>0$}+\underbrace{\frac{\partial D_A^*}{\partial p_A^*}}_{=0}\frac{\partial p_A^*}{\partial a}+\underbrace{(p_A^*-c)\frac{\partial D_A^*}{\partial p_B^*}\frac{\partial p_B^*}{\partial a}}_\textnormal{strategic effect $<0$}
    \end{aligned}
    \nonumber
\end{equation}
Which effect dominates depends on the model. In this model, the strategic effect dominates the direct effect. That is, $a=b=0$. (If allowing negative values, $a=b=-\frac{1}{4}$.)

\begin{note}
    If the transportation cost is $tx$, the equilibrium may not exist.
\end{note}


Other models:
\begin{enumerate}
    \item vertical differentiation (S. Moorthy);
    \item defender model (John Hauser);
    \item logit/limited defender model, $u_{ij}=v_j - \alpha p_j+\epsilon_{ij}$, where $\epsilon_{ij}\sim EVI$. The outside option is modeled as $u_{i0}=\epsilon_{i0}\sim EVI$. The market share is given as
    \begin{equation}
        \begin{aligned}
            s_{j}=\textnormal{Pr}(u_{ij}\geq u_{ij'},\forall j')=\frac{\exp(v_j - \alpha p_j)}{1+\sum_{j'}\exp(v_{j'}-\alpha p_{j'})}
        \end{aligned}
        \nonumber
    \end{equation}
    The demand is given by
    \begin{equation}
        \begin{aligned}
            D_j=s_j\cdot \textnormal{Market Size}
        \end{aligned}
        \nonumber
    \end{equation}
    Estimated by taking inversion,
    \begin{equation}
        \begin{aligned}
            \ln s_j - \ln s_0 = v_j - \alpha p_j
        \end{aligned}
        \nonumber
    \end{equation}
\end{enumerate}

\textit{Proliferation of Products to Deter Entry}: Entry with products locating uniformly to deter entry.

\textit{Spoke Model}.


\chapter{Search}
\section{Optimal Search}
\subsection{Individual Choice}
\begin{enumerate}
    \item \textit{Simple example}, where the cost of search for each price is $c$ and a consumer buys one or zero unit.
    \begin{definition}[Optimal Stopping Rule]
        The optimal rule of searching is an \textbf{optimal stopping rule}: Stop and buy once the consumer finds a price less or equal to a reservation price, $R$.
    \end{definition}
    The $R$ is constructed as the critical value when the expected return from an extra search equals to the marginal cost:
    \begin{equation}
        \begin{aligned}
            c&=\mathbb{E}[R-p\mid p<R] \textnormal{Pr}(p<R)\\
            &=\int_0^R(R-p)f(p)dp
        \end{aligned}
        \nonumber
    \end{equation}
    where $f(\cdot)$ is the density distribution of prices.
    \item \textit{Consuming multiple units (general case, the one unit case can be modeled by assuming the demand function)}, Remind that the consumer surplus given price $p$ is given as
    \begin{equation}
        \begin{aligned}
            s(p)=\max_q\left\{U(q)-p\cdot q\right\}
        \end{aligned}
        \nonumber
    \end{equation}
    and its derivatives are
    \begin{equation}
        \begin{aligned}
            s'(p)=-D(p),\ s''(p)=-D'(p)>0 \textnormal{ (convexity)}
        \end{aligned}
        \nonumber
    \end{equation}
    In this case, the optimal stopping rule $R$ that maximizing the consumer surplus is given as
    \begin{equation}
        \begin{aligned}
            c=\int_0^R[s(p)-s(R)]f(p)dp
        \end{aligned}
        \nonumber
    \end{equation}
    \begin{note}
        Given a greater variance of the price distribution, the $R$ decreases.
    \end{note}
\end{enumerate}

\subsection{Homogeneous Markets}
Can we find a fixed point of search behavior (i.e., $f(p)$)?

\begin{assumption}
    All firms are identical with marginal cost $c_f$. All consumers are identical with search costs $c$.
\end{assumption}
Define the ``monopoly price'':
\begin{equation}
    \begin{aligned}
        p^M=\argmax_{p}\left\{p\cdot D(p)-c_fD(p)\right\}
    \end{aligned}
    \nonumber
\end{equation}
\begin{assumption}
    The consumers want to visit the first store in equilibrium, $s(p^M)>c$.
\end{assumption}

\begin{theorem}[Diamond Theorem]
    The unique equilibrium is for all firms to charge monopoly price $p^M$ and consumers do not search.
\end{theorem}
\begin{proof}[Sketch]
    Firstly, we can prove this proposed equilibrium exists: all monopoly prices $\Rightarrow$ no search; no search $\Rightarrow$ all monopoly prices.

    Secondly, we prove the uniqueness: Given $f(p)$, the corresponding reservation price of search is $R$. All firms charge $p=R$ and the consumers do not search.
    \begin{enumerate}
        \item Firstly, no firms should charge below $R$: if $p<R$, the consumer purchases once he visits the store. So, there is no firm charging $p<R$.
        \item Secondly, we prove no firms charge above $R$: Suppose by the way of contradiction that there is a firm charging $p=R+\epsilon$. Once a consumer visits the store, the consumer's highest surplus from an extra search is $s(R)-s(R+\epsilon)-c$. There always exists an $\epsilon>0$ such that purchases at $R+\epsilon$ is profitable for the consumer.
    \end{enumerate}
    Therefore, the consumers do not search, and then all firms charge $p^M$.
\end{proof}

\subsection{Solutions for Diamond Paradox}
\begin{enumerate}
    \item Firms are different.
    \begin{enumerate}
        \item Different costs, which give different $p^M$.
        \item Inflation + menu costs.
    \end{enumerate}
    \item Consumers are different.
    \begin{enumerate}
        \item Different search costs.
        \begin{enumerate}
            \item If mass zero at zero search cost, then everything is the same.
            \item If no mass zero at zero, then equal breaks down.
        \end{enumerate}
        \item Different preferences \cite{choi2018consumer}.
        \item Lack of common knowledge \cite{kuksov2006search}.
    \end{enumerate}
\end{enumerate}

\subsection{\cite{choi2018consumer}}
Suppose the utility of consumer from consuming product $j$ is
\begin{equation}
    \begin{aligned}
        U(v_j,z_j,p_j,N)=v_j+z_j-p_j-\underbrace{\sum_{k\in N}s_k}_\textnormal{search costs based on his search history}
    \end{aligned}
    \nonumber
\end{equation}
where $(v_j,z_j)$ are the parameters of the consumer that reflects his preference, h $v_j$ is known before search, $z_j$ is learning during search, $p_j$ is the price, $N$ is set of the products searched by the consumer, $s_k$ is the search cost of product (firm) $k$.

\begin{assumption}
    $v_j\sim F_j\in \Delta[\underline{v}_j,\bar{v}_j]$ and $z_j\sim G_j\in \Delta[\underline{z}_j,\bar{z}_j]$.
\end{assumption}

\begin{definition}[Gittins index]
    Weitzman index (Gittins index) $z_j^*$ is given by solving
    \begin{equation}
        \begin{aligned}
            s_j=\int_{z_j^*}^{\bar{z}_j}[(1-G_j(z_j))]dz_j
        \end{aligned}
        \nonumber
    \end{equation}
    which is equivalent to $s_j=\int_{z_j^*}^{\bar{z}_j}(z_j-z_j^*)g(z_j) dz_j$ (integration by part).
\end{definition}
\begin{remark}
    Upper Confidence Bound (UCB) Algorithm
\end{remark}
\begin{proposition}[Optimal Search Policy]
    \textit{The optimal policy}: Visit stores in descending order of $v_j+z_j^*-p_j$ and stop if
    \begin{equation}
        \begin{aligned}
            \max\left\{u_0,\max_{j\in N}v_j+z_j-p_j\right\}>\max_{j\notin N} v_j+z_j^*-p_j
        \end{aligned}
        \nonumber
    \end{equation}
    where $u_0$ is the utility from the outside option.
    \begin{note}
        Since it is descending order of $v_j+z_j^*-p_j$, stop after visiting $i$ if $z_i\geq z_i^*$ holds.
    \end{note}

    Define $w_i:= v_i+\min\{z_i,z_i^*\}$. Then, shopping outcome is that consumer buys product $i$ \underline{iff}
    \begin{enumerate}
        \item $w_i-p_i>u_0$ and
        \item $w_i-p_i>w_j-p_j$ for all $j\neq i$.
    \end{enumerate}
    \begin{note}
        Latent utility framework that maximizes $w_i-p_i$.
    \end{note}
\end{proposition}
\begin{proof}
    Prove the sufficiency: Given $w_i-p_i> v_0$,
    \begin{equation}
        \begin{aligned}
            w_i-p_i> v_0 \Rightarrow v_i+z_i^*-p_i> u_0
        \end{aligned}
        \nonumber
    \end{equation}
    By the search policy, we must have $v_i+z_i-p_i>u_0$.

    Given $w_i-p_i>w_j-p_j$.
    \begin{enumerate}
        \item If $z_j^*\leq z_j$, we have $w_j=v_j+z_j^*-p_j$. As $v_i+z_i^*-p_i\geq w_i-p_i>w_j-p_j=v_j+z_j^*-p_j$ and the order is descending in $v_j+z_j^*-p_j$, $i$ must be visited before $j$. The consumer does not want to visit $j$ after visit $i$ because $v_i+z_i-p_i\geq w_i-p_i>v_j+z_j^*-p_j$.
        \item If $z_j^*>z_j$, ...
    \end{enumerate}
    Prove the necessary:
\end{proof}
\paragraph*{Solving the Equilibrium}
Let $H_i(\cdot)$ be the CDF of $w_i:=v_i+\min\{z_i,z_i^*\}$.
\begin{equation}
    \begin{aligned}
        H_i(w_i):=\int_{\underline{z}_i}^{z_i^*}F_i(w_i-z_i)d G(z_i)+\int_{z_i^*}^{\bar{z}_i}F_i(w_i-z_i^*)d G(z_i)
    \end{aligned}
    \nonumber
\end{equation}
The best alternative is defined as $x_i=\max\{u_0,\max_{j\neq i}w_j-p_j\}$. Its CDF is given as
\begin{equation}
    \begin{aligned}
        \tilde{H}(x)=\textnormal{Prob}[x_i\leq x]
    \end{aligned}
    \nonumber
\end{equation}
Note that $\tilde{H}(x)$ depends on $p_{-i}$ but not $p_i$.

Therefore, the demand for the product $i$ given the price vector $p$ can be given as
\begin{equation}
    \begin{aligned}
        D_i(p)=\int\underbrace{[1-H_i(x_i+p_i)]}_{\textnormal{Prob}[w_i-p_i\geq x_i]}d\tilde{H}(x_i)
    \end{aligned}
    \nonumber
\end{equation}
Thus, the optimization problem of pricing is given by
\begin{equation}
    \begin{aligned}
        \max_{p_i} (p_i-MC_i) D_i(p_i,p_{-i})
    \end{aligned}
    \nonumber
\end{equation}

\paragraph*{Main Results}
\begin{enumerate}
    \item \underline{Observable prices}: as the search costs increase $\Rightarrow$ the benefit from search decreases $\Rightarrow$ attract consumers to search first is more important $\Rightarrow$ Lower prices.
    \item \underline{Unobservable prices}: as the search costs increase $\Rightarrow$ the benefit from search increases $\Rightarrow$ try to exploit consumers $\Rightarrow$ Higher prices.
\end{enumerate}
Given \underline{more information pre-search}, there are two effects:
\begin{enumerate}
    \item Less benefit from search $\Rightarrow$ attract consumers to search first is more important $\Rightarrow$ Lower prices.
    \item More dispersed consumer preferences $\Rightarrow$ Larger differentiation among products $\Rightarrow$ Higher prices.
\end{enumerate}



\subsection{\cite{kuksov2006search}: Lack of Common Knowledge}
Suppose there are identical buyers have valuation $v$ for products (1 unit). Sellers are uncertain about the valuation $v$ and choose their own prices based on private signals.

Suppose a seller $j$ gets a signal
\begin{equation}
    \begin{aligned}
        x_j=v+\eta_j=\left\{\begin{matrix}
            v+0,& \textnormal{ with prob }\frac{1}{2}\\
            v-\delta(v),& \textnormal{ with prob }\frac{1}{2}
        \end{matrix}\right.
    \end{aligned}
    \nonumber
\end{equation}
The cost of first search is zero and the cost of sequential search is $s$.

If the signal $v$ I received is the high signal, then others who receive the low signal gets $f(v):=v-\delta(v)$

If the signal $v$ I received is the low signal, then others who receive the high signal gets $g(v):=f^{-1}(v)$

Thus, if a seller receives signal $x_j$, he believes the possible other sellers' signals are $\{f(x_j),x_j,g(x_j)\}$.

\paragraph*{Equilibrium}
Consumers are facing two kinds of prices set by firms $P(v)$ and $P(f(v))$.
\begin{enumerate}
    \item If consumers see $P(f(v))$, they stop.
    \item If consumers see $P(v)$, they buy the product if
    \begin{equation}
        \begin{aligned}
            \frac{1}{2}(P(v)-P(f(v)))\leq s
        \end{aligned}
        \nonumber
    \end{equation}
\end{enumerate}
Therefore, the equilibrium prices should satisfy
\begin{equation}
    \begin{aligned}
        P(v)=P(f(v))+2s
    \end{aligned}
    \nonumber
\end{equation}
\paragraph*{Results}
\begin{enumerate}
    \item Prices are below the monopoly price.
    \item As $s$ goes to $0$, the price $P(v)$ goes to $MC$.
    \item As $\delta(v)$ goes to zero, the price $P(v)$ goes to the monopoly price.
\end{enumerate}


\subsection{\cite{kuksov2010more}: Alternative Overload}
Consumers can either search with some costs or buy at random.

The value of consumers is defined as
\begin{equation}
    \begin{aligned}
        V(v,x,I,N)=\max\bigg\{\underbrace{0}_\textnormal{not choose an alternative},\underbrace{\max_{i\in I}(v-t|x-z_i|)}_\textnormal{choose the best checked product},\\
        \underbrace{\sum_{i\notin I}\frac{1}{N-M}\left(v-t|x-z_i|\right)}_\textnormal{randomly choose from unchecked products},\\
        \underbrace{-c+\sum_{i\notin I}\frac{1}{N-M}V\left(v,x,I\cup\{i\},N\right)}_\textnormal{(randomly) search one more}\bigg\}
    \end{aligned}
    \nonumber
\end{equation}
where $v$ is the basic value of choosing an alternative, $x$ is the location of the consumer, $I$ is the set of products checked, $M$ is the number of products checked, $N$ is the number of total products, and $z_i$ is the location of the alternative $i$.

%\paragraph*{One Alternative:} the product is placed at $z=\frac{1}{2}$, which gives disutility $\frac{t}{4}$ on average for consumers.

\paragraph*{Infinite Alternatives:} In this case the expected values from different strategies are given as
\begin{enumerate}
    \item Random purchase: $x$ gets expected disutility $d=\int_0^1 t|x-z| dz$.
    \item Search: the search strategy can be given as
    \begin{enumerate}
        \item Stop search if $z\in (x-\delta,x+\delta)$. This gives disutility $d=\int_{x-\delta}^{x+\delta}t|x-z|dz$, which is minimized for consumers located at $x=\frac{1}{2}$ who gets disutility $d=\frac{t}{4}$.
        \item Search otherwise. This requires a search cost of $c$.
    \end{enumerate}
    In equilibrium, the search costs equal to the disutility from stopping:
    \begin{equation}
        \begin{aligned}
            c=\int_{x-\delta}^{x+\delta}t|x-z|dz \Rightarrow \delta=\sqrt{\frac{c}{t}}
        \end{aligned}
        \nonumber
    \end{equation}
    The corresponding expected disutility is $$d(\infty)=t\delta=\sqrt{ct}$$
\end{enumerate}
Therefore, all consumers choose to search instead of randomly purchase if and only if $\frac{c}{t}\leq \frac{1}{16}$.


\paragraph*{Results} There can be too many products. To show this, we compare $\hat{N}$ products with infinity. ($\hat{N}$ products uniformly locate on $[0,1]$ in equilibrium).

Let's consider a strategy (not optimal): \underline{Search until find the closest alternative}: without search costs, it gives the expected disutility $\frac{t}{4\hat{N}}$ for a consumer on average.
\begin{enumerate}
    \item Without search costs, it gives the expected disutility $\frac{t}{4\hat{N}}$ for a consumer on average.
    \item The expected search cost to find this product is
    \begin{equation}
        \begin{aligned}
            c\frac{1}{\hat{N}}+2c\frac{\hat{N}-1}{\hat{N}}\frac{1}{\hat{N}-1}+\cdots \left[(N-1)c\frac{\hat{N}-1}{\hat{N}}\frac{\hat{N}-2}{\hat{N-1}}\cdots\frac{1}{2}\right]=(\hat{N}-1)\left(\frac{1}{2}+\frac{1}{\hat{N}}\right)c
        \end{aligned}
        \nonumber
    \end{equation}
\end{enumerate}
The expected disutility for a consumer on average is
\begin{equation}
    \begin{aligned}
        \frac{t}{4\hat{N}}+(\hat{N}-1)\left(\frac{1}{2}+\frac{1}{\hat{N}}\right)c
    \end{aligned}
    \nonumber
\end{equation}
its derivate of $\hat{N}$ is $-\frac{t}{4\hat{N}^2}+\left(\frac{1}{2}+\frac{1}{\hat{N}}\right)c-\frac{(\hat{N}-1)}{\hat{N}^2}c\propto -\frac{t}{4}+\left(\frac{\hat{N}^2}{2}+1\right)c$.

For some $\hat{N}\approx \frac{1}{2\sqrt{c/t}}$ such that
\begin{equation}
    \begin{aligned}
        \frac{t}{4\hat{N}}+(\hat{N}-1)\left(\frac{1}{2}+\frac{1}{\hat{N}}\right)c<\frac{3}{4}\sqrt{ct}+c\leq \sqrt{ct}<d(\infty)
    \end{aligned}
    \nonumber
\end{equation}
Therefore, there is a finite number of products $\hat{N}$ that benefits consumers compared to infinity.

\section{Search for Information}
\subsection{Multiple Attributes}
\begin{example}
    Suppose there is one attribute. The search cost is $c$. The initial value is $v=-1$. The attribute is $z=-10$ or $z=10$ with equal probability.\\
    The expected payoff from search is $\frac{9}{2}-c$. That is, search if and only if $c<\frac{9}{2}$.
\end{example}

\paragraph*{Two Attributes}
Suppose you get $v+z$ (positive) or $v-z$ (negative) after the first search. Should you search second attribute? (assume $v<0$)
\begin{enumerate}
    \item Given $v+z$, the expected payoff from searching the second attribute is
    \begin{equation}
        \begin{aligned}
            \frac{1}{2}\left(v+2z\right)+\frac{1}{2}\max\{v,0\}-c=\frac{v}{2}+z-c
        \end{aligned}
        \nonumber
    \end{equation}
    and the expected payoff from not searching is $v+z$. Therefore, search if and only if $c<-\frac{v}{2}$.
    \item Given $v-z$, the expected payoff from searching the second attribute is
    \begin{equation}
        \begin{aligned}
            \frac{1}{2}\max\{v,0\}+\frac{1}{2}\max\{v-z,0\}-c=-c
        \end{aligned}
        \nonumber
    \end{equation}
    and the expected payoff from not searching is $0$. Therefore, never search in this case.
\end{enumerate}
Therefore, the expected payoff from the first search is
\begin{equation}
    \begin{aligned}
        &\frac{1}{2}0+\frac{1}{2}\max\{\frac{v}{2}+z-c,v+z\}-c\\
        =&\max\{\frac{v}{4}+\frac{z}{2}-\frac{3c}{2},\frac{v+z}{2}-c\}
    \end{aligned}
    \nonumber
\end{equation}
Thus, search if and only if $c<\max\{\frac{v+z}{2},\frac{v+2z}{6}\}$.

\paragraph*{More Attributes}
Every search $X_i$ gives $+z$ or $-z$ $\mathbb{E}[X_i]=0$. $U=v+\sum_{i=1}^N X_i$.
After search $n$ attributes, $x_i,i=1,...,n$. The expected payoff is given by
\begin{equation}
    \begin{aligned}
        u=v+\sum_{i=1}^n x_i + \sum_{i=n+1}^N \mathbb{E}[X_i]=v+\sum_{i=1}^n x_i
    \end{aligned}
    \nonumber
\end{equation}
In a continuous form,
\begin{equation}
    \begin{aligned}
        d u=\sigma dW
    \end{aligned}
    \nonumber
\end{equation}
where $W$ is the standard Brownian motion. Let $z=\sigma\sqrt{dt}$.

At each point in search process, the decision maker chooses among: 1. Continue to search (gather information) at cost $c$ per unit of attributes; 2. Stop search and buy the product; 3. Stop search without buying the product.

The expected payoff of keeping on searching is given by
\begin{equation}
    \begin{aligned}
        V(u,t)=-cdt+\mathbb{E}[V(u+du,t+dt)]
    \end{aligned}
    \nonumber
\end{equation}
where $u$ is the current expected payoff if buying now and $r$ is the mass of attributes searched.

By the Taylor expansion,
\begin{equation}
    \begin{aligned}
        V(u+du,t+dt)=V(u,t)+V_u du+V_t dt+V_{uu}\frac{(du)^2}{2}+V_{ut}dudt+o((dt)^2)
    \end{aligned}
    \nonumber
\end{equation}
where $\mathbb{E}du=0$, $\mathbb{E}(du)^2=\sigma^2dt$, and $V_t=0$. Thus,
\begin{equation}
    \begin{aligned}
        V(u,t)&=-cdt+V(u,t)+V_u\mathbb{E}du+V_t dt+\frac{V_{uu}}{2}\mathbb{E}[(du)^2]+V_{ut}dt\mathbb{E}du\\
        &=-cdt+V(u,t)+\frac{V_{uu}}{2}\sigma^2 dt\\
    \end{aligned}
    \nonumber
\end{equation}
Therefore,
\begin{equation}
    \begin{aligned}
        V_{uu}=\frac{2c}{\sigma^2}
    \end{aligned}
    \nonumber
\end{equation}
That is the function form of $V(u)$ must be
\begin{equation}
    \begin{aligned}
        V(u)=A_2+A_1 u +\frac{c}{\sigma^2}u^2
    \end{aligned}
    \nonumber
\end{equation}
for some constants $A_1,A_2$. Now, we solve for the $A_1$ and $A_2$.

The rule of searching can be given by $\overline{U}$ and $\underline{U}$, where $\overline{U}$ is the utility that the DM buys the product if $u\geq \overline{U}$ and $\underline{U}$ is the utility that the DM stops search without buying if $u\leq \underline{U}$.
\begin{enumerate}
    \item \textit{Value matching conditions} are
    \begin{equation}
        \begin{aligned}
            V(\overline{U})=\overline{U}\\
            V(\underline{U})=0
        \end{aligned}
        \nonumber
    \end{equation}
    \item \textit{Smooth pasting conditions} are
    \begin{equation}
        \begin{aligned}
            V'(\overline{U})=1\\
            V'(\underline{U})=0
        \end{aligned}
        \nonumber
    \end{equation}
    \begin{remark}
        Intuition for $V'(\overline{U})=1$: Suppose the DM chooses to search given $u=\overline{U}$, the expected payoff is
    \begin{equation}
        \begin{aligned}
            \frac{1}{2}\left(\overline{U}+\mathbb{E}[du\mid du>0]\right)+\frac{1}{2}\mathbb{E}\left[V(\overline{U}+du)\mid du<0\right]-cdt
        \end{aligned}
        \nonumber
    \end{equation}
    where $V(\overline{U}+du)=\bar{U}+V'(\overline{U})du$, $\mathbb{E}[du\mid du>0]=\sigma\frac{\sqrt{dt}}{\sqrt{2\pi}}$, and $\mathbb{E}[du\mid du<0]=-\sigma\frac{\sqrt{dt}}{\sqrt{2\pi}}$. Then, the expected payoff is equal to
    \begin{equation}
        \begin{aligned}
            \frac{1}{2}\left(\overline{U}+\sigma\frac{\sqrt{dt}}{\sqrt{2\pi}}\right)+\frac{1}{2}\left[\bar{U}-V'(\overline{U})\sigma\frac{\sqrt{dt}}{\sqrt{2\pi}}\right]-cdt=\overline{U}+\frac{1}{2}\left(1-V'(\overline{U})\right)\sigma\frac{\sqrt{dt}}{\sqrt{2\pi}}-cdt
        \end{aligned}
        \nonumber
    \end{equation}
    By the definition of $\overline{U}$, the expected payoff must be equal to $\overline{U}$. As $dt$ has a higher order than $\sqrt{dt}$, we need to have $V'(\overline{U})=1$.
    \end{remark}
\end{enumerate}
The solution is $\overline{U}=-\underline{U}=\frac{\sigma^2}{4c}$
\begin{equation}
    \begin{aligned}
        V(u)=\frac{\sigma^2}{16c}+\frac{1}{2}u +\frac{c}{\sigma^2}u^2
    \end{aligned}
    \nonumber
\end{equation}

The \textbf{probability of purchase} is
\begin{equation}
    \begin{aligned}
        \textnormal{Prob}(v)=\frac{v-\underline{U}}{\overline{U}-\underline{U}}=\frac{1}{2}+\frac{2cv}{\sigma^2}
    \end{aligned}
    \nonumber
\end{equation}
The \textbf{expected number of attributes searched}:
\begin{equation}
    \begin{aligned}
        &V(v)=\textnormal{Prob}(v)\overline{U}-c \mathbb{E}[\textnormal{number of attributes searched}]\\
        \Rightarrow &\mathbb{E}[\textnormal{number of attributes searched}]=\frac{\textnormal{Prob}(v)\overline{U}-V(u)}{c}=\frac{\sigma^2}{16c^2}-\frac{v^2}{\sigma^2}
    \end{aligned}
    \nonumber
\end{equation}

\textbf{Optimal Price}: Suppose the marginal cost is $g$. The $v$ becomes $v-p$. That is probability of purchase with $v-p\in [\underline{U},\overline{U}]$ is
\begin{equation}
    \begin{aligned}
        \textnormal{Prob}(v)=\frac{1}{2}+\frac{2c(v-p)}{\sigma^2}
    \end{aligned}
    \nonumber
\end{equation}
Since we always make $v-p\in [\underline{U},\overline{U}]$, the profit is given by
\begin{equation}
    \begin{aligned}
        \Pi=(p-g)\textnormal{Prob}(v)=(p-g)\left[\frac{1}{2}+\frac{2c(v-p)}{\sigma^2}\right]
    \end{aligned}
    \nonumber
\end{equation}
\begin{enumerate}
    \item If $v$ is extremely high ($v>g+3\overline{U}$),
    \begin{equation}
        \begin{aligned}
            p^*=v-\overline{U}\\
            \Pi^*=v-\overline{U}-g
        \end{aligned}
        \nonumber
    \end{equation}
    \item If $v$ is not extremely high,
    \begin{equation}
        \begin{aligned}
            p^*=\frac{1}{2}\left(v+\overline{U}+g\right)\\
            \Pi^*=\frac{(v-g+\overline{U})^2}{8\overline{U}}
        \end{aligned}
        \nonumber
    \end{equation}
\end{enumerate}
Intuition about surplus, as $c$ decreases $\Rightarrow$ $p$ can increase $\Rightarrow$ consumer surplus can fall.
\subsection{Discounting}
Consider the discounting, the expected payoff of keeping on searching is given by
\begin{equation}
    \begin{aligned}
        V(u)=-cdt+e^{-\pi dt}\mathbb{E}[V(u+du)]
    \end{aligned}
    \nonumber
\end{equation}
By Taylor's expansion, we have
\begin{equation}
    \begin{aligned}
        V(u)=-cdt+e^{-\pi dt}[V(u)+\frac{\sigma^2}{2}V_{uu}(u)dt]\\
        V(u)\frac{1-e^{-\pi dt}}{dt}=-c+e^{-\pi dt}\frac{\sigma^2}{2}V_{uu}(u)
    \end{aligned}
    \nonumber
\end{equation}
As $dt \rightarrow 0$,
\begin{equation}
    \begin{aligned}
        V(u)\pi=-c+\frac{\sigma^2}{2}V_{uu}(u)
    \end{aligned}
    \nonumber
\end{equation}
Then, we can solve
\begin{equation}
    \begin{aligned}
        V(\mu)=A_1\exp\left(\frac{2\pi}{\sigma^2}\mu\right)+A_2\exp\left(-\frac{2\pi}{\sigma^2}\mu\right)-\frac{c}{\pi}
    \end{aligned}
    \nonumber
\end{equation}
We also have the same conditions that
\begin{enumerate}
    \item \textit{Value matching conditions} are
    \begin{equation}
        \begin{aligned}
            V(\overline{U})=\overline{U}\\
            V(\underline{U})=0
        \end{aligned}
        \nonumber
    \end{equation}
    \item \textit{Smooth pasting conditions} are
    \begin{equation}
        \begin{aligned}
            V'(\overline{U})=1\\
            V'(\underline{U})=0
        \end{aligned}
        \nonumber
    \end{equation}
\end{enumerate}
Thus, we can solve that
\begin{equation}
    \begin{aligned}
        \overline{U}=\sqrt{\frac{\sigma^2}{\pi^2}+\frac{\sigma^2}{2\pi}}-\frac{c}{\pi}
    \end{aligned}
    \nonumber
\end{equation}
Intuition: discounting $\Rightarrow$ search less, $\overline{U}$ decreases and $\underline{U}$ increases.

In previous case, we have $|\overline{U}|=|\underline{U}|$. However, in this case, $|\overline{U}|<|\underline{U}|$ (because the option of not purchasing, the consumer is more likely to search with discounting at low value).

As $c \rightarrow 0$, $\underline{U} \rightarrow -\infty$. If the outside option is positive (i.e. there exists a cost from receiving outside option later), $\underline{U} \nrightarrow -\infty$.

\subsection{Signals of Value of Products (heterogeneous attributes)}
In this case, we have
\begin{equation}
    \begin{aligned}
        d u = \sigma_t dW,
    \end{aligned}
    \nonumber
\end{equation}
where $\sigma_t$ changes over time.

Consider the case of normal learning, that is the signal $s_i=U+\epsilon_i$, where $U\sim \mathcal{N}\left(v,e^2\right)$ and $\epsilon_i\sim \mathcal{N}\left(0,s^2\right)$. In discrete form, the posterior belief of $u$ after receiving $t$ number of signals is a normal distribution with
\begin{equation}
    \begin{aligned}
        u(t)=\frac{\frac{v}{e^2}+\sum_{j=1}^t\frac{s_j}{s^2}}{\frac{1}{e^2}+\frac{t}{s^2}}, \sigma(t)^2=\frac{1}{\frac{1}{e^2}+\frac{t}{s^2}}
    \end{aligned}
    \nonumber
\end{equation}
$\sigma_t$ in continuous form is induced by the equations above.

Most induction is similar to previous case, but the $V$ is related to $t$ this time. Then, we have
\begin{equation}
    \begin{aligned}
        -c+V_t(u,t)+\frac{\sigma_t^2}{2}V_{uu}(u,t)=0
    \end{aligned}
    \nonumber
\end{equation}
Combing with
\begin{enumerate}
    \item \textit{Value matching conditions} are
    \begin{equation}
        \begin{aligned}
            V(\overline{U}_t,t)=\overline{U}_t\\
            V(\underline{U}_t,t)=0
        \end{aligned}
        \nonumber
    \end{equation}
    \item \textit{Smooth pasting conditions} are
    \begin{equation}
        \begin{aligned}
            V'(\overline{U}_t,t)=1\\
            V'(\underline{U}_t,t)=0
        \end{aligned}
        \nonumber
    \end{equation}
\end{enumerate}
Intuition: as time goes, the marginal information can be obtained is less, so $\overline{U}_t$ and $\underline{U}_t$ converges to zero.


\subsection{Finite Mass of Attributes: $T$}
In this case, $V(u,T)=\max\{u,0\}$. We have $\overline{U}(T)=\underline{U}(T)=0$.
\begin{equation}
    \begin{aligned}
        -c+V_t(u,t)+\frac{\sigma^2}{2}V_{uu}(u,t)=0
    \end{aligned}
    \nonumber
\end{equation}


\subsection{Finite Mass and Heterogeneous Attributes}


\subsection{Choosing Search Intensity}
Suppose the consumer can choose the search intensity, $\sigma^2$, with search cost $c(\sigma^2)$ such that $c'>0, c''>0$.

With discounting, we have
\begin{equation}
    \begin{aligned}
        \pi V = -c + \frac{\sigma^2}{2}V_{uu}
    \end{aligned}
    \nonumber
\end{equation}
Thus, $\sigma^2V_{uu}$ is increasing in $u$.
\begin{equation}
    \begin{aligned}
        \max_{\sigma^2} V
    \end{aligned}
    \nonumber
\end{equation}
The F.O.C. of this problem is
\begin{equation}
    \begin{aligned}
        c'(\sigma^2)=\frac{V_{uu}}{2}
    \end{aligned}
    \nonumber
\end{equation}
Therefore, the $\sigma^2$ is increasing in $u$.

\subsection{Information Overload}
\begin{assumption}
    \begin{enumerate}
        \item The firm decides how much information to provide, $T$.
        \item Information is not structured.
        \item Given a certain amount of information, the firm provides the more relevant information.
    \end{enumerate}
\end{assumption}
Let $\sigma_i$ denote the importance of attribute $i$, which is decreasing in $i$ and $\sigma_{i \rightarrow \infty}\sigma_i=0$. Let the average of $\sigma$ over $T$ be
\begin{equation}
    \begin{aligned}
        \bar{\sigma}_T=\frac{1}{T}\int_0^T\sigma_i d i,
    \end{aligned}
    \nonumber
\end{equation}
which is decreasing in $T$. The $T\bar{\sigma}_T$ is the total information perceived, which is increasing in $T$.

\underline{Stationary}: Each attribute has a constant hazard rate (with parameter $1$) of running out. With information available $T$: Probability of running out of attribute $T\sigma$ check is $\frac{dt}{T}$. (Consider an attribute I am checking, I do not know whether the next marginal checking is still available. The attribute is ``running' out' if the next marginal checking is not available.)

\begin{equation}
    \begin{aligned}
        V(u)&=-cdt+\frac{dt}{T}\max\{u,0\}+\left(1-\frac{dt}{T}\right)\mathbb{E}[u+du]\\
        &=-cdt+\frac{dt}{T}\max\{u,0\}+\left(1-\frac{dt}{T}\right)\left(V(u)+\frac{\bar{\sigma}_T^2}{2}V_{uu}dt\right)
    \end{aligned}
    \nonumber
\end{equation}
Then, we have
\begin{equation}
    \begin{aligned}
        -cT+\max\{u,0\}-V+\frac{T\bar{\sigma}_T^2}{2}V_{uu}=0
    \end{aligned}
    \nonumber
\end{equation}
\begin{enumerate}
    \item \textit{Value matching conditions} are
    \begin{equation}
        \begin{aligned}
            V(\overline{U})&=\overline{U}\\
            V(\underline{U})&=0\\
            V(0^+)&=V(0^-)
        \end{aligned}
        \nonumber
    \end{equation}
    \item \textit{Smooth pasting conditions} are
    \begin{equation}
        \begin{aligned}
            V'(\overline{U})&=1\\
            V'(\underline{U})&=0\\
            V'(0^+)&=V'(0^-)
        \end{aligned}
        \nonumber
    \end{equation}
\end{enumerate}
We have $\overline{U}=-\underline{U}$ with $\overline{U}$ increases in $T$ firstly and then decreases.

With initial value $v<0$. As the $T$ increases, the purchase probability increases firstly and the decreases.

As the initial value ($v<0$) close to zero, the optimal information $T$ increases.

As the $T$ increases, the payoff of the consumer $V(v)$ increases firstly and then decreases.

The optimal information provided by the firm is more than the optimal information for the consumer.

\subsection{Search for information of Multiple Products}
Suppose there are two alternatives. The consumer can
\begin{enumerate}
    \item search information for alternative 1,
    \item search information for alternative 2,
    \item purchase alternative 1,
    \item purchase alternative 2,
    \item exit without purchases.
\end{enumerate}
The optimal policy is not going to be a (single) index-based policy (what we do in multi-armed bandit, Gittins index policy). Gittins index: $K_i:=V(I_i,K_i)$: Value of playing arm $i$ with information $I_i$ with the possibility of exiting and getting $K_i$.
\begin{claim}
    Single index-based policy may not be optimal in search problem.
\end{claim}

\begin{example}[ (An example that Gittins index policy is not optimal, Bergman)]
    There are two alternatives, $A$ and $B$. The initial expected values are $v_A=10$ and $v_B=4$.
    \begin{enumerate}
        \item The first search: no learning.
        \item The second search: $A$ becomes $20$ or $0$ with equal probability. $B$ becomes $18$ or $-10$ with equal probability.
    \end{enumerate}
    The search cost of each search is $c=1$.
    \begin{enumerate}[(i).]
        \item \textbf{Gittins index policy}: Gittins indexes are given by
        \begin{equation}
            \begin{aligned}
                K_A=\underbrace{-2}_\textnormal{search cost}+\frac{1}{2}20+\frac{1}{2}K_A \Rightarrow K_A^*=16
            \end{aligned}
            \nonumber
        \end{equation}
        \begin{equation}
            \begin{aligned}
                K_B=\underbrace{-2}_\textnormal{search cost}+\frac{1}{2}18+\frac{1}{2}K_B \Rightarrow K_B^*=14
            \end{aligned}
            \nonumber
        \end{equation}
        Since $K_A^*>K_B^*$, we search $A$ first. The expected payoff is
        \begin{equation}
            \begin{aligned}
                -2+\frac{1}{2}20+\frac{1}{2}\left(-2+\frac{1}{2}18+\frac{1}{2}0\right)=11.5
            \end{aligned}
            \nonumber
        \end{equation}
        \item \textbf{Optimal Policy}: Search $B$ first is better. The expected payoff is
        \begin{equation}
            \begin{aligned}
                -2+\frac{1}{2}18+\frac{1}{2}10=12
            \end{aligned}
            \nonumber
        \end{equation}
    \end{enumerate}
\end{example}

\paragraph*{Continuous Search}
Suppose the expected value of product $i$ from searching follows $d\mu_i=\sigma dW_i$. The search phase depends on the value function of keep searching.
\begin{equation}
    \begin{aligned}
        V(\mu_1,\mu_2)=-cdt+\max\left\{\mathbb{E}_{d\mu_1}\left[V(\mu_1+d\mu_1,\mu_2)\right]+\mathbb{E}_{d\mu_2}\left[V(\mu_1,\mu_2+d\mu_2)\right]\right\}
    \end{aligned}
    \nonumber
\end{equation}
By the Taylor expansion, we have $\mathbb{E}_{d\mu_1}\left[V(\mu_1+d\mu_1,\mu_2)\right]=V(\mu_1,\mu_2)+V_{\mu_1\mu_1}\left(\mu_1,\mu_2\right)\frac{\sigma^2}{2}dt$. Then,
\begin{equation}
    \begin{aligned}
        \frac{2c}{\sigma^2}=\max\left\{V_{\mu_1\mu_1},V_{\mu_2\mu_2}\right\}
    \end{aligned}
    \nonumber
\end{equation}
The consumer searches $A$ if $V_{\mu_1\mu_1}=\frac{2c}{\sigma^2}>V_{\mu_2\mu_2}$.


The optimal search policy can be defined as $\mu_i=\overline{U}_i(\mu_j)$ (Buy $i$), $\mu_i=\underline{U}_i(\mu_j)$ (Buy $j$ or zero).

The value matching moment is
\begin{equation}
    \begin{aligned}
        V(\mu_1,\mu_2)\bigg|_{\mu_i=\overline{U}_i(\mu_j)}=\overline{U}_i(\mu_j)\\
        V(\mu_1,\mu_2)\bigg|_{\mu_i=\underline{U}_i(\mu_j)}=\max\{0,\mu_j\}\\
    \end{aligned}
    \nonumber
\end{equation}
Smooth pasting moment is
\begin{equation}
    \begin{aligned}
        V_{\mu_1}(\mu_1,\mu_2)\bigg|_{\mu_1=\overline{U}_1(\mu_2)}=1\\
        V_{\mu_1}(\mu_1,\mu_2)\bigg|_{\mu_2=\underline{U}_2(\mu_1)}=0
    \end{aligned}
    \nonumber
\end{equation}
...
The solutions are
\begin{equation}
    \begin{aligned}
        \underline{U}(\mu)=-\frac{\sigma^2}{4c}, \overline{U}(\mu)=...\geq \frac{\sigma^2}{4c}
    \end{aligned}
    \nonumber
\end{equation}

Observations:
\begin{enumerate}
    \item Search alternative with the highest $\mu_i$.
    \item Endogenous consideration sets.
    \item Require more information of an alternative if $\mu$ of the other alternative is high.
    \item Purchase thresholds $\overline{U}(\mu)$ narrow as $\mu$'s increase and converge to $\frac{\sigma^2}{4c}$.
    \item The expected utility is increasing in $\sigma^2$ and decreasing in $c$.
\end{enumerate}
\underline{Purchase Probability:} The probability of purchase alternative $i$ is increasing in $\mu_i$. The probability of purchase one of the alternatives can be decreased by increasing one of the $\mu_i$ (which induces more searches).

\underline{Number of Alternatives:} For $N=2$, the thresholds converge to $\frac{\sigma^2}{4c}$. For $N=3$, the thresholds converge to $\frac{4}{3}\frac{\sigma^2}{4c}$. For $N=\infty$, the thresholds converge to $2\frac{\sigma^2}{4c}$.

\underline{Heterogeneous Search Cost or $\sigma^2$:} Search more (larger thresholds) for the alternative with higher $\sigma^2$ or lower cost.



























\chapter{Dynamic Competition}
\section{Competition with Multiple Products}
\subsection{Bertrand Supertraps}
Products interact on demand and supply.
\begin{equation}
    \begin{aligned}
        \frac{P-MC}{P}=\frac{1}{E}
    \end{aligned}
    \nonumber
\end{equation}
If products are complementary, $P$'s are lower. $\frac{P-MC}{P}<\frac{1}{E}$. In competition, the complementary properties decrease the prices, giving lower expected profits. (Bertrand Supertraps)

Consider two products $A$ and $B$.\\
\underline{No Bundling}: Reminds that in Bertrand the price is $p_A=p_B=c+t$, where $t$ is the parameter for transportation cost. The profits are $\Pi_A=\Pi_B=\frac{t}{2}$ in both A and B markets. The total profit of a firm is $\Pi_A+\Pi_B=t$.

\underline{Bundling:} A and B sold together. In this case, the bundled price is $p_{BD}=2c+t$ with the total profit being $\Pi_{BD}=\frac{t}{2}$. (Bundling induces more intensive competition, giving lower profits).

\subsection{Two-Sided Markets}
Interaction between two end users: Sellers (Suppliers) (S) and Buyers (B).
\begin{example}
    Video Games: In the platform, buyers (gamers) buy games and sellers (game publishers) sell games.\\
    OS: App developers and users.\\
    Advertising: Viewers and Ad companies in media.
\end{example}

There exists membership charges $A^S,A^B$ that gives membership externalities (benefits/hurts the other part) and usage charges $a^S,a^B$ that gives usage externalities.

\paragraph*{Monopoly Platform}
Suppose there is a fixed cost $C_i$ per member on side $i$ of market and a marginal cost $c$ per interaction between 2 members. The average benefit per transaction is $b_i$ and the fixed benefit of membership is $B_i$.
\begin{assumption}
    Transaction involves no payments between end-users.
\end{assumption}
Let $N^S$ and $N^B$ be the numbers of sellers and buyers, respectively. The number of potential transactions: $N^B\cdot N^S$. The expected payoff of agent $i$ is
\begin{equation}
    \begin{aligned}
        U^i=(b^i-a^i)N^j+B^i-A^i
    \end{aligned}
    \nonumber
\end{equation}
where $N^j=\textnormal{Prob}\left(U^j\geq 0\right)$.

Define ``per-interaction price'' as
\begin{equation}
    \begin{aligned}
        P_i:=a^i+\frac{A^i-C^i}{N^j}
    \end{aligned}
    \nonumber
\end{equation}
\begin{equation}
    \begin{aligned}
        U^i\geq 0\Leftrightarrow (b^i-a^i)N^j+B^i-A^i\geq 0\\
        \Leftrightarrow b_i+\frac{B^i-C^i}{N^j}\geq P_i
    \end{aligned}
    \nonumber
\end{equation}
Thus,
\begin{equation}
    \begin{aligned}
        N^i=\textnormal{Prob}\left(U^i\geq 0\right)=\textnormal{Prob}\left(b_i+\frac{B^i-C^i}{N^j}\geq P_i\right):=D^i\left(P_i,N^j\right)
    \end{aligned}
    \nonumber
\end{equation}
Similarly, we can have
\begin{equation}
    \begin{aligned}
        \left\{\begin{matrix}
            N^S&=D^S\left(P_S,N^B\right)\\
            N^B&=D^B\left(P_B,N^S\right)
        \end{matrix}\right. \Rightarrow \left\{\begin{matrix}
            N^S&=n^S\left(P_S,P_B\right)\\
            N^B&=n^B\left(P_S,P_B\right)
        \end{matrix}\right.
    \end{aligned}
    \nonumber
\end{equation}
The platform's profit is given by
\begin{equation}
    \begin{aligned}
        \Pi&=\left(A^B-C^B\right)N^B+\left(A^S-C^S\right)N^S+(a^B+a^S-c)N^SN^B\\
        &=\left(P_B+P_S-c\right)n^S\left(P_S,P_B\right)n^B\left(P_S,P_B\right)
    \end{aligned}
    \nonumber
\end{equation}

\section{Dynamic Competition}
\begin{enumerate}
    \item Detection lags incurs more competition.
    \item Firm asymmetry leads to more competition.
    \item Multi-market compact may increase the collusion.
    \item Larger number of competitors / Longer horizon increases the competition.
\end{enumerate}


Suppose there are two firms with marginal cost $c$ selling homogeneous products. The demand follows
\begin{equation}
    \begin{aligned}
        D_i(p_i,p_j)=\left\{\begin{matrix}
            D_i(p_i),& p_i<p_j,\\
            \frac{1}{2}D_i(p_i),& p_i=p_j,\\
            0,& p_i>p_j
        \end{matrix}\right.
    \end{aligned}
    \nonumber
\end{equation}
Suppose there is a finite time $T$. The payoff of firm $i$ is
\begin{equation}
    \begin{aligned}
        \sum_{t=1}^T\delta^{t-1} \Pi^i(p_{it},p_{jt})
    \end{aligned}
    \nonumber
\end{equation}
($\delta$ can be written as $\delta=\frac{1}{1+r}$).

When deciding $p_{it}$, the firm $i$ knows history until $t-1$.

$T=1$ (only one period): it is exactly the static case. $p_1=p_2=c$.

For any finite $T$ (finite periods): Since we must have $p_{1T}=p_{2T}=c$ in the period T, we must have $p_{1t}=p_{2t}=c$ in any period.

\subsection{Infinite Periods}
$p_{1t}=p_{2t}=c$ is still an equilibrium.

Another equilibrium can be achieved by the strategy: Collude at monopoly price $p^M:=\argmax_{p}(p-c)D(p)$ (denote the total profit in each period as $\Pi^M:=\max_{p}(p-c)D(p)$) and move to $p=c$ if the competitor deviates. The existence of this equilibrium requires
\begin{equation}
    \begin{aligned}
        \sum_{t=1}^\infty\delta^{t-1}\frac{\Pi^M}{2}\geq \Pi^M
        \Leftrightarrow \delta\geq \frac{1}{2}
    \end{aligned}
    \nonumber
\end{equation}
(Folk Theorem)

\paragraph*{$N$ Competitors}
In the case of $N$ competitors, the condition becomes
\begin{equation}
    \begin{aligned}
        \sum_{t=1}^\infty\delta^{t-1}\frac{\Pi^M}{N}\geq \Pi^M
        \Leftrightarrow \delta\geq 1-\frac{1}{N}
    \end{aligned}
    \nonumber
\end{equation}

\paragraph*{Detection Lags}
The deviation is not detected at once, suppose after two periods. Then the condition becomes
\begin{equation}
    \begin{aligned}
        \sum_{t=1}^\infty\delta^{t-1}\frac{\Pi^M}{2}\geq (1+\delta)\Pi^M
        \Leftrightarrow \delta\geq \frac{1}{\sqrt{2}}
    \end{aligned}
    \nonumber
\end{equation}

\paragraph*{Barriers to Entry}

\paragraph*{Differentiation} Higher differentiation makes collusion less likely, as it decreases the punishment.

\paragraph*{Fluctuations in Demand}
When demand is high, firms have a higher temptation to deviate $\Rightarrow$ Price Wars during booms.

\paragraph*{Secrete Price Cuts with Demand Fluctuations} Suppose the prices are not observed but the demand fluctuates. Given a decrease in sales, you do not know whether it is incurred by the competitor's deviation or the fluctuation of the demand. Firms give punishment in some degree $\Rightarrow$ Price wars during recessions.

\paragraph*{Cost Asymmetries}

\paragraph*{Multi-Market Contract}
Suppose there are two markets: Market A meets every period and Market B meet every two periods.

As we know the collusion occurs in A if $\delta\geq \frac{1}{2}$ and in B if $\delta\geq \frac{1}{\sqrt{2}}$, if $A$ and $B$ are separated.

If we consider these two markets together, the condition is
\begin{equation}
    \begin{aligned}
        \sum_{t=1}^\infty\delta^{t-1}\frac{\Pi^M}{2}+\sum_{t=1}^\infty\delta^{2t-2}\frac{\Pi^M}{2}\geq 2\Pi^M
    \end{aligned}
    \nonumber
\end{equation}
Therefore, there exists $\delta\in \left[\frac{1}{2},\frac{1}{\sqrt{2}}\right]$ that can make collusion in both markets.


\subsection{Empirical Analysis}
Suppose the demand in time $t$ at market $s$ is given by
\begin{equation}
    \begin{aligned}
        P_{ts}=f\left(q_{1ts}+q_{2ts},Z_{ts}\right)
    \end{aligned}
    \nonumber
\end{equation}
where $q$'s are quantities and $Z$ is IV.

The cost of production is given by
\begin{equation}
    \begin{aligned}
        C_{its}=F_{its}+C^{VC}\left(q_{its},w_{ts}\right)
    \end{aligned}
    \nonumber
\end{equation}
where $F_{its}$ is the fixed cost, $C^{VC}(\cdot)$ is the variance cost, and $w_{ts}$ are prices of inputs.
\begin{enumerate}
    \item In perfect competition, $P_{ts}=MC_{its}$.
    \item In perfect collusion, $(q_{1ts},q_{2ts})=\argmax P_{ts}Q_{ts}-C_{1ts}-C_{2ts}$, where $Q_{ts}=q_{1ts}+q_{2ts}$. The F.O.C. is
    \begin{equation}
        \begin{aligned}
            P_{ts}+Q_{ts}\frac{\partial P_{ts}}{\partial q_{1ts}}-MC_{its}=0\\
            P_{ts}=MC_{its}-Q_{ts}\frac{\partial P_{ts}}{\partial q_{1ts}}
        \end{aligned}
        \nonumber
    \end{equation}
    \item Cournot competition: $P_{ts}=MC_{its}-\frac{1}{2}Q_{ts}\frac{\partial P_{ts}}{\partial q_{1ts}}$.
\end{enumerate}

Then, testing the perfect competition or collusion, can be given by estimating the $\theta$:
\begin{equation}
    \begin{aligned}
        P_{ts}=MC_{its}-\theta Q_{ts}\frac{\partial P_{ts}}{\partial q_{1ts}}
    \end{aligned}
    \nonumber
\end{equation}
\begin{enumerate}
    \item Perfect competition: $H_0: \theta=0$
    \item Perfect collusion: $H_0: \theta=1$
    \item Cournot competition: $H_0: \theta=\frac{1}{2}$
\end{enumerate}

\subsection{Market Continue with Probability $x$}
Suppose the market continues with probability $x$. The condition becomes
\begin{equation}
    \begin{aligned}
        \sum_{t=1}^\infty (x\delta)^{t-1}\frac{\Pi^M}{2}\geq \Pi^M
        \Leftrightarrow \delta\geq \frac{1}{2x}
    \end{aligned}
    \nonumber
\end{equation}

\section{Repeated Game}
\subsection{Reputation}
\begin{center}
    \begin{tabular}{ccc}
        \hline
            &$C$ &$\tilde{C}$\\
        \hline
            $C$& 1,1&-1,1.5\\
            $\tilde{C}$&1.5,-1 &0,0\\
        \hline
    \end{tabular}
\end{center}
There are two periods.

Suppose there is a probability $\frac{1}{2}$ that the player 1 is ``irrational'' such that it always chooses $C$ in the first period, and it chooses $C$ in the second period if the player 2 chooses $C$ in the first period and chooses $\tilde{C}$ in the second period if the player 2 chooses $\tilde{C}$ in the first period.

``Rational'' player may choose either $C$ or $\tilde{C}$ in the first period and always choose $\tilde{C}$ in the second period.

The payoffs of two rational players (each player has an equal probability to meet the other rational player or the irrational player) from the game is
\begin{center}
    \begin{tabular}{ccc}
        \hline
            &$C$ &$\tilde{C}$\\
        \hline
            $C$& $\frac{7}{4},\frac{7}{4}$ & $\frac{3}{4},\frac{3}{2}$\\
            $\tilde{C}$& $\frac{3}{2},\frac{3}{4}$& $\frac{3}{4},\frac{3}{4}$\\
        \hline
    \end{tabular}
\end{center}
There are two Nash equilibrium, $(C,C)$ and $(\tilde{C},\tilde{C})$.

\subsection{Multiple Equilibria}
\begin{center}
    \begin{tabular}{cccc}
        \hline
            &$p_H$ & $p_M$ & $p_L$\\
        \hline
        $p_H$& 1,1 & -1,1.5 & -2,1\\
        $p_M$& 1.5,-1& 0,0 & -2,0\\
        $p_L$& 1,-2 &0,-2 &-1,-1\\
        \hline
    \end{tabular}
\end{center}
For two periods of game, we have the following equilibrium:
\begin{enumerate}
    \item In first period, play $(p_H,p_H)$.
    \item In second period, play $(p_M,p_M)$ if $(p_H,p_H)$ is played in the first period, and play $(p_L,p_L)$ if other outcome is played in the first period.
\end{enumerate}

\section{Markov Perfect Equilibrium}
\begin{enumerate}
    \item Switching cost / loyalty effect.
    \item Consumer learning.
    \item Durable goods.
    \item Network effects.
    \item Learning by doing.
    \item Advertising dynamics.
    \item Entry with costs.
\end{enumerate}
A refinement: you can only condition actions on payoff relevant state variables. Thus, if $\Pi(a,h_1)=\Pi(a,h_2),\forall a$, we have $h_1$ and $h_2$ in the same state.

Player $i$'s payoff with action $a_t$ at time $t$ is written as $\Pi_t^i(a_t,s_t)$, where $s_t$ is the payoff relevant state at time $t$. The value function is
\begin{equation}
    \begin{aligned}
        V^i(s_t)=\max_{a^i_t}\Pi_t^i((a^i_t,a^{-i}_t),s_t)+\delta V^i(h(s_t,a_t))
    \end{aligned}
    \nonumber
\end{equation}
where $s_{t+1}=h(s_t,a_t)$ is the transition function and the solution $a^i_t(s_t)$ is called policy function.

\subsubsection*{Application}
\paragraph*{Uncertainty of evolution process}
Transportation probability is $\textnormal{Prob}\left(s_{t+1}\mid s_t,a_t\right)$. Then,
\begin{equation}
    \begin{aligned}
        V^i(s_t)=\max_{a^i_t}\Pi_t^i((a^i_t,a^{-i}_t),s_t)+\sum_{s'}\delta V^i(s')\textnormal{Prob}\left(s'\mid s_t,a_t\right)
    \end{aligned}
    \nonumber
\end{equation}
Numerical Computation:
\begin{enumerate}
    \item Value iteration: (usually slow) start at $[V^i]^{(0)}$.
    \begin{equation}
        \begin{aligned}
            [V^i(s_t)]^{(k+1)}=\max_{a^i_t}\Pi_t^i((a^i_t,a^{-i}_t),s_t)+\sum_{s'}\delta [V^i(s')]^{(k)}\textnormal{Prob}\left(s'\mid s_t,a_t\right)
        \end{aligned}
        \nonumber
    \end{equation}
    until $[v^i]^{(k)}$ converges.
    \item Policy iteration: start at $a^{(0)}(s_t)$. There are $N$ states and $J$ players.
    \begin{equation}
        \begin{aligned}
            \underbrace{\begin{pmatrix}
                \\
                V^i(s_t)\\
                \\
            \end{pmatrix}}_{N\times 1}
            &=
            \underbrace{\begin{pmatrix}
                \\
                \Pi_t^i(a^{(k)}(s_t),s_t)\\
                \\
            \end{pmatrix}}_{N\times 1}
            +\delta \underbrace{\begin{pmatrix}
                \\
                \textnormal{Prob}\left(s'\mid s_t,a^{(k)}(s_t)\right)\\
                \\
            \end{pmatrix}}_{N\times N}
            \underbrace{\begin{pmatrix}
                \\
                V^i(s')\\
                \\
            \end{pmatrix}}_{N\times 1}\\
            V&=\Pi+\delta P V\\
            V&=(I-\delta P)^{-1}\Pi
        \end{aligned}
        \nonumber
    \end{equation}
    \begin{equation}
        \begin{aligned}
            [a^i]^{(k+1)}=\argmax_{a^i_t}\Pi_t^i((a^i_t,a^{-i}_t),s_t)+\sum_{s'}\delta [V^i(s')]^{(k)}\textnormal{Prob}\left(s'\mid s_t,a_t\right)
        \end{aligned}
        \nonumber
    \end{equation}
    Obtain empirically policy functions:
    We can regress
    \begin{enumerate}
        \item $a_t$ on $s_t$ to get policy function.
        \item $s_{t+1}$ on $a_t$ and $s_t$.
    \end{enumerate}
    Plug in $V=(I-\delta P)^{-1}\Pi$ to obtain $V$.
\end{enumerate}

\begin{enumerate}
    \item $s_t$ is discrete or continuous (interpolation).
    \item Uncertainty on profit function $\Pi^i(a_t,s_t,\epsilon_t)$. That is, the profit also depends on a noise $\epsilon_t$:
    \subitem Known $\epsilon_t$:
    \begin{equation}
        \begin{aligned}
            V^i(s_t,\epsilon_t)=\max_{a^i_t}\Pi_t^i((a^i_t,a^{-i}_t),s_t,\epsilon_t)+\sum_{s'}\delta V^i(s',\epsilon_{t+1})\textnormal{Prob}\left(s',\epsilon_{t+1}\mid s_t,a_t,\epsilon_t\right)
        \end{aligned}
        \nonumber
    \end{equation}
    \subitem Unknown $\epsilon_t$:
    \begin{equation}
        \begin{aligned}
            \Pi_t^i(a_t,s_t)=\mathbb{E}_{\epsilon_t}\tilde{\Pi}_t^i(a_t,s_t,\epsilon_t)
        \end{aligned}
        \nonumber
    \end{equation}
    \subitem Private information on $\epsilon_t$.
\end{enumerate}

\subsection{Switching Costs}
\subsubsection*{2-Period Model}
Suppose consumers are uniformly distributed on $[0,1]$. $t\sim \textnormal{Unif}[0,1]$ is the transportation cost, $s$ is he switching cost, $f$ is the fraction of consumers that have the same preferences from period to period, and $\delta_F,\delta_S$ are the discount factors.

Consider the second period first. Suppose $q_1^i$ consumers bought $i$ in the first period, where $q_1^i=1-q_1^{-i}$.\\
Assume the consumers who bought from one firm in the first period and did not change preferences continue to buy from that firm in the second period. Then,
\begin{equation}
    \begin{aligned}
        q_2^i=yq_1^i+(1-y)q_1^i\frac{t+s+p_2^{-i}-p_2^i}{2t}+(1-y)(1-q_1^i)\frac{t-s+p_2^{-i}-p_2^i}{2t}
    \end{aligned}
    \nonumber
\end{equation}
Thus, the best response to $p_2^{-i}$ is given by
\begin{equation}
    \begin{aligned}
        \max_{p_2^i}p_2^iq_2^i
    \end{aligned}
    \nonumber
\end{equation}
By FOC, we have
\begin{equation}
    \begin{aligned}
        p_2^i=t+\frac{2t}{3}\frac{y}{1-y}\left(1+q_1^i\right)+\frac{s}{3}\left(2q_1^i-1\right)
    \end{aligned}
    \nonumber
\end{equation}
Then, $p_2^i=\frac{t}{1-y}$ if $q_1^i=\frac{1}{2}$.
Moreover, we have
\begin{equation}
    \begin{aligned}
        \frac{\partial q_2^i}{\partial p_2^{-i}}=\frac{1-y}{2t},\
        \frac{\partial q_2^i}{\partial q_1^i}=y+(1-y)\frac{s}{t}\\
        \frac{\partial p_2^{-i}}{\partial q_1^i}=-\frac{2}{3}\left(\frac{ty}{1-y}+s\right),\ \frac{\partial \Pi_2^i}{\partial q_1^i}=\frac{2}{3}\left(\frac{yt}{1-y}+s\right)
    \end{aligned}
    \nonumber
\end{equation}

In the first period, the indifferent consumer $t$ should be
\begin{equation}
    \begin{aligned}
        p_1^i+tq_1^i+y\delta_C\left[p_2^i+t q_1^i\right]+(1-y)\delta_C M^i=p_1^{-i}+t(1-q_1^i)+y\delta_C\left[p_2^{-i}+tq_1^{-i}\right]+(1-y)\delta_C M^{-i}
    \end{aligned}
    \nonumber
\end{equation}
where $M^i:=\mathbb{E}_z\min\left\{p_2^i+tz,p_2^j+t(1-z)+s\right\}$ is the expected cost in second period if consumers bought $i$ in the first period and change preferences. ... We can have $M^{i}+M^{-i}=\frac{s}{t}\left[p_2^i-p_2^{-i}\right]$.

Firm $i$'s problem is $\Pi_1^i=\max_{p_1^i} p_1^i q_1^i +\delta_F\Pi_2^i(q_1^i)$.

As $\delta_F$ increases, $p_1^i$ and $\Pi_1^i$ decrease. As $\delta_C$ increases, $p_1^i$ and $\Pi_1^i$ increase.


\subsubsection*{Infinite Horizon}
\underline{Overlapping Generations:}
Consumers only live for 2 periods.
\begin{equation}
    \begin{aligned}
        V(q^i_{t-1})=\max_{p^i_t} p^i_t\cdot (q^i_t+q^i_{ot})+V(q^i_t)
    \end{aligned}
    \nonumber
\end{equation}
where the $q^i_{t-1}$ is the number of new consumers in period $t-1$ who chooses $i$ and the $q^i_{ot}$ is a function about $q^i_{t-1}$ which represents the number consumers in previous period who chooses $i$ in this period. ($q^i_{ot}$ is the $q_2^i=yq_1^i+(1-y)q_1^i\frac{t+s+p_2^{-i}-p_2^i}{2t}+(1-y)(1-q_1^i)\frac{t-s+p_2^{-i}-p_2^i}{2t}$ in previous case.)

Given the linear demand function. The relationship between the demand and prices is constructed as
\begin{equation}
    \begin{aligned}
        \frac{d q^i_t}{d p^i_t}=-\frac{d q^j_t}{d p^i_t}=-\frac{1}{\Delta}
    \end{aligned}
    \nonumber
\end{equation}
$\Delta$ is some function (omitted). The F.O.C.s' forms are $p^i_t=c+d q^i_{t-1}$. ...

The results are: $\frac{\partial p^i_t}{\partial \delta^F}<0$; as $\delta_P=\delta_C$ increases, $\Pi$ decreases.

\subsection{Experience Goods}


\section{Price Discrimination}
The problem of price discrimination: arbitrage
\begin{enumerate}
    \item Transfer of goods, which can be impeded by transaction costs.
    \item Transfer of demands.
\end{enumerate}
Price discrimination can increase the quantity supplied.


\subsection{First Degree Price Discrimination (Perfect)}
Consider a linear demand of homogeneous consumers, the perfect price discrimination can be implemented as:
Tariff, total payment
\begin{equation}
    \begin{aligned}
        T=A+p\cdot q
    \end{aligned}
    \nonumber
\end{equation}
So that the average price can vary according to the demand of the consumer.

Consider a 1 unit demand for heterogeneous consumers, the perfect price discrimination can be implemented as setting individual prices.


\subsection{Third Degree Price Discrimination (External Signal)}
Given exogenous information, the price is based on that exogenous information. (e.g. student ID, covid or not, age, gender, income).

Pricing based on $\frac{p-MC}{p}=\frac{1}{E_d}$ in each submarket.


\begin{example}[ (Spatial Discrimination)]
    Suppose the firm locates at $0$ and consumers located at $x$. There is a transportation cost $t\cdot x$ incurred by the firm to sell to consumers\\
    The optimal price for consumers at $x$ is charged as
    \begin{equation}
        \begin{aligned}
            p^*(x)=\argmax_p (p-tx-c) D(p)=\frac{a+btx+bc}{2b}
        \end{aligned}
        \nonumber
    \end{equation}
    where $D(p)=a-bp$.
    Thus,
    \begin{equation}
        \begin{aligned}
            \frac{\partial p^*(x)}{\partial x}=\frac{t}{2}
        \end{aligned}
        \nonumber
    \end{equation}
    The discrimination against consumers close-by.

    Now, consider $D(p)=a e^{-bp}$. The optimal price for consumers at $x$ is charged as
    \begin{equation}
        \begin{aligned}
            p^*(x)=\argmax_p (p-tx-c)D(p)=tx+c+\frac{1}{b}
        \end{aligned}
        \nonumber
    \end{equation}
    Thus,
    \begin{equation}
        \begin{aligned}
            \frac{\partial p^*(x)}{\partial x}=t
        \end{aligned}
        \nonumber
    \end{equation}
    There is no discrimination compared to cost.

    Now, consider $D(p)=a p^{-b}$. The optimal price for consumers at $x$ is charged as
    \begin{equation}
        \begin{aligned}
            p^*(x)=\argmax_p (p-tx-c)D(p)=\frac{b}{b-1}(tx+c)
        \end{aligned}
        \nonumber
    \end{equation}
    Thus,
    \begin{equation}
        \begin{aligned}
            \frac{\partial p^*(x)}{\partial x}=\frac{b}{b-1}t>t
        \end{aligned}
        \nonumber
    \end{equation}
    There is price discrimination against consumers far-away. (Which may not happen, if there is a competitive transportation industry.)
\end{example}

\begin{example}[ (2-Vertical Intergration, PS3 \#4)]
    Suppose there is a manufacturer with two downstream industries $C_1$ and $C_2$ such that the elasticity follows $\epsilon_2>\epsilon_1$. Thus, the monopoly prices are $p_1^*=\frac{c}{1-1/\epsilon_1}>p_2^*=\frac{c}{1-1/\epsilon_2}$. The manufacturer integrates $C_2$ can achieve third-price discrimination. (He sets the monopoly price that is lower than his competitor's MC in the industry as he sets the monopoly price in $C_1$.)
\end{example}

\subsection{Second Degree Price Discrimination (Self-Selection)}
Offer a menu for price-quantity (or price-quality) options.

\underline{Two-part Tariffs:} $T(q)=A+p\cdot q$ (quantity discount).

Suppose the utility of consumers is defined as
\begin{equation}
    \begin{aligned}
        U=\left\{\begin{matrix}
            \theta V(q)-T,& \textnormal{ if buy}\\
            0,& \textnormal{ if not buy}
        \end{matrix}\right.
    \end{aligned}
    \nonumber
\end{equation}























































































\bibliography{ref}

\end{document}