\documentclass[11pt]{elegantbook}
\usepackage{graphicx}
%\usepackage{float}
\definecolor{structurecolor}{RGB}{40,58,129}
\linespread{1.6}
\setlength{\footskip}{20pt}
\setlength{\parindent}{0pt}
\newcommand{\argmax}{\operatornamewithlimits{argmax}}
\newcommand{\argmin}{\operatornamewithlimits{argmin}}
\elegantnewtheorem{proof}{Proof}{}{Proof}
\elegantnewtheorem{claim}{Claim}{prostyle}{Claim}
\DeclareMathOperator{\col}{col}
\title{Miguel Class}
\author{Wenxiao Yang}
\institute{Haas School of Business, University of California Berkeley}
\date{2024}
\setcounter{tocdepth}{2}
\extrainfo{All models are wrong, but some are useful.}

\cover{cover.png}

% modify the color in the middle of titlepage
\definecolor{customcolor}{RGB}{32,178,170}
\colorlet{coverlinecolor}{customcolor}
\usepackage{cprotect}


\bibliographystyle{apalike_three}

\begin{document}
\maketitle

\frontmatter
\tableofcontents

\mainmatter



\chapter{Pricing}
\section{Monopoly}
\subsection{Base Case}
The firm decides its price $p$ to maximize $\Pi(p)=p\cdot D(p)-C(D(p))$, where $D(\cdot)$ is the demand function and $C(\cdot)$ is the cost function.

The monopoly problem is maximizing the profit
\begin{equation}
    \begin{aligned}
        \max_p \Pi(p)=p\cdot D(p)-C(D(p))
    \end{aligned}
    \nonumber
\end{equation}
The F.O.C. (first-order condition) is
\begin{equation}
    \begin{aligned}
        \frac{\partial \Pi(p)}{\partial p}=D(p)+pD'(p)-C'(D(p))D'(p)=0\\
    \end{aligned}
    \nonumber
\end{equation}
and the S.O.C. (second-order condition) is
\begin{equation}
    \begin{aligned}
        \frac{\partial \Pi^2(p)}{\partial p^2}<0
    \end{aligned}
    \nonumber
\end{equation}

The F.O.C. gives that
\begin{equation}
    \begin{aligned}
        (p-C')D'&=-D\\
        p-C'&=-\frac{D}{D'}\\
        \underbrace{\frac{p-C'}{p}}_\text{Lerner Index}&=-\frac{D}{D'p}\\
        &=-\frac{1}{\frac{d D}{d p}\frac{p}{D}}=-\frac{1}{\frac{\frac{d D}{D}}{\frac{d p}{p}}}:=\frac{1}{E}
    \end{aligned}
    \nonumber
\end{equation}
where $\frac{\frac{d D}{D}}{\frac{d p}{p}}<0$ is the elasticity of demand with respect to price. The absolute value of the elasticity is denoted by $E$.

$E$ is supposed to be greater than $1$, otherwise, the optimal price is negative.

In the demand function $D(p)=kp^{-E}$, where the elasticity is constant. Its elasticity is $-E$.


The monopolist gives the production that is lower than social-optimal to maximize the profit (dead weight loss). Rent dissipation can give larger dead weight loss.

\subsection{Multiple Products}
\begin{equation}
    \begin{aligned}
        \max_{p}\sum_{i=1}^N p_i D_i(p)-C(D_1(p),...,D_N(p))
    \end{aligned}
    \nonumber
\end{equation}
\paragraph*{Related Demand and Separable Costs:} $C(D_1(p),...,D_N(p))=C_1(D_1(p))+...+C_N(D_N(p))$. The optimal pricing in this case satisfies
\begin{equation}
    \begin{aligned}
        \frac{p_i-C'_i}{p_i}=\frac{1}{E_{ii}}-\sum_{j\neq i}\frac{(p_j-C'_j)D_j E_{ij}}{R_i E_{ii}}
    \end{aligned}
    \nonumber
\end{equation}
where $E_{ij}=\frac{\partial D_i}{\partial p_j}\frac{p_j}{D_i}$ and $R_i$ is the revenue.

\textbf{Intuition:} In the case of substitutes/complements, we want to increase/decrease the price of products compared to the one product case. (Positive/negative externality by increasing price of substitutes).


\textbf{Similar Intuition:}
Consider a two-period model that the demand at second period depends on the price at first period (assuming $\frac{\partial D_2}{\partial p_1}<0$).
\begin{enumerate}
    \item $q_1=D_1(p_1)$; $C_1(q_1)$
    \item $q_2=D_2(p_2,p_1)$; $C_2(q_2)$
\end{enumerate}
Then, $\frac{p_1-C'_1}{p_1}<\frac{1}{E_1}$ (the negative externality of prices).

\paragraph*{Independent Demands and Related Costs:}
\begin{example}
    Different intensity of demand across periods.
    \begin{enumerate}
        \item Period 1: Low demand. $q_1=D_1(p_1)$.
        \item Period 2: High demand. $q_2=D_2(p_2)$, where $D_1(p)=\lambda D_2(p)$ for some $\lambda<1$.
        \item Marginal cost of Production is $c$ and the Marginal cost of capacity is $\gamma$.
    \end{enumerate}
\end{example}
Intuition: if $\lambda \rightarrow 0$, the marginal cost at period $\rightarrow c+\gamma$ and the marginal cost at period 1 $=c$. Then, we have
\begin{equation}
    \begin{aligned}
        \frac{p_2-(c+\gamma)}{p_2}=\frac{1}{E_2},\ \frac{p_1-c}{p_1}=\frac{1}{E_1}
    \end{aligned}
    \nonumber
\end{equation}
Now, let's consider a not too small $\lambda$. The problem is given as
\begin{equation}
    \begin{aligned}
        \max_{p_1,p_2,k}&\ (p_1-c)D_1(p_1)+(p_2-c)D_2(p_2)-\gamma k\\
        s.t.\ & D_1(p_1)\leq k\\
        & D_2(p_2)\leq k
    \end{aligned}
    \nonumber
\end{equation}
The Lagrangian is given by
\begin{equation}
    \begin{aligned}
        \mathcal{L}=(p_1-c)D_1(p_1)+(p_2-c)D_2(p_2)-\gamma k+\lambda_1(k-D_1(p_1))+\lambda_2(k-D_2(p_2))
    \end{aligned}
    \nonumber
\end{equation}
\begin{equation}
    \begin{aligned}
        \frac{\partial \mathcal{L}}{\partial k}=-\gamma+\lambda_1+\lambda_2=0 \Leftrightarrow \gamma=\lambda_1+\lambda_2
    \end{aligned}
    \nonumber
\end{equation}
Skip the process: $\frac{p_1-(c+\lambda_1)}{p_1}=\frac{1}{E_1}$, $\frac{p_2-(c+\lambda_2)}{p_2}=\frac{1}{E_2}$. Example: If $\lambda_1=0, k>D_1(p_1)$, the second period pays all the capacity cost.

\begin{example}[ (Learning by Doing)]
    Suppose there are two periods $t=1,2$. The demand is $q_t=D_t(p_t)$. The cost in period one is $c_1(q_1)$ and $c_2(q_2,q_1)$ ($\frac{\partial c_2}{\partial q_1}<0$, the more you produce in period one, the lower the cost you are facing in period two).
\end{example}
In continuous form, the cost form is
\begin{equation}
    \begin{aligned}
        C(w(t))
    \end{aligned}
    \nonumber
\end{equation}
where $\dot{w}(t)=\frac{d w}{d t}=q(t)$. We want to maximize
\begin{equation}
    \begin{aligned}
        \max_{q(t),w(t)}\ &\int_0^\infty e^{-\pi t}[q(t)p(q(t))-C(w(t))q(t)]dt\\
        &\textnormal{s.t. }\dot{w}(t)=q(t)
    \end{aligned}
    \nonumber
\end{equation}
By Hamiltonian (skip), average of future marginal costs is
\begin{equation}
    \begin{aligned}
        A(t)=\int_t^\infty C(w(s))\pi e^{-\pi(s-t)}ds
    \end{aligned}
    \nonumber
\end{equation}
\begin{equation}
    \begin{aligned}
        \frac{P(t)-A(t)}{P(t)}=\frac{1}{E(t)}
    \end{aligned}
    \nonumber
\end{equation}

\subsection{Durable Goods}
The demand in one period is substitute to demand in other periods.
\begin{example}
    Two periods $t=1,2$. Three consumers: $v_1=1$ per period, $v_2=2$ per period, and $v_3=3$ per period. The cost of production is zero. The seller chooses $p_1,p_2$.
\end{example}
(Consumer may forward-looking).


Moorthy (1988), Levinthal, D. A., \& Purohit, D. (1989).

$t=1,2$ and zero production cost. The values of consumers $v\sim U[0,1]$ and the discount factor is $\delta<1$. The selling price $p_1,p_2$.

Suppose the consumers bought in first period have $v\geq v_1^*$, which must satisfies
\begin{equation}
    \begin{aligned}
        \delta (v_1^* - p_2)&= v_1^* - p_1 + \delta v_1^*\\
        v_1^*&=p_1-\delta p_2
    \end{aligned}
    \nonumber
\end{equation}
The price in second period should be $p_2=\frac{v_1^*}{2}$. Then, $v_1^*=p_1-\delta \frac{v_1^*}{2} \Rightarrow v_1^*=\frac{2p_1}{2+\delta}$. The price in the first period is given by
\begin{equation}
    \begin{aligned}
        \max_{p_1}\ p_1(1-v_1^*)+\delta \frac{(v_1^*)^2}{4}=p_1\frac{2+\delta-2p_1}{2+\delta}+\delta \frac{p_1^2}{(2+\delta)^2}
    \end{aligned}
    \nonumber
\end{equation}

\paragraph*{Leasing (instead of selling):} The leasing price is $p$ in each period, which is given by $\argmax_p p(1-p)=\frac{1}{2}$.

Leasing may generate more profits than selling for the seller.

\paragraph*{Intuition:}
\begin{enumerate}
    \item Too much flexibility for seller $\Rightarrow$ losses of capital of first period buyers.
    \item Intertemporal price discrimination (first period buyers pay higher price) ``price skimming''.
\end{enumerate}
\begin{theorem}[Coase Conjecture]
    Suppose the seller can change the price faster and faster. What happens to the profits of the seller? The profit goes to zero.
\end{theorem}

Why there is selling in the world?
\begin{enumerate}
    \item Moral hazard of leasing.
    \item Leasing is not anonymous. Reveal reservation price $\Rightarrow$ Price discrimination in leasing. (Even worse than selling.) (Long-term contract + renegotiation = selling).
    \item Commit to sequences of prices.
    \begin{enumerate}
        \item Deposit to third party
        \item Reputation
    \end{enumerate}
    \item Increasing cost
    \item ``Most-favored Nation'' clause.
    \item Consumers are not informed about the production costs.
    \item New consumers coming into the market.
\end{enumerate}

\subsection{Learning Demand}
Firms may not be able to learn the demand function perfectly.

It is relatively easy to learn a quasi-concave profit function. In the case that the profit function is not quasi-concave, the firm may not be able to learn the profit function. (stay at local maximum because of the loss from learning).

Learning the optimal features: By assuming the distribution of marginal increase $\frac{\partial \pi_j}{\partial x_j}$ is symmetric about $0$, we can use Brownian motion to model the continuous profit function.


\section{Short-run Competition}
\subsection{Bertrand Paradox}
Consider two firms with marginal cost $c$:
\begin{equation}
    \begin{aligned}
        \Pi^i(p_i,p_j)=(p_i-c)D_i(p_i,p_j)
    \end{aligned}
    \nonumber
\end{equation}
where
\begin{equation}
    \begin{aligned}
        D_i(p_i,p_j)=\left\{\begin{matrix}
            D_i(p_i),& p_i<p_j,\\
            \frac{1}{2}D_i(p_i),& p_i=p_j,\\
            0,& p_i>p_j
        \end{matrix}\right.
    \end{aligned}
    \nonumber
\end{equation}
The NE is $p_i=p_j=c$. $\Pi^i=\Pi^j=0$.

\subsection{Static Solution to Bertrand Paradox}
\subsubsection*{Capacity Constraints}
Edgeworth: there may exist some constraints of the capacity.
\begin{enumerate}
    \item Firms choose capacity $K_i,K_j$.
    \item Firms choose prices $p_i,p_j$.
\end{enumerate}
Solving by backward induction: That is, firstly solve $p_i^*(K_i,K_j)$ such that
\begin{equation}
    \begin{aligned}
        \max_{p_i}\ & (p_i-c)D_i(p_i,p_j)\\
        \textnormal{s.t. }&D_i(p_i,p_j)\leq K_i
    \end{aligned}
    \nonumber
\end{equation}
and then solve
\begin{equation}
    \begin{aligned}
        \max_{K_i}\ \left(p_i^*(K_i,K_j)-c\right)D_i(p_i^*(K_i,K_j),p_j^*(K_i,K_j))-\gamma K_i
    \end{aligned}
    \nonumber
\end{equation}
where $\gamma$ is the marginal cost of capacity.

Best response in prices: positive correlated, which is called ``strategic complements''.

Best response in quantities (Cournot competition): negative correlated, which is called ``strategic substitutes''. (Quantity competition gives higher profits.)
\begin{example}[ (Simple Example of Couront Competition)]
    $P(q_1,q_2)=1-q_1-q_2$ and $\gamma\in (\frac{3}{4},1)$.
    \begin{equation}
        \begin{aligned}
            \max_{q_1}\ q_1\left(1-q_1-q_2-\gamma\right)\\
            \Rightarrow q_1^*(q_2)=\frac{1-q_2-\gamma}{2}
        \end{aligned}
        \nonumber
    \end{equation}
    Similarly, $q_2^*(q_1)=\frac{1-q_1-\gamma}{2}$. Thus, $q_1^*=q_2^*=\frac{1-\gamma}{3}$.
\end{example}

Similar to the Cournot competition, we can get positive profits with capacity constraints.

\subsubsection*{Differentiation}
\textbf{Idea:} it is easier to change prices than to change products.

\paragraph*{Basic case:} Spatial Competition: There are consumers in $[0,1]$ (uniform distribution). The position chosen by both firms is $\frac{1}{2}$ (the center of the market).

\paragraph*{With price competition:} Transportation cost is $t x^2$, where $x$ is the distance. Suppose the firm $A$ locates at $0$ and the firm $B$ locates at $1$. The profit of consumer $x$ from purchasing $A$ is $v-p_A-tx^2$ and the profit of consumer $x$ from purchasing $B$ is $v-p_B-t(1-x)^2$. The indifferent consumer is
\begin{equation}
    \begin{aligned}
        v-p_A-tx^2=v-p_B-t(1-x)^2\\
        p_A-p_B=t(1-2x)\\
        \Rightarrow x=\frac{1}{2}-\frac{p_A-p_B}{2t}
    \end{aligned}
    \nonumber
\end{equation}
Therefore, the demand of $A$ is $$D_A(p_A,p_B)=\frac{1}{2}-\frac{p_A-p_B}{2t}$$ and the demand of $B$ is $$D_B(p_A,p_B)=\frac{1}{2}+\frac{p_A-p_B}{2t}$$
\begin{note}
    If the transportation cost is $tx$, the demand function is the same as above.
\end{note}
The $p_A^*$ and $p_B^*$ are given by
\begin{equation}
    \begin{aligned}
        \left.\begin{matrix}
            p_A^*(p_B)=\argmax_{p_A}(p_A-c)\left(\frac{1}{2}-\frac{p_A-p_B}{2t}\right)=\frac{c+t+p_B}{2}\\
            p_B^*(p_A)=\argmax_{p_B}(p_B-c)\left(\frac{1}{2}+\frac{p_A-p_B}{2t}\right)=\frac{c+t+p_A}{2}
        \end{matrix}\right\} \Rightarrow p_A^*=p_B^*=c+t
    \end{aligned}
    \nonumber
\end{equation}
Then, the profits are $\Pi_A^*=\Pi_B^*=\frac{t}{2}$.

\paragraph*{Endogenous Differentiation:}
Denote the position of firm $A$ as $a$ and the position of firm $B$ as $1-b$. Then, the indifferent consumer is
\begin{equation}
    \begin{aligned}
        p_A+t(x-a)^2=p_B+t(1-b-x)^2\\
        \Rightarrow x=\frac{p_B-p_A+t\left[(1-b)^2-a^2\right]}{2t(1-b-a)},
    \end{aligned}
    \nonumber
\end{equation}
and the demands are
\begin{equation}
    \begin{aligned}
        D_A(p_A,p_B)=\frac{p_B-p_A+t\left[(1-b)^2-a^2\right]}{2t(1-b-a)},\
        D_B(p_A,p_B)=1-D_A(p_A,p_B)
    \end{aligned}
    \nonumber
\end{equation}
Then, the equilibrium prices given $a$ and $b$ are
\begin{equation}
    \begin{aligned}
        p_A^*=t(1-a-b)\left(1+\frac{a-b}{3}\right)\\
        p_B^*=t(1-a-b)\left(1+\frac{b-a}{3}\right)
    \end{aligned}
    \nonumber
\end{equation}
where $c:=0$. The corresponding profits are
\begin{equation}
    \begin{aligned}
        \Pi_A(a,b)&=p_A^*D_A^*=\left(1+\frac{a-b}{3}\right)\frac{t(1-b-a)(1-b+a)}{2}\\
        \Pi_B(a,b)&=p_B^*D_B^*=\left(1+\frac{b-a}{3}\right)\frac{t(1-b-a)(1-a+b)}{2}
    \end{aligned}
    \nonumber
\end{equation}
\begin{equation}
    \begin{aligned}
        \frac{\partial \Pi_A(a,b)}{\partial a}=\underbrace{(p_A^*-c)\frac{\partial D_A^*}{\partial a}}_\textnormal{direct effect $>0$}+\underbrace{\frac{\partial D_A^*}{\partial p_A^*}}_{=0}\frac{\partial p_A^*}{\partial a}+\underbrace{(p_A^*-c)\frac{\partial D_A^*}{\partial p_B^*}\frac{\partial p_B^*}{\partial a}}_\textnormal{strategic effect $<0$}
    \end{aligned}
    \nonumber
\end{equation}
Which effect dominates depends on the model. In this model, the strategic effect dominates the direct effect. That is, $a=b=0$. (If allowing negative values, $a=b=-\frac{1}{4}$.)

\begin{note}
    If the transportation cost is $tx$, the equilibrium may not exist.
\end{note}


Other models:
\begin{enumerate}
    \item vertical differentiation (S. Moorthy);
    \item defender model (John Hauser);
    \item logit/limited defender model, $u_{ij}=v_j - \alpha p_j+\epsilon_{ij}$, where $\epsilon_{ij}\sim EVI$. The outside option is modeled as $u_{i0}=\epsilon_{i0}\sim EVI$. The market share is given as
    \begin{equation}
        \begin{aligned}
            s_{j}=\textnormal{Pr}(u_{ij}\geq u_{ij'},\forall j')=\frac{\exp(v_j - \alpha p_j)}{1+\sum_{j'}\exp(v_{j'}-\alpha p_{j'})}
        \end{aligned}
        \nonumber
    \end{equation}
    The demand is given by
    \begin{equation}
        \begin{aligned}
            D_j=s_j\cdot \textnormal{Market Size}
        \end{aligned}
        \nonumber
    \end{equation}
    Estimated by taking inversion,
    \begin{equation}
        \begin{aligned}
            \ln s_j - \ln s_0 = v_j - \alpha p_j
        \end{aligned}
        \nonumber
    \end{equation}
\end{enumerate}

\textit{Proliferation of Products to Deter Entry}: Entry with products locating uniformly to deter entry.

\textit{Spoke Model}.


\subsubsection*{Search Costs}
\textit{Individual Choice}:
\begin{enumerate}
    \item \textit{Simple example}, where the cost of search for each price is $c$ and a consumer buys one or zero unit.
    \begin{definition}[Optimal Stopping Rule]
        The optimal rule of searching is an \textbf{optimal stopping rule}: Stop and buy once the consumer finds a price less or equal to a reservation price, $R$.
    \end{definition}
    The $R$ is constructed as the critical value when the expected return from an extra search equals to the marginal cost:
    \begin{equation}
        \begin{aligned}
            c&=\mathbb{E}[R-p\mid p<R] \textnormal{Pr}(p<R)\\
            &=\int_0^R(R-p)f(p)dp
        \end{aligned}
        \nonumber
    \end{equation}
    where $f(\cdot)$ is the density distribution of prices.
    \item \textit{Consuming multiple units (general case, the one unit case can be modeled by assuming the demand function)}, Remind that the consumer surplus given price $p$ is given as
    \begin{equation}
        \begin{aligned}
            s(p)=\max_q\left\{U(q)-p\cdot q\right\}
        \end{aligned}
        \nonumber
    \end{equation}
    and its derivatives are
    \begin{equation}
        \begin{aligned}
            s'(p)=-D(p),\ s''(p)=-D'(p)>0 \textnormal{ (convexity)}
        \end{aligned}
        \nonumber
    \end{equation}
    In this case, the optimal stopping rule $R$ that maximizing the consumer surplus is given as
    \begin{equation}
        \begin{aligned}
            c=\int_0^R[s(p)-s(R)]f(p)dp
        \end{aligned}
        \nonumber
    \end{equation}
    \begin{note}
        Given a greater variance of the price distribution, the $R$ decreases.
    \end{note}
\end{enumerate}

\textit{Homogeneous Markets}:\\
Can we find a fixed point of search behavior (i.e., $f(p)$)?

\begin{assumption}
    All firms are identical with marginal cost $c_f$. All consumers are identical with search costs $c$.
\end{assumption}
Define the ``monopoly price''
\begin{equation}
    \begin{aligned}
        P^M=\argmax_{p}\left\{p\cdot D(p)-c_fD(p)\right\}
    \end{aligned}
    \nonumber
\end{equation}








\section{}





\bibliography{ref}

\end{document}