\documentclass[11pt]{elegantbook}
\usepackage{graphicx}
%\usepackage{float}
\definecolor{structurecolor}{RGB}{40,58,129}
\linespread{1.6}
\setlength{\footskip}{20pt}
\setlength{\parindent}{0pt}
\newcommand{\argmax}{\operatornamewithlimits{argmax}}
\newcommand{\argmin}{\operatornamewithlimits{argmin}}
\elegantnewtheorem{proof}{Proof}{}{Proof}
\elegantnewtheorem{claim}{Claim}{prostyle}{Claim}
\DeclareMathOperator{\col}{col}
\title{Miguel Class}
\author{Wenxiao Yang}
\institute{Haas School of Business, University of California Berkeley}
\date{2024}
\setcounter{tocdepth}{2}
\extrainfo{All models are wrong, but some are useful.}

\cover{cover.png}

% modify the color in the middle of titlepage
\definecolor{customcolor}{RGB}{32,178,170}
\colorlet{coverlinecolor}{customcolor}
\usepackage{cprotect}


\bibliographystyle{apalike_three}

\begin{document}
\maketitle

\frontmatter
\tableofcontents

\mainmatter



\chapter{Pricing}
\section{Monopoly}
\subsection{Base Case}
The firm decides its price $p$ to maximize $\Pi(p)=p\cdot D(p)-C(D(p))$, where $D(\cdot)$ is the demand function and $C(\cdot)$ is the cost function.

The monopoly problem is maximizing the profit
\begin{equation}
    \begin{aligned}
        \max_p \Pi(p)=p\cdot D(p)-C(D(p))
    \end{aligned}
    \nonumber
\end{equation}
The F.O.C. (first-order condition) is
\begin{equation}
    \begin{aligned}
        \frac{\partial \Pi(p)}{\partial p}=D(p)+pD'(p)-C'(D(p))D'(p)=0\\
    \end{aligned}
    \nonumber
\end{equation}
and the S.O.C. (second-order condition) is
\begin{equation}
    \begin{aligned}
        \frac{\partial \Pi^2(p)}{\partial p^2}<0
    \end{aligned}
    \nonumber
\end{equation}

The F.O.C. gives that
\begin{equation}
    \begin{aligned}
        (p-C')D'&=-D\\
        p-C'&=-\frac{D}{D'}\\
        \underbrace{\frac{p-C'}{p}}_\text{Lerner Index}&=-\frac{D}{D'p}\\
        &=-\frac{1}{\frac{d D}{d p}\frac{p}{D}}=-\frac{1}{\frac{\frac{d D}{D}}{\frac{d p}{p}}}:=\frac{1}{E}
    \end{aligned}
    \nonumber
\end{equation}
where $\frac{\frac{d D}{D}}{\frac{d p}{p}}<0$ is the elasticity of demand with respect to price. The absolute value of the elasticity is denoted by $E$.

$E$ is supposed to be greater than $1$, otherwise, the optimal price is negative.

In the demand function $D(p)=kp^{-E}$, where the elasticity is constant. Its elasticity is $-E$.


The monopolist gives the production that is lower than social-optimal to maximize the profit (dead weight loss). Rent dissipation can give larger dead weight loss.

\subsection{Multiple Products}
\begin{equation}
    \begin{aligned}
        \max_{p}\sum_{i=1}^N p_i D_i(p)-C(D_1(p),...,D_N(p))
    \end{aligned}
    \nonumber
\end{equation}
\paragraph*{Related Demand and Separable Costs:} $C(D_1(p),...,D_N(p))=C_1(D_1(p))+...+C_N(D_N(p))$. The optimal pricing in this case satisfies
\begin{equation}
    \begin{aligned}
        \frac{p_i-C'_i}{p_i}=\frac{1}{E_{ii}}-\sum_{j\neq i}\frac{(p_j-C'_j)D_j E_{ij}}{R_i E_{ii}}
    \end{aligned}
    \nonumber
\end{equation}
where $E_{ij}=\frac{\partial D_i}{\partial p_j}\frac{p_j}{D_i}$ and $R_i$ is the revenue.

\textbf{Intuition:} In the case of substitutes/complements, we want to increase/decrease the price of products compared to the one product case. (Positive/negative externality by increasing price of substitutes).


\textbf{Similar Intuition:}
Consider a two-period model that the demand at second period depends on the price at first period (assuming $\frac{\partial D_2}{\partial p_1}<0$).
\begin{enumerate}
    \item $q_1=D_1(p_1)$; $C_1(q_1)$
    \item $q_2=D_2(p_2,p_1)$; $C_2(q_2)$
\end{enumerate}
Then, $\frac{p_1-C'_1}{p_1}<\frac{1}{E_1}$ (the negative externality).

\paragraph*{Independent Demands and Related Costs:}
\begin{example}
    Different intensity of demand across periods.
    \begin{enumerate}
        \item Period 1: Low demand. $q_1=D_1(p_1)$.
        \item Period 2: High demand. $q_2=D_2(p_2)$, where $D_1(p)=\lambda D_2(p)$ for some $\lambda<1$.
        \item Marginal cost of Production is $c$ and the Marginal cost of capacity is $\gamma$.
    \end{enumerate}
\end{example}
Intuition: if $\lambda \rightarrow 0$, the marginal cost at period $\rightarrow c+\gamma$ and the marginal cost at period 1 $=c$. Then, we have
\begin{equation}
    \begin{aligned}
        \frac{p_2-(c+\gamma)}{p_2}=\frac{1}{E_2},\ \frac{p_1-c}{p_1}=\frac{1}{E_1}
    \end{aligned}
    \nonumber
\end{equation}
Now, let's consider a not too small $\lambda$. The problem is given as
\begin{equation}
    \begin{aligned}
        \max_{p_1,p_2,k}&\ (p_1-c)D_1(p_1)+(p_2-c)D_2(p_2)-\gamma k\\
        s.t.\ & D_1(p_1)\leq k\\
        & D_2(p_2)\leq k
    \end{aligned}
    \nonumber
\end{equation}
The Lagrangian is given by
\begin{equation}
    \begin{aligned}
        \mathcal{L}=(p_1-c)D_1(p_1)+(p_2-c)D_2(p_2)-\gamma k+\lambda_1(k-D_1(p_1))+\lambda_2(k-D_2(p_2))
    \end{aligned}
    \nonumber
\end{equation}
\begin{equation}
    \begin{aligned}
        \frac{\partial \mathcal{L}}{\partial k}=-\gamma+\lambda_1+\lambda_2=0 \Leftrightarrow \gamma=\lambda_1+\lambda_2
    \end{aligned}
    \nonumber
\end{equation}
Skip the process: $\frac{p_1-(c+\lambda_1)}{p_1}=\frac{1}{E_1}$, $\frac{p_2-(c+\lambda_2)}{p_2}=\frac{1}{E_2}$. Example: If $\lambda_1=0, k>D_1(p_1)$, the second period pays all the capacity cost.

\begin{example}[ (Learning by Doing)]
    Suppose there are two periods $t=1,2$. The demand is $q_t=D_t(p_t)$. The cost in period one is $c_1(q_1)$ and $c_2(q_2,q_1)$ ($\frac{\partial c_2}{\partial q_1}<0$, the more you produce in period one, the lower the cost you are facing in period two).
\end{example}
In continuous form, the cost form is
\begin{equation}
    \begin{aligned}
        C(w(t))
    \end{aligned}
    \nonumber
\end{equation}
where $\dot{w}(t)=\frac{d w}{d t}=q(t)$. We want to maximize
\begin{equation}
    \begin{aligned}
        \max_{q(t),w(t)}\ &\int_0^\infty e^{-r t}[q(t)p(q(t))-C(w(t))q(t)]dt\\
        &\textnormal{s.t. }\dot{w}(t)=q(t)
    \end{aligned}
    \nonumber
\end{equation}
By Hamiltonian (skip), average of future marginal costs is
\begin{equation}
    \begin{aligned}
        A(t)=\int_t^\infty C(w(s))\pi e^{-\pi(s-t)}ds
    \end{aligned}
    \nonumber
\end{equation}
\begin{equation}
    \begin{aligned}
        \frac{P(t)-A(t)}{P(t)}=\frac{1}{E(t)}
    \end{aligned}
    \nonumber
\end{equation}






\bibliography{ref}

\end{document}