\documentclass[11pt]{elegantbook}
\usepackage{graphicx}
%\usepackage{float}
\definecolor{structurecolor}{RGB}{40,58,129}
\linespread{1.6}
\setlength{\footskip}{20pt}
\setlength{\parindent}{0pt}
\newcommand{\argmax}{\operatornamewithlimits{argmax}}
\newcommand{\argmin}{\operatornamewithlimits{argmin}}
\elegantnewtheorem{proof}{Proof}{}{Proof}
\elegantnewtheorem{claim}{Claim}{prostyle}{Claim}
\DeclareMathOperator{\col}{col}
\title{Ganesh Class}
\author{Wenxiao Yang}
\institute{Haas School of Business, University of California Berkeley}
\date{2025}
\setcounter{tocdepth}{2}
\extrainfo{All models are wrong, but some are useful.}

\cover{cover.png}

% modify the color in the middle of titlepage
\definecolor{customcolor}{RGB}{32,178,170}
\colorlet{coverlinecolor}{customcolor}
\usepackage{cprotect}


\bibliographystyle{apalike_three}

\begin{document}
\maketitle

\frontmatter
\tableofcontents

\mainmatter



\chapter{Self-Control/Addiction}
\section{Hyperbolic Discounting}
The classical discounting model discounts as $1,\delta,\delta^2,...$, i.e., $f(t)=\delta^t$ which discounts at a constant rate. However, the hyperbolic discounting model discounts faster in short run and slower in long run (i.e., we care more about now than future, and the difference of two future dates is small.)
\begin{equation}
    \begin{aligned}
        u_\textnormal{day1}\succ u_\textnormal{day2}\succ \cdots u_\textnormal{day365}\sim u_\textnormal{day366}
    \end{aligned}
    \nonumber
\end{equation}
The functional form can be given by
\begin{equation}
    \begin{aligned}
        f(t)=\left(1+\alpha t\right)^{-\frac{\gamma}{\alpha}}
    \end{aligned}
    \nonumber
\end{equation}
The discount rate is declining with time, $-\frac{f'(t)}{f(t)}=\frac{\gamma}{1+\alpha t}$.

\paragraph*{Quasi Hyperbolic Discounting}
A good approximation of the hyperbolic discounting model is
\begin{equation}
    \begin{aligned}
        U_t=u_t+\beta\left[\delta u_{t+1}+\delta^2 u_{t+2}+\cdots+\right]
    \end{aligned}
    \nonumber
\end{equation}
where $\beta$ is the uniformly discount rate for future utility.

\paragraph*{Dynamic Inconsistency} Preference hold at $t$ do not agree with preference hold at $t+1$.


\paragraph*{Application to Procrastination}
Cost Minimization Model:

Each period $t$ is consisted of three parts, 1. Pay cost $L$ because task is not done, 2. Observe current cost to do the task, $c\sim \textnormal{Unif}[0,1]$, 3. Choose to do task in $t$ or not.

Agent's decision follows a quasi hyperbolic discounting model, where we have $\beta<1$ and $\delta=1$. Agent has a ``sophisticated'' belief about her future behavior if $\hat{\beta}=\beta$ and has a ``naive'' belief about her future behavior if $\hat{\beta}=1$.

The optimal strategy is given by a critical cost $c^*$ such that the agent do the task if and only if $c\leq c^*$. Then, the expected un-discounted cost $v$ follows
\begin{equation}
    \begin{aligned}
        v=L+c^*\underbrace{\frac{c^*}{2}}_\textnormal{average cost conditional on $c\leq c^*$}+(1-c^*)v
    \end{aligned}
    \nonumber
\end{equation}
The sophisticated agent should be indifferent between doing the task and delaying:
\begin{equation}
    \begin{aligned}
        c^*=\beta v = \beta[L+c^*\frac{c^*}{2}+(1-c^*)v]
    \end{aligned}
    \nonumber
\end{equation}
which gives us
\begin{equation}
    \begin{aligned}
        c^*=\sqrt{\frac{L}{1/\beta-1/2}}
    \end{aligned}
    \nonumber
\end{equation}
which is increasing in $L$ and $\beta$.

The naive agent with $\hat{\beta}=1$ now using $c^*=\sqrt{\frac{L}{1/\hat{\beta}-1/2}}=\sqrt{2L}$ in the future. We have the critical cost at current time follows
\begin{equation}
    \begin{aligned}
        \tilde{c}=\beta[L+c^*\frac{c^*}{2}+(1-c^*)v]=\beta\sqrt{2L}
    \end{aligned}
    \nonumber
\end{equation}


\paragraph*{Doing it Now or Later (Matthew Rabin Lecture)}
A 2 hours effort today can save 10 mins every day in the future. The expected profit from doing the task today:
\begin{equation}
    \begin{aligned}
        U_t(today)=-120+\beta \delta 10 + \beta \delta^2 10 + \cdots=-120+\beta\frac{\delta}{1-\delta}10
    \end{aligned}
    \nonumber
\end{equation}
The expected profit from doing the task tomorrow with belief $\hat{\beta}$:
\begin{equation}
    \begin{aligned}
        U_t(tomorrow)=\beta\delta\left(-120+\hat{\beta}\frac{\delta}{1-\delta}10\right)
    \end{aligned}
    \nonumber
\end{equation}
With biased belief $\hat{\beta}>\beta$, there can exist infinite delay.



\section{\cite{fudenberg2006dual}: Dual-Self Model}
\subsubsection*{History:}
\begin{enumerate}
    \item $A$: The set of actions for the short-run selves (the set of probability measures is $\mathbf{A}$).
    \item $R$: The set of self-control actions for the long-run selves (the set of probability measures is $\mathbf{R}$). $0\in R$ is taken to mean that no self-control is used.
    \item $Y$: The set of states that encode the effects of history on current and future payoff possibilities (the set of probability measures is $\mathbf{Y}$).
    \item A finite history of play, $h_t\in H$, can consist of the past states and actions $h_t=(y_1,r_1,a_1,\cdots,y_t,r_t,a_t)$ or the null history $0$.
\end{enumerate}

\subsubsection*{Long-run self vs. short-run self:}
The game is played between the long-run self and a sequence of short-run selves. Each short-run self plays in only one period and observes the self-control action chosen by the long-run self prior to moving.
\begin{enumerate}
    \item \textbf{The long-run self's action $r$ only affects the short-run player's pay-off function $$u(y,r,a)\footnote{The utility is allowed to be $-\infty$ but not $+\infty$}$$}\\
    The mixed strategies of the long-run self are maps from histories $h$ and the current state $y$ to self-control actions $r$: $$\sigma_{LR}: H\times Y \rightarrow \mathbf{R}$$
    \item \textbf{All interactions with the outside world are handled by the short-run self:} $\mu(y_t,a_t)$ is the probability distribution over states at time $t+1$, which depends on the time-$t$ state $y_t$ and action $a_t$. Specifically, $\mu(y_t,a_t)[Y']$ denotes the probability of the set $Y'$ at time $t+1$.\\
    The strategy for the time-$t$ short-run self is a map $$\sigma_t:H_{t-1}\times Y\times R \rightarrow \mathbf{A},$$ where $H_{t}$ is the set of $t$-length histories. We denote the collection of all of these strategies by $\sigma_{SR}$.
\end{enumerate}
The utility of the long-run self is thus given by
\begin{equation}
    \begin{aligned}
        U_{LR}(\sigma_{LR},\sigma_{SR})=\sum_{t=1}^\infty\delta^{t-1}\int u\left(y(h_t),r(h_t),a(h_t)\right)d\pi_t(h_t),
    \end{aligned}
    \nonumber
\end{equation}
where $\pi_t(h_t):=\pi_t(h_t;\sigma_{LR},\sigma_{SR})$ is measure over histories of length $t$ for every $t$ that considers the strategies $(\sigma_{LR},\sigma_{SR})$ together with the measure $\mu$. The $(\sigma_{LR},\sigma_{SR})$ is suppressed for lighten notation.

\subsubsection*{Assumptions:}
We need to guarantee the expected utility to be finite:
\begin{assumption}[Upper Bounded on Utility Growth]
    For all initial conditions,
    \begin{equation}
        \begin{aligned}
            \sum_{t=1}^\infty\delta^{t-1}\int \max\{0,u(h_t)\}d\pi_t(h_t)<\infty.
        \end{aligned}
        \nonumber
    \end{equation}
\end{assumption}
Self-control is costly:
\begin{assumption}[Costly Self-Control]
    If $r\neq 0$ then $u(y,r,a)<u(y,0,a)$.
\end{assumption}
Every optimal short-run self action corresponds to a finite self-control action:
\begin{assumption}[Unlimited Self-Control]\label{ass:unlimited_self_control}
    For all $y,a$, there exists $r$ such that for all $a'$, $u(y,r,a)\geq u(y,r,a')$.
\end{assumption}
The cost of self-control can be defined as
\begin{equation}
    \begin{aligned}
        C(y,a)\equiv u(y,0,a)-\sup_{\{r|u(y,r,a)\geq u(y,0,\cdot)\}}u(y,r,a)
    \end{aligned}
    \nonumber
\end{equation}
Assumption~\ref{ass:unlimited_self_control} can avoid the above self-control cost going to infinity.

\begin{assumption}[Continunity]
    $u(y,r,a)$ is continuous in $(r,a)$.
\end{assumption}
This assumption assures the supremum in the definition
of $C$ can be replaced with a maximum. Then, the cost function has the following properties:
\begin{lemma}[Strict Cost of Self-Control]\label{lem:strict_cost}
    Property 1:
    \begin{enumerate}
        \item If $a\in\argmax_{a'}u(y,r,a')$, then $C(y,a)=0$.
        \item $C(y,a)>0$ for all $a$ such that $a\notin\argmax_{a'}u(y,0,a')$.
    \end{enumerate}
\end{lemma}
Conversely, given continuous functions $u(y,0,a)$ and $C(y,a)$ satisfying Property 1 in Lemma~\ref{lem:strict_cost}, we can extend that utility function to a function $u(y,r,a)$ that generates $C$ and satisfies all assumptions.


\bibliography{ref}

\end{document}