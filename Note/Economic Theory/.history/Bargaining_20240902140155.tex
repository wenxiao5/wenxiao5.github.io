\documentclass[11pt]{elegantbook}
\usepackage{graphicx}
%\usepackage{float}
\definecolor{structurecolor}{RGB}{40,58,129}
\linespread{1.6}
\setlength{\footskip}{20pt}
\setlength{\parindent}{0pt}
\newcommand{\argmax}{\operatornamewithlimits{argmax}}
\newcommand{\argmin}{\operatornamewithlimits{argmin}}
\elegantnewtheorem{proof}{Proof}{}{Proof}
\elegantnewtheorem{claim}{Claim}{prostyle}{Claim}
\DeclareMathOperator{\col}{col}
\title{Bargaining}
\author{Wenxiao Yang}
\institute{Haas School of Business, University of California Berkeley}
\date{2024}
\setcounter{tocdepth}{2}
\extrainfo{All models are wrong, but some are useful.}

\cover{cover.png}

% modify the color in the middle of titlepage
\definecolor{customcolor}{RGB}{32,178,170}
\colorlet{coverlinecolor}{customcolor}
\usepackage{cprotect}


\bibliographystyle{apalike_three}

\begin{document}
\maketitle

\frontmatter
\tableofcontents

\mainmatter

\chapter{\cite{fuchs2022dynamic}: Dynamic Bargaining with Private Information}

\section{``Classic'' Coase Conjecture}
Consider a seller facing a buyer has private value of the good $v\in[1,2]$ that is distributed according to an atomless distribution with full support, $F(v)$. The seller has cost $c\geq 0$ to serve the buyer. Every period of an infinite horizon game, the uninformed seller makes an offer $p_t$. If $p_t$ is accepted, the game ends with payoffs $v-p_t$ and $p_t-c$ for the buyer and the seller, respectively. If $p_t$ is rejected, the seller makes another offer at time $t+\Delta$. Both buyer and seller discount future with the rate $r$. Thus, $\Delta$ can be thought of as the commitment power of the seller.

Since higher-value buyers lose more from delay, the Skimming Property holds in any equilibrium.
\begin{lemma}[Skimming Property]
    If type $v$ is willing to accept an offer $p_t$, then all higher types strictly prefer to accept this offer.
\end{lemma}
Denote by $k_t$ the highest type remaining before the offer at time $t$ is made. Then, the seller beliefs at time $t$ are $v\in[1,k_t]$ with truncated distribution $F(v)/F(k_t)$, where $k_0=2$.


A distinction arises in this game: (1). $c<1$ (the gap case); (2). $c=1$ (the no-gap case).

As shown by \cite{fudenberg1985infinite} and \cite{gul1986foundations}, the game must end in a finite number of rounds (uniformly bounded for all $\Delta$). The intuition is: the difference between distinguishing $k_t$ and an immediate trading $1$ is $k_t-1$. If the $k_t-1$ is relative small to $1-c$. The seller prefers an immediate trade rather than keep bargaining. So, there is a critical $k_t$ that is close enough to $1$ and the seller offers $p_t=1$ to get an immediate trade. These equilibria have the property that the seller value depends only on the state of the game, $k_t$, a property called ``stationarity'' that is similar to the equilibria being Markov.

























































































\bibliography{ref}




\end{document}