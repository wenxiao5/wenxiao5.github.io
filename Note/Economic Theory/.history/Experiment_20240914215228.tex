\documentclass[11pt]{elegantbook}
\usepackage{graphicx}
%\usepackage{float}
\definecolor{structurecolor}{RGB}{40,58,129}
\linespread{1.6}
\setlength{\footskip}{20pt}
\setlength{\parindent}{0pt}
\newcommand{\argmax}{\operatornamewithlimits{argmax}}
\newcommand{\argmin}{\operatornamewithlimits{argmin}}
\elegantnewtheorem{proof}{Proof}{}{Proof}
\elegantnewtheorem{claim}{Claim}{prostyle}{Claim}
\DeclareMathOperator{\col}{col}
\title{Experimental Economics}
\author{Wenxiao Yang}
\institute{Haas School of Business, University of California Berkeley}
\date{2024}
\setcounter{tocdepth}{2}
\extrainfo{All models are wrong, but some are useful.}

\cover{cover.png}

% modify the color in the middle of titlepage
\definecolor{customcolor}{RGB}{32,178,170}
\colorlet{coverlinecolor}{customcolor}
\usepackage{cprotect}


\bibliographystyle{apalike_three}

\begin{document}
\maketitle

\frontmatter
\tableofcontents

\mainmatter



\chapter{Becker-DeGroot-Marschak Mechanism}
\cite{karni1987preference} showed that the BDM is not incentive compatible when the object being valued is a lottery. The BDM can elicit the certainty equivalents of given lotteries if and only if the respondent's preferences can be represented by expected utility functional.


\chapter{}
\section{\cite{simonson1989choice}: Choice Based on Reasons: The Case of Attraction and Compromise Effects}
Two effects about consumers' choices are introduced:
\begin{enumerate}
    \item \textbf{Attraction Effect} (asymmetric dominance effect): When an asymmetrically dominated or relatively inferior alternative is added to a set, the attractiveness and choice probability of the dominating alternative increase.
    \item \textbf{Compromise Effect}: An alternative would tend to gain market share when it becomes a compromise or middle option in the set.
\end{enumerate}

This paper use the framework that decision makers make the choice that is supported by the best overall reasons to analyze these two effects.

That is good that this paper want to explain two effects by one theory. However, the exploitation the paper proposed to explain these two effects is not completely proven by experiments. Decision makers concern others' evaluations and they make decisions based on their expectation on evaluator's preferences. So, what if the evaluators' preferences are determined. I would design an experiment that the participants are told to be evaluated by evaluators with a common extreme preference. Then, based on the theory the paper proposed, the participants would tend to choose an extreme alternative that fit the preferences more instead of a middle alternative.














































































\bibliography{ref_BE}




\end{document}