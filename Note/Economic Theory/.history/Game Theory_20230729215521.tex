\documentclass[11pt]{elegantbook}
\definecolor{structurecolor}{RGB}{40,58,129}
\linespread{1.6}
\setlength{\footskip}{20pt}
\setlength{\parindent}{0pt}
\newcommand{\argmax}{\operatornamewithlimits{argmax}}
\newcommand{\argmin}{\operatornamewithlimits{argmin}}
\elegantnewtheorem{proof}{Proof}{}{Proof}
\elegantnewtheorem{claim}{Claim}{prostyle}{Claim}
\DeclareMathOperator{\col}{col}
\title{\textbf{Game Theory}}
\author{Wenxiao Yang}
\institute{Haas School of Business, University of California Berkeley}
\date{2023}
\setcounter{tocdepth}{2}
\cover{cover.png}
\extrainfo{All models are wrong, but some are useful.}

% modify the color in the middle of titlepage
\definecolor{customcolor}{RGB}{9,119,119}
\colorlet{coverlinecolor}{customcolor}
\usepackage{cprotect}

\addbibresource[location=local]{reference.bib} % bib

\begin{document}

\maketitle
\frontmatter
\tableofcontents
\mainmatter



\chapter{Signalling Game}
Based on
\begin{enumerate}[$\circ$]
    \item "Kreps, D. M., \& Sobel, J. (1994). Signalling. \textit{Handbook of game theory with economic applications}, 2, 849-867."
    \item 
\end{enumerate}

\section{Canonical Game}
\begin{definition}[Canonical Game]
    \normalfont
    \begin{enumerate}
        \item There are two players: $S$ (sender) and $R$ (receiver).
        \item $S$ holds more information than $R$: the value of some random variable $t$ with support $T$.
    \end{enumerate}
\end{definition}











\end{document}