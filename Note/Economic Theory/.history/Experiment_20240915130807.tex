\documentclass[11pt]{elegantbook}
\usepackage{graphicx}
%\usepackage{float}
\definecolor{structurecolor}{RGB}{40,58,129}
\linespread{1.6}
\setlength{\footskip}{20pt}
\setlength{\parindent}{0pt}
\newcommand{\argmax}{\operatornamewithlimits{argmax}}
\newcommand{\argmin}{\operatornamewithlimits{argmin}}
\elegantnewtheorem{proof}{Proof}{}{Proof}
\elegantnewtheorem{claim}{Claim}{prostyle}{Claim}
\DeclareMathOperator{\col}{col}
\title{Experimental Economics}
\author{Wenxiao Yang}
\institute{Haas School of Business, University of California Berkeley}
\date{2024}
\setcounter{tocdepth}{2}
\extrainfo{All models are wrong, but some are useful.}

\cover{cover.png}

% modify the color in the middle of titlepage
\definecolor{customcolor}{RGB}{32,178,170}
\colorlet{coverlinecolor}{customcolor}
\usepackage{cprotect}


\bibliographystyle{apalike_three}

\begin{document}
\maketitle

\frontmatter
\tableofcontents

\mainmatter



\chapter{Becker-DeGroot-Marschak Mechanism}
\cite{karni1987preference} showed that the BDM is not incentive compatible when the object being valued is a lottery. The BDM can elicit the certainty equivalents of given lotteries if and only if the respondent's preferences can be represented by expected utility functional.


\chapter{}
\section{\cite{simonson1989choice}: Choice Based on Reasons: The Case of Attraction and Compromise Effects}
Two effects about consumers' choices are introduced:
\begin{enumerate}
    \item \textbf{Attraction Effect} (asymmetric dominance effect): When an asymmetrically dominated or relatively inferior alternative is added to a set, the attractiveness and choice probability of the dominating alternative increase.
    \item \textbf{Compromise Effect}: An alternative would tend to gain market share when it becomes a compromise or middle option in the set.
\end{enumerate}
This paper use the framework that decision makers make the choice that is supported by the best overall reasons to analyze these attraction and compromise effects.

It’s good that this paper aims to explain both the attraction and compromise effects through a single framework. However, the mechanism and explanation the paper proposes for these two effects is not fully validated by the experiments.

Decision-makers take into account others’ evaluations and make choices based on their expectations of the evaluators’ preferences. One reason for choosing middle alternatives may be the uncertainty surrounding these preferences. But what if the evaluators’ preferences are clearly defined and extreme? I would propose an experiment where participants are told their decisions will be evaluated by evaluators with a common, extreme preference. According to the theory presented in the paper, participants would then be more likely to choose an extreme alternative that aligns with the evaluators’ preferences, rather than opting for a middle option. This could help further test and refine the proposed mechanism and explanation.


\section{\cite{simonson1992choice}: Choice in Context: Tradeoff Contrast and Extremeness Aversion}
This paper uses context effect based approach to explain trade-off contrast and extremeness aversion, which also provides a new mechanism for the attraction and compromise effects in the first paper. It is a nice perspective to understand the decision makers' choices from reference dependence and loss aversion. This explanation is power and often ignored in value maximization framework in classical economic theory. This approach can help a lot in product line design and product attributes design for firms. In the future (or maybe have been done), it would be really nice to have formal models that formalize this framework in economic theory.













































































\bibliography{ref_BE}




\end{document}