\documentclass[11pt]{elegantbook}
\definecolor{structurecolor}{RGB}{40,58,129}
\linespread{1.6}
\setlength{\footskip}{20pt}
\setlength{\parindent}{0pt}
\newcommand{\argmax}{\operatornamewithlimits{argmax}}
\newcommand{\argmin}{\operatornamewithlimits{argmin}}
\elegantnewtheorem{proof}{Proof}{}{Proof}
\elegantnewtheorem{claim}{Claim}{prostyle}{Claim}
\DeclareMathOperator{\col}{col}
\title{\textbf{ECON 201B}}
\author{Wenxiao Yang}
\institute{Haas School of Business, University of California Berkeley}
\date{2024}
\setcounter{tocdepth}{2}
\cover{cover.png}
\extrainfo{Seeking what is true is not seeking what is desirable.}

% modify the color in the middle of titlepage
\definecolor{customcolor}{RGB}{9,119,119}
\colorlet{coverlinecolor}{customcolor}
\usepackage{cprotect}

\addbibresource[location=local]{reference.bib} % bib

\begin{document}

\maketitle
\frontmatter
\tableofcontents
\mainmatter




\chapter{Correspondence: $\Psi : X \rightarrow 2^Y$ \small{(@ Lec 07 of ECON 204)}}
\begin{definition}[Correspondence]
    \normalfont
    A \textbf{correspondence} $\Psi : X \rightarrow 2^Y$ from $X$ to $Y$ is a function from $X$ to $2^Y$, that is, $\Psi(x) \subseteq Y$ for every $x \in X$. ($2^Y$ is the set of all subsets of $Y$)
\end{definition}
\begin{example}
Let $u : \mathbb{R}_+^n \rightarrow \mathbb{R}$ be a continuous utility function, $y > 0$ and $p \in \mathbb{R}_{++}^n$, that is, $p_i > 0$ for each $i$. Define $\Psi : \mathbb{R}_{++}^n \times \mathbb{R}_{++} \rightarrow 2^{\mathbb{R}_{+}^n}$ by
\begin{equation}
    \begin{aligned}
        \Psi(p,y)=&\argmax u(x)\\
        \textnormal{s.t. }&x\geq 0\\
        &p\cdot x\leq y
    \end{aligned}
    \nonumber
\end{equation}
$\Psi$ is the demand correspondence associated with the utility function $u$; typically $\Psi(p, y)$ is multi-valued.
\end{example}

\section{Continuity of Correspondences}
\subsection{Upper/Lower Hemicontinuous}
Let $X\subseteq \mathbb{E}^n$, $Y\subseteq \mathbb{E}^m$, and $\Psi: X \rightarrow 2^Y$.
\begin{definition}[Upper Hemicontinuous]
    \normalfont
    $\Psi$ is \textbf{upper hemicontinuous} (uhc) at $x_0 \in X$ if, for every \underline{open set} $V$ with $\Psi(x_0)\subseteq V$, there is an \underline{open set} $U$ with $x_0 \in U$ s.t.
    $$x\in U \Rightarrow \Psi(x)\subseteq V$$
\end{definition}
Upper hemicontinuity reflects the requirement that $\Psi$ doesn't “jump down/implode in the limit” at $x_0$. \textit{(A set to “jump down” at the limit $x_0$: It should mean the set suddenly gets smaller -- it “implodes in the limit” -- that is, there is a sequence $x_n \rightarrow x_0$ and points $y_n \in \Psi(x_n)$ that are far from every point of $\Psi(x_0)$ as $n \rightarrow \infty$.)}
\begin{definition}[Lower Hemicontinuous]
    \normalfont
    $\Psi$ is \textbf{lower hemicontinuous} (lhc) at $x_0 \in X$ if, for every \underline{open set} $V$ with $\Psi(x_0)\cap V \neq \emptyset$, there is an \underline{open set} $U$ with $x_0 \in U$ s.t.
    $$x\in U \Rightarrow \Psi(x)\cap V\neq \emptyset$$
\end{definition}
Lower hemicontinuity reflects the requirement that $\Psi$ doesn't “jump up/explode in the limit” at $x_0$. \textit{(A set to “jump up” at the limit $x_0$: It should mean that the set suddenly gets bigger -- it “explodes in the limit” -- that is, there is a sequence $x_n \rightarrow x_0$ and a point $y_0\in\Psi(x_0)$ that is far from every point of $\Psi(x_n)$ as $n \rightarrow \infty$.)}

\begin{definition}[Continuous Correspondence]
    \normalfont
    $\Psi$ is \textbf{continuous} at $x_0 \in X$ if it is both \textbf{uhc} and \textbf{lhc} at $x_0$.
\end{definition}

\begin{proposition}
    $\Psi$ is upper hemicontinuous (respectively lower hemicontinuous, continuous) if it is uhc (respectively lhc, continuous) at every $x \in X$.
\end{proposition}

\begin{center}\begin{figure}[htbp]
    \centering
    \includegraphics[scale=0.2]{uhc.png}
    \caption{The correspondence $\Psi$ “implodes in the limit” at $x_0$. $\Psi$ is not upper hemicontinuous at $x_0$.}
    \label{}
\end{figure}\end{center}

\begin{center}\begin{figure}[htbp]
    \centering
    \includegraphics[scale=0.2]{lhc.png}
    \caption{The correspondence $\Psi$ “explodes in the limit” at $x_0$. $\Psi$ is not lower hemicontinuous at $x_0$.}
    \label{}
\end{figure}\end{center}

\subsection{Theorem: $\Psi(x)=\{f(x)\}$ is uhc $\Leftrightarrow$ $f$ is continuous}
\begin{theorem}[$\Psi(x)=\{f(x)\}$ is uhc $\Leftrightarrow$ $f$ is continuous]
    Let $X \subseteq \mathbb{E}^n$, $Y \subseteq \mathbb{E}^m$ and $f : X \rightarrow Y$. Let $\Psi : X \rightarrow  2^Y$ be defined by $\Psi(x) = \{f(x)\}$ for all $x \in X$. Then $\Psi$ is \textbf{uhc} \underline{if and only if} $f$ is \textbf{continuous}.
\end{theorem}


\subsection{Berge's Maximum Theorem: the set of maximizers is uhc with non-empty compact values}
\begin{theorem}[Berge's Maximum Theorem]\label{thm:Berge's Maximum Theorem}
    Let $X \subseteq \mathbb{R}^n$ and $Y \subseteq \mathbb{R}^m$. Consider the function $f : X \times Y \rightarrow \mathbb{R}$ and the correspondence $\Gamma : Y \rightarrow 2^X$. Define $v(y) = \max_{x\in\Gamma(y)} f(x, y)$ and the set of maximizers $$\Omega(y) = \argmax_{x\in\Gamma(y)} f(x, y)=\{x:f(x,y)=v(y)\}$$
    Suppose $f$ and $\Gamma$ are continuous, and that $\Gamma$ has non-empty compact values. Then, $v$ is continuous and $\Omega$ is uhc with non-empty compact values.
\end{theorem}




\section{Graph of Correspondence}
An alternative notion of continuity looks instead at properties of the graph of the correspondence.
\begin{definition}[Graph of Correspondence]
    \normalfont
    The \textbf{graph} of a correspondence $\Psi : X \rightarrow 2^Y$ is the set
    $$\textnormal{graph}\Psi=\{(x,y)\in X\times Y:y\in\Psi(x)\}$$
\end{definition}

\subsection{Closed Graph}
By the definition of continuous function $f:\mathbb{R}^n \rightarrow \mathbb{R}$,  each convergent sequence $\{(x_n, y_n)\}$ in graph $f$ converges to a point $(x, y)$ in graph $f$, that is, graph $f$ is closed.

\begin{definition}[Closed Graph]
    \normalfont
    Let $X\subseteq \mathbb{E}^n$, $Y\subseteq \mathbb{E}^m$. A correspondence $\Psi: X \rightarrow 2^Y$ has closed graph if its graph is a closed subset of $X \times Y$, that is, if for any sequences $\{x_n\} \subseteq X$ and $\{y_n\} \subseteq Y$ such that $x_n \rightarrow x \in X$, $y_n \rightarrow y \in Y$ and $y_n \in \Psi(x_n)$ for each $n$, then $y \in \Psi(x)$.
\end{definition}
\begin{example}
    Consider the correspondence $\Psi(x)=\left\{\begin{matrix}
        \{\frac{1}{x}\},&\textnormal{ if }x\in(0,1]\\
        \{0\},&\textnormal{ if }x=0
    \end{matrix}\right.$ ("implode in the limit")\\
    Let $V = (-0.1, 0.1)$. Then $\Psi(0) = \{0\} \subseteq V$, but no matter how close $x$ is to $0$, $\Psi(x)=\{\frac{1}{x}\}\nsubseteq V$, so $\Psi$ is not uhc at $0$. However, note that $\Psi$ has closed graph.
\end{example}

\section{Closed-valued, Compact-valued, and Convex-valued Correspondences}
\begin{definition}[Closed-valued, Compact-valued, and Convex-valued Correspondences]
    \normalfont
    Given a correspondence $\Psi : X \rightarrow 2^Y$,
    \begin{enumerate}
        \item $\Psi$ is \textbf{closed-valued} if $\Psi(x)$ is a closed subset of $Y$ for all $x$;
        \item $\Psi$ is \textbf{compact-valued} if $\Psi(x)$ is compact for all $x$.
        \item $\Psi$ is \textbf{convex-valued} if $\Psi(x)$ is convex for all $x$.
    \end{enumerate}
\end{definition}

\subsection{Closed-valued, uhc and Closed Graph}
For closed-valued correspondences these concepts can be more tightly connected. A closed-valued and upper hemicontinuous correspondence must have closed graph. For a closed-valued correspondence with a compact range, upper hemicontinuity is equivalent to closed graph.

\begin{theorem}[uhc and Closed Graph]
    Let $X\subseteq \mathbb{E}^n$, $Y\subseteq \mathbb{E}^m$, and $\Psi: X \rightarrow 2^Y$.
    \begin{enumerate}
        \item $\Psi$ is \textbf{closed-valued} and \textbf{uhc} $\Rightarrow$ $\Psi$ has \textbf{closed graph}.
        \item $\Psi$ is \textbf{closed-valued} and \textbf{uhc} $\Leftarrow$ $\Psi$ has \textbf{closed graph}. (If $Y$ is \textbf{compact})
    \end{enumerate}
\end{theorem}

\begin{theorem}
    Let $X\subseteq \mathbb{E}^n$, $Y\subseteq \mathbb{E}^m$, and $\Psi: X \rightarrow 2^Y$. If $\Psi$ has \textbf{closed graph} and there is an \textbf{open set} $W$ with $x_0 \in W$ and a \textbf{compact set} $Z$ such that $x \in W \cap X \Rightarrow \Psi(x) \subseteq Z$, then $\Psi$ is \textbf{uhc} at $x_0$.
\end{theorem}



\subsection{Theorem: compact-valued, uhs correspondence of compact set is compact}
\begin{theorem}\label{thm:compact-valued, uhs correspondence of compact set is compact}
    Let $X$ be a compact set and $\Psi : X \rightarrow 2^X$ be a non-empty, compact-valued upper-hemicontinuous correspondence. If $C \subseteq X$ is compact, then $\Psi(C)$ is compact.
\end{theorem}
\begin{proof}
    Given the compact-valued $\Psi$, we can have an open cover of $\Psi(C)$, $\{U_\lambda:\lambda\in\Lambda\}$. So $\forall x\in C$, there exists $U_{l(x)},l(x)\in\Lambda$ such that $U_{l(x)}$ is an open cover of $\Psi(x)$.

    Consider a $c\in C$. Since $\Psi$ is uhs and $\Psi(c)\subseteq U_{l(c)}$, there exists open set $V_c$ s.t. $c\in V_c$ and $\Psi(x)\subseteq U_{l(c)}, \forall x\in V_c\cap C$.

    $\{V_c:c\in C\}$ is an open cover of $C$. Because $C$ is compact, there is a finite subcover $\{V_{c_i}: i=1,...,m\},m\in \mathbb{N}$, where $\{c_i:i=1,...,m\}\subseteq C$.

    Because $\Psi(x)\subseteq U_{l(c_i)}, \forall x\in V_{c_i}\cap C$ and $\{V_{c_i}: i=1,...,m\},m\in \mathbb{N}$ is a open cover for $C$, we can infer $\{U_{l(c_i)}:i=1,...,m\}$ is a finite subcover of $\{U_{l(c)}:c\in C\}$ for $\Psi(C)$. Hence, $\Psi(C)$ is compact.
\end{proof}

\section{Fixed Points for Correspondences \small{(@ Lec 13 of ECON 204)}}
\subsection{Definition}
\begin{definition}[Fixed Points for Correspondences]
    \normalfont
    Let $X$ be nonempty and $\psi : X \rightarrow 2^X$ be a correspondence. A point $x^* \in X$ is a fixed point of $\psi$ if $x^* \in \psi(x^*)$.
\end{definition}
\begin{note}
    We only need $x^*$ to be in $\psi(x^*)$, not $\{x^*\} = \psi(x^*)$. That is, $\psi$ need not be single-valued at $x^*$. So $x^*$ can be a fixed point of $\psi$ but there may be other elements of $\psi(x^*)$ different from $x^*$.
\end{note}



\subsection{Kakutani's Fixed Point Theorem: uhs, compact, convex values correspondence has a fixed point over compact convex set}
\begin{theorem}[Kakutani's Fixed Point Theorem]\label{thm:Kakutani's Fixed Point Theorem}
    Let $X \subseteq \mathbb{R}^n$ be a non-empty, \textbf{compact}, \textbf{convex} set and $\psi : X \rightarrow 2^X$ be an \textbf{upper hemi-continuous} correspondence with non-empty and \textbf{convex} values. Then $\psi$ has a fixed point in $X$.
\end{theorem}


\subsection{Theorem: $\exists$ compact set $C = \cap_{i=0}^\infty \Psi^i(X)$ s.t. $\Psi(C)=C$}
\begin{theorem}
    Let $(X, d)$ be a compact metric space and let $\Psi(x) : X \rightarrow 2^X$ be a upper-hemicontinuous, compact-valued correspondence, such that $\Psi(x)$ is non-empty for every $x \in X$. There exists a compact non-empty subset $C\subseteq X$, such that $\Psi(C) \equiv \cup_{x\in C}\Psi(x) = C$.
\end{theorem}
\begin{proof}
    Let's construct a sequence $\{C_n\}$ such that $C_0=X$, $C_1=\Psi(C_0)$, ..., $C_n=\Psi(C_{n-1}),...$ We claim that $C=\cap_{i=0}^\infty C_i$ is a non-empty compact set and satisfies $\Psi(C)=C$.
    \begin{enumerate}
        \item Because we can infer $\Psi(X_1)\subseteq \Psi(X_2)$ if $X_1\subseteq X_2$, $X=C_0\supseteq C_1 \Rightarrow C_1=\Psi(C_0)\supseteq C_2=\Psi(C_1)$,...., so $C_0\supseteq C_1\supseteq \cdots C_n\supseteq \cdots$. Hence, $C$ is not empty.
        \item Because $X$ is compact, by the theorem \ref{thm:compact-valued, uhs correspondence of compact set is compact}, we can infer $C_n$ is compact for all $n$. Then, $C_n$ is closed for all $n$, so $C$ is closed. Because $C$ is a closed set of compact set $X$, $C$ is compact.
        \item $C\subseteq C_n,\forall n \Rightarrow \Psi(C)\subseteq \Psi(C_n),\forall n \Rightarrow \Psi(C)\subseteq C$
        \item Assume $C\subseteq \Psi(C)$ doesn't hold, that is $\exists y\in C$ s.t. $y\notin \Psi(C)$. Because $y\in C$ and $C_0\supseteq C_1\supseteq \cdots C_n\supseteq \cdots$, there exists $k\in C_n$ for all $n$ s.t. $y\in\Psi(k)$. $k\in \cap_{i=1}^\infty C_i=C$, so $\Psi(k)\subseteq \Psi(C)$, which contradicts to $y\notin \Psi(C)$. Hence, $C\subseteq \Psi(C)$.
    \end{enumerate}
    All in all the claim "$C=\cap_{i=0}^\infty C_i$ is a non-empty compact set and satisfies $\Psi(C)=C$" is proved.
\end{proof}





\chapter{Game Theory}
Based on
\begin{enumerate}[$\circ$]
    \item "Kreps, D. M., \& Sobel, J. (1994). Signalling. \textit{Handbook of game theory with economic applications}, 2, 849-867."
    \item Mas-Colell, Whinston, and Green, Microeconomic Theory, Oxford University Press (1995).
    \item UIUC ECON 530 21Fall, Nolan H. Miller
    \item UC Berkeley ECON 201A 23Fall, 201B 24Spring
    \item UC Berkeley MATH 272 23Fall, Alexander Teytelboym
    \item  Jehle, G., Reny, P.: Advanced Microeconomic Theory. Pearson, 3rd ed. (2011). Ch. 6.
\end{enumerate}



\section{Basic Game Theory}
\subsection{Action and Domination Theorem}
Let $A$ be the finite set of possible actions and $\Omega$ be the finite set of possible states. A function can map the action and state to a value, $u(a,\omega)$. It can be represented by $\vec{u}(a)=\{u(a,\omega)\}_{\omega\in\Omega}$. It is common in game theory to assume the utility function is given or known.

A \textbf{mixed action} is a probability distribution over $A$, $\sigma\in\Delta(A)$.

A \textbf{belief} of the agent is a probability distribution over $\Omega$, $\mu\in\Delta(\Omega)$.

\begin{definition}[Optimal and Justifiable Mixed Action]
    \normalfont
    A mixed action $\sigma\in\Delta(A)$ is \textbf{optimal} given $\mu\in\Delta(\Omega)$ if $$\mathbb{E}_\mu u(\sigma,\tilde{\omega})\geq \mathbb{E}_\mu u(\sigma',\tilde{\omega}),\ \forall \sigma'\in \Delta(A)$$
    A mixed action $\sigma\in\Delta(A)$ is \textbf{justifiable} if it is optimal for some belief $\mu\in\Delta(\Omega)$.
\end{definition}

\begin{definition}[Dominant and Dominated Action]
    \normalfont
    A mixed action $\sigma\in\Delta(A)$ is \textbf{dominant} if $$u(\sigma,\omega)>u(\sigma',\omega),\ \forall \omega\in \Omega, \sigma'\in \Delta(A),\sigma\neq\sigma'$$
    A mixed action $\sigma\in\Delta(A)$ is \textbf{dominated} if $$u(\sigma,\omega)<u(\sigma',\omega),\ \forall \omega\in \Omega, \text{ and for some } \sigma'\in \Delta(A)$$
    In this case we say $\sigma'$ dominates $\sigma$.
\end{definition}

\begin{theorem}[Domination Theorem: Justifiable $=$ Not Dominated]
    A mixed action is justifiable \underline{if and only if} it is not dominated.
\end{theorem}
\begin{proof}
    $\Rightarrow$ is easily proved by the definition. We focus on proving $\Leftarrow$:
    
    Let $\mathcal{U}=\{\vec{u}(\sigma):\sigma\in\Delta(A)\}$ and $\sigma^*$ be an undominated mixed action. Then, we have $\mathcal{U}\cap(\{\vec{u}(\sigma^*)\}+\mathbb{R}_{++}^\Omega)=\emptyset$. Because $\mathcal{U}$ and $\{\vec{u}(\sigma^*)\}+\mathbb{R}_{++}^\Omega$ are disjoint, convex, and nonempty, we can use the Separating Hyperplane Theorem \ref{SHT}: $\exists p\in \mathbb{R}^\Omega,p\neq 0$ such that $p\cdot a\leq p\cdot b, \forall a\in\mathcal{U}, b\in (\{\vec{u}(\sigma^*)\}+\mathbb{R}_{++}^\Omega)$.

    \begin{claim}
        $p\cdot \vec{u}(\sigma)\leq p\cdot \vec{u}(\sigma^*), \forall \sigma\in\Delta(A)$.
    \end{claim}
    \begin{proof}
        For any positive number $m$, $\vec{u}(\sigma^*)+(\frac{1}{m},....,\frac{1}{m})\in \{\vec{u}(\sigma')\}+\mathbb{R}_{++}^\Omega$. So, for any $\sigma\in\Delta(A)$, $p\cdot \vec{u}(\sigma)\leq p\cdot\left(\vec{u}(\sigma^*)+(\frac{1}{m},....,\frac{1}{m})\right)$. By taking limit, $p\cdot \vec{u}(\sigma^*)=\lim_{m \rightarrow \infty}p\cdot\left(\vec{u}(\sigma^*)+(\frac{1}{m},....,\frac{1}{m})\right)\geq p\cdot \vec{u}(\sigma)$.
    \end{proof}
    \begin{claim}
        $p>0$.
    \end{claim}
    \begin{proof}
        Prove by the contradiction. Suppose $p_\omega<0$ for some $\omega\in\Omega$. Let $z=(\epsilon,...,\epsilon)+M\mathbb{1}_\omega, M>0,\epsilon>0$. So, $\vec{u}(\sigma^*)+z\in (\{\vec{u}(\sigma^*)\}+\mathbb{R}_{++}^\Omega)$. We have $p\cdot\vec{u}(\sigma^*)\leq p\cdot (\vec{u}(\sigma^*)+z)$ by the result of SHT. There is a contradiction since $p_\omega<0$. So, we have $p\geq 0$. Because $p\neq 0$, $p>0$ is proved.
    \end{proof}
    Finally, we normalize $p$ to $\mu=\frac{1}{\sum_{\omega}p_\omega}p$. Then, $\sigma^*$ is optimal for the belief $\mu$, which means $\sigma^*$ is justifiable.
\end{proof}










\subsection{Extensive Game}
\begin{definition}[History]
    \normalfont
    The sequences of actions are called \textbf{histories}. $h'=(\underbrace{a_1,...,a_n}_{h: \text{prefix of }h'},a_{n+1},...)\in H$. We call $h'$ is the \textbf{continuation} of $h$. $h$ is a \textbf{terminal} of $H$ if there is no continuation of $h$ in $H$. ($\emptyset\in H$.)
\end{definition}

\begin{definition}[Extensive form Perfect Information Game]
    \normalfont
    Am extensive form game with prefect information is defined as $G=\{N,A,H,Z,P,O,o,\succ_{n\in N}\}$, where $N$ is the set of players, $A$ is the set of actions, $H$ is the set of all histories, $Z$ is the set of all histories that are terminals, $P:H/Z \rightarrow N$ is a mapping from a non-terminal histories to a player (who moves after a non-terminal history), $O$ is the set of outcomes, and $o$ is a function from $Z$ to $O$.\\
    A PIG is \underline{finite horizon} if there is a bound on the length of its histories.
\end{definition}


We denote $A(h)$ as the actions available to player $P(h)$ after a history $h$.

Let $H_i=\{h\in H/Z:i=P(h)\}$ be the set of histories that player $i$ moves after.

\begin{definition}[Strategy]
    \normalfont
    A \textbf{strategy} is defined as a function $s_i:H_i \rightarrow A$ for which $s_i(h)\in A(h),\forall h\in H_i$. Let $S_i$ be the set of all strategies available to the player $i$. A \textbf{strategy profile} is a collection of strategy $s=(s_i)_{i\in N}$.
\end{definition}



\begin{definition}[Subgame]
\normalfont
    A \textbf{subgame} of a PIG $G=\{N,A,H,Z,P,O,o,\succ_{n\in N}\}$ is a game (a PIG) that starts after a given finite history $h\in H$. Formally, the subgame $G(h)$ associated with $h=(h_1,...,h_n)\in H$ is $G(h)=\{N,A,H_h,Z,P_h,O,o_h,\succ_{n\in N}\}$, where
    \begin{equation}
        \begin{aligned}
            H_h=\{(a_1,a_2,...):(h_1,...,h_n,a_1,a_2,...)\in H\}\\
            o_h(h')=o(hh'), P_h(h')=P(hh')
        \end{aligned}
        \nonumber
    \end{equation}
    A strategy $s$ of $G$ defines a strategy $s_h$ of $G(h)$ by $s_h(h')=s(hh')$.
\end{definition}

\begin{definition}[Subgame Perfect Equilibrium (SPNE)]
    \normalfont
    A \textbf{subgame perfect equilibrium (SPNE)} of $G$ is a strategy profile $s^*$ such that for every subgame $G(h)$ it holds that $h^{\prime} \mapsto s_i^*\left(h h^{\prime}\right)$ is an optimal strategy in $G(h)$, given beliefs that the rest of the players behave according to $s_{-i}^*$ (or its restriction to $G(h)$).
\end{definition}

\begin{definition}[Profitable Deviation]
    \normalfont
    Let $s$ be a strategy profile. We say that $s_i^{\prime}$ is a \textbf{profitable deviation} from $s$ for player $i$ at history $h$ if $s_i^{\prime}$ is a strategy for $G$ such that
    $$
    o_h\left(s_i^{\prime}, s_{-i}\right) \succ_i o_h(s)
    $$
\end{definition}
Note that a strategy profile has no profitable deviations iff it's a SPNE.

\begin{theorem}[The one-deviation principle]
    Let $G=\left(N, A, H, O, o, P,\left\{\preceq_i\right\}_{i \in N}\right)$ be a finite horizon, extensive form game with perfect information. Let $s$ be a strategy profile that is \underline{not} a subgame perfect equilibrium. There exists some history $h$ and a profitable deviation $\bar{s}_i$ for player $i=P(h)$ in $G(h)$ such that $\bar{s}_i(k)=s_i(k)$ for all $k \neq h$.
\end{theorem}
\begin{enumerate}[$\circ$]
    \item Let $G=\left(N, A, H, O, o, P,\left\{\preceq_i\right\}_{i \in N}\right)$ be a PIG.
    \item $A(\emptyset)$ is the set of allowed initial actions for player $i=P(\emptyset)$. For each $b \in A(\emptyset)$, let $s^{G(b)}$ be some strategy profile for the subgame $G(b)$.
    \item Given some $a \in A(\emptyset)$, we denote by $s^a$ the strategy profile for $G$ in which player $i=P(\emptyset)$ chooses the initial action $a$, and for each action $b \in A(\emptyset)$ the subgame $G(b)$ is played according to $s^{G(b)}$.
    \item So $s_i^a(\emptyset)=a$ and for every player $j, b \in A(\emptyset)$ and $b h \in H \backslash Z$, $s_j^a(b h)=s_j^{G(b)}(h)$.
\end{enumerate}
\begin{lemma}[Backward Induction]
    Let $G=\left(N, A, H, Z, O, o, P,\left\{\preceq_i\right\}_{i \in N}\right)$ be a finite PIG. Assume that for each $b \in A(\emptyset)$ the subgame $G(b)$ has a subgame perfect equilibrium $s^{G(b)}$. Let $i=P(\emptyset)$ and let $a$ be the $\succ_i$-maximizer over $A(\emptyset)$ of $o_a\left(s^{G(a)}\right)$. Then $s^a$ is a subgame perfect equilibrium of $G$.
\end{lemma}


\subsection{Strategic Form Game}
\begin{definition}[Normal Form Game]
    \normalfont
    A game in \textbf{normal form} is denoted by $$G =\left(\underbrace{N}_{\textnormal{players}},\underbrace{\{S_i\}_{i\in N}}_{\textnormal{Strategy Set}},\underbrace{\{u_i(\cdot)\}_{i\in N}}_{\textnormal{VNM utility}}\right)$$

    $u_i:\prod_{i\in I}S_i \rightarrow \mathbb{R}$ is the utility function that maps all players' strategies to a player's utilities.

    A \underline{finite} game is a normal-form game in which the set of players $N$ is a finite set, and the set of strategy profiles $S$ is finite.
\end{definition}

\begin{definition}[Mixed/Pure Strategy]
\normalfont
A mixed strategy  for player $i$ is a probability distribution $\sigma_i\in\Delta(S_i)$.\\
Elements of $S_i$ are called pure strategies.
\end{definition}

\begin{definition}[Dominant/Dominated Strategy]
    \normalfont
    A strategy $\sigma_i\in \Delta(S_i)$ is a \textbf{dominant strategy} for $i$ in $G$, if we have $u_i(\sigma_i,\sigma_{-i})> u_i(\sigma'_i,\sigma_{-i}), \forall \sigma'_i\neq \sigma_i, \sigma_{-i}\in\times_{j\neq i}\Delta(S_j)$.\\
    A strategy $\sigma_i\in \Delta(S_i)$ is a \textbf{dominated strategy} for $i$ in $G$, if $\exists \sigma'_i\neq \sigma_i$, $u_i(\sigma_i,\sigma_{-i})<u_i(\sigma'_i,\sigma_{-i}), \forall \sigma_{-i}\in\times_{j\neq i}\Delta(S_j)$.\\
    A strategy $\sigma_i\in \Delta(S_i)$ is a \textbf{weakly dominated strategy} for $i$ in $G$, if $\exists \sigma'_i\neq \sigma_i$, $u_i(\sigma_i,\sigma_{-i})\leq u_i(\sigma'_i,\sigma_{-i}), \forall \sigma_{-i}\in\times_{j\neq i}\Delta(S_j)$ and there is a $\sigma_{-i}\in\times_{j\neq i}\Delta(S_j)$, $u_i(\sigma_i,\sigma_{-i})< u_i(\sigma'_i,\sigma_{-i})$
\end{definition}

\begin{lemma}
    1. A dominant strategy is always pure.\\
    2. A strategy $\sigma'_i$ dominates $\sigma_i$ iff $u_i(\sigma'_i,s_{-i})> u_i(\sigma_i,s_{-i})$, for all pure strategy profiles $s_{-i}\in S_{-i}$.
\end{lemma}


\begin{definition}[Belief, Best Response]
    \normalfont
    A \textbf{belief} for player $i$ is a probability distribution $\mu\in\Delta(S_{-i})$.\\
    A strategy $\sigma_i \in \Delta(S_i)$ is the \textbf{best response} to beliefs $\mu$ if it solves the problem of $\max_{\sigma_i\in\Delta(S_i)}u_i(\sigma_i,s_{-i})$.\\
    Denote the set of all best responses to $\mu$ by $\beta_i(\mu)$.
\end{definition}
\begin{lemma}[Mixed Strategy is BR iff its Pure Strategies are Indifferent]
    A mixed strategy $\sigma_i$ is in $\beta_i(\mu)$ iff every pure strategy in the support of $\sigma_i$ is in $\beta_i(\mu)$. In particular, every strategy in the support of $\sigma_i$ yields the same payoff to $i$.
\end{lemma}

\begin{theorem}[Domination Theorem rephrased]
    In a finite game, a strategy is dominated iff there is no belief to which it is a best response.
\end{theorem}


\begin{definition}[Algorithm: Iterated Elimination of Dominated Strategies (IEDS)]
    \normalfont
    Let $\left(N,\left(S_i\right),\left(u_i\right)\right)$ be a finite game; $N=[n]$.
    \begin{enumerate}[$\bullet$]
        \item We define (inductively) $n$ sequences of sets of mixed strategies.
        \item Let $D_i^0=\Delta\left(S_i\right)$.
        \item Given $D_1^{k-1}, \ldots, D_n^{k-1}$, let
        $$
        D_i^k=\left\{\sigma_i: \nexists \bar{\sigma}_i: u_i\left(\sigma_i, \sigma_{-i}\right)<u_i\left(\bar{\sigma}_i, \sigma_{-i}\right) \forall \sigma_{-i} \in \times_{j \neq i} D_j^{k-1}\right\} .
        $$
        \item Note that $\left\{D_i^k\right\}$ is a decreasing sequence of sets.
        \item Let $D_i = \cap_{k=0}^\infty D_i^k$.
        \item The set $D = \times_{i=1}^n D_i$ be the set of strategies that survive the iterated elimination of dominated strategies.
    \end{enumerate}
    A game is called \textbf{dominance-solvable} if $D$ is a singleton.
\end{definition}


\begin{definition}[Rationalizable Strategies]
\normalfont
    \begin{enumerate}[$\bullet$]
        \item $R_i^0=\Delta\left(S_i\right)$.
        \item Given $R_1^{k-1}, \ldots, R_n^{k-1}$, Let
        $$
        \begin{aligned}
        & Z_i^k=\left\{s_i \in S_i: \sigma_i\left(s_i\right)>0 \text { for some } \sigma_i \in R_i^{k-1}\right\} \\
        & R_i^k=\left\{\sigma_i \in \Delta\left(S_i\right): \exists \mu \in \Delta\left(\times_{j \neq i} Z_j^k\right) \text { s.t. } \sigma_i \in \beta_i(\mu)\right\}
        \end{aligned}
        $$
    \end{enumerate}
    Note: $\left\{R_i^k\right\}_{k=0}^{\infty}$ is a decreasing sequence of sets.\\
    Let $R_i=\cap_{k=0}^{\infty} R_i^k$.\\
    The \textbf{rationalizable strategies} are the elements of $R=\times_{i=1}^n R_i$.
\end{definition}

\begin{lemma}
    In a finite game, $R$ is always non-empty and contains a pure strategy profile.
\end{lemma}

\begin{proposition}
    $\sigma_i \in \Delta(S_i)$ is \textbf{rationalizable} iff there are sets $Z_1,..., Z_n, Z_j \subseteq S_j$ such that
    \begin{enumerate}
        \item $\sigma_i \in \beta_i(\mu_i)$ for some $\mu_i \in \Delta(\times_{h\neq i}Z_h)$.
        \item for every $s_j \in Z_j$ there is $\mu_j \in \Delta(\times_{h\neq j}Z_h)$ such that $s_j\in \beta_j(\mu_j)$.
    \end{enumerate}
\end{proposition}

\begin{corollary}[Rationalizable = IEDS]
    Rationalizable strategies are exactly the strategies survive the iterated elimination of dominated strategies, $$R=D$$
\end{corollary}
















\subsection{Nash Equilibrium and Existence}
\begin{definition}[Nash Equilibrium]
    \normalfont
    A strategy profile $\Sigma=(\sigma_1,...,\sigma_I)$ is a \textbf{Nash} equilibrium of the game $G$ if for every $i\in I$, we have: $u_i(\sigma^*_i,\sigma^*_{-i})\geq u_i(\sigma'_i,\sigma^*_{-i}), \forall \sigma'_i\in \Delta(S_i)$ (no profitable deviation). In other words,
    \begin{enumerate}
        \item $\sigma_i$ is the \underline{best response} to beliefs $\mu_i\in \Delta (S_{-i})$
        \item $\mu_i=\sigma_{-i}$ (correct beliefs).
    \end{enumerate}
\end{definition}
\begin{enumerate}
    \item In rationalizable strategies, beliefs can be incorrect.
    \item In a Nash equilibrium, beliefs are correct. Any strategy in a Nash equilibrium is rationalizable.
\end{enumerate}

\begin{definition}[Best Response Correspondence]
    \normalfont
    In a Nash equilibrium the player $i$'s best response correspondence $\beta_i:\Delta(S_{-i})\rightarrow 2^{\Delta(S_i)}$ is defined as $\beta_i(\sigma_{-i})=\arg\max_{\sigma_i\in\Delta(S_i)}u_i(\sigma_i,\sigma_{-i})$. Let $\beta(\sigma)=\times_{i\in I}\beta_i(\sigma_{-i})$. Then $\sigma$ is a Nash equilibrium iff $\beta(\sigma)=\sigma$. $\beta$ is called the \textbf{best response correspondence} of the game.
\end{definition}

\begin{theorem}[Existence of Nash Equilibrium]
    A Nash equilibrium exists in a finite game $\Gamma$, if for all $i\in I$,
    \begin{enumerate}[(i).]
        \item $S_i$ is non-empty, convex, compact, subset of $\mathbb{R}^m$ (i.e., for some finite dimensions of real numbers).
        \item $u_i(s_i,...,s_I)$ is continuous in $(s_i,...,s_I)$ and quasi-concave in any $s_i$.
    \end{enumerate}
\end{theorem}
\begin{proof}
    We prove a lemma for the best response correspondence $\beta_i(s_{-i})=\argmax_{s_i\in S_i}u(s_i,s_{-i})$ firstly.
    \begin{lemma}
        Suppose $\{S_i\}_{i\in I}$ are non-empty. Suppose that $S_i$ is compact and convex and $u_i$ is continuous in $(s_i,...,s_I)$ and quasi-concave in any $s_i$, then best response correspondence $\beta_i(s_{-i})$ is non-empty, convex-valued and uhc.
    \end{lemma}
    \begin{proof}
        This lemma is proved by Berge's Maximum Theorem (Theroem \ref{thm:Berge's Maximum Theorem}).
    \end{proof}
    Consider the best response correspondence of the game $\beta$ with $\beta(s_i,...,s_I)=\{\beta_1(s_{-1}),...,\beta_I(s_{-I})\}$.

    As we proved $\beta$ is non-empty, convex-valued and uhc from $S$ to $S$ where $S$ is non-empty, compact, and convex. By the Kakutani's Fixed Point Theorem (Theorem \ref{thm:Kakutani's Fixed Point Theorem}), we have $\beta$ has a fixed point $s\in S$, which should be the Nash equilibrium.
\end{proof}

\subsection{Bayesian Game}
\begin{definition}[Bayesian Game]
    \normalfont
    A \textbf{Bayesian game} is defined by $$\Gamma=(I, \Omega, \{A_i\}_{i\in I}, \{u_i(\cdot)\}_{i\in I},\{\Theta_i\}_{i\in I}, \{F_i\}_{i\in I})$$
    where $\Omega$ is the state space, $u_i:A\times \Omega$ is $i$'s payoff function, and $F_i\in\Delta\left(\Omega\times\Theta_i\right)$ is the (prior) distribution of the player $i$'s type.
\end{definition}

\begin{definition}[Normal-form Bayesian game]
    \normalfont
    Assume a finite game. The \textbf{normal-form game} can be represented by $$\left(I,(S_i,U_i)_{i\in I}\right)$$ defined by letting $S_i$ be the set of strategies based on types $s_i:\Theta_i \rightarrow A_i$ and
    \begin{equation}
        \begin{aligned}
            U_i(s)=\sum_{\omega\in\Omega}\sum_{(\theta_i)_{i\in I}\in \Theta}p(\omega,\theta_1,...,\theta_I)u_i(s_1(\theta_1),..., s_I(\theta_I),\omega)
        \end{aligned}
        \nonumber
    \end{equation}
    for all $s\in S$.

    A \textbf{Bayesian Nash equilibrium} (BNE) of a Bayesian game is a strategy profile $(s_1,...,s_n)$ that is a Nash equilibrium of the derived normal-form game.
\end{definition}

\begin{definition}[Best Response, Interim Payoff]
    \normalfont
    $s_i$ is a BR to $s_{-i}$ iff for all $\theta_i$, $s_i(\theta_i)$ maximizes the \textbf{interim payoff} of player $i$. The interim payoff is defined by the expected payoff given the type $\theta_i$ of player $i$ by playing action $a_i$.
    \begin{equation}
        \begin{aligned}
            \mathbb{E}_{\omega\in\Omega,\tilde{\theta}_{-i}\in\Theta_{-i}}[u_i(a_i,s_{-i}(\tilde{\theta}_{-i}),\omega)|\theta_i]
        \end{aligned}
        \nonumber
    \end{equation}
\end{definition}


\section{Adverse Selection}
Consider a labor market that has many identical firms. In competitive equilibrium, firms' profits are $0$. Firms are price (wage) takers, risk-neutral, and CRS. There are continuum of workers with productivity levels $\theta\in\left[\underline{\theta},\overline{\theta}\right]$ (Assume workers work if it is indifferent for them between employment and non-employment).
\begin{enumerate}
    \item $\theta\sim F$, $F(\cdot)$ is a c.d.f. over $\left[\underline{\theta},\overline{\theta}\right]$.
    \item $N$ is the total mass of workers.
    \item Type $\theta$ worker has a reservation utility $r(\theta)$.
\end{enumerate}

\begin{enumerate}[$\circ$]
    \item Suppose the competitive equilibrium wages are $\theta=w^*(\theta)$.
    \item An allocation is denoted by $I:\left[\underline{\theta},\overline{\theta}\right] \rightarrow \{0,1\}$, where $I(\theta)=0$ denotes $\theta$ is unemployed and $I(\theta)=1$ denotes $\theta$ is employed.
    \item Aggregate welfare = sum of utilities of all participants
    \begin{equation}
        \begin{aligned}
            =N\int_{\underline{\theta}}^{\overline{\theta}} \left[I(\theta)\times\theta+[1-I(\theta)]r(\theta)\right]dF(\theta)
        \end{aligned}
        \nonumber
    \end{equation}
    Then we have the optimal allocation satisfies
    \begin{equation}
        \begin{aligned}
            I^*(\theta)=\left\{\begin{matrix}
                1,&\theta>r(\theta)\\
                0,1&\theta=r(\theta)\\
                0,&\theta<r(\theta)
            \end{matrix}\right.
        \end{aligned}
        \nonumber
    \end{equation}
\end{enumerate}
In the asymmetric information case,
\begin{definition}
\normalfont
$w$ is CE wage if $w=\mathbb{E}[\theta|r(\theta)\leq w]$.
\end{definition}

\subsection{Adverse Selection}
\begin{assumption}
    \begin{enumerate}[({A}1).]
        \item $r$ is strictly increasing in $\theta$.
        \item $F(\cdot)$ has a strictly positive density, $F(\theta)>0, \forall \theta\in \left[\underline{\theta},\overline{\theta}\right]$.
        \item $r(\theta)\leq\theta$ (outside option is worse than productivity, i.e., full employment is optimal).
    \end{enumerate}
\end{assumption}

\begin{lemma}
    Under A1-A3, $\Phi(w):=\mathbb{E}[\theta|r(\theta)\leq w]$ is well-defined, continuous, and non-decreasing.
\end{lemma}

Hence, there exists underemployment, which makes $1^{st}$ welfare theorem fails. There may exist multiple CEs, where the one with the highest wage Pareto dominates others.

\begin{example}
    Suppose $\theta\in[0,2]$, $F(\theta)=\frac{\theta}{2}$, $f(\theta)=\frac{1}{2}$, $r(\theta)=\alpha\theta,\alpha\in(0,1)$.
    \begin{equation}
        \begin{aligned}
            \mathbb{E}[\theta|r(\theta)\leq w]=\mathbb{E}\left[\theta|\theta \leq \frac{w}{\alpha}\right]=\left\{\begin{matrix}
                1,&w\geq 2\alpha\\
                \frac{1}{F\left(\frac{w}{\alpha}\right)}\int_0^{\frac{w}{\alpha}}\theta f(\theta)d\theta=\frac{w}{2\alpha},&w\leq 2\alpha
            \end{matrix}\right.
        \end{aligned}
        \nonumber
    \end{equation}
    CEs are given by $\mathbb{E}[\theta|r(\theta)\leq w]=w$. $w^*=0$ is always CE and $w^*=1$ is CE if $\alpha\leq\frac{1}{2}$.
\end{example}

\subsection{Game Theoretical Approach}
\begin{enumerate}
    \item Suppose there are two firms setting wages simultaneously.
    \item Workers observe the wages in stage 1 and make an employment decision.
\end{enumerate}
Let $W^*$ be the set of CE wages and $w^*:=\max W^*$.
\begin{lemma}\label{lemma:ad_l2}
    $\forall w'\in\left(w^*,\overline{\theta}\right]$: $\mathbb{E}[\theta|r(\theta)\leq w']<w'$.
\end{lemma}
\begin{proof}
    Suppose by the contradiction that $\exists w'\in \left(w^*,\overline{\theta}\right]$ s.t. $\mathbb{E}[\theta|r(\theta)\leq w']\geq w'$. Since $\mathbb{E}[\theta|r(\theta)\leq \overline{\theta}]<\overline{\theta}$, there must exist a $w''\in [w',\overline{\theta})$ s.t. $\mathbb{E}[\theta|r(\theta)\leq w'']=w''$ by intermediate value theorem, which contradicts to the definition of $w^*$.
\end{proof}

\begin{proposition}
    \begin{enumerate}[(i).]
        \item If $w^*>r(\underline{\theta})$ and $\exists \epsilon>0$ s.t. $\mathbb{E}[\theta|r(\theta)\leq w']>w',\forall w'\in \left(w^*-\epsilon,w^*\right)$. Then, there is a unique SPE where both firms set wage $=w^*$.
        \item If $w^*=r(\underline{\theta})$ (complete market shutdown at $w^*$), there are multiple SPE that all give the same outcome as complete market shutdown where both firms set wage $=w^*$.
    \end{enumerate}
\end{proposition}
\begin{proof}
    \begin{lemma}\label{lemma:p1}
        In all SPE, firms make zero profits.
    \end{lemma}
    \begin{proof}
        Suppose not, i.e., at least one firm makes strictly positive profits. Then, the total profits of firms $1\&2$, $$\Pi=M(\bar{w})\left[\mathbb{E}[\theta|r(\theta)\leq\bar{w}]-\bar{w}\right]>0$$
        where $\bar{w}$ is the max wage set by the two firms and $M(\bar{w})$ is the mass of workers willing to work at $\bar{w}$. At least one firm, $i$, makes profit $\leq\frac{\Pi}{2}$. Then, $i$'s profits from setting $\bar{w}+\delta$, with $\delta \rightarrow 0^+$, is higher:
        \begin{equation}
            \begin{aligned}
                &M(\bar{w}+\delta)\left[\mathbb{E}[\theta|r(\theta)\leq\bar{w}+\delta]-\bar{w}-+\delta\right]\\
                \geq &M(\bar{w})\left[\mathbb{E}[\theta|r(\theta)\leq\bar{w}+\delta]-\bar{w}-+\delta\right] \rightarrow \Pi \textnormal{ as }\delta \rightarrow 0
            \end{aligned}
            \nonumber
        \end{equation}
        Hence, the $i$ has incentive to deviate.
    \end{proof}
    \begin{lemma}\label{lemma:p2}
        In all SPE, firm $i$ sets $w_i\leq w^*, i\in\{1,2\}$.
    \end{lemma}
    \begin{proof}
        Directly given by Lemma \ref{lemma:ad_l2} and Lemma \ref{lemma:p1}.
    \end{proof}
    \begin{enumerate}[(i):]
        \item In SPE, no firm $i$ sets $w_i<w^*$: suppose $w_i<w^*$ and let $j\neq i$, take any $w'_j$ s.t. $w'_j\in\left(w_i,w^*\right)$ and $w'_j>w^*-\epsilon$. Then, $j$ gets profit: $M(w'_j)\left[\mathbb{E}[\theta|r(\theta)\leq w'_j]-w'_j\right]>0$ (by Case (i)'s conditions).
        \item By Lemma \ref{lemma:p2}, both firms set $w_i\leq w^*=r(\underline{\theta})$. Check that $\{(w_1,w_2):w_1,w_2\leq w^*\}$ is SPE wage profiles.
    \end{enumerate}
\end{proof}

\subsection{Planner Intervention}
Planner can't observe the true type $\theta$.

The planner's tools:
\begin{enumerate}
    \item Take over the firms.
    \item $w_e$, employment wage.
    \item $w_u$, unemployment wage.
\end{enumerate}
s.t. budget balanced.

\begin{definition}[Constrained Efficient]
    \normalfont
    A CE $w$ is \textbf{constrained efficient} if it cannot be Pareto improved upon by an intervention by the planner.
\end{definition}

\begin{proposition}[$w^*:=\max W^*$ is constrained efficient]
    Let $W^*$ be the set of CE wages. $w^*:=\max W^*$ is constrained efficient.
\end{proposition}
\begin{proof}
    Note that both firms are making zero profits by the Lemma \ref{lemma:p1}. Any CE wage $w\neq w^*$ can be Pareto improved by $\{w_e=w^*,w_u=0\}$. Then, we prove $w^*$ can't be Pareto improved.
    \begin{enumerate}
        \item Case 1: if $w^*$ gives full-employment in CE, then $w^*$ is Pareto efficient.
        \item Case 1: suppose $w^*$ doesn't give full-employment in CE.
        
        Consider taking an intervention $w_e\&w_u$. Then, $\{\theta:r(\theta)+w_u\leq w_e\}=[\underline{\theta},\hat{\theta}]$ for some $\hat{\theta}\in[\underline{\theta},\overline{\theta}]$ such that
        \begin{equation}
            \begin{aligned}
                r(\hat{\theta})+w_u=w_e
            \end{aligned}
            \label{con:1}
        \end{equation}
        The budget balanced gives
        \begin{equation}
            \begin{aligned}
                w_e F(\hat{\theta})+w_u (1-F(\hat{\theta}))=\int_{\underline{\theta}}^{\hat{\theta}}\theta d F(\theta)
            \end{aligned}
            \label{con:2}
        \end{equation}
        Plug \eqref{con:1} into \eqref{con:2}:
        \begin{equation}
            \begin{aligned}
                \left\{\begin{matrix}
                    &w_u(\hat{\theta})=\int_{\underline{\theta}}^{\hat{\theta}}\theta d F(\theta)-r(\hat{\theta})F(\hat{\theta})=F(\hat{\theta})\left(\mathbb{E}[\theta|\theta\leq\hat{\theta}]-r(\hat{\theta})\right)\\
                    &w_e(\hat{\theta})=\int_{\underline{\theta}}^{\hat{\theta}}\theta d F(\theta)+r(\hat{\theta})(1-F(\hat{\theta}))
                \end{matrix}\right.
            \end{aligned}
            \nonumber
        \end{equation}
        Let $\theta^*$ be s.t. $r(\theta^*)=w^*$. Because $w^*$ is a CE price, $\mathbb{E}[\theta|\theta\leq\theta^*]=r(\theta^*)=w^*$. So, CE with $w^*$ can be implemented by $w_u(\theta^*)=0$ and $w_e(\theta^*)=w^*$.
        \begin{enumerate}
            \item If $\hat{\theta}<\theta^*$. $\underline{\theta}$ is worse off under the intervention since $w_e(\hat{\theta})<w^*$.
            \item If $\hat{\theta}>\theta^*$. $\overline{\theta}$ is worse off under the intervention since $w_u(\hat{\theta})=F(\hat{\theta})\left(\mathbb{E}[\theta|\theta\leq\hat{\theta}]-r(\hat{\theta})\right)<0$ by the Lemma \ref{lemma:ad_l2}
        \end{enumerate}
    \end{enumerate}
\end{proof}


\subsection{Signaling}\label{sec:signaling}
Suppose the worker $\theta\in[\underline{\theta},\overline{\theta}]$ can properly and costlessly reveal his type to the firms. Then,
\begin{lemma}
    All workers revel their types.
\end{lemma}
\paragraph*{Spence's Job Market Signaling Model} One worker has productivity $\theta\in\{\theta_L,\theta_H\}$ with $P(\theta_H)=\lambda$. The worker signal through his education with cost $e>0$. The education doesn't change his productivity. The payoff of the worker is the wage minus the cost:
\begin{equation}
    \begin{aligned}
        u(w,e|\theta)=w-c(e,\theta)
    \end{aligned}
    \nonumber
\end{equation}
where $c(0,\theta)=0,c_e(e,\theta):=\frac{\partial c(e,\theta)}{\partial e}>0, c_\theta(e,\theta):=\frac{\partial c(e,\theta)}{\partial \theta}<0$, and $c_{e\theta}(e,\theta):=\frac{\partial^2 c(e,\theta)}{\partial e\partial \theta}<0$ (Single-Crossing Property, the difference $c(e,\theta_L)-c(e,\theta_H)$ is increasing in $e$ (i.e., $c_e(e,\theta_L)-c_e(e,\theta_H)>0$), which means if $c(e,\theta_L)$ and $c(e,\theta_H)$ intersect as functions of $e$, they only intersect at one time.)
\begin{enumerate}[]
    \item \underline{Stage 0}: Nature chooses the $\theta\in\{\theta_L,\theta_H\}$ with $P(\theta_H)=\lambda$.
    \item \underline{Stage 1}: The worker learns $\theta$ and chooses $e(\theta)\geq 0$.
    \item \underline{Stage 2}: Firms observe $e(\theta)$. Then, they simultaneously make wage offers $w_1$ and $w_2$.
    \item \underline{Stage 3}: The worker observes $w_1,w_2$ and makes employment decision.
\end{enumerate}
Let $r(\theta_L)$ and $r(\theta_H)=0$. Let $\mu(e)\in[0,1]$ be the probability that in the beginning of stage 2, firms think that the worker is $\theta_H$ type with probability $\mu(e)$ when observing $e$. The corresponding expected productivity (the highest wage) that the firm can pay is
\begin{equation}
    \begin{aligned}
        w(e)=\mu(e)\theta_H+(1-\mu(e))\theta_L
    \end{aligned}
    \nonumber
\end{equation}
In stage 2, both firm will set $w(e)$ (complete competition).

\begin{definition}[Perfect Bayesian Equilibrium]
    \normalfont
    A PBE is a strategy profile ($e^*(\theta_L)$, $e^*(\theta_H)$, $w^*_1: \mathbb{R}_+ \rightarrow \mathbb{R}$, $w^*_2: \mathbb{R}_+ \rightarrow \mathbb{R}$), and a belief $\mu^*: \mathbb{R} \rightarrow [0,1]$ such that
    \begin{enumerate}
        \item $\forall \theta\in\{\theta_L,\theta_H\}$, the worker strategy optimal given firm strategies.
        \item The belief $\mu^*(e)$ is derived from $\lambda, e^*(\theta_L), e^*(\theta_H)$ via Bayes' rule whenever possibly (on the equilibrium path). Outside the equilibrium path the belief $\mu^*(e)$ is arbitrarily.
        \item Firms offer wages that form a NE of the stage 2 game, where their belief $\mu^*(e)$ about their workers' type. (sequential rationality).
    \end{enumerate}
\end{definition}
We simplify the game by backward induction:
\begin{enumerate}
    \item \underline{Stage 3}: The worker chooses the highest wage off if it is $\geq 0$.
    \item \underline{Stage 2}: After observing $e(\theta)$, firms chooses the wage as the expected productivity in NE,
    \begin{equation}
        \begin{aligned}
            w^*(e)=\mu^*(e)\theta_H+(1-\mu^*(e))\theta_L
        \end{aligned}
        \nonumber
    \end{equation}
    because it is a Bertrend competition.
\end{enumerate}
\paragraph*{Separating Equilibrium}
In separating equilibrium, $e^*(\theta_L)\neq e^*(\theta_H)$.
\begin{lemma}
    In any separating PBE, $w^*(e^*(\theta))=\theta, \forall \theta\in\{\theta_L,\theta_H\}$.
\end{lemma}
\begin{proof}
    By Bayes' rule, after firm observe $e^*(\theta_L)$, $\mu^*(e^*(\theta_L))=0$. Then, $w^*(e^*(\theta_L))=\theta_L$. ($e^*(\theta_H)$ is similar.)
\end{proof}

\begin{lemma}
    In separating PBE, low type always chooses zero education, $\theta^*(\theta_L)=0$.
\end{lemma}
\begin{proof}
    If not, the low type worker always has profitable deviation, $\theta^*(\theta_L)=0$.
\end{proof}

\begin{lemma}
    Define $\underline{e}$ and $\overline{e}$ such that
    \begin{enumerate}
        \item $\theta_L=\theta_H-c(\underline{e},\theta_L)$ (the lowest effort can prevent the low type from mimicking high type) and
        \item $\theta_L=\theta_H-c(\overline{e},\theta_H)$ (the highest effort can prevent the high type from pooling with low type).
    \end{enumerate}
    Then, in all separating PBEs, $e\in \left[\underline{e},\overline{e}\right]$.\\
    Conversely, $\forall \hat{e}\in \left[\underline{e},\overline{e}\right]$, there is a separating PBE where $e^*(\theta_H)=\hat{e}$.
\end{lemma}
These different PBEs are Pareto ranked. High type prefers the PBE with a lower $e$ (the best is the one with $e^*(\theta_H)=\underline{e}$.)

\paragraph*{Pooling PBE}
$e^*(\theta)=e^*,\theta\in\{\theta_L,\theta_H\}$, $\mu^*(e^*)=\lambda$, and $w^*(e^*)=\mathbb{E}[\theta]$.

\begin{lemma}
    Define $e'$ by $\theta_L=\mathbb{E}[\theta]-c(e',\theta_L)$ (the highest effort can prevent the low type from choosing $e=0$ and get $w=\theta_L$.)\\
    Then, for all pooling PBE, $e^*(\theta_L)=e^*(\theta_H)=e^*\in[0,e']$. Conversely, for all $\hat{e}\in [0,e']$, there is a pooling PBE with $e^*=\hat{e}$.
\end{lemma}


\subsection{Cho-Kreps Intuitive Criterion}
\begin{definition}[Equilibrium Dominated Message]
    \normalfont
    A message is \textbf{equilibrium dominated} for a type if the type must do strictly worse by sending the message than it does in equilibrium (i.e., payoff in eq. is strictly better than maximum payoff from deviating).
\end{definition}

\begin{definition}[Cho-Kreps Intuitive Criterion]
    \normalfont
    If an information set is off the eq. path and a message is eq. dominated for a type, then beliefs should assign zero probability to the message coming from that type (if possible).
\end{definition}

Fix a PBE $e^*(\theta), \theta\in\{\theta_L,\theta_H\}, \mu^*(\cdot)$ (We know $w_1^*(e)=w_2^*(e)=\mu(e)\theta_H+(1-\mu(e))\theta_L$). Let $u^*(\theta),\theta\in\{\theta_L,\theta_H\}$ be the PBE utility of the type $\theta$ worker.

The criterion requires the (off-path) belief $\mu^*(e):=P(\tilde{\theta}=\theta_H|e)=1-P(\tilde{\theta}=\theta_L|e)$ satisfies $$P(\tilde{\theta}=\theta|e)=0,\forall e,\theta$$ such that
\begin{enumerate}
    \item $u^*(\theta)>\max_{w\in[\underline{\theta},\overline{\theta}]}[w-c(e,\theta)]$
    \item $\exists \theta'$ s.t. $u^*(\theta')\leq \max_{w\in[\underline{\theta},\overline{\theta}]}[w-c(e,\theta')]$ (make sure the sum of beliefs given $e$ is nonzero.)
\end{enumerate}
In this application, the only PBE that survives Intuitive Criterion is the best separating PBE, $e^*(\theta_H)=\underline{e}$ (the lowest effort).


\subsection{Screening Model}
Workers can undertake a contractible/observable task level $t\geq 0$. The utility of a worker is defined by $u(w,t,\theta):=w-c(t,\theta)$, where $c(\cdot,\cdot)$ satisfies the same assumption as in signaling model \ref{sec:signaling}.

The Game follows
\begin{enumerate}[]
    \item \underline{Stage 1}: Two firms simultaneously determine sets of contracts, $(w,t)$.
    \item \underline{Stage 2}: The worker observes all offer contracts and makes employment decision.
    (If indifference, choose lower task contract, favor employment over unemployment. If contracts of firms are indifferent, choose each with probability 1/2.)
\end{enumerate}

The null contract is $(w,t)=(0,0)$. Assume WLOG at stage 1, each firm appears a non-empty set of contracts.

\subsubsection*{Perfect Information}
\begin{proposition}[Perfect Information]
    If firms can observe the worker types, then in SPE firms make zero profit and type $\theta_i$ worker signs $(w^*_i,t^*_i)=(\theta_i,0)$.
\end{proposition}
\begin{proof}
    \begin{claim}
        Firms make zero profits from this contract.
    \end{claim}
    \begin{proof}
        Suppose not,
    \begin{enumerate}[$\circ$]
        \item $w^*_i>\theta_i$ $\Rightarrow$ negative profits, firms benefit from offering null contract.
        \item $w^*_i<\theta_i$ $\Rightarrow$ Let $\Pi$ be the total profits of the firms. Then one of the firms makes profit $\leq \frac{\Pi}{2}$. Then, this firm can benefit from offering $(w^*_i+\Delta,t^*_i)$, where $\Delta \rightarrow 0^+$.
    \end{enumerate}
    \end{proof}
    Then, we prove the firms must choose $(w^*_i,t^*_i)=(\theta_i,0)$. Suppose by the way of contradiction that $t_i^*>0$. Then, one firm can profitably deviate by offering $(w^*_i,0)$.
\end{proof}

\subsubsection*{Asymmetric Information}
\begin{lemma}\label{lemma:zero_profit}
    In any SPE, firms obtain zero profits,
\end{lemma}
\begin{proof}
    Firms must make profits $\geq 0$. Suppose by the way of contradiction that the total profit $\Pi>0$. Let $(w_L,t_L)$ be the contract signed by $\theta_L$ and $(w_H,t_H)$ be the contract signed by $\theta_H$. One firm can profitably deviate by offering $(w_L+\Delta,t_L)$ and $(w_H+\Delta,t_H)$, where $\Delta \in (0,\Pi)$.
\end{proof}

\begin{lemma}
    There is \textbf{no} pooling SPE.
\end{lemma}
\begin{proof}
    Suppose for a contradiction, $\exists$ an SPE where both worker types sign $(w_p=\mathbb{E}[\theta],t_p)$. Suppose one firm offers $(w_p,t_p)$, then another firm can only employ high type workers by offering $(\tilde{w},\tilde{t})$, where $\tilde{w}-c(\tilde{t},\theta_H)>\mathbb{E}[\theta]-c(t_p,\theta_H)$, $\tilde{w}-c(\tilde{t},\theta_L)<\mathbb{E}[\theta]-c(t_p,\theta_L)$, and $\tilde{w}<\theta_H$. (The existence is given by $\frac{\partial^2 c(t,\theta)}{\partial t\partial \theta}<0$.)
\end{proof}

\begin{lemma}
    Let $(w_L,t_L)$ be the contract signed by $\theta_L$ and $(w_H,t_H)$ be the contract signed by $\theta_H$ in separating SPE. Then, $w_L=\theta_L$ and $w_H=\theta_H$.
\end{lemma}
\begin{proof}
    Suppose $w_i>\theta_i,i\in\{L,H\}$, firms benefit from not offering this contract. So, $w_L\leq \theta_L$ and $w_H\leq \theta_H$.
    \begin{enumerate}
        \item \underline{$w_L=\theta_L$:} Suppose $w_L<\theta_L$. Either firm can profitably deviate by setting $(w'_L,t_L)$ such that $w_L<w'_L<\theta_L$. This offer can win all low-type workers and get a positive profit from hiring them. If $w'_L-c(t_L,\theta_H)\geq w_H-c(t_H,\theta_H)$, the offer can also hire high-type workers, which also give positive profit for the firm. Hence, there is a contradiction.
        \item \underline{$w_H=\theta_H$:} Suppose $w_H<\theta_H$, firms get positive profits, which contradicts to the Lemma \ref{lemma:zero_profit}.
    \end{enumerate}
\end{proof}

\begin{lemma}
    $\theta_L$ signs the contract $(\theta_L,0)$ in SPE.
\end{lemma}
\begin{proof}
    Suppose $t_L>0$. One firm can profitably deviate by offering $(\theta_L-\Delta,0)$.
\end{proof}

\begin{proposition}
    In any (pure strategy) SPE, $\theta_L$ signs $(w_L,t_L)=(\theta_L,0)$ and $\theta_H$ signs $(w_H,t_H)=(\theta_H,t_H)$, where $t_H$ solves
    \begin{equation}
        \begin{aligned}
            \theta_H-c(t_H,\theta_L)=\theta_L
        \end{aligned}
        \nonumber
    \end{equation}
\end{proposition}

If $\lambda:=P(\theta_H)$ is high, the pure SPE may not exist (exist $(\tilde{w},\tilde{t})$ can attract both types and make positive profit).

Cross subsidizing deviation by a firm (prices one product above its market value to fund another product), $(\tilde{w},\tilde{t})$ (signed by low type) and $(\tilde{\tilde{w}},\tilde{\tilde{t}})$ (signed by high type), is a profitable deviation if $\lambda$ is large enough.


\chapter{Mechanism Design}

\section{Mechanism Design}
Design incentives for agents to reveal their types or achieve particular society outcomes.

Given
\begin{enumerate}
    \item the set of agents $I$ with utility function $u_i(x;\theta_i),i\in I$,
    \item alternatives (outcomes for the society) $X$,
    \item  types (of agents) $\Theta=(\Theta_1,...,\Theta_I)$ with prior probability $\phi$ over $\Theta$,
    \item and a \textbf{social choice function} (SCF) $f:\Theta\rightarrow X$.
\end{enumerate}

\begin{definition}[Mechanism $\Gamma=(S,g)$]
    \normalfont
    A \textbf{mechanism} is represented as $$\Gamma=\left(S, g\right)$$
    where $S\triangleq(S_1,...,S_I)(S_1,...,S_I)$ represents the set of strategies, $S_i$ represents the strategy set of agent $i$, and $g:S\triangleq(S_1,...,S_I) \rightarrow X$ is the outcome function that determines the social outcome.
\end{definition}

A \textbf{Bayesian game induced by} $\Gamma$ is $(I,S,\Theta,\phi,\tilde{u})$, where the payoffs functions are
\begin{equation}
    \begin{aligned}
        \tilde{u}_i(s;\theta_i)=u_i(g(s);\theta_i)
    \end{aligned}
    \nonumber
\end{equation}
for all $i\in I, s\in S$, and $\theta_i\in\Theta_i$.


\subsection{Implement in Dominant Strategies}
\begin{definition}[$\Gamma$ Implements $f$]
    \normalfont
    A mechanism $\Gamma$ (indirectly) \textbf{implements} a social choice function (SCF) $f$ if there exists an ``equilibrium'' $s^*(\cdot)=\left(s_1^*(\cdot),...,s_I^*(\cdot)\right)$ of the Bayesian game induced by $\Gamma$ such that $$g(s_1^*(\theta_1),...,s_I^*(\theta_I))=f(\theta_1,...,\theta_I)$$ for all $(\theta_1,...,\theta_I)\in \Theta$. Here the ``equilibrium'' is a dominant strategy equilibrium or BNE.
\end{definition}
That is, the equilibrium in a game induced by $\Gamma$ gives the same outcome as the outcome of $f$ given by revealing agents' true types.

\begin{definition}[Direct Mechanism]
    \normalfont
    A mechanism is \textbf{direct mechanism} if agents directly report their types (types are observable). $S_i=\Theta_i$ for all $i\in I$ and $g(\theta)=f(\theta)$ for all $\theta=(\theta_1,...,\theta_I)\in\Theta$. So, a direct mechanism can be represented by $\Gamma=(\Theta,f(\cdot))$.
\end{definition}
In indirect mechanism agents don't report their types directly. Types can be observed only indirectly through signals or behavior.

A strategy is weakly dominant if for all $\theta_i\in\Theta_i$ and all $s_{-i}(\cdot)\in S_{-i}$, we have $u_i(g(s_i(\theta_i),s_{-i}),\theta_i)\geq u_i(g(s'_i,s_{-i}),\theta_i)$ for all $s'_i\neq s_i$.


\begin{definition}[Dominant Strategy Equilibrium]
    \normalfont
    Strategy profile $s^*=(s_1^*(\cdot),...,s_I^*(\cdot))$ is a \textbf{dominant strategy (D-S) equilibrium} of $\Gamma=(S,g(\cdot))$ if for all $i\in I$ and $\theta_i\in \Theta$, we have, for all $s'_i\in S_i$ and $s_{-i}\in S_{-i}$:
    \begin{equation}
        \begin{aligned}
            u_i(g(s_i^*(\theta_i),s_{-i}),\theta_i)&\geq u_i(g(s'_i,s_{-i}),\theta_i)
        \end{aligned}
        \nonumber
    \end{equation}
    equivalently, in the Bayesian game induced by $\Gamma$,
    \begin{equation}
        \begin{aligned}
            \tilde{u}_i(s_i^*(\theta_i),s_{-i},\theta_i)\geq \tilde{u}_i(s'_i,s_{-i},\theta_i)
        \end{aligned}
        \nonumber
    \end{equation}
\end{definition}

\begin{definition}[Implement in dominant strategies]
    \normalfont
    $\Gamma$ \textbf{implements} $f$ in \textbf{dominant strategies} if $\exists$ a dominant strategy (D-S) equilibrium $s^*$ of $\Gamma$ such that $g(s^*(\theta))=f(\theta)$.
\end{definition}


\subsection{Dominant-Strategy-Incentive-Compatible (DSIC)/Strategy-Proof}
\begin{definition}[Strategy-Proof, DSIC]
    \normalfont
    $f$ is \textbf{strategy-proof} (also called dominant-strategy-incentive-compatible, \textbf{DSIC}) if $$s^*_i(\theta_i)=\theta_i,\quad \forall \theta_i\in\Theta_i,i\in I$$ is a dominant strategy (D-S) equilibrium of the direct mechanism $\Gamma=(\Theta,f(\cdot))$.
\end{definition}

\begin{theorem}[Revelation Principle]
    If $\exists$ a mechanism $\Gamma=(S,g(\cdot))$ that implements $f$ in dominant strategies. Then $f$ is strategy-proof (DSIC).
\end{theorem}
\begin{proof}
    Given $\Gamma$ implements $f$ in dominant strategies, there is a  D-S equilibrium $s^*=\left(s_1^*(\cdot),...,s_I^*(\cdot)\right)$ such that $g(s^*(\theta))=f(\theta)$.\\
    By the definition of D-S equilibrium,
    \begin{equation}
        \begin{aligned}
            u_i(g(s_i^*(\theta_i),s_{-i}),\theta_i)&\geq u_i(g(s'_i,s_{-i}),\theta_i)
        \end{aligned}
        \nonumber
    \end{equation}
    By substituting $g(s^*(\theta))=f(\theta)$, we have
    \begin{equation}
        \begin{aligned}
            u_i(f(\theta_i,\theta_{-i}),\theta_i)&\geq u_i(f(\theta'_i,\theta_{-i}),\theta_i),\quad \forall \theta'_i\in\Theta_i
        \end{aligned}
        \nonumber
    \end{equation}
    which gives that $f$ is DSIC.
\end{proof}

\begin{center}\begin{figure}[htbp]
    \centering
    \begin{tikzpicture}[domain=0:3.25]
        \node at (0,0) {Types: $\Theta$};
        \draw[->](1,0)--(7,0) node[right] {Alternatives: $X$};
        \draw[dashed,->](0.5,-0.5)--(3.5,-3);
        \node at (4,-3) {$s^*(\theta)$};
        \draw[dashed,->](4.5,-3)--(7.5,-0.5);
        \draw[->](0,-0.5)--(3.5,-3.5);
        \node at (4,-3.5) {$s^*=\theta$};
        \draw[->](4.6,-3.5)--(8.1,-0.5);
        \node at (0.5,-1.8) {Desirable};
        \node at (0.5,-2.2) {Situation};
        \node at (2.5,-1.3) {Mechanism};
        \node at (2.5,-1.7) {$\Gamma$};
        \node at (6.5,-2.5) {$f(\theta)$};
        \node at (5.5,-1.5) {$g(s^*(\theta))$};
    \end{tikzpicture}
    \caption{How Mechanism Design works}
    \label{}
\end{figure}\end{center}

\subsection{Bayesian-Incentive-Compatible (BIC)}
\begin{definition}[BIC]
    \normalfont
    $f$ is Bayesian-incentive-compatible (B.I.C.) if $$s^*_i(\theta_i)=\theta_i,\quad \forall \theta_i\in\Theta_i,i\in I$$ is a BNE of the Bayesian game induced by the direct mechanism $\Gamma=(\Theta,f(\cdot))$.
\end{definition}

\begin{theorem}[Revelation Principle (BIC)]
    If $\exists$ a mechanism $\Gamma=(S,g(\cdot))$ that implements $f$ in BNE. Then $f$ is BIC.
\end{theorem}

BIC is a weaker condition than DSIC.


\subsection{Negative Results: dictatorial SCF $f$}
\begin{theorem}[Gibbard-Satterthwaite Theorem]
    Suppose that $|X|\geq 3$ and a social choice function $f$ is surjective (i.e., $\forall x\in X$ $\exists (\theta_1,...,\theta_I)\in\Theta$ s.t. $f(\theta_1,...,\theta_I)=x$). Then, $f$ is strategy-proof (DSIC) $\Leftrightarrow$ $f$ is dictatorial (\ref{SWF_properties}, i.e., $\exists i^*\in\{1,...,I\}$ such that $f(\theta)\in \argmax_{x\in X}u_{i^*}(x;\theta_{i^*})$ for all $\theta\in \Theta$).
\end{theorem}
Note: By the revelation principle, under the conditions of the Theorem, there is no mechanism that implements a non-dictatorial SCF $f$ in dominant strategies.

\begin{comment}
Some lemmas can help to prove the theorem.
\begin{lemma}
    If $f$ is strategy-proof (DSIC) and $f(\succeq)=x$ and $x\succeq_i y \Rightarrow x\succeq'_i y$ for all $i\in I$ and all $x\neq y\in X$, then $f(\succeq')=x$.
\end{lemma}

\begin{lemma}[Pareto Effeciency]
    If $f$ is strategy-proof (DSIC) and $x\succ_i y$ for all $i\in I$, then $f(\succeq')\neq y$.
\end{lemma}

\begin{example}
    Define $\succeq=\begin{pmatrix}
        x&y\\
        y&x\\
        z&z
    \end{pmatrix}$ and $\succeq'=\begin{pmatrix}
        x&y\\
        y&z\\
        z&x
    \end{pmatrix}$, each column 1/2 represents player 1/2's preferences.
\end{example}
\end{comment}


\section{Quasi-linear Model}
Consider $x=(k,\underbrace{t_1,...,t_I}_{t})\in X=K\times \mathbb{R}^I$, in our example, $K$ represents a set of choices for projects and $\mathbb{R}^I$ represents the set of transfers for all agents.

Each agent has a quasi-linear function that represents her utility:
\begin{equation}
    \begin{aligned}
        u_i(k,t,\theta_i)=v(k,\theta_i)+t_i
    \end{aligned}
    \nonumber
\end{equation}
where $v: K\times\Theta_i \rightarrow \mathbb{R}$ represents the utility without transfers.

Let $p(\cdot)=\left(k(\cdot),t(\cdot)\right)$ represents the ``project-choice rule'' $k: \Theta \rightarrow K$ and the ``transfer rule'' $t: \Theta \rightarrow \mathbb{R}^I$.
\begin{definition}[ex-post efficient]
    \normalfont
    $k(\cdot):\Theta \rightarrow K$ is \textbf{ex-post efficient} if $\nexists \left(\theta\in\Theta, k'\in K, t=(t_1,...,t_I)\in \mathbb{R}^I\right)$ such that
    \begin{enumerate}[(1).]
        \item $\sum_{i=1}^I t_i=0$
        \item $v_i(k',\theta_i)+t_i> v_i(k(\theta),\theta_i)$, $\forall i\in I$
    \end{enumerate}
    i.e., we can't get a higher total social welfare. (Because of the transfers, a higher social welfare can make everyone better off.)
\end{definition}

\begin{proposition}[ex-post efficient $\Leftrightarrow$ maximizing the sum of utilities]
    $\forall$ project-choice rule $k(\cdot)$, $k(\cdot)$ is ex-post efficient \underline{if and only if} $k(\cdot)$ maximizes the sum of utilities, i.e., $\forall \theta\in\Theta$ and $\forall k'\in K$,
    \begin{equation}
        \begin{aligned}
            \sum_{i=1}^I v_i(k(\theta),\theta_i)\geq \sum_{i=1}^I v_i(k',\theta_i)
        \end{aligned}
        \nonumber
    \end{equation}
\end{proposition}
\begin{proof}
    ``$\Leftarrow$'': Suppose by the way of contradiction that there exists $\left(\theta, k', t\right)$ such that $\sum_{i=1}^I t_i=0$ and $v_i(k',\theta_i)+t_i> v_i(k(\theta),\theta_i)$, $\forall i\in I$. Sum together, there is a contradiction.\\
    ``$\Rightarrow$'': Suppose by the way of contradiction that there exists $(\theta,k')$, $\sum_{i=1}^I v_i(k(\theta),\theta_i)<\sum_{i=1}^I v_i(k',\theta_i)$. Then, we can define a $t$ such that satisfies $\sum_{i=1}^I t_i=0$ and $v_i(k',\theta_i)+t_i> v_i(k(\theta),\theta_i)$, $\forall i\in I$. Let $\Delta=\sum_{i=1}^I v_i(k',\theta_i)-\sum_{i=1}^I v_i(k(\theta),\theta_i)$, then $t_i=v_i(k(\theta),\theta_i)-v_i(k',\theta_i)+\frac{\Delta}{I},\forall i\in I$ is the transfer-choice we want.
\end{proof}

\subsection{Vickrey-Clarke-Groves Mechanism}
\begin{proposition}[VCG Mechanism]
    Suppose $k^*(\cdot)$ is ex-post efficient project choice rule. For each $i\in \{1,...,I\}$, let $h_i:\Theta_{-i} \rightarrow \mathbb{R}$ be an arbitrary function.\\
    Define the transfer rule $t(\cdot)$ as follows
    \begin{equation}
        \begin{aligned}
            t_i(\theta_i,\theta_{-i})=\sum_{j\neq i}v_j(k^*(\theta_i,\theta_{-i}),\theta_j)+h_i(\theta_{-i})
        \end{aligned}
        \nonumber
    \end{equation}
    Then the SCF $f(\cdot)=\left(k^*(\cdot),t(\cdot)\right)$ is DSIC.
\end{proposition}
\begin{proof}
    Take any $i,\theta\in\Theta$ and let $\hat{\theta}_i\in\Theta_i$.
    Reporting truthfully gives higher profits than misreporting $\hat{\theta}_i$:
        \begin{equation}
            \begin{aligned}
                v_i(k^*(\theta_i,\theta_{-i}),\theta_i)+t_i(\theta_i,\theta_{-i})&=v_i(k^*(\theta_i,\theta_{-i}),\theta_i)+\sum_{j\neq i}v_j(k^*(\theta),\theta_j)+h_i(\theta_{-i})\\
                &=\sum_{j=1}^Iv_j(k^*(\theta),\theta_j)+h_i(\theta_{-i})\\
                &\geq \sum_{j=1}^Iv_j(k^*(\hat{\theta}_i,\theta_{-i}),\theta_j)+h_i(\theta_{-i})\\
                &=v_i(k^*(\hat{\theta}_i,\theta_{-i}),\theta_i)+t_i(\hat{\theta}_i,\theta_{-i})
            \end{aligned}
            \nonumber
        \end{equation}
    Hence, VCG mechanism with SCF $f(\cdot)=\left(k^*(\cdot),t(\cdot)\right)$ is DSIC.
\end{proof}

\begin{definition}[Pivotal VCG Mechanism (Special Case)]
    \normalfont
    Let $h_i(\theta_{-i})=\max_{k\in K}\sum_{j\neq i}v_j(k,\theta_j)$.
    \begin{equation}
        \begin{aligned}
            t_i(\theta_i,\theta_{-i})=\sum_{j\neq i}v_j(k^*(\theta_i,\theta_{-i}),\theta_j)-\max_{k\in K}\sum_{j\neq i}v_j(k,\theta_j)\leq 0
        \end{aligned}
        \nonumber
    \end{equation}
    \begin{enumerate}
        \item $i$ is \textbf{pivotal} if $k^*(\theta)$ doesn't maximize $\max_{k\in K}\sum_{j\neq i}v_j(k,\theta_j)$.
        \item $i$ is \textbf{not pivotal} if $k^*(\theta)$ maximizes $\max_{k\in K}\sum_{j\neq i}v_j(k,\theta_j)$.
    \end{enumerate}
    \begin{note}
        $i$ is \textbf{not pivotal} $\Rightarrow$ $t_i(\theta)=0$.
    \end{note}
\end{definition}

\begin{example}
    Suppose $k\in\{0,1\}, \theta\in \Theta\subset \mathbb{R}^I$, $v_i(k,\theta_i)=k\theta_i$. Since $\sum_{i=1}^Iv_i(k,\theta_i)=k\sum_{i=1}^I\theta_i$, $k^*(\cdot)$ is ex-post efficient: $k^*(\theta)=1 \Leftrightarrow \sum_{i=1}^I\theta_i\geq 0$. The pivotal VCG transfers:
    \begin{equation}
        \begin{aligned}
            t_i(\theta)=\left\{\begin{matrix}
                \sum_{j\neq i}\theta_j-0&\textnormal{ if }\sum_{j=1}^I \theta_j\geq 0>\sum_{j\neq i}\theta_j\\
                0-\sum_{j\neq i}\theta_j&\textnormal{ if }\sum_{j=1}^I \theta_j< 0\leq\sum_{j\neq i}\theta_j\\
                0&\textnormal{ otherwise}
            \end{matrix}\right.
        \end{aligned}
        \nonumber
    \end{equation}
\end{example}

\begin{example}[ (Second Price Auction)]
    One indivisible object to be allocated to one of $1,...,I$. Social decision is deciding who gets the object, $K=\{1,...,I\}$, $k=i$ means ``i receives the object''. $\Theta_i\subseteq \mathbb{R}_+$, $\theta_i\in\Theta_i$ denotes $i$'s valuation for the object $v_i(k,\theta_i)=\left\{\begin{matrix}
        \theta_i,& \textnormal{ if }k=i\\
        0,& \textnormal{ if }k\neq i
    \end{matrix}\right.$.
    Ex-post efficient $k^*(\cdot)$ allocates the object to the individual with the highest valuation. The pivotal VCG transfers:
    \begin{equation}
        \begin{aligned}
            t_i(\cdot)=\left\{\begin{matrix}
                0-\theta^{(2)}\textnormal{(the second highest)},& \textnormal{ if }k^*(\theta)=i \textnormal{ ($i$ is pivotal)}\\
                0=\theta^{(1)}-\theta^{(1)},&\textnormal{ if }k^*(\theta)\neq i
            \end{matrix}\right.
        \end{aligned}
        \nonumber
    \end{equation}
\end{example}


\begin{example}[ (Uniform-Price Auction)]
    $m$-identical indivisible objects ($m<I$). Each agent can consume $0$ or $1$ object. $K=\{M\subset \{1,...,I\}\mid |M|=m\}$, where $k=M$ is the set of agents who receive an object. $v_i(k,\theta_i)=\left\{\begin{matrix}
        \theta_i,&i\in k\\
        0,&i\notin k
    \end{matrix}\right.$. The ex-post efficient $k^*(\cdot)$ allocates the objects to top $m$-valuation agents. The pivotal VCG transfers:
    \begin{equation}
        \begin{aligned}
            t_i(\theta)=\left\{\begin{matrix}
                \left(\sum_{j=1}^m\theta_{(j)}-\theta_i\right)-\left(\sum_{j=1}^{m+1}\theta_{(j)}-\theta_i\right)=-\theta^{(m+1)}&,i\in k^*(\theta)\\
                0&,i\notin k^*(\theta)
            \end{matrix}\right.
        \end{aligned}
        \nonumber
    \end{equation}
\end{example}

\begin{example}[ (Package Auction)]
    $2$ identical indivisible objects to be allocated $I=3$ agents. Each agent can consume $0$, $1$, or $2$ units. $K=\{(1,1,0),(1,0,1),(0,1,1),(2,0,0),(0,2,0),(0,0,2)\}$. $\Theta_i=\{\theta_i=(v_1,v_2)\in \mathbb{R}^2_+\mid v_2\geq v_1\geq 0\}$. Consider an example,
    \begin{center}
        \begin{tabular}{c|c|c|c}
            \hline
                &$\theta_1$ &$\theta_2$&$\theta_3$\\
            \hline
                $v_1$&$3$ &$4$&$1$\\
            \hline
                $v_2$&$4$ &$5$&$6$\\
            \hline
        \end{tabular}
    \end{center}
    The ex-post efficient $k^*(\theta)=(1,1,0)$. Then,
    \begin{equation}
        \begin{aligned}
            t_1(\theta)=4-6=-2, t_2(\theta)=3-6=-3, t_3(\theta)=7-7=0
        \end{aligned}
        \nonumber
    \end{equation}
\end{example}


\subsection{Uniqueness of VCG Mechanism}
\begin{assumption}
    $K$ is a compact subset of a topological space which all singletons are closed (metric spaces $K\subset \mathbb{R}^n$, $K$ compact, or any finite $K$.)
\end{assumption}

Let $V_{usc}$ be the set of upper hemicontinuous functions $v: K \rightarrow \mathbb{R}$. ($v$ is upper hemicontinuous if $\forall \alpha\in \mathbb{R}: \{k\in K\mid v(k)\geq \alpha\}$ is closed.)

\textbf{Facts}: A upper hemicontinuous function attains the maximum over a compact set. Sum of upper hemicontinuous functions is upper hemicontinuous.

\begin{proposition}[Green \& Laffont 1979]\label{prop:Uniq_VCG_mechanism}
    Suppose that $\forall i: \{v_i(\cdot,\theta_i): K \rightarrow \mathbb{R}\mid \theta_i\in\Theta_i\}=V_{usc}$. Then, any ex-post efficient and DSIC direct mechanism is a VCG mechanism.
\end{proposition}
\begin{proof}
    Take any $f(\cdot)=\left(k^*(\cdot),t_i(\cdot)\right)$ such that it is ex-post efficient and DSIC. We prove it is VCG mechanism by showing there is a $h_{i}$ satisfies the definition of VCG mechanism.\\
    Define $\forall i, h_i:\Theta \rightarrow \mathbb{R}$ such that
    \begin{equation}
        \begin{aligned}
            h_i(\theta)=-\sum_{j\neq i}v_j(k^*(\theta_i,\theta_{-i}),\theta_j)+t_i(\theta_i,\theta_{-i})
        \end{aligned}
        \nonumber
    \end{equation}
    \textbf{We want to show $h_i(\theta)$ is independent of $\theta_i$ and is actually $h_i(\theta_{-i})$.}\\
    That is, $\forall \theta_i,\hat{\theta}_i,\theta_{-i}$, we want to show $h_i(\theta_i,\theta_{-i})=h_i(\hat{\theta}_i,\theta_{-i})$.
    \begin{lemma}\label{lemma:k_h}
        If $k^*(\theta_i,\theta_{-i})=k^*(\hat{\theta}_i,\theta_{-i})$, then $h_i(\theta_i,\theta_{-i})=h_i(\hat{\theta}_i,\theta_{-i})$.
    \end{lemma}
    \begin{proof}
        $k^*(\theta_i,\theta_{-i})=k^*(\hat{\theta}_i,\theta_{-i})$ requires
        \begin{equation}
            \begin{aligned}
                v_i(k^*(\theta_i,\theta_{-i}),\theta_i)+t_i(\theta_i,\theta_{-i})&\geq v_i(k^*(\hat{\theta}_i,\theta_{-i}),\theta_i)+t_i(\hat{\theta}_i,\theta_{-i})\\
                v_i(k^*(\hat{\theta}_i,\theta_{-i}),\hat{\theta}_i)+t_i(\hat{\theta}_i,\theta_{-i})&\geq v_i(k^*(\theta_i,\theta_{-i}),\hat{\theta}_i)+t_i(\theta_i,\theta_{-i})
            \end{aligned}
            \nonumber
        \end{equation}
        Since $v_i(k^*(\hat{\theta}_i,\theta_{-i}),\hat{\theta}_i)=v_i(k^*(\theta_i,\theta_{-i}),\hat{\theta}_i)$, we have $t_i(\hat{\theta}_i,\theta_{-i})=t_i(\theta_i,\theta_{-i})$. Hence, $h_i(\theta_i,\theta_{-i})=h_i(\hat{\theta}_i,\theta_{-i})$.
    \end{proof}
    \begin{enumerate}
        \item \underline{Case 1:} ``$k^*(\theta_i,\theta_{-i})=k^*(\hat{\theta}_i,\theta_{-i})$'', $h_i(\theta_i,\theta_{-i})=h_i(\hat{\theta}_i,\theta_{-i})$ is given by Lemma \ref{lemma:k_h}.
        \item \underline{Case 2:} ``$k^*(\theta_i,\theta_{-i})\neq k^*(\hat{\theta}_i,\theta_{-i})$''\\
        Suppose by the way of contradiction $h_i(\theta_i,\theta_{-i})\neq h_i(\hat{\theta}_i,\theta_{-i})$, WLOG, we consider $h_i(\theta_i,\theta_{-i})>h_i(\hat{\theta}_i,\theta_{-i})$. There is an $\epsilon>0$ s.t. $h_i(\theta_i,\theta_{-i})>h_i(\hat{\theta}_i,\theta_{-i})+\epsilon$.\\
        Define $v: K \rightarrow \mathbb{R}$ such that
        \begin{equation}
            \begin{aligned}
                v(k)=\left\{\begin{matrix}
                    -\sum_{j\neq i}v_j(k^*(\theta_i,\theta_{-i}),\theta_j),& \textnormal{ if }k=k^*(\theta_i,\theta_{-i})\\
                    -\sum_{j\neq i}v_j(k^*(\hat{\theta}_i,\theta_{-i}),\theta_j)+\epsilon,& \textnormal{ if }k=k^*(\hat{\theta}_i,\theta_{-i})\\
                    -C,&\textnormal{ otherwise}\
                \end{matrix}\right.
            \end{aligned}
            \nonumber
        \end{equation}
        where $C>\max_{k\in K}\sum_{j\neq i}v_j(k^*(k,\theta_{-i}),\theta_j)$.\\
        Hence, $v$ is upper hemicontinuous, $v\in V_{usc}$.

        By the assumption that $\forall i: \{v_i(\cdot,\theta_i): K \rightarrow \mathbb{R}\mid \theta_i\in\Theta_i\}=V_{usc}$, we know $\exists \theta^\epsilon_i\in\Theta_i$ s.t. $v_i(\cdot,\theta^\epsilon_i)=v(\cdot)$.
        \begin{enumerate}[$\circ$]
            \item Because $k^*(\cdot)$ is ex-post efficient,
            \begin{equation}
                \begin{aligned}
                    v_i(k^*(\theta^\epsilon_i,\theta_{-i}),\theta^\epsilon_i)+\sum_{j\neq i} v_j(k^*(\theta^\epsilon_i,\theta_{-i}),\theta_j)&\geq v_i(k^*(\hat{\theta}_i,\theta_{-i}),\theta^\epsilon_i)+\sum_{j\neq i} v_j(k^*(\hat{\theta}_i,\theta_{-i}),\theta_j)\\
                    &=v(k^*(\hat{\theta}_i,\theta_{-i}))+\sum_{j\neq i} v_j(k^*(\hat{\theta}_i,\theta_{-i}),\theta_j)=\epsilon
                \end{aligned}
                \nonumber
            \end{equation}
            By the definition of $v(\cdot)$, we have
            \begin{equation}
                \begin{aligned}
                    v_i(k^*(\theta^\epsilon_i,\theta_{-i}),\theta^\epsilon_i)+\sum_{j\neq i} v_j(k^*(\theta^\epsilon_i,\theta_{-i}),\theta_j)=v(k^*(\theta^\epsilon_i,\theta_{-i}))+\sum_{j\neq i} v_j(k^*(\theta^\epsilon_i,\theta_{-i}),\theta_j)\leq \epsilon
                \end{aligned}
                \nonumber
            \end{equation}
            Hence, we can conclude $k^*(\theta^\epsilon_i,\theta_{-i})=k^*(\hat{\theta}_i,\theta_{-i})$. Then, by the Lemma \ref{lemma:k_h}, $h_i(\theta^\epsilon_i,\theta_{-i})=h_i(\hat{\theta}_i,\theta_{-i})$.
            \item Because $f(\cdot)=\left(k^*(\cdot),t_i(\cdot)\right)$ is DSIC, the agent with $\theta^\epsilon_i$ gets the highest profit from truthfully reporting
            \begin{equation}
                \begin{aligned}
                    v_i(k^*(\theta^\epsilon_i,\theta_{-i}),\theta^\epsilon_i)+t_i(\theta^\epsilon_i,\theta_{-i})&\geq v_i(k^*(\theta_i,\theta_{-i}),\theta^\epsilon_i)+t_i(\theta_i,\theta_{-i})\\
                    \Leftrightarrow -\sum_{j\neq i}v_j(k^*(\theta^\epsilon_i,\theta_{-i}),\theta_j)+\epsilon+t_i(\theta^\epsilon_i,\theta_{-i})&\geq-\sum_{j\neq i}v_j(k^*(\theta_i,\theta_{-i}),\theta_j)+t_i(\theta_i,\theta_{-i})\\
                    \Leftrightarrow h_i(\theta^\epsilon_i,\theta_{-i})+\epsilon&\geq h_i(\theta_i,\theta_{-i})\\
                    \Leftrightarrow h_i(\hat{\theta}_i,\theta_{-i})+\epsilon&\geq h_i(\theta_i,\theta_{-i})
                \end{aligned}
                \nonumber
            \end{equation}
            There is a contradiction.
        \end{enumerate}
    \end{enumerate}
\end{proof}


\subsection{Budget Balancedness of VCG Mechanism}
\begin{definition}[Budget-Balanced VCG Mechanism]
    \normalfont
    A VCG mechanism is \textbf{budget-balanced} if $\sum_{i=1}^It_i(\theta)=0$.
\end{definition}

Based on the Proposition \ref{prop:Uniq_VCG_mechanism}, we can show the following corollary.
\begin{corollary}
    Suppose $I\geq 2$, $|K|\geq 2$, and $\forall i: \{v_i(\cdot,\theta_i): K \rightarrow \mathbb{R}\mid \theta_i\in\Theta_i\}=V_{usc}$. Then, there does not exist a budget-
    balanced VCG mechanism.
\end{corollary}


\begin{example}
    $K=\{0,1\},\Theta_i=[-1,1], v_i(k,\theta_i)=k\theta_i$. Take a VCG mechanism $k^*(\cdot)$ ex-post efficient, $h_1:\Theta_2 \rightarrow \mathbb{R}$, $h_2:\Theta_1 \rightarrow \mathbb{R}$.
    \begin{equation}
        \begin{aligned}
            t_1(\theta)+t_2(\theta)&=v_2(k^*(\theta),\theta_2)+h_1(\theta_2)+v_1(k^*(\theta),\theta_1)+h_2(\theta_1)\\
            &=\max\{0,\theta_1+\theta_2\}+h_1(\theta_2)+h_2(\theta_1)
        \end{aligned}
        \nonumber
    \end{equation}
    Suppose by the contradiction that it is a budget-balanced VCG mechanism.
    \begin{equation}
        \begin{aligned}
            \max\{0,\theta_1+\theta_2\}+h_1(\theta_2)+h_2(\theta_1)=0
        \end{aligned}
        \nonumber
    \end{equation}
    We have
    \begin{equation}
        \begin{aligned}
            h_2(1)-h_2(0)=\max\{0,\theta_1\}-\max\{0,\theta_1+1\}
        \end{aligned}
        \nonumber
    \end{equation}
    The LHS is constant and the RHS is a function of $\theta_1$, which gives a contradiction.
\end{example}


\subsection{Expected-externality Mechanism (BIC)}
\begin{definition}[EE Mechanism]
\normalfont
    $(k^*(\cdot),t(\cdot))$ is an Expected-externality (EE/AGV) mechanism if $k^*(\cdot)$ is ex-post efficient and there are functions $h_i:\Theta_i: \mathbb{R}$ for all $i$ s.t.
    \begin{equation}
        \begin{aligned}
            t_i(\theta)=\underbrace{\mathbb{E}_{\tilde{\theta}_{-i}}\left[\sum_{j\neq i}v_j(k^*(\theta_i,\tilde{\theta}_{-i}),\theta_j)\right]}_{\triangleq \xi_i(\theta_i)}+h_i(\theta_{-i})
        \end{aligned}
        \label{EE}
    \end{equation}
    where $\xi_i(\theta_i)\triangleq\mathbb{E}_{\tilde{\theta}_{-i}}\left[\sum_{j\neq i}v_j(k^*(\theta_i,\tilde{\theta}_{-i}))\right]$ is the expected externality $i$ imposes on others, from $i$'s interim perspective when her type is $\theta_i$.
\end{definition}

\begin{proposition}
    EE mechanisms are BIC.
\end{proposition}
\begin{proof}
    Take any $i$ and $\theta_i,\hat{\theta}_i\in\Theta_i$. $i$'s expected payoff from truthfully reproting is
    \begin{equation}
        \begin{aligned}
            &\mathbb{E}_{\tilde{\theta}_{-i}}\left[v_i(k^*(\theta_i,\tilde{\theta}_{-i}),\theta_i)+t_i(\theta_i,\tilde{\theta}_{-i})\right]\\
            \textnormal{(substitute \eqref{EE}) }=&\mathbb{E}_{\tilde{\theta}_{-i}}\left[\sum_{j=1}^I v_j(k^*(\theta_i,\tilde{\theta}_{-i}),\theta_j)\right]+\mathbb{E}_{\tilde{\theta}_{-i}}[h_i(\tilde{\theta}_{-i})]\\
           \textnormal{($k^*$ is ex-post efficient) } \geq& \mathbb{E}_{\tilde{\theta}_{-i}}\left[\sum_{j=1}^I v_j(k^*(\hat{\theta}_i,\tilde{\theta}_{-i}),\theta_j)\right]+\mathbb{E}_{\tilde{\theta}_{-i}}[h_i(\tilde{\theta}_{-i})]\\
           =& \mathbb{E}_{\tilde{\theta}_{-i}}\left[\sum_{j=1}^I v_j(k^*(\hat{\theta}_i,\tilde{\theta}_{-i}),\theta_j)\right]+\mathbb{E}_{\tilde{\theta}_{-i}}[t_i(\hat{\theta}_i,\tilde{\theta}_{-i})-\xi_i(\hat{\theta_i})]\\
           =&\mathbb{E}_{\tilde{\theta}_{-i}}\left[v_i(k^*(\hat{\theta}_i,\tilde{\theta}_{-i}),\theta_i)+t_i(\hat{\theta}_i,\tilde{\theta}_{-i})\right]
        \end{aligned}
        \nonumber
    \end{equation}
\end{proof}


\subsection{Budget-Balanced EE Mechanism}
Budget balancedness requires
\begin{equation}
    \begin{aligned}
        0=\sum_{i=1}^I t_i(\theta)=\sum_{i=1}^I[\xi_i(\theta_i)+h_i(\theta_{-i})] \Leftrightarrow \sum_{i=1}^I h_i(\theta_{-i})=-\sum_{i=1}^I \xi_i(\theta_i)
    \end{aligned}
    \nonumber
\end{equation}
Suppose $h_i(\theta_{-i})$ is in the form of $h_i(\theta_{-i})=c\sum_{j\neq i} \xi_j(\theta_j)$. Then,
\begin{equation}
    \begin{aligned}
        \sum_{i=1}^I h_i(\theta_{-i})=c(I-1)\sum_{i=1}^I \xi_i(\theta_i) \Rightarrow c=-\frac{1}{I-1}
    \end{aligned}
    \nonumber
\end{equation}
\begin{proposition}
    The EE mechanism where $h_i(\theta_{-i})=-\frac{1}{I-1}\sum_{j\neq i} \xi_j(\theta_j)$ is budget-balanced, $$t_i(\theta)=\xi_i(\theta_i)-\frac{1}{I-1}\sum_{j\neq i} \xi_j(\theta_j)$$
\end{proposition}

\begin{corollary}
    $\exists$ a BIC, ex-post efficient and budget-balanced direct mechanism.
\end{corollary}


\begin{example}
    Project choice with $K=\{0,1\}$, $\theta_i\sim U[-1,1]$, $v_j(k,\theta_j)=k\theta_j$. Let $k^*(\cdot)$ be:
    \begin{equation}
        \begin{aligned}
            k^*(\theta)=\left\{\begin{matrix}
                1,&\textnormal{ if }\theta_1+\theta_2\geq 0\\
                0,&\textnormal{ if }\theta_1+\theta_2<0
            \end{matrix}\right.
        \end{aligned}
        \nonumber
    \end{equation}
    Then,
    \begin{equation}
        \begin{aligned}
            \xi_i(\theta_i)&\triangleq\mathbb{E}_{\tilde{\theta}_{-i}}\left[v_{-i}(k^*(\theta_i,\tilde{\theta}_{-i}))\right]\\
            &=\int_{-1}^{-\theta_i}0\times \frac{1}{2} d\tilde{\theta}_{-i}+\int_{-\theta_i}^1 \tilde{\theta}_{-i}\times \frac{1}{2} d\tilde{\theta}_{-i}=\frac{1}{4}(1-\theta_i^2)
        \end{aligned}
        \nonumber
    \end{equation}
    Hence, the budget-balanced EE mechanism is given by
    \begin{equation}
        \begin{aligned}
            t_i(\theta_i)=\xi_i(\theta_i)-\xi_j(\theta_j)=\frac{1}{4}(\theta_j^2-\theta_i^2)
        \end{aligned}
        \nonumber
    \end{equation}
\end{example}

\subsection{Linear Utility Model}
Suppose types are real numbers $\Theta_i=[\underline{\theta}_i,\overline{\theta}_i]$ and $v_i(k,t,\theta_i)=\theta_i\cdot v_i(k)+t_i$, where $v_i:K \rightarrow \mathbb{R}_i$.

Given a direct mechanism $\left(\Theta,k(\cdot),t(\cdot)\right)$, define interim expected values of $v_i(\cdot)$ and $t_i(\cdot)$:
\begin{equation}
    \begin{aligned}
        \bar{v}_i(\theta_i)=\mathbb{E}_{\theta_{-i}}\left[v_i(\theta_i,\theta_{-i})\right]\\
        \bar{t}_i(\theta_i)=\mathbb{E}_{\theta_{-i}}\left[t_i(\theta_i,\theta_{-i})\right]
    \end{aligned}
    \nonumber
\end{equation}
The expected utility of agent $i$ when all agents report truthfully,
\begin{equation}
    \begin{aligned}
        U_i(\theta_i)&=\mathbb{E}_{\theta_{-i}}\left[\theta_iv_i(\theta_i,\theta_{-i})+t_i(\theta_i,\theta_{-i})\right]\\
        &=\theta_i \bar{v}_i(\theta_i)+\bar{t}_i(\theta_i)
    \end{aligned}
    \nonumber
\end{equation}

\begin{proposition}\label{Linear utility model_BIC}
    A direct mechanism $\left(\Theta,k(\cdot),t(\cdot)\right)$ is BIC \underline{iff} $\forall i\in\{1,...,I\}$
    \begin{enumerate}[(1).]
        \item $\bar{v}_i(\theta_i)$ is non-decreasing in $\theta_i$.
        \item $\forall \theta_i\in\Theta_i$, $U_i(\theta_i)=U_i(\underline{\theta}_i)+\int_{\underline{\theta}_i}^{\theta_i}\bar{v}_i(s)ds$
    \end{enumerate}
\end{proposition}
\begin{proof}
    ``$\Rightarrow$'': Given the direct mechanism is BIC. Take any $i$, $\theta_i,\hat{\theta}_i\in\Theta$, agents with $\theta_i,\hat{\theta}_i$ both report truthfully
    \begin{equation}
        \begin{aligned}
            \theta_i \bar{v}_i(\theta_i)+\bar{t}_i(\theta_i)&\geq \theta_i \bar{v}_i(\hat{\theta}_i)+\bar{t}_i(\hat{\theta}_i)\\
            \hat{\theta}_i \bar{v}_i(\hat{\theta}_i)+\bar{t}_i(\hat{\theta}_i)&\geq \hat{\theta}_i \bar{v}_i(\theta_i)+\bar{t}_i(\theta_i)\\
            \Rightarrow [\theta_i-\hat{\theta}_i][\bar{v}_i(\theta_i)-\bar{v}_i(\hat{\theta}_i)]&\geq 0
        \end{aligned}
        \nonumber
    \end{equation}
    Hence, $\bar{v}_i(\theta_i)$ is non-decreasing.\\
    $U_i(\theta_i)=\max_{\hat{\theta}_i\in\Theta_i}\left[\theta_i\bar{v}_i(\hat{\theta}_i)+t_i(\hat{\theta}_i)\right]$ by BIC is maximized at $\hat{\theta}_i=\theta_i$.\\
    By Envelope Theorem,
    \begin{equation}
        \begin{aligned}
            U_i(\theta_i)&=U_i(\underline{\theta}_i)+\int_{\underline{\theta}_i}^{\theta_i}U'_i(s)ds\\
            &=U_i(\underline{\theta}_i)+\int_{\underline{\theta}_i}^{\theta_i}\bar{v}_i(s)ds
        \end{aligned}
        \nonumber
    \end{equation}
    ``$\Leftarrow$'': Take any $i$, $\theta_i,\hat{\theta}_i\in\Theta$. $i$'s expected interim payoff from reporting $\hat{\theta}_i$ instead of $\theta_i$ is
    \begin{equation}
        \begin{aligned}
            &\underbrace{\theta_i \bar{v}_i(\theta_i)+\bar{t}_i(\theta_i)}_{U_i(\theta_i)}-\left[\theta_i \bar{v}_i(\hat{\theta}_i)+\bar{t}_i(\hat{\theta}_i)\right]\\
            =&U_i(\theta_i)-\left[U_i(\hat{\theta}_i)+(\theta-\hat{\theta}_i)\bar{v}_i(\hat{\theta}_i)\right]\\
            =&U_i(\underline{\theta}_i)+\int_{\underline{\theta}_i}^{\theta_i}\bar{v}_i(s)ds-[U_i(\underline{\theta}_i)+\int_{\underline{\theta}_i}^{\hat{\theta}_i}\bar{v}_i(s)ds]-(\theta-\hat{\theta}_i)\bar{v}_i(\hat{\theta}_i)\\
            =&\int_{\theta_i}^{\hat{\theta}_i}(\bar{v}_i(\hat{\theta}_i)-\bar{v}_i(s))ds\geq 0
        \end{aligned}
        \nonumber
    \end{equation}
    So the direct mechanism is BIC.
\end{proof}

\section{Auction}
Based on
\begin{enumerate}[$\circ$]
    \item Klemperer, P. (1998). Auctions with almost common values: The Wallet Game'and its applications. \textit{European Economic Review}, 42(3-5), 757-769.
\end{enumerate}


\subsection{Examples: Auctions with Common-value}
\begin{enumerate}[(1).]
    \item Financial assets;
    \item Oilfields;
    \item A takeover target has a common value if the bidders are financial acquirers (e.g. LBO firms) who will follow similar management strategies if successful;
    \item The Personal Communications Spectrum (PCS) licenses sold by the U.S. Government in the 1995 "Airwaves Auction".
\end{enumerate}

\subsection{First / Second Price Sealed-bid Auction}
\begin{enumerate}[$\circ$]
    \item A seller sells an indivisible object.
    \item There are $N=\{1,...,n\}$ bidders, $i\in N$.
    \item Each bidder has a valuation for the object, $X_i\sim F$, $x_i\in[\underline{x},\overline{x}]$. p.d.f. $f(\cdot)$ is strictly positive and continuous.
    \item Strategy of $i$: $b_i:[\underline{x},\overline{x}] \rightarrow \mathbb{R}$, a \underline{bid function}.
\end{enumerate}

\begin{assumption}
    1. Independence; 2. Symmetry; 3. Private Values; 4. Risk-neutrality.
\end{assumption}

Let $X=(X_1,...,X_n)$. The $k^{th}$-order statistic, $X^k$, is the $k^{th}$ the highest value in $X_1,...,X_n$.

\begin{definition}[Second Price Auction]
    \normalfont
    Highest bidder wins and pays the second-highest bid. (If more than one bidders bid the highest value, they win with equal probability.)\\
    It can be written as the form of Bayesian game: a bidder $i$'s utility function is
    \begin{equation}
        \begin{aligned}
            u_i(b_1,...,b_n;x_i)=\left\{\begin{matrix}
                \frac{1}{|\{j\in N:b_j=b_i\}|}x_i-b^2,&b_i=b^1\\
                0,&b_i\neq b^1
            \end{matrix}\right.
        \end{aligned}
        \nonumber
    \end{equation}
    where $b^k$ is the $k$-th highest bid.
\end{definition}

\begin{theorem}[Second Price Auction: Bid Truthfully]
    In the second-price sealed-bid auction, it is a (weakly) dominant strategy to bid your valuation, i.e., $\forall i\in N,\forall x_i\in[\underline{x},\overline{x}]$, $b_i(x_i)=x_i$.\\
    That is, $\forall i, \forall b'_i\in \mathbb{R}$,
    \begin{equation}
        \begin{aligned}
            u_i(x_i,b_{-i};x_i)\geq u_i(b'_i,b_{-i};x_i), \forall b_{-i}\in \mathbb{R}^{n-1}
        \end{aligned}
        \nonumber
    \end{equation}
    (Moreover, if $\exists b'_{i}\neq x_i$, then $\exists b_{-i}\in \mathbb{R}^{n-1}$ such that $u_i(x_i,b_{-i};x_i)> u_i(b'_{i},b_{-i};x_i)$.)
\end{theorem}
\begin{proof}
    Player $i$ has value $x_i$ and treats $b^1_{-i}$ as a random variable. The payoff conditional on winning is $$x_i-b^1_{-i}$$ By bidding $b_i=x_i$, $i$ ensures that $i$ wins if $b_i=x_i>b^1_{-i}\Leftrightarrow x_i-b^1_{-i}>0$ and $i$ loses if $b_i=x_i<b^1_{-i}\Leftrightarrow x_i-b^1_{-i}<0$.
\end{proof}

\begin{definition}[First Price Auction]
    \normalfont
    Highest bidder wins and pays her bid. (If more than one bidder bid the highest value, they win with equal probability.)\\
    It can be written as the form of Bayesian game: a bidder $i$'s utility function is
    \begin{equation}
        \begin{aligned}
            u_i(b_1,...,b_n;x_i)=\left\{\begin{matrix}
                \frac{1}{|\{j\in N:b_j=b_i\}|}x_i-b_i,&b_i=b^1\\
                0,&b_i\neq b^1
            \end{matrix}\right.
        \end{aligned}
        \nonumber
    \end{equation}
    where $b^k$ is the $k$-th highest bid.
\end{definition}


\subsubsection*{Bayesian Nash Equilibrium Analysis of First Price Auction}
Conjecture that $\exists$ a BNE with the following properties:
\begin{enumerate}
    \item Symmetry: $b_1(\cdot)=b_2(\cdot)=\cdots=b_n(\cdot):=b(\cdot)$.
    \item $b(\cdot)$ is differentiable.
    \item $b'(\cdot)>0$.
\end{enumerate}
Take any bidder $i$ with valuation $x_i$. Assume $i$ knows $b(\cdot)$ and knows that the other bidder use the same $b(\cdot)$. Take any $b_i\in \mathbb{R}$ ($b_i:=b(x_i)$). (Not that, by the continuity of $X_i$, it is impossible to tie in this case.)

Then, $i$'s expected payoff from bidding $b_i$ is
\begin{equation}
    \begin{aligned}
        P(b(X_j)\leq b_i,\forall j\neq i)(x_i-b_i)=F^{n-1}(b^{-1}(b_i))(x_i-b_i)
    \end{aligned}
    \nonumber
\end{equation}
The necessary F.O.C. gives that optimal $b_i$ satisfies
\begin{equation}
    \begin{aligned}
        (n-1)f(b^{-1}(b_i))\frac{1}{b'(b^{-1}(b_i))}F^{n-2}(b^{-1}(b_i))(x_i-b_i)-F^{n-1}(b^{-1}(b_i))=0
    \end{aligned}
    \nonumber
\end{equation}
Since $b(\cdot)$ is a symmetric BNE, the optimal $b_i$ must be $b(x_i)$, then $b^{-1}(b_i)=x_i$.
\begin{equation}
    \begin{aligned}
        (n-1)f(x_i)\frac{1}{b'(x_i)}F^{n-2}(x_i)(x_i-b(x_i))-F^{n-1}(x_i)=0
    \end{aligned}
    \nonumber
\end{equation}
Hence,
\begin{equation}
    \begin{aligned}
        \underbrace{(n-1)f(x_i)F^{n-2}(x_i)x_i+F^{n-1}(x_i)}_{\frac{\partial F^{n-1}(x_i)x_i}{\partial x_i}}-F^{n-1}(x_i)=\underbrace{(n-1)f(x_i)F^{n-2}(x_i)b(x_i)+b'(x_i)F^{n-1}(x_i)}_{\frac{\partial F^{n-1}(x_i)b(x_i)}{\partial x_i}}
    \end{aligned}
    \nonumber
\end{equation}
Taking integral at both sides in $[\underline{x},x]$,
\begin{equation}
    \begin{aligned}
        F^{n-1}(x)x-\int_{\underline{x}}^x F^{n-1}(t) dt= F^{n-1}(x)b(x)
    \end{aligned}
    \nonumber
\end{equation}
That is,
\begin{equation}
    \begin{aligned}
        b(x)=x-\frac{1}{F^{n-1}(x)}\int_{\underline{x}}^x F^{n-1}(t) dt
    \end{aligned}
    \label{FPA:symmetric BNE}
\end{equation}
Note $b(\cdot)$ is differentiable and $b'(\cdot)>0$. We can extend $b(\cdot)$ to $[\underline{x},\overline{x}]$ by setting $b(\underline{x})=\lim_{x \rightarrow \underline{x}} b(x)=\underline{x}$.

\begin{proposition}[Symmetric BNE of First Price Auction]
    $b(x)=x-\frac{1}{F^{n-1}(x)}\int_{\underline{x}}^x F^{n-1}(t) dt$ is a symmetric BNE of First Price Auction.
\end{proposition}
\begin{proof}
    Any bid higher than $b(\overline{x})$ is suboptimal, and any bid lower than $b(\underline{x})$ is indifferent to the $b(\underline{x})$.\\
    We prove that, for a bidder with $x_i$, she prefers to bid $b(x_i)$ than $b(y),\forall y$.\\
    Bidding $b(y)$ gives expected payoff
    \begin{equation}
        \begin{aligned}
            F^{n-1}(y)(x_i-b(y))&=F^{n-1}(y)(y-b(y))+F^{n-1}(y)(x_i-y)\\
            (\textnormal{by \eqref{FPA:symmetric BNE}})&=\int_{\underline{x}}^y F^{n-1}(t) dt+F^{n-1}(y)(x_i-y)\\
            &=\int_{\underline{x}}^{x_i} F^{n-1}(t) dt-\underbrace{\int_y^{x_i} [F^{n-1}(t)-F^{n-1}(y)] dt}_{\geq 0, \textnormal{ minimized at }y=x_i}
        \end{aligned}
        \nonumber
    \end{equation}
\end{proof}

\begin{theorem}[Lebrun, 1999]
    Consider the bid function $b(\cdot)$ in \eqref{FPA:symmetric BNE}, the First Price Auction has essentially unique. (Bidders with types $x>\underline{x}$ bid $b(x)$ and bid for type $x=\underline{x}$ is not pinned down any further than the support $b_i(\underline{x})$ must lie in $(-\infty,b(\underline{x})]$.)
\end{theorem}

The equilibrium expected payoffs of a bidder in first-price auction and second price auction with valuation $x$ is
\begin{equation}
    \begin{aligned}
        \int_{\underline{x}}^x F^{n-1}(t)dt
    \end{aligned}
    \nonumber
\end{equation}
The equilibrium expected revenue of the seller in first-price auction and second price auction is
\begin{equation}
    \begin{aligned}
        \overline{x}-\int_{\underline{x}}^{\overline{x}} [nF^{n-1}(t)-(n-1)F^n(t)]dt
    \end{aligned}
    \nonumber
\end{equation}




\section{Revenue Equivalence Theorem}
Consider the Optimal Auctions in an Independent Private Values Setting. There is one object and $N$ bidders.
\begin{enumerate}
    \item Bidders are risk-neutral;
    \item Bidders have private valuations;
    \item each bidder $i$'s valuation independently drawn from a strictly increasing c.d.f. $F_i(v)$ (with p.d.f. $f_i(v),v\in \mathcal{X}_i$) that is continuous and bounded below;
    \item Seller knows each $F_i$ (use $F$ and $f$ to represent all distributions) and have no value for the object.
\end{enumerate}

\begin{definition}[General Auction]
    \normalfont
    A general auction mechanism: bidders have values $x$ and \textbf{strategies} $\beta: \mathcal{X}\triangleq \prod_i^N \mathcal{X}_i \rightarrow \mathcal{B}$ generate message (bids) based on their values, then there is an \textbf{allocation rule} based on bids $\pi: \mathcal{B} \rightarrow \Delta N$ generates a distribution over all bidders and a \textbf{payment rule} $\mu: \mathcal{B} \rightarrow \mathbb{R}^N$ generates payment for all bidders.
\end{definition}

\begin{definition}[Direct Mechansim]
    \normalfont
    Consider a situation that bidders follow \textit{revelation principle} that provide their true values. Then the outcome can directly base on the true values.\\
    Then a \textbf{direct mechanism} can be represented as $(Q,T)$, where $Q: \mathcal{X} \rightarrow \Delta N$ is the allocation rule and $T: \mathcal{X} \rightarrow \mathbb{R}^N$ is the payment rule, such that
    \begin{equation}
        \begin{aligned}
            Q(x)=\pi(\beta(x)),\quad T(x)=\mu(\beta(x))
        \end{aligned}
        \nonumber
    \end{equation}
\end{definition}

\begin{proposition}[Revelation Principle]
    Take any equilibrium of any auction mechanism $(\mathcal{B},\pi,\mu)$. There is a distinct direct mechanism $(Q,T)$ that produces the same outcome.
\end{proposition}


Consider a direct mechanism $(Q,T)$, an agent $i$ reports $v_i$ while others report their values.
\begin{equation}
    \begin{aligned}
        \textbf{Expected allocation: }&q_i(z_i)=\int_{\mathcal{X}_{-i}}Q_i(z_i,x_{-i})dF_{-i}(x_{-i})\\
        \textbf{Expected payment: }&t_i(z_i)=\int_{\mathcal{X}_{-i}}T_i(z_i,x_{-i})dF_{-i}(x_{-i})
    \end{aligned}
    \nonumber
\end{equation}
where $Q_i, T_i$ are $i^\textnormal{th}$ item of $Q,T$.\\
The bidder wants to maximize
\begin{equation}
    \begin{aligned}
        q_i(z_i) x_i - t_i(z_i)
    \end{aligned}
    \nonumber
\end{equation}
Define the maximum value is
\begin{equation}
    \begin{aligned}
        u_i(x_i)=\max_{z_i\in \mathcal{X}_i}\{q_i(z_i) x_i - t_i(z_i)\}
    \end{aligned}
    \nonumber
\end{equation}

\begin{assumption}
    The condition for direct mechanism being incentive competitive (IC) is:
    \begin{equation}
        \begin{aligned}
            u_i(x_i)\equiv q_i(x_i) x_i - t_i(x_i)\geq q_i(z_i) x_i - t_i(z_i), \forall x_i, z_i\in \mathcal{X}_i
        \end{aligned}
        \tag{Ass 1}
        \label{Ass 1}
    \end{equation}
\end{assumption}


Firstly, we can compute, for any $z_i\in \mathcal{X}_i$
\begin{equation}
    \begin{aligned}
        &q_i(x_i)z_i-t_i(x_i)\\
        =&q_i(x_i)x_i-t_i(x_i)+q_i(x_i)(z_i-x_i)\\
        =&u_i(x_i)+q_i(x_i)(z_i-x_i)
    \end{aligned}
    \nonumber
\end{equation}
Based on the assumption \ref{Ass 1}, we have
\begin{equation}
    \begin{aligned}
        u_i(z_i)\geq q_i(x_i)z_i-t_i(x_i)=u_i(x_i)+q_i(x_i)(z_i-x_i)
    \end{aligned}
    \nonumber
\end{equation}
which shows that $u_i(\cdot)$ is convex.

If $u_i$ is differentiable, $u'_i(x_i)=q_i(x_i)$. Then, we can write the \textbf{Envelope theorem/condition}:
\begin{equation}
    \begin{aligned}
        u_i(x_i)=u_i(0)+\int_0^{x_i} q_i(y_i)dy_i
    \end{aligned}
    \nonumber
\end{equation}
which only depends on the allocation rule.

\begin{theorem}[Revenue Equivalence Theorem]
    If the direct mechanism $(Q,T)$ is incentive competitive (IC), then for all $i,x$, the \textbf{expected payment} is
    \begin{equation}
        \begin{aligned}
            t_i(x_i)=\underbrace{t_i(0)}_{e.g.=0}+q_i(x_i)x_i-\int_0^{x_i} q_i(y_i)dy_i
        \end{aligned}
        \nonumber
    \end{equation}
\end{theorem}
\begin{proof}
    \begin{equation}
        \begin{aligned}
            u_i(x_i)= q_i(x_i) x_i - t_i(x_i)=u_i(0)+\int_0^{x_i} q_i(y_i)dy_i\\
            \Rightarrow t_i(x_i)=q_i(x_i)x_i-u_i(0)-\int_0^{x_i} q_i(y_i)dy_i
        \end{aligned}
        \nonumber
    \end{equation}
    Set $u_i(0)=-t_i(0)$, that is, if $i$'s value is zero, he pays zero.
\end{proof}

\begin{corollary}[Standard Revenue Equivalence Theorem]
    Suppose that values are \underline{i.i.d.} and bidders are \underline{risk-neutral}.\\
    Consider any auction and its \underline{symmetric} and \underline{increasing} equilibrium, in which the expected payment of bidders have $0$ value is $0$. Then the expected revenue to the seller is the \underline{same}.
\end{corollary}
\begin{proof}
    If equilibrium, is symmetric and increasing, then object is \underline{always} allocated to the bidder with the highest value. Set $t_i(0)=0$.
\end{proof}

Standard Revenue Equivalence Theorem is based on \underline{symmetric}, \underline{independent}, and \underline{private} (uncorrelated) values.































\section{Optimal Auctions}
\underline{Goal:} Find the \textbf{optimal auction} that maximizes the seller's expected revenue subject to individual rationality (IR) and Bayesian incentive compatibility for the buyers.

Given a direct mechanism $\left(\Theta,y(\cdot),t(\cdot)\right)$, define
\begin{equation}
    \begin{aligned}
        \bar{y}_i(\theta_i)=\mathbb{E}_{\theta_{-i}}\left[y_i(\theta_i,\theta_{-i})\right]\\
        \bar{t}_i(\theta_i)=\mathbb{E}_{\theta_{-i}}\left[t_i(\theta_i,\theta_{-i})\right]
    \end{aligned}
    \nonumber
\end{equation}
The expected utility of agent $i$ when all agents report truthfully,
\begin{equation}
    \begin{aligned}
        U_i(\theta_i)&=\mathbb{E}_{\theta_{-i}}\left[\theta_iy_i(\theta_i,\theta_{-i})+t_i(\theta_i,\theta_{-i})\right]\\
        &=\theta_i \bar{y}_i(\theta_i)+\bar{t}_i(\theta_i)
    \end{aligned}
    \nonumber
\end{equation}

\begin{definition}[Individual Rationality]
    \normalfont
    A direct mechanism $\left(\Theta,y(\cdot),t(\cdot)\right)$ is \textbf{individual rationality (IR)} if $\forall i,\theta_i\in\Theta_i$, $U_i(\theta_i)\geq 0$.
\end{definition}

\begin{corollary}[Corollary of Proposition \ref{Linear utility model_BIC}]
    A direct mechanism $\left(\Theta,y(\cdot),t(\cdot)\right)$ is \textbf{BIC and IR} iff $\forall i\in\{1,...,I\}$
    \begin{enumerate}[(1).]
        \item $\bar{y}_i(\theta_i)$ is non-decreasing in $\theta_i$.
        \item $\forall \theta_i\in\Theta_i$, $U_i(\theta_i)=U_i(\underline{\theta}_i)+\int_{\underline{\theta}_i}^{\theta_i}\bar{y}_i(s)ds$
        \item $U_i(\underline{\theta}_i)\geq 0$
    \end{enumerate}
\end{corollary}
For a BIC $\&$ IR mechanism, $\bar{t}_i(\theta_i)$ can be represented as
\begin{equation}
    \begin{aligned}
        U_i(\theta_i)&=U_i(\underline{\theta}_i)+\int_{\underline{\theta}_i}^{\theta_i}\bar{y}_i(s)ds\\
        \theta_i \bar{y}_i(\theta_i)+\bar{t}_i(\theta_i)&=U_i(\underline{\theta}_i)+\int_{\underline{\theta}_i}^{\theta_i}\bar{y}_i(s)ds
    \end{aligned}
    \nonumber
\end{equation}
\begin{equation}
    \begin{aligned}
        \bar{t}_i(\theta_i)&=-\theta_i \bar{y}_i(\theta_i)+U_i(\underline{\theta}_i)+\int_{\underline{\theta}_i}^{\theta_i}\bar{y}_i(s)ds
    \end{aligned}
    \label{t_star}
\end{equation}

For a BIC $\&$ IR mechanism, the \textit{seller's expected revenues} from bidder $i$:
\begin{equation}
    \begin{aligned}
        \mathbb{E}_\theta[-t_i(\theta)]&=-\int_\Theta t_i(\theta)f(\theta) d\theta\\
        &=-\int_{\Theta_i} \underbrace{\left(\int_{\Theta_{-i}}t_i(\theta_i,\theta_{-i})f_{-i}(\theta_{-i})d\theta_{-i}\right)}_{\bar{t}_i(\theta_i)}f_i(\theta_i) d\theta_i\\
        &=-\int_{\underline{\theta}_i}^{\overline{\theta}_i}\left(-\theta_i \bar{y}_i(\theta_i)+U_i(\underline{\theta}_i)+\int_{\underline{\theta}_i}^{\theta_i}\bar{y}_i(s)ds\right)f_i(\theta_i) d\theta_i\\
        &=-\underbrace{\int_{\underline{\theta}_i}^{\overline{\theta}_i}\int_{\underline{\theta}_i}^{\theta_i}\bar{y}_i(s)dsf_i(\theta_i) d\theta_i}_{\triangleq \star}+\int_{\underline{\theta}_i}^{\overline{\theta}_i}\theta_i \bar{y}_i(\theta_i)f_i(\theta_i) d\theta_i-U_i(\underline{\theta}_i)
    \end{aligned}
    \nonumber
\end{equation}
applying integration by parts
\begin{equation}
    \begin{aligned}
        \star&=\int_{\underline{\theta}_i}^{\overline{\theta}_i}\left(\int_{\underline{\theta}_i}^{\theta_i}\bar{y}_i(s)ds\right)d F_i(\theta_i)\\
        &=\left(\int_{\underline{\theta}_i}^{\theta_i}\bar{y}_i(s)ds\right)F_i(\theta_i)\bigg|_{\underline{\theta}_i}^{\overline{\theta}_i}-\int_{\underline{\theta}_i}^{\overline{\theta}_i}F_i(\theta_i)d\left(\int_{\underline{\theta}_i}^{\theta_i}\bar{y}_i(s)ds\right)\\
        &=\int_{\underline{\theta}_i}^{\overline{\theta}_i}\bar{y}_i(s)ds-\int_{\underline{\theta}_i}^{\overline{\theta}_i}\bar{y}_i(\theta_i)F_i(\theta_i)d\theta_i\\
        &=\int_{\underline{\theta}_i}^{\overline{\theta}_i}(1-F_i(s))\bar{y}_i(s) ds
    \end{aligned}
    \nonumber
\end{equation}
Hence,
\begin{equation}
    \begin{aligned}
        \mathbb{E}_\theta[-t_i(\theta)]&=-\int_{\underline{\theta}_i}^{\overline{\theta}_i}(1-F_i(\theta_i))\bar{y}_i(\theta_i) d\theta_i+\int_{\underline{\theta}_i}^{\overline{\theta}_i}\theta_i \bar{y}_i(\theta_i)f_i(\theta_i) d\theta_i-U_i(\underline{\theta}_i)\\
        &=\int_\Theta y_i(\theta)\left[\theta_i-\frac{1-F_i(\theta_i)}{f_i(\theta_i)}\right]f(\theta)d\theta-U_i(\underline{\theta}_i)
    \end{aligned}
    \nonumber
\end{equation}
The \textit{total expected revenue of the seller} is
\begin{equation}
    \begin{aligned}
        \int_\Theta \sum_{i=1}^I y_i(\theta)\left[\theta_i-\frac{1-F_i(\theta_i)}{f_i(\theta_i)}\right]f(\theta)d\theta-\sum_{i=1}^IU_i(\underline{\theta}_i)
    \end{aligned}
    \label{eq:revenue}
\end{equation}

\begin{theorem}[Revenue Equivalence Theorem]
    \begin{enumerate}
        \item Direct, BIC, and IR auction mechanisms that have the same allocation rule $y(\cdot)$ and the same utilities $(U_i(\theta_i))_{i=1,...,I}$ generate the same revenues \eqref{eq:revenue}.
        \item Take two auction formats $A$ and $B$, fix a BNE of $A$ and a BNE of $B$, interim expected payoff of $\underline{\theta}_i$ is the same for both $A$ and $B$.
    \end{enumerate}
\end{theorem}


The optimal auction design is given by
\begin{equation}
    \begin{aligned}
        \max_{\textnormal{BIC and IR }f(\cdot)}\int_\Theta \sum_{i=1}^I y_i(\theta)\left[\theta_i-\frac{1-F_i(\theta_i)}{f_i(\theta_i)}\right]f(\theta)d\theta-\sum_{i=1}^IU_i(\underline{\theta}_i)
    \end{aligned}
    \nonumber
\end{equation}

\begin{definition}[Virtual Valuation]
    \normalfont
    Define bidder $i$'s \textbf{virtual valuation} is $c_i(v_i)=v_i-\frac{1-F_i(v_i)}{f(v_i)}$.
\end{definition}

\begin{assumption}[Regularity Condition]
    Any bidder $i$'s virtual valuation $c_i(v_i)=v_i-\frac{1-F_i(v_i)}{f(v_i)}$ is strictly increasing.
\end{assumption}

\begin{corollary}[Optimal Auction Mechanism]
    Assume regularity. Then the expected revenue maximizing direct auction mechanism $(y(\cdot),t(\cdot))$ can be described as follows
    \begin{enumerate}[(1).]
        \item $y(\cdot): \Theta \rightarrow K$ is defined as follows. For any $\theta\in \Theta$, $\max_{i\in\{1,...,I\}}c_i(\theta_i)<0$, the seller keeps the object ($y_i(\theta)=0,\forall _i$). Otherwise, the object is allocated to a highest virtual valuation bidder.
        \item Define $t(\cdot):\Theta \rightarrow K$,
        \begin{equation}
            \begin{aligned}
                t_i(\theta):=-\theta_i y_i(\theta_i,\theta_{-i})+U_i(\underline{\theta}_i)+\int_{\underline{\theta}_i}^{\theta_i}y_i(s,\theta_{-i})ds
            \end{aligned}
            \nonumber
        \end{equation}
        which satisfies \eqref{t_star}.
    \end{enumerate}
\end{corollary}

\begin{example}
    Suppose $\Theta_i=[0,1],\theta_i\sim U[0,1]$. $c_i(\theta_i)=\theta_i-\frac{1-\theta_i}{1}=2\theta_i-1$, which is strictly increasing in $\theta_i$ (regularity satisfied). Then, the optimal auction mechanism is allocating the object to the highest (virtual) valuation bidder (iff his value $\theta_i\geq\frac{1}{2}$).
\end{example}

\begin{definition}[Bidder-Specific Reserve Price]
    \normalfont
    Bidder $i$'s bidder-specific reserve price $r_i^*$ is the value for which $c_i(r_i^*)=0$.
\end{definition}

\begin{theorem}[Myerson (1981)]
    The optimal (single-good) auction in terms of a direct mechanism: The good is sold to the agent $i=\arg\max_i\phi_i(\hat{v}_i)$, as long as $v_i\geq r_i^*$. If the good is sold, the winning agent $i$ is charged the smallest valuation that he could have declared while still remaining the winner:
    \begin{equation}
        \begin{aligned}
            \inf\{v_i^*:c_i(v_i^*)\geq 0 \textnormal{ and }\forall j\neq i, c_i(v_i^*)\geq c_j(\hat{v}_j)\}
        \end{aligned}
        \nonumber
    \end{equation}
\end{theorem}



\chapter{Market Design}
Based on
\begin{enumerate}[$\circ$]
    \item Two-Sided Matching: A Study in Game-Theoretic Modeling and Analysis, Roth, Alvin E.\& Sotomayor, Matilda, 1990.
    \item Fleiner, T. (2003). A fixed-point approach to stable matchings and some applications. \textit{Mathematics of Operations research}, 28(1), 103-126.
    \item Hatfield, J. W., \& Kominers, S. D. (2017). Contract design and stability in many-to-many matching. \textit{Games and Economic Behavior}, 101, 78-97.
\end{enumerate}
\section{Matching One-to-One}
Suppose there are doctors ($D$) and hospitals ($H$). For a doctor $d$, define a relation $\succeq_d$ over $H\cup\{d\}$; for a hospital $h$, define a relation $\succeq_h$ over $D\cup\{h\}$. A matching market is defined by $$\left(D,H,\{\succeq_i\}_{i\in D\cup H}\right)$$

\begin{note}
    Given a matching $\mu: D\cup H \rightarrow D\cup H$, we would call $\mu(d)$ be "$d$'s match".
\end{note}

\begin{definition}[Involution]
    \normalfont
    A matching $\mu: D\cup H \rightarrow D\cup H$ is \textbf{involution} such that $$\mu (d)\neq d \Rightarrow \mu(d)\in H, \forall d\in D$$ and $$\mu (h)\neq h \Rightarrow \mu(h)\in D, \forall h\in H$$
\end{definition}

\begin{definition}[Stable]
    \normalfont
    A matching $\mu: D\cup H \rightarrow D\cup H$ is \textbf{stable} if it is
    \begin{enumerate}[$\circ$]
        \item Individually Rational: $\nexists$ $i$ for whom $i>\mu(i)$.
        \item (Pairwise) Unblocked: $\nexists$ $(d,h)$ such that $d\succ_h \mu(h)$ and $h\succ_d \mu(d)$.
    \end{enumerate}
\end{definition}

\begin{theorem}[Gale-Shapley, 1962]
    For any matching market, a stable matching $\mu$ exists.
\end{theorem}
\begin{proof}
    We prove it by an algorithm:
    \begin{definition}[Deferred Acceptance Algorithm (DA)]
        \normalfont
        At each round, every doctor applies for his most preferred hospital among those haven't rejected him. Each hospital chooses its most preferred doctors among its applicants and the one on the previous waitlist, and then rejects others.
    \end{definition}
    Observation: DA terminates $\mu$. We want to prove
    \begin{enumerate}
        \item $\mu$ is IR (obviously);
        \item $\mu$ is unblocked.
        \subitem Suppose there is a block $(d,h)$ such that $d\succ_h \mu(h)$ and $h\succ_d \mu(d)$. That is impossible, because the $d\neq \mu(h)$, the $d$ must be rejected by $h$, which means $h\preceq_d \mu(d)$.
    \end{enumerate}
\end{proof}

\begin{note}
    We call "$h$ is \textbf{achievable} for $d$" if $\mu(d)=h$ for some stable matching $\mu$.
\end{note}


\subsection{Matching Markets: One-to-One}
\begin{definition}[$D$-Optimal Matching]
    \normalfont
    A matching $\mu: D\cup H \rightarrow D\cup H$ is \textbf{$D$-optimal}, denoted by $\mu^D$, if for any stable $\mu'$ we have that $\mu^D\succeq_D \mu'$ (the best stable matching for all doctors).
\end{definition}

\begin{theorem}[Deferred Acceptance Algorithm $\Rightarrow$ $D$-Optimal Matching]
    Deferred Acceptance Algorithm (with D proposing) terminates in $\mu^D$.
\end{theorem}
\begin{proof}
    %Suppose $d$ proposes to some $h$.
    %\begin{enumerate}
        %\item If $d$ is unacceptable ($d$ is below $\{h\}$) in $h$'s ranking, then $h$ is unachievable anyway.
        %\item Suppose $\exists d'\succ_h d$. If $h$ is achievable for $d'$, we have $h\succ_{d'} h'$
    %\end{enumerate}
    ...Theorem 2.12 (Gale and Shapley)
\end{proof}

\begin{theorem}[Lone-Wolf Theorem]
    The set of matched agent is identical in every stable $\mu$.
\end{theorem}
\begin{proof}
    $|\mu^D(H)|\geq |\mu(H)|\geq |\mu^H(H)|$; by symmetry, $|\mu^H(D)|\geq |\mu(D)|\geq |\mu^D(D)|$. Because $|\mu^D(H)|=|\mu^D(D)|$ and $|\mu^H(H)|=|\mu^H(D)|$ by one-to-one, so everything is equal.
\end{proof}

\subsection{Joint and Meet}
\begin{definition}[Joint and Meet]
    \normalfont
    \begin{enumerate}
        \item \textbf{Join $\mu \vee_D \mu'$} assign the more preferred match to every $d$ and the less preferred match to every $h$, that is,
        \begin{equation}
            \begin{aligned}
                \mu \vee_D \mu'(d)=\left\{\begin{matrix}
                    \mu(d),&\textnormal{ if }\mu(d)>_d\mu'(d)\\
                    \mu'(d),&\textnormal{ otherwise}
                \end{matrix}\right., \forall d\in D
            \end{aligned}
            \nonumber
        \end{equation}
        \begin{equation}
            \begin{aligned}
                \mu \vee_D \mu'(h)=\left\{\begin{matrix}
                    \mu(h),&\textnormal{ if }\mu(h)<_h\mu'(h)\\
                    \mu'(h),&\textnormal{ otherwise}
                \end{matrix}\right., \forall h\in H
            \end{aligned}
            \nonumber
        \end{equation}
        \item \textbf{Meet $\mu\wedge_D\mu'$} assign the less preferred match to every $d$ and the more preferred match to every $h$, that is,
        \begin{equation}
            \begin{aligned}
                \mu \wedge_D \mu'(d)=\left\{\begin{matrix}
                    \mu(d),&\textnormal{ if }\mu(d)<_d\mu'(d)\\
                    \mu'(d),&\textnormal{ otherwise}
                \end{matrix}\right., \forall d\in D
            \end{aligned}
            \nonumber
        \end{equation}
        \begin{equation}
            \begin{aligned}
                \mu \wedge_D \mu'(h)=\left\{\begin{matrix}
                    \mu(h),&\textnormal{ if }\mu(h)>_h\mu'(h)\\
                    \mu'(h),&\textnormal{ otherwise}
                \end{matrix}\right., \forall h\in H
            \end{aligned}
            \nonumber
        \end{equation}
    \end{enumerate}
\end{definition}

\begin{theorem}[Join and Meet of Stable Matchings are Stable]
    If $\mu$ and $\mu'$ are stable, then $\mu\vee_D\mu'$ and $\mu\wedge_D\mu'$ are stable.
\end{theorem}

\subsection{Strategic Incentives}
\begin{enumerate}[$\circ$]
    \item Type $=$ preference list.
    \item SCF: $f: \Theta \rightarrow \mathcal{M}$, where $\mathcal{M}$ is a set of stable matchings;
    \item Is $f$ strategy-proof?
    \item Does there exist a stable and strategy-proof (direct) mechanism?
\end{enumerate}

\begin{definition}
    \normalfont
    We say a mechanism $\varphi$ is strategy-proof (SP) if $\varphi(\succ_i,\succ_{-i})\geq \varphi (\succ'_i,\succ_{-i})$ for all $i\in I$ and $\succ'_i$ and $\succ_{-i}$.
\end{definition}

\begin{theorem}[Impossibility theorem (Roth)]
    There is no stable and strategy-proof (SP) mechanism.
\end{theorem}

The mechanism that yields the D-optimal stable matching (in terms of the stated preferences) makes it a dominant strategy for each doctor to state his true preferences. (Similarly, the mechanism that yields the H-optimal stable matching makes it a dominant strategy for every hospital to state its true preferences.)
\begin{theorem}[Dubins and Freedman; Roth]
    The doctor($D$)-optimal stable mechanism is strategy-proof for doctors.
\end{theorem}
\begin{proof}
    %Suppose under truthful $\succ$ (all doctors), a doctor $d$ has $\mu(d)=h$. $d$ changes his report to $\succ'_d$ such that $\mu'(d)=h'\succ_d h$.\\
    %Consider $\succ''_d$ which $\succ'_d$ truncated below $h'$.\\
    %Now, run a doctor-proposing DA (conside $\mu^D$) under $\succ''_d$. $d$ is unmatched.
\end{proof}


\section{Matching Many-to-Many}
Contracts are denoted by $x\in X$, $x_D\in D$, $x_H\in H$. $F\triangleq D\cup H$.

Consider a set of contracts $Y\subseteq X$,
\begin{enumerate}[$\circ$]
    \item $Y_D$ = doctors listed in $Y$;
    \item $Y_d$ = the contract in $Y$ that list the doctor $d$;
    \item $\succ_d$ over set of contracts that name the doctor $d$;
    \item The set of contracts $f\in F$ chooses from $Y$: $C_f(Y)=\max_{\succ_f}\{Z\subseteq X:Z\subseteq Y_f\}\subseteq Y_f$;
    \item The set of contracts doctors choose from $Y$: $C_D(Y)=\cup_{d\in D}C_d(Y)$.
    \item The set of contracts doctors reject from $Y$: $R_D(Y)=Y\backslash C_D(Y)$.
\end{enumerate}
The outcome of matching is $Y\subseteq X$.

\begin{definition}[Stable Contracts]
    \normalfont
    $A\subseteq X$ is \textbf{stable} if
    \begin{enumerate}[$\circ$]
        \item Individually Rational (IR): for all $f\in F$: $C_f(A)=A_f$;
        \item Unblocked: $\nexists$ non-empty $Z\subseteq X$ such that $Z\cap A=\emptyset $ and for all $f\in F$, $Z_f\subseteq C_f(A\cup Z)$.
    \end{enumerate}
\end{definition}

\begin{example}
    Preferences over doctor $d$: $\{x,y\}>\{x\}>\emptyset>\{y\}$; Preferences over hospital $h$: $\{y\}>\{x,y\}>\{x\}>\emptyset$.\\
    $\{x\}$ $\Rightarrow$ $\{x,y\}$ $\Rightarrow$ $\{y\}$ $\Rightarrow$ $\emptyset$ $\Rightarrow$ $\{x\}$.
\end{example}

\begin{definition}[Substitutability Condition]
    \normalfont
    Preference of $f$ satisfies the \textbf{substitutability condition} if for all $Y\subseteq X$ and $x,z\in X\backslash Y$:
    $$z\notin C_f(Y\cup\{z\}) \Rightarrow z\notin C_f(Y\cup\{z\}\cup\{x\})$$
    ($Y'\subseteq Y\subseteq X \Rightarrow R_f(Y')\subseteq R_f(X)$, where $R$ is the rejection choice.)
\end{definition}
If $z$ is rejected given a set, then it should also be rejected given a larger set.


\begin{theorem}
    If contracts are substitutes, then $Y\subseteq X$ is stable \underline{if and only if} pairwise stable.
\end{theorem}
\begin{proof}
    Prove $\Leftarrow$:
    (If not pairwise stable $\Rightarrow$ not stable)\\
    Suppose that $Z$ is a block. So, $Z\subseteq C_f(A\cup Z)$ for all $f$ listed in $Z$.\\
    We can pick a $z\in Z$ such that $z\in C_f(A\cup Z)$. By the substitutability condition, $z\in C_f(A\cup \{z\})$. So, it is stable.
\end{proof}

\begin{theorem}
    If contracts are substitutes, then a stable outcome exists.
\end{theorem}

\begin{definition}[Lattice]
    \normalfont
    On a \textbf{lattice}, $L=(X,<,\wedge,\vee)$ (or we just use $L=(X,<)$), $<$ is a partial order on $X$ in such a way that any two elements $x$ and $y$ of $X$ have a unique greatest lower bound (glb) $x \wedge y$ (meet) and a unique lowest upper bound (lub) $x \vee y$ (join).
\end{definition}



\begin{definition}[Complete Lattice]
    \normalfont
    A lattice $L=(X,<)$ is \textbf{complete} if there are both a meet (i.e. a greatest lower bound) and a join (i.e. a least upper bound) for any subset $Y\subseteq X$.\\
    These generalized meet and join operations on $Y$ are denoted by $\wedge Y$ and $\vee Y$.
\end{definition}

\begin{definition}[Monotone Function over Lattice]
    \normalfont
    A function from one lattice to another lattice $f:(X,<) \rightarrow (X',<')$ is \textbf{monotone} if $x\leq y \Rightarrow f(x)\leq' f(y)$ for any $x,y\in X$.
\end{definition}

\begin{theorem}[Tarski 1955]
    Let $L=(X,<)$ be a complete lattice and $f: L \rightarrow L$ be monotone ($\leq$) function on $L$. Then, the set $\{x\in L: f(x)=x\}$ of fixed points is a non-empty, complete lattice with order $\leq$.
\end{theorem}
\begin{proof}
    Fleiner, T. (2003). A fixed-point approach to stable matchings and some applications. \textit{Mathematics of Operations research}, 28(1), 103-126.
\end{proof}


%Given sets of contracts $X^D$ and $X^H$.

%Operator $f(X^D,X^H)$ produces $\left(X\backslash R_H(X^H), X\backslash R_D(X^D)\right)$. Fixed point: $\left\{\begin{matrix}
    %X^D=&X\backslash R_H(X^H)\\
    %X^H=&X\backslash R_D(X^D)
%\end{matrix}\right.$. $X^D\cap X^H=A$ is stable.\\
%(Find the intercection of contracts that are not rejected in both $X^D$ and $X^H$, which is stable.)




%Operator $g(X^D,X^H)$ produces $g_H(X^H)=\{x\in X: x\in C_H(X^H\cup\{x\})\}$ and $g_D(X^D)=\{x\in X: x\in C_D(X^D\cup\{x\})\}$. Fixed point: $\left\{\begin{matrix}
    %X^D=&g_H(X^H)\\
    %X^H=&g_D(X^D)
%\end{matrix}\right.$. $X^D\cap X^H=A$ is stable.\\
%(Find the intercection of contracts that are accepted both $X^D$ and $X^H$, which is stable.)


%Define partial order, $(X^D,X^H)\geq (\bar{X}^D,\bar{X}^H)$ if $X^D\subseteq \bar{X}^D$ and $X^H \supseteq  \bar{X}^H$.

%If $(X^D,X^H)\geq (\bar{X}^D,\bar{X}^H)$, then $g(X^D,X^H)\geq g(\bar{X}^D,\bar{X}^H)$


%Check if $X^D\subseteq \bar{X}^D$, then $g(X^D)\supseteq g(\bar{X}^D)$, by substitutability.


%Prove $X^D\cap X^H=A$ is stable:\\
%Given the claim $C_D(X^D)=A$ (prove later)
%\begin{enumerate}[1).]
    %\item IR: $C_D(X^D)=A$ by $C(A)=A$;
    %\item Unblocked: $z\in X\backslash A$ that blocks. Then $z\notin X^H \Rightarrow z\in C_D(A\cup\{z\})$ and $z\notin C_D(X^D\cup\{z\})$, but $C_D(X^D)=A \Rightarrow z\notin C_D(A\cup\{z\})$.
%\end{enumerate}


If some contracts are not substitute, there are no stable outcomes exist.

\section{Matching Many-to-One}
\underline{Settings}
\begin{enumerate}[$\circ$]
    \item Doctors, $D$; Hospitals, $H$; Contracts $X=D\times H\times \textnormal{terms}$;
    \item Hospitals preference $\succ_h$ over $2^X$;
    \item Doctors preference $\succ_d$ over $X$ (compare one contract with another one contract, not compare over sets of contracts);
    \item Outcome is $Y\subseteq X$ s.t. $|Y_d|\leq 1$ for all $d\in D$ (a doctor signs at most one contract).
\end{enumerate}

What restriction do we need to have a stable matching? Not as strong as substitute.

\begin{corollary}
    Doctor-proposing DA algorithm produces a doctor-optimal stable matching.
\end{corollary}

\begin{example} The preferences of agents are
    \begin{enumerate}[$\circ$]
        \item $d_1: h_1\succ h_2$; $d_2: h_1\succ h_2$; $d_3: h_2\succ h_1$;
        \item $h_1: d_3\succ d_1,d_2\succ d_1\succ d_2$; $h_2: d_1\succ d_2\succ d_3$.
    \end{enumerate}
    There are two stable outcomes
    \begin{enumerate}
        \item $(d_1,h_2)$, $(d_3,h_1)$;
        \item $(d_1,h_1), (d_2,h_1), (d_3,h_2)$.
    \end{enumerate}
    \begin{remark}
        Lone-Wolf Theorem doesn't hold.
    \end{remark}

    Assume the $d_2$'s true preference is $h_2\succ h_1$ and he reveals it, there is only one stable matching: $(d_1,h_2)$, $(d_3,h_1)$. So, the $d_2$ may benefit from lying.\\
    \begin{remark}
        Strategy-proof doesn't hold.
    \end{remark}
\end{example}

\begin{definition}[Law of Aggagate Demand/ Cardianlity Monotomicity (CM)]
    \normalfont
    For $h$, $Y\subseteq Y'\subseteq X \Rightarrow |C_h(Y)|\leq |C_h(Y')|$
\end{definition}

\begin{theorem}
    Under substitutes and CM, doctor-proposing DA is strategy-proof and LWT holds.
\end{theorem}

\begin{theorem}[Rural Hosptial Theorem]
    Under substitutes / CM, hospitals have same numbers of contracts in every stable outcome.
\end{theorem}

\subsubsection*{Cadets-branch matching}
Can be found in:
\begin{enumerate}[$\circ$]
    \item Jagadeesan, R. (2019). Cadet-branch matching in a Kelso-Crawford economy. \textit{American Economic Journal: Microeconomics}, 11(3), 191-224.
\end{enumerate}


\begin{remark}
    Contracts are not substitutes.
\end{remark}

\begin{definition}[Unilateral Substitute]
    \normalfont
    Contracts are \textbf{unilateral substitutes} if for all $z,x\in X$ and $Y\subseteq X$ \underline{such that $z_D\notin Y_D$} if $z\notin C_h(Y\cup\{z\}) \Rightarrow z\notin C_h(Y\cup\{z\}\cup\{x\})$
\end{definition}

\begin{remark}
    Preferences of branches satisfying unilateral substitute.
\end{remark}

\begin{remark}
    The outcome of doctor-proposing DA algorithm is doctor-optimal and stable.
\end{remark}

\section{Networks}
Based on
\begin{enumerate}[$\circ$]
    \item Fleiner, T., Jankó, Z., Tamura, A., \& Teytelboym, A. (2015). Trading networks with bilateral contracts. arXiv preprint arXiv:1510.01210.
    \item Fleiner, T., Jankó, Z., Schlotter, I., \& Teytelboym, A. (2023). Complexity of stability in trading networks. \textit{International Journal of Game Theory}, 1-20.
\end{enumerate}

Considering a trading network represented by a directed graph, where nodes are firms $F$ and edges $X$ are contracts (income arrow can be understood as buying products and outcome arrow can be understood as selling products).

The choice function of $f\in F$ is represented by $C^f$, the choice of $f$ over $Y_f\subseteq X_f$ is $C^f(Y_f)\subseteq Y_f$, where $X_f$ is the set of contracts involving $f$.

The choice sets of buyer side (B) and seller side (S) are defined as
\begin{equation}
    \begin{aligned}
        C_B^f(Y|Z)&\triangleq C^f(Y_f^B\cup Z_f^S)\cap X_f^B\\
        C_S^f(Z|Y)&\triangleq C^f(Z_f^S\cup Y_f^B)\cap X_f^S
    \end{aligned}
    \nonumber
\end{equation}
where $Y$ is the contracts from buyer side and $Z$ is the contratcts from seller side.


\begin{definition}[Irrelevance of Rejected Contracts]
    \normalfont
    Irrelevance of Rejected Contracts (IRC): $C(A)\subseteq B\subseteq A \Rightarrow C(A)=C(B)$
\end{definition}

\begin{definition}[Fully Substitute]
    \normalfont
    $C^f$ is \textbf{fully substitute} if for $Y'\subseteq Y\subseteq X$ and $Z'\subseteq Z\subseteq X$,
    \begin{equation}
        \begin{aligned}
            R_B^f(Y'|Z)\subseteq R_B^f(Y|Z)\\
            R_S^f(Z'|Y)\subseteq R_S^f(Z|Y)
        \end{aligned}
        \nonumber
    \end{equation}
    and
    \begin{equation}
        \begin{aligned}
            R_B^f(Y|Z)\subseteq R_B^f(Y|Z')\\
            R_S^f(Z|Y)\subseteq R_S^f(Z|Y')
        \end{aligned}
        \nonumber
    \end{equation}
\end{definition}
Define partial order, $(Y,Z)\geq (Y',Z')$ if $Y\subseteq Y'$ and $Z\supseteq  Z'$.


\begin{definition}[Stable Outcome, Hatfield and Kominers (2012)]
    \normalfont
    An outcome $A\subseteq X$ is stable if it is
    \begin{enumerate}
        \item Individual Rational: $\forall f\in F$, $C^f (A_f)=A_f$;
        \item Unblocked: there is no non-empty set $Z\subseteq X$ s.t. $Z\cap A=\emptyset$ and $\forall f\in F(Z)$, $Z_f\subseteq C^f(A\cup Z)$, where $F(Z)$ is the set of the firms are lined to $Z$.
    \end{enumerate}
\end{definition}

\begin{definition}[Trail]
    \normalfont
    \textbf{Trail} is the set of distinct edges $T=(X^1,X^2,...,X^M)$ such that the buyer side (the firm who is the buyer in the edge) of $X^i$ is exactly the seller side (the firm who is the seller in the edge) of $X^{i+1}$, which is denoted by $b(X^i)=s(X^{i+1})$, $i=1,...,M-1$.
\end{definition}


\begin{definition}[Trail-stable Outcome]
    \normalfont
    An outcome $A\subseteq X$ is \textbf{trail-stable} if its is
    \begin{enumerate}
        \item Individual Rational;
        \item There is no locally blocking trail $T=(X^1,X^2,...,X^M)$ such that
        \subitem $X^1\in C^{S(X^1)}(A\cup X^1)$;
        \subitem $\{X^i,X^{i+1}\}\in C^{b(X^{i})}(A\cup X^i\cup X^{i+1})$;
        \subitem $X^M\in C^{b(X^M)}(A\cup X^M)$.
    \end{enumerate}
\end{definition}

\begin{theorem}[Fleiner et al. 2016]
    If $C^f$ is fully substitute and IRC for all $f\in F$, then a trail-stable outcome exists.
\end{theorem}
\begin{proof}
    $Y\subseteq X$ and $Z\subseteq X$,
    \begin{equation}
        \begin{aligned}
            \Phi (Y,Z)=\left(X\backslash R_S(Z|Y), X\backslash R_B(Y|Z)\right)
        \end{aligned}
        \nonumber
    \end{equation}
    where $R_B(Y|Z)=\cup_{f\in F}R_B^f(Y|Z)$.
    \begin{claim}
        If $(Y,Z)$ is a fixed point of $\Phi$, then $A=Y\cap Z$ is trail-stable outcome.
    \end{claim}
    \begin{lemma}
        $C^f$ is fully substitute and IRC, and $(Y,Z)$ such that $Y \cap Z=A$, $C_S(Z|Y)=A$, $C_B(Y|Z)=A$. Then, for a contract $x\in X\backslash A$ and $A\subseteq A'\subseteq X$ if $C_S^{S(x)}(A\cup x|A')$ then $x\in C_S^{S(x)}(Z\cup x|A')$.
    \end{lemma}
    $\Phi$ will be monotone for the partial order $\geq$. As $(Y,Z)\geq (Y',Z')$, then $\Phi(Y,Z)\geq \Phi (Y',Z')$. Using Tarski fixed-point theorem, there is a $(Y,Z)$ fixed point.
    .....


    \textbf{Read} \textnormal{Fleiner, T., Jankó, Z., Tamura, A., \& Teytelboym, A. (2015). Trading networks with bilateral contracts. arXiv preprint arXiv:1510.01210.}
\end{proof}


\begin{proposition}
    $A$ is trail-stable $\Rightarrow$ $\exists$ $(Y,Z)$ such that $Y\cap Z=A$ and $(Y,Z)$ is a fixed point of $\Phi$.
\end{proposition}


\section{Corporate Game Theory}
There is a set of players $N=\{1,...,n\}$. The subset of players $S\subseteq N$ is called coalition.

There is a value function about coalition $v: 2^N \rightarrow \mathbb{R}$, which assumes $v(N)\geq \max_{S\subseteq N}v(S)$.

\begin{definition}[Cooperative Game]
    \normalfont
    A cooperative game is described by the pair $\left<N,v\right>$.
\end{definition}

\begin{definition}[Transferable Utility]
    \normalfont
    Utility is transferable if one player can losslessly transfer part of its utility to another player.
\end{definition}
Assume a TU (transferable utility) Economy. Consider a payoffs vector for all players, $x\in \mathbb{R}^n$. The efficiency requires $\sum_{i\in N}x_i=v(N)$. Individual Rational (IR) requires $x_i\geq v(\{i\})$.

\subsection{Core of Corporate Game and Farkas' lemma}
\begin{definition}[Core]
    \normalfont
    The \textbf{core} is the set of feasible allocations where no coalition of agents can benefit by breaking away from the grand coalition.
    $$C(v,N)=\left\{x\in \mathbb{R}^n: \sum_{i\in N}x_i=v(N), \sum_{i\in S}x_i\geq v(S), \forall S\subseteq N\right\}$$
\end{definition}

\begin{theorem}[Bondareva-Shapley Theorem]\ref{BST}
    The core of $\left<N,v\right>$ is non-empty ($C(v,N)\neq \emptyset$) \underline{if and only if} for every function $\alpha: 2^N\backslash\{\emptyset\} \rightarrow [0,1]$ where $\forall i\in N: \sum_{S:i\in S}\alpha(S)=1$, the following condition holds:
    \begin{equation}
        \begin{aligned}
            \sum_{S\in 2^N\backslash\{\emptyset\}}\alpha(S) v(S)\leq v(N)
        \end{aligned}
        \nonumber
    \end{equation}
\end{theorem}
Consider $B(N)$ be the solutions to: $\left\{\begin{matrix}
    \sum_{S:i\in S}y_S=1,&\forall i\in N\\
    y_S\geq 0,& \forall S\subseteq N
\end{matrix}\right.$

\begin{lemma}[Farkas' lemma]
    Let $A\in \mathbb{R}^{m\times n}$ and $b\in \mathbb{R}^m$. Then, \textbf{exactly one} of the following statement is true
    \begin{enumerate}[(1).]
        \item There exists $x\in \mathbb{R}^n$ such that $Ax=b$ and $x\geq 0$
        \item There exists $y\in \mathbb{R}^n$ such that $A^Ty\geq 0$ and $b^T y<0$.
    \end{enumerate}
\end{lemma}

\begin{lemma}

\end{lemma}

\begin{proof}
    \begin{lemma}[(Alternative) Farkas' lemma]
        Let $A$ be $m\times n$ matrix, $b\in \mathbb{R}^m$ and $F=\{x\in \mathbb{R}^n: Ax\geq b,x\geq 0\}$. Then, either $Cx=d$ or $\exists z$ such that for $y_S\geq 0$, $C^Tz-A^Ty_S=0$ and such that $d^Tz-b^Ty_S<0$, but not both.
    \end{lemma}
    By using this lemma, we can conclude $\left\{\begin{matrix}
        v(N)z-\sum_S v(S)y_S<0\\
        z-\sum{y_S}=0\\
        y_S\geq 0
    \end{matrix}\right.$ must hold at the same time, (let $z=1$, the last two lines are $B(N)$).

    Hence, $\forall y_S\in B(N)$, we have $v(N)\geq \sum_S v(S)y_S$.
\end{proof}


\subsection{Doubly stochastic matrix and Birkhoff-von Neumann Theorem}

Consider a matching game between sellers and buyers: $v(\{i,j\})=v_{ij}$, $v(\{i\})=0$ for buyer $i$ and $v(\{j\})\geq 0$ for seller $j$.

\underline{Core:}
\begin{equation}
    \begin{aligned}
        \max_{\alpha}\quad &\sum_i\sum_j v_{ij}\alpha_{ij}\\
        \textnormal{s.t.}\quad&\sum_{i}\alpha_{ij}=1,\forall j\\
        &\sum_{j}\alpha_{ij}=1,\forall i\\
        &\alpha_{ij}\geq 0
    \end{aligned}
    \nonumber
\end{equation}

\begin{definition}[Doubly Stochastic Matrix]
    \normalfont
    A \textbf{doubly stochastic matrix} is a square matrix $X=(x_{ij})$ of non-negative real numbers, each of whose rows and columns sums to $1$.\\
    The class of $n\times n$ doubly stochastic matrices is a convex polytope (convex set in euclidean space) known as the \textbf{Birkhoff polytope}.
\end{definition}

\begin{theorem}[Birkhoff-von Neumann Theorem]
    A matrix is doubly stochastic if and only if it is a convex combination of permutation matrices.
\end{theorem}
By this theorem, we can set efficient "integer" assignment.

Can the efficient allocation be competitive equilibrium (CE)?
\begin{theorem}
    The core of assignment of game is non-empty.
\end{theorem}
\begin{proof}
    The duality of core can be written as
    \begin{equation}
        \begin{aligned}
            \min \quad&  \sum_{j}u_j^S + \sum_{i}u_i^B\\
            \textnormal{s.t.}\quad& u_j^S+u_i^B\geq v_{ij}, \forall i,j
        \end{aligned}
        \nonumber
    \end{equation}
    By strong duality, the minimum value should be equal to $V(N)$.

    Hence, $\sum_{j\in T}u_j^S + \sum_{i\in T}u_i^B\geq V(T)$ for a subset $T\subseteq N$. That is, the core is non-empty.
\end{proof}

\begin{corollary}
    For an assignment game, outcome is in the core \underline{if and only if} the outcome is CE outcome.
\end{corollary}



\section{Constrained Demand Theory}
\subsection{Substitutes Valuation}
There are buyers $i\in N$ and goods $j\in J$ with quantities $S\in \mathbb{Z}^J$ sold by a seller.

A buyer's utility is $v(x)-p\cdot x$, where $v(0)=0$, $p\in \mathbb{R}^J$, and $x\in \{0,1\}^J$. The buyer's demand is represented by $D(p)\argmax_{x}\left\{v(x)-p\cdot x\right\}$.

The competitive equilibrium $\left(p^*,(x^{*i})_{i\in N}\right)$ here are
\begin{enumerate}
    \item $x^{*i}\in D^i(p^*)$ for every $i\in N$ and
    \item $\sum_{i}x^{*i}\leq S_i$, where the equality holds for $p_i>0$.
\end{enumerate}

\begin{definition}[Substitutes Valuation]
    \normalfont
    A valuation $v_i$ is a \textbf{substitutes valuation} if $\forall p: p'=p+\lambda e^j$ ($\lambda>0$), where $D^i(p)=\{x\}$ and $D^i(p')=\{x'\}$, we have that $x'_k\geq x_k$ for all $k\neq j$. (The increase of product $j$'s price increases other product's demand).
\end{definition}

\begin{theorem}[Substitutes Valuation $\Rightarrow$ Competitive Equilibrium Exists]\label{thm:substitute}
    If agents have substitutes valuations, then a competitive equilibrium exists.
\end{theorem}

\begin{theorem}
    If there exists an agent without substitutes valuation, then we can construct \underline{unit-demand preferences} for other agents such that no competitive equilibrium exists.
\end{theorem}


\subsection{Income Effect}
There are buyers $i\in N$ and goods $j\in J$. The endowments (money and goods) of agents are denoted by $w=(w_0,w_I)$.

\underline{Outcome:} The indivisible (bought) goods is represented by $x_I\in\{0,1\}^J$ and the (left) divisible money is represented by $x_0\in(\underline{m},\infty)$. $$w_0=x_0+p_I\cdot x_I$$ must hold, where $p_I$ is the vector of prices of goods.

\underline{Utility Function:} An agent's utility function is defined by $u^i:(\underline{m},\infty)\times \{0,1\}^J \rightarrow (-\infty, +\infty)$ with assumptions of strictly increasing in $x_0$, $\lim_{x_0 \rightarrow \underline{m}}u^i(x_0,x_I)=-\infty$, and $\lim_{x_0 \rightarrow \infty}u^i(x_0,x_I)=+\infty$.

\begin{example}
    Examples of feasible utility functions:
    \begin{enumerate}
        \item $u^i(x)=v(x)-p\cdot x$ with $\underline{m}=-\infty$;
        \item $u^i(x_0,x_I)=\log(x_0)-\log(-V_Q^i(x_I))$ with $V_Q^i:\{0,1\}^J \rightarrow (-\infty,0)$.
    \end{enumerate}
\end{example}

\underline{Demand:}
\begin{enumerate}[$\circ$]
    \item $D_\textnormal{Marshallian}^i(p,w)=\{x^*:x^*\in\arg\max_{x} u^i(x) \textnormal{ s.t. }p\cdot x\leq p\cdot w\}$
    \item $D_\textnormal{Hicksian}^i(p,u)=\{x^*:x^*\in\arg\min_{x} p\cdot x \textnormal{ s.t. }u^i(x)\geq u\}$ which is the dual of $D_\textnormal{Marshallian}^i$.
\end{enumerate}
\begin{definition}[Competitive Equilibrium]
    \normalfont
    Given $(w^i)_{i\in I}$ s.t. $\sum_{i\in N}w_I^i=y_I$. A \textbf{competitive equilibrium} is a price vector $p_I^*\in \mathbb{R}^J$ and ${x_I^i}^*\in D_\textnormal{Marshallian}(p_I^*,w^i)$ for each $i\in N$ such that $\sum_{i\in N}{x_I^i}^*=y_I$.
\end{definition}

Based on the idea of duality, we can analyze problem based on the dual demand, Hicksian demand.
\begin{definition}[Hicksian Valuation]
    \normalfont
    Hicksian valuation is defined by $-1$ times "the money that can lead to the utility $u$ with goods $x_I$": $$V_\textnormal{Hicksian}^i(x_I,u)=-(u^i(\cdot,x_I))^{-1}(u)$$
\end{definition}


\begin{proposition}[Using Hicksian Valuation to Represent Hicksian Demand]
    $D_\textnormal{Hicksian}^i(p_I,u)=\arg\max_{x_I}\left\{v_\textnormal{Hicksian}^i(x_I,u)-p_I\cdot x_I\right\}$
\end{proposition}
\begin{proof}
    $D_\textnormal{Hicksian}^i(p_I,u)=\arg\min_{x_I}\{(u^i(\cdot,x_I))^{-1}(u)+p_I\cdot x_I\}=\arg\max_{x_I}\left\{V_\textnormal{Hicksian}^i(x_I,u)-p_I\cdot x_I\right\}$
\end{proof}


\begin{definition}[Hicksian Economy]
    \normalfont
    Hicksian economy: for a profile $(u^i)_{i\in N}$ is a transferable utility (TU) economy in which each agent's "valuation" is a Hicksian valuation $V_\textnormal{Hicksian}^i$.
\end{definition}
Hicksian Economy works in finding Competitive Equilibrium
\begin{theorem}[Equilibrium Existence Duality(EED)]\label{EED}
    Competitive Equilibrium exists for all feasible endowment profiles \underline{if and only if} Competitive Equilibrium exists in the Hicksian economies for all profiles of utility levels.
\end{theorem}

\begin{center}
    \begin{tabular}{ccc}
        \hline
            Marshallian& Hicksian\\
        \hline
            Housing Market & Assignment Game\\
            Utility is not Quasi-linear & Utility is Quasi-linear\\
            Unit Demand& Unit Demand\\
            Existence in Housing Market& Existence in Assignment Game\\
            $\times$& Lattice structure and Convexity of structure of CE prices\\
            Net-substitutes& $\Rightarrow$ Substitutes\\
        \hline
    \end{tabular}
\end{center}

Like the Theorem \ref{thm:substitute}, we want the Hicksian valuations be "substitutes".
\begin{definition}[Net-Substitutes]
    \normalfont
    A agent's utility $u^i$ is \underline{net-substitutes} if $\forall u$, $D^i_H(p;u)=\{x\}$ and $D^i_H(p'_j,p_{-j};u)=\{x'\}$, $p'_j>p_j \Rightarrow x'_k\geq x_k$ for all $k\neq j$.
\end{definition}

\begin{theorem}
    Net-Substitutes Valuation $\Rightarrow$ competitive equilibrium exists.
\end{theorem}
\begin{proof}
    Net-substitutes $\Rightarrow$ substitutes holds in Hicksian economy. Hence, CE exists. By \ref{EED}, CE exists in original economy.
\end{proof}

\begin{definition}[Gross-Substitutes]
    \normalfont
    A agent's utility $u^i$ is \underline{gross-substitutes} if $\forall w$, $D^i_M(p;w)=\{x\}$ and $D^i_M(p'_j,p_{-j};w)=\{x'\}$, $p'_j>p_j \Rightarrow x'_k\geq x_k$ for all $k\neq j$.
\end{definition}

\begin{example}
    In quasi housing market, we consider an example, of holding a house which price increases, the demand of another bad house doesn't change under Hicksian demand, which makes net-substitutes hold. But, the Marshallian demand decreases, which makes gross-substitutes don't hold.
\end{example}

\begin{example}
\textbf{Net, but not gross}:\\
Suppose there is a firm $f$ thinking about workers $s_1,s_2$. $f$ values worker at $\$ 5$ each, and the hiring budget is $\$ 6$;
\begin{enumerate}[$\circ$]
    \item $p_1=2,p_2=4$;
    \item $p_1=3,p_2=4$
\end{enumerate}
Obviously, the gross-substitutes (Marshallian Demand) leads to hiring both under $p_1=2,p_2=4$ and only hiring $s_1$ under $p_1=3,p_2=4$.\\
Let's consider the net-substitutes (Hicksian Demand): As the utility given under $p_1=2,p_2=4$ is $\$ 10$. We can find hiring two workers is still the optimal strategy.
\end{example}


\begin{example}
    \textbf{Net, but no auction:}\\
    Suppose there are two identical firms $f_1,f_2$ and workers $s_1,s'_1,s_2$. The value of workers is $\$ 5$ each, but a firm want at most one of $s_1,s'_1$ and has hiring budget $\$ 6$. A worker has reservation wage of $\$ 1$.\\
    \underline{Equilibrium:} $\$1$ for worker $s_1,s'_1$ and $\$ 5$ for $s_2$; One firm hires one of $s_1,s'_1$ and the other hires $s_2$.
\end{example}


\section{Object Allocation}
Exchange: $i\in N$ agent; Agents have strict preference $\succ_i$ over objects. (We use $\succ$ denote $\{\succ_i\}_{i\in N}$).

\underline{Two settings:}
\begin{enumerate}
    \item Exchange: an agent shows up with exactly one object.
    \item Allocation: One planner owns $N$ objects; agents have $\emptyset$.
\end{enumerate}

A \textbf{mechanism} $\Phi(\succ)$ gives a outcome $\mu$.
We want the final outcome $\mu$ be
\begin{enumerate}
    \item Individual Rationality (IR): for all $i\in N$, $\mu_i\succeq i$ (Exchange) and $\mu_i\succeq \emptyset$ (Allocation).
    \item Pareto Efficient (PE): $\nexists \mu'$ such that $\mu'_i\succeq \mu_i$ for all $i\in N$, strict for at least one.
    \item Strategy-Proof (SP): $\Phi$ induces a game. We want that, in this game, truth-telling is a weakly dominant strategy for all agent $i\in N$.
\end{enumerate}

\subsection{Allocation}
(Random) Serial Dictatorship: Randomly order the agents, ask one by one, and allocate a remaining object. $\Rightarrow$ it satisfies IR, PE, SP, but \underline{unfair}(?).

\subsection{Exchange}
\begin{definition}[Core]
    \normalfont
    The \textbf{core} is the set of all allocations $\mu$ such that there is no $S\subseteq N$ and $\mu'$ for which:
    \begin{enumerate}[$\circ$]
        \item for $i\in S$, $\mu'_i=j$ for some $j\in S$
        \item $\mu'_i\succeq \mu_i$ for all $i\in S$, at least one strict.
    \end{enumerate}
    Core: IR+PE.
\end{definition}
\begin{theorem}[Core is a Singleton]
    There is a unique element in the core.
\end{theorem}
\begin{proof}
    Run the algorithm: Top Trading Cycles (TTC).
\end{proof}
\begin{definition}[Top Trading Cycles (TTC)]
    \normalfont
    Agent = node.
    \begin{enumerate}
        \item Step 1: every agent point at her favorite object/agent.
        \subitem (1A): Find cycles.
        \subitem (1B): Allocate object to agent who is pointing at it in cycle.
        \subitem (1C): Remove the cycle.
        \item Step 2: every (remaining) agent point at her favorite object/agent.
        \subitem (2A): Find cycles.
        \subitem (2B): Allocate object to agent who is pointing at it in cycle.
        \subitem (2C): Remove the cycle.
        \item Repeat $\cdots$
    \end{enumerate}
\end{definition}
\begin{proposition}
    TTC produces an allocation that satisfies IR, PE, SP.
\end{proposition}

\begin{theorem}[TTC $\Leftrightarrow$ IR, PE, SP (Ma, 1999)]
    There is at most $1$ IR, PE, SP mechanism (TTC).
\end{theorem}
\begin{proof}
    \begin{definition}
        \normalfont
        The \textbf{size} of a preference profile $\succ$ is the total number of objects agents find acceptable in $\succ$:
        \begin{equation}
            \begin{aligned}
                S(\succ)=\sum_{i\in N}\# \textnormal{acceptable objects in }\succ_i
            \end{aligned}
            \nonumber
        \end{equation}
    \end{definition}
    Consider two $\Phi$ and $\Psi$ that disagree for some $\succ$, the $\succ$ is defined to be \underline{bad}.\\
    We define the \underline{minimal bad profile} as a bad profile of minimal size.
    Consider the two outcomes given by these mechanisms:
    \begin{center}
        \begin{tabular}{ccc}
            \hline
                $\Phi(\succ)$&\textnormal{same} & $A(\Phi)$\\
            \hline
                $\Psi(\succ)$&\textnormal{same} & $A(\Psi)$\\
            \hline
        \end{tabular}
    \end{center}
    the sum of different parts are $A\triangleq A(\Phi)+A(\Psi)$.
    \begin{lemma}
        If $\Phi$ and $\Psi$ are SP, and $\succ$ is a minimal bad profile, then each agent in $A$ has exactly two acceptable objects.
    \end{lemma}
    \begin{proof}
        Suppose there exists $i\in A$ such that she has $>2$ acceptable objects.\\
        Without losing generality, we consider $\Phi_i(\succ)\succ_i\Psi_i(\succ)$.\\
        Remove all objects from his preference list except $\Phi_i(\succ)$ and endowment of $i$ (call it $\{i\}$). The new preference profile is denoted by $\succ'_i$.\\
        Since $\Phi$ is SP, $\Phi_i(\succ')=\Phi_i(\succ)$; since $\Psi$ is SP, $\Psi_i(\succ')\prec_i\Phi_i(\succ)$.\\
        So, we have $\succ'$ is a bad profile and $S(\succ')<S(\succ)$, a contradiction.
    \end{proof}
\end{proof}


\section{School Choice}
\underline{Model:}
\begin{enumerate}
    \item There is a set of school $S$; a school is denoted by $s\in S$; Quota for each $s$ is $q_s$;
    \item $I$ is the set of all students; A student is denoted by $i\in I$; Student $i$ has preference $\succ_i$.
    \item School places = objects.
    \item Each school has a priority order over students $\pi_s$.
    \item Matching $\mu: I \rightarrow S$ such that $\forall s\in S: \# \mu^{-1}(s)\leq q_s$.
    \item Matching violates priority if $\exists s\in S$ such that
    \begin{enumerate}[(i).]
        \item $s\succ_i\mu(i)$ and
        \item either ``Wastefulness: $\# \mu^{-1}(s)< q_s$'' or ``Justified Envy: $i\pi_s j$ for some $j\in \mu^{-1}(s)$''
    \end{enumerate}
    $\approx$ existence of a blocking pair.\\
    \underline{A matching is \textbf{stable}} if there are no priority violates.\\
    (As we don't consider the preference of $j$ in (ii), it is not true stable $\Rightarrow$ (Pareto) efficient.)
\end{enumerate}

\begin{example}
    Boston (Immediate Acceptance)
    \begin{enumerate}[(1).]
        \item Step 1: students apply for favorite schools; school accepts applicants up to capacity and reject rest permanently.
        \item Step k: students apply for favorite schools among those with capacity and hasn't already rejected them; schools accept applicants up to capacity $q_s$ and reject rest permanently.
    \end{enumerate}
\end{example}

\begin{proposition}
DA gives a matching that satisfies \underline{stability and SP} (not PE).
\end{proposition}

Run TTC:
\begin{definition}[Top Trading Cycles (TTC)]
    \normalfont
    Schools and Students (agents) = nodes.
    \begin{enumerate}
        \item Step 1: every agent point at her favorite object/agent.
        \subitem (1A): Find cycles.
        \subitem (1B): Allocate object (school) to agent (student) who is pointing at it in cycle. (Usually based on the students' preference.)
        \subitem (1C): Remove the cycle.
        \item Step 2: every (remaining) agent point at her favorite object/agent.
        \subitem (2A): Find cycles.
        \subitem (2B): Allocate object to agent who is pointing at it in cycle.
        \subitem (2C): Remove the cycle.
        \item Repeat $\cdots$
    \end{enumerate}
\end{definition}

\begin{proposition}
    TTC produces an allocation that satisfies \underline{PE and SP} (not stable).
\end{proposition}
Hence, we need to make a trade-off between priority violation and efficiency.

\begin{theorem}[Keslen]
    For all $S, \{q_s\}_{s\in S}$, there exists $I,\succ_i,\{\pi_s\}_{s\in S}$ s.t. in the SOSM, every student gets either their last choice or second-last choice.
\end{theorem}


\begin{theorem}[Abdulkadiroğlu, Pathak, Roth, AER]
    There is no (PE+)SP mechanism that Pareto-dominates SOSM.
\end{theorem}

\begin{theorem}
    There is no PE+SP mechanism that selects a PE+stable matching whenever it exists.
\end{theorem}

\begin{definition}[Kesten/Tang+Yu Algorithm]
    \normalfont
    Suppose the number of student is not larger than the total capacity $\# I\leq \sum_s q_s$.
    \begin{enumerate}[(i).]
        \item Step 0: Run DA, set SOSM $\mu_0$. Find under-demanded schools = a school that doesn't reject any students.\\
        Assign $\mu^{-1}(s)$ permanently. Call these schools/students ``settled''. Remove all settled schools and students.
        \item Step k: Rerun DA on everyone unsettled.
    \end{enumerate}
\end{definition}

\begin{definition}[Priority-Neutral(PN), Reny 2022]
    \normalfont
    $\mu$ is \textbf{priority-neutral}(PN) iff $\exists$ no matching $u$ that can make any student whose priority is violated at $\mu$ better off \underline{unless} $u$ violates the priority of some student and make them worse off.\\
    We call $\mu$ is \textbf{priority-efficient} if it is PN and PE.
\end{definition}

\begin{theorem}[Reny 2022]
    \begin{enumerate}
        \item $\exists$ a unique Priority-efficient matching;
        \item Priority efficient $\Leftrightarrow$ SO priority neutral matching;
        \item It can be found by the \underline{CUTE Algorithm};
        \item $\mu$ is priority efficient $\Leftrightarrow$ no matching $u$ can make \underline{any student better off} unless $u$ \underline{unless} $u$ violates the priority of some student and make them worse off.
    \end{enumerate}
\end{theorem}


\section{School Choice with Reserves}
Consider a school choice model, students can be divided into majority ($M$) and minority ($m$), $I=I^M\cup I^m$. Quotas of schools are represented by $q_s=(q,q^M), s\in S$, where $q^M$ is the quota for majority.

\begin{definition}[Stability]
    \normalfont
    A matching is stable if, for all $s\in S$ such that $s\succ_i \mu(i)$,
    \begin{enumerate}
        \item Either: ``No Wastefulness: $|\mu^{-1}(s)| = q_s$'' and ``No Justified Envy: $i'\pi_s i$ for all $i'\in \mu^{-1}(s)$''
        \item Or: $i\in I^M$, ``$|\mu^{-1}(s)\cap I^M| = q_s^M$'' and ``$i'\pi_s i$ for all $i'\in \mu^{-1}(s)\cap I^M$''
    \end{enumerate}
\end{definition}

\begin{definition}[Stronger Quota]
    \normalfont
    A $\tilde{setting}$ (with $\tilde{q}_s$) has \textbf{stronger quota} than setting (with $q_s$) if $\tilde{q}_s=q_s$ but $q_s^M\geq \tilde{q}_s^M$.
\end{definition}

\begin{definition}[Good Mechanism]
    \normalfont
    Mechanism $\Phi$ is \textbf{good}, if whenever a $\tilde{setting}$ has stronger quotas than its setting, it doesn't make all \underline{minority} students worse off.
\end{definition}


\begin{theorem}[Kojima 2012]
    There is no stable good mechanism.
\end{theorem}

\subsection{Minority Reserves (slot-specific priority)}
Suppose $r_s^m$ is reserved for minority only. That is $q_s=q_s^M+r_s^m$.
\begin{definition}[Minority Reserves]
    \normalfont
     School has minority reserve $r_s^m$ whenever $\#$ of admitted minority students is less than $r_s^m$, then any minority students is admitted ahead of majority students.
\end{definition}
\begin{definition}[No Blocking Pair]
    \normalfont
    \textbf{No blocking pair} if $s\succ_i \mu(i)$, then $|\mu(s)|=q_s$ and,
    \begin{enumerate}
        \item Either: $i\in I^m$ and ``$i'\pi_s i$ for all $i'\in \mu^{-1}(s)$''
        \item Or: $i\in I^M$, ``$|\mu^{-1}(s)\cap I^m| > r_s^m$'' and ``$i'\pi_s i$ for all $i'\in \mu^{-1}(s)$''
        \item Or: $i\in I^M$, ``$|\mu^{-1}(s)\cap I^m| \leq r_s^m$'' and ``$i'\pi_s i$ for all $i'\in \mu^{-1}(s)\cap I^M$''
    \end{enumerate}
\end{definition}

\begin{theorem}[Smart Reserves]
    Suppose $\mu$ is a stable matching without affirmative action. Let $r_s^m$ be such that $$r_s^m\geq |\mu^{-1}(s)\cap I^m|, \forall s\in S$$
    Then, either $\mu$ is stable w.r.t. $r^m$ or $\exists$ stable matching under $r^m$ that Pareto-dominates $\mu$ for $I^m$.
\end{theorem}


\section{Random Assignment}
Suppose there are agents $i\in I$ and objects $j\in J$, where $|I|=|J|$. Agents have preferences $\succ_i$ over objects, and objects have priorities $\rhd_j$ over agents.

An allocation is represented by a matrix that each row and each column has sum to $1$ probability.

There are two mechanism can be used:
\begin{enumerate}[(i).]
    \item RSD (Random: draw a priority order $\rhd$ uniformly.)
    \item TTC with uniform random endowment.
\end{enumerate}
\begin{theorem}
    These two mechanisms are equivalent (bijection).
\end{theorem}


RSD is not Pareto-efficiently.

\begin{proposition}
    For a row of an allocation matrix ($\tilde{\mu}$) for agent $i$, $\tilde{\mu}_i\succ_i\tilde{\mu}'_i$
    \begin{enumerate}[$\circ$]
        \item \underline{if and only if} $\tilde{\mu}_i\succ_{FOSD}\tilde{\mu}'_i$ (first-order stochastic dominance).
        \item \underline{if and only if} $\mathbb{E}U(\tilde{\mu}_i)\geq\mathbb{E}U(\tilde{\mu}'_i)$ under expected utility.
    \end{enumerate}
\end{proposition}

\begin{definition}
    \normalfont
    $\tilde{\mu}$ is \textbf{ordinally efficient (sd-efficient)} if there is no $\tilde{\mu}'$ which is $\succ_{FOSD}$ by all agents. (\textit{ex-ante efficient} with respect to cardinal utility)\\
    $\tilde{\mu}$ is \textbf{ex-post efficient} if those are only Pareto efficient outcome in the support.
\end{definition}

\begin{definition}
    \normalfont
    $\tilde{\mu}$ is \textbf{ordinally envy-free} if $\tilde{\mu}_i\succ_{FOSD}\tilde{\mu}_j, \forall i,j$.
\end{definition}
RSD is not envy-free.

There exists ordinally efficient and envy-free mechanism.
\begin{definition}[Probabilitistic Serial Algorithm]
    \normalfont
    Based on the preference of agents:
    \begin{enumerate}
        \item Give each agent his most preferred object with the same proportion such that the sum of each object is at most 1.
        \item Repeat by using remaining objects.
    \end{enumerate}
    \begin{example}
        Preference: A: $Obj1\succ Obj3\succ Obj2$; B: $Obj1\succ Obj2 \succ Obj3$; C: $Obj2\succ Obj3\succ Obj1$
        \begin{enumerate}
            \item [$t=\frac{1}{2}$] A: $\frac{1}{2} Obj 1$; B: $\frac{1}{2} Obj 1$; C: $\frac{1}{2} Obj 2$.
            \item [$t=\frac{3}{4}$] A:$ \frac{1}{2} Obj 1+\frac{1}{4} Obj 3$; B: $\frac{1}{2} Obj 1+\frac{1}{4} Obj 2$; C: $\frac{3}{4} Obj 2$.
            \item [$t=1$] :$ \frac{1}{2} Obj 1+\frac{1}{2} Obj 3$; B: $\frac{1}{2} Obj 1+\frac{1}{4} Obj 2+\frac{1}{4} Obj 3$; C: $\frac{3}{4} Obj 2+\frac{1}{4} Obj 3$.
        \end{enumerate}
    \end{example}
\end{definition}
\begin{theorem}[Welfare Theorem]
    Probabilistic Serial Algorithm gives ordinally efficient and envy-free outcome.
\end{theorem}

\begin{definition}[Equal Treatment of Equals (ETE)]
    \normalfont
    Equal Treatment of Equals: if same preference $\succ_i$ $\Rightarrow$ the same bundle $\tilde{\mu}_i$.
\end{definition}


\begin{proposition}
    For $n=3$, RSD is \textit{ordinally efficient, ETE, Strategy-Proof}. (These three properties are incompatible when $n>3$).
\end{proposition}

\section{Random Assignment in School Choice}
\begin{example}
    \begin{enumerate}[$\circ$]
        \item Preference of Agents: $A: s_2\succ s_3\succ s_1$; $B: s_2\succ s_3\succ s_1$; $C: s_1\succ s_2\succ s_3$.
        \item Priority of Schools: $s_1: A\succ B\succ C$, $s_2: C\succ (A,B)$, $s_3: C\succ B\succ A$.
    \end{enumerate}
    There are two stable outcomes: $\mu: A-s_2, B-s_3, C-s_1$; $\mu': A-s_3, B-s_2, C-s_1$.

    It can't be strategy proof. In $\mu$, $B$ can lie: $s_2\succ s_1\succ s_3$, to make the outcome become $\mu'$. In $\mu'$, $A$ can lie: $s_2\succ s_1\succ s_3$, to make the outcome become $\mu$.
\end{example}

\begin{definition}[Stable Imporovement Cycle (S.I.C.)]
    \normalfont
    Each student points at schools they prefer and where he doesn't have a lower priority among those students who prefer students to their assignment.
\end{definition}

\begin{theorem}
    If a stable matching is not in the student-optimal stable set, then $\exists$ a S.I.C.
\end{theorem}
\begin{example}
    \begin{enumerate}[$\circ$]
        \item Preference of Agents: $A: s_2\succ s_1\succ s_3$; $B: s_3\succ s_2\succ s_1$; $C: s_2\succ s_3\succ s_1$.
        \item Priority of Schools: $s_1: A\succ (B,C)$, $s_2: B\succ (A,C)$, $s_3: C\succ (A,B)$.
    \end{enumerate}
    DA: $A:s_1, B:s_2, C:s_3$.
    Another allocation: $A:s_1, B:s_3, C:s_2$.

    Consider DA, $A$ wants $s_2$: $C$ also wants $s_2$, which has the same priority as $A$, so $A$ can point at $s_2$. $B$ points at $s_3$. $C$ can also point at $s_2$. So, there is a S.I.C.
\end{example}


\section{Pseduomarket (FF)}
Consider an example that agent $A_1$ wants $a,b$ for $0.9$, $A_2$ wants $a,c$ for $0.9$, $A_3$ wants $b,c$ for $2$. Suppose the budget for each agent is $1$.

Reminds that utility is only meaningful for the agent itself. Here, as the budget is the same, the demand of each agent is the same.

\subsection{Problem of Implementability}
An equilibrium (but can't be implemented): $A_1$ gets $\{\frac{1}{2}:\emptyset; \frac{1}{2}:a+b\}$; $A_2$ gets $\{\frac{1}{2}:\emptyset; \frac{1}{2}:a+c\}$; $A_3$ gets $\{\frac{1}{2}:\emptyset; \frac{1}{2}:b+c\}$.


\begin{center}
    \begin{tabular}{ccc}
        \hline
            Transfer Utility Economy& Pseduomarket\\
        \hline
            Allocation $x_j\in X_j,j=1,...,J$& Lottery $\tilde{x}_j\in \mathcal{L}(X_j)$\\
            Price $p\in \mathbb{R}^I$& Budget $b_j$ and Price $p\in \mathbb{R}^I$\\
            $u_j(x)=v_j(x)-p\cdot x$& $V_j(\tilde{x}_j)=\sum_x v_j(x) \mathbb{P}(\tilde{x}_j=x)$\\
            Demand $D_j(p)=\arg\max_x u_j(x)$& $\tilde{D}_j(p)=\arg\max_{\tilde{x}:p\cdot\tilde{x}\leq b_j} V_j(\tilde{x})$\\
            CE: $(p^*,x^*): x_j^*\in D_j(p^*), \sum_{j}x_j^*\leq S$& RE: $(p^*,\tilde{x}^*): \tilde{x}_j^*\in \tilde{D}_j(p^*), \sum_{j}\tilde{x}_j^*\leq S$\\
            (equality holds for no zero $p^*$)\\
        \hline
    \end{tabular}\\
    $S$ is supply, which equals to $\sum_i \omega_i$ if the economy with endowments.
\end{center}

We want an allocation being implementable that an allocation (a set of lotteries over agents) $\{w_1,...,w_J\}=\mathcal{W}\in \mathcal{L}(\prod_{j}X_j)$ (feasible bundles for each agent).

Define $\bar{w}_j=\mathbb{E}[w_j]$ and $\bar{\mathcal{W}}=\mathbb{E}[\mathcal{W}]$

\begin{definition}[Implementable]
    \normalfont
    A random equilibrium $(p^*,\tilde{x}^*)$ is \textbf{implementable} if there exists $\mathcal{W}$ over feasible allocations such that $w_j\in D_j(p^*)$ and $\bar{x}^*_j=\bar{w}_j, \forall j=1,...,J$.
\end{definition}
can be implemented by a distribution of allocations. (BvN)

\begin{proposition}
    Random equilibrium always exists.
\end{proposition}

\begin{definition}[Rich]
    \normalfont
    A set of valuations $\mathcal{V}^j=\{v_j(x):x\in X_j\}$ is \textbf{rich} if whenever $v_j(x)\in \mathcal{V}^j$ then $v_j(x)+a\cdot x\in \mathcal{V}^j$ for all $a\in \mathbb{R}^I$. That is $\exists x'$ such that $v_j(x')=v_j(x)+a\cdot x$.
\end{definition}
Complement may induce unimplementable problem.

Suppose value functions live in $V$ and are \underline{rich}.
\begin{theorem}
    CE exists for all valuations in $V$ $\Leftrightarrow$ RE is implementable for all budgets profiles and all valuations in $V$.
\end{theorem}



\end{document}