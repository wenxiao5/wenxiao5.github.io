\documentclass[11pt]{elegantbook}
\definecolor{structurecolor}{RGB}{40,58,129}
\linespread{1.6}
\setlength{\footskip}{20pt}
\setlength{\parindent}{0pt}
\newcommand{\argmax}{\operatornamewithlimits{argmax}}
\newcommand{\argmin}{\operatornamewithlimits{argmin}}
\elegantnewtheorem{proof}{Proof}{}{Proof}
\elegantnewtheorem{claim}{Claim}{prostyle}{Claim}
\DeclareMathOperator{\col}{col}
\title{\textbf{ECON 201B}}
\author{Wenxiao Yang}
\institute{Haas School of Business, University of California Berkeley}
\date{2024}
\setcounter{tocdepth}{2}
\cover{cover.jpg}
\extrainfo{Seeking what is true is not seeking what is desirable.}

% modify the color in the middle of titlepage
\definecolor{customcolor}{RGB}{9,119,119}
\colorlet{coverlinecolor}{customcolor}
\usepackage{cprotect}

\addbibresource[location=local]{reference.bib} % bib

\begin{document}

\maketitle
\frontmatter
\tableofcontents
\mainmatter



\chapter{Geometric Programming (GP)}

\section{Arithmetic Mean-Geometric Mean Inequality}
\subsection{($x_1+x_2+...+x_n)/n$}
\begin{theorem}[AM-GM inequality]
For any $x_1,x_2,..,x_n\geq 0$, $$\frac{x_1+x_2+\cdots+x_n}{n}\geq \sqrt[n]{x_1x_2\cdots x_n}$$
Equality is only achieved when $x_1=x_2=\cdots=x_n$
\end{theorem}
\begin{enumerate}[$\bullet$]
    \item The LHS is the arithmetic mean (average) of $x_1,x_2,...,x_n$.
    \item The RHS is the geometric mean of $x_1,x_2,...,x_n$.
\end{enumerate}

\begin{theorem}[Weighted AM-GM inequality]
    For any $x_1,x_2,...,x_n\geq 0$ with $\delta_1,\delta_2,\cdots,\delta_n>0$ with $\delta_1+\cdots+\delta_n=1$, $$\delta_1x_2+\delta_2x_2+\cdots+\delta_nx_n\geq x_1^{\delta_1}x_2^{\delta_2}\cdots x_n^{\delta_n}$$
    Equality is only achieved if $x_1=x_2=\cdots=x_n$.
\end{theorem}
When $\delta_1=\cdots=\delta_n=\frac{1}{n}$, the inequality recovers to unweighted AM-GM inequality.
\begin{proof}Prove by Jensen's Inequality:\\
    Let $f(t)=-\ln(t)$ which is strictly convex in $(0,\infty)$. Take $\lambda_1,\lambda_2,\cdots,\lambda_n>$ such that $\delta_1+\delta_2+\cdots+\delta_n=1$. According to Jensen's Inequality: $$f(\sum_{i=1}^n\delta_i x_i)\leq \sum_{i=1}^n\delta_i f(x_i)$$
    By substituting $f$:
    \begin{equation}
        \begin{aligned}
            -\ln(\sum_{i=1}^n\delta_i x_i)&\leq -\sum_{i=1}^n\delta_i \ln(x_i)\\
            e^{\ln(\sum_{i=1}^n\delta_i x_i)}&\geq e^{\sum_{i=1}^n\delta_i \ln(x_i)}\\
            \sum_{i=1}^n\delta_i x_i&\geq x_1^{\delta_1}x_2^{\delta_2}\cdots x_n^{\delta_n}
        \end{aligned}
        \nonumber
    \end{equation}
\end{proof}

\section{Unconstrained Geometric Programs}
\subsection{Def: Posynomial}
\begin{definition}
    A \textbf{posynomial term} in variables $t_1,...,t_m$ is a function of the form $$Ct_1^{\alpha_1}t_2^{\alpha_2}\cdots t_m^{\alpha_m}$$ where $\alpha_1,...,\alpha_m\in \mathbb{R}$ and $C>0$ is a positive real number.
\end{definition}
\begin{definition}
    A \textbf{posynomial} is a sum of posynomial terms.
\end{definition}

\subsection{General Strategy: AM-GM inequality}
\begin{definition}
    An unconstrained geometric program (GP) is the problem of minimizing a posynomial over positive real inputs.
    $$\min_{(t_1,\cdots,t_m\in \mathbb{R}^m_{>0}}g(t_1,\cdots,t_m)$$
    where $g(t_1,\cdots,t_m)$ is a sum of posynomial terms. $g(t_1,\cdots,t_m)=\sum_{i=1}^n{Term}_i(t_1,\cdots,t_m)$, where ${Term}_i(t_1,\cdots,t_m)=C_i t_1^{\alpha_{i,1}}t_2^{\alpha_{i,2}}\cdots t_m^{\alpha_{i,m}}$
\end{definition}

\begin{center}
    \fcolorbox{black}{gray!10}{\parbox{.9\linewidth}{\textbf{General Strategy:}\\
    Choose weights $\delta_1,...,\delta_n>0$ with $\delta_1+\cdots+\delta_n=1$ and use the inequality
    \begin{equation}
        \begin{aligned}
            \sum_{i=1}^n{Term}_i(t_1,\cdots,t_m)&=\sum_{i=1}^n\delta_i\left(\frac{{Term}_i(t_1,\cdots,t_m)}{\delta_i}\right)\\
            &\geq \left(\frac{{Term}_1(t_1,\cdots,t_m)}{\delta_1}\right)^{\delta_1}\cdots \left(\frac{{Term}_n(t_1,\cdots,t_m)}{\delta_n}\right)^{\delta_n}
        \end{aligned}
        \nonumber
    \end{equation}}}
\end{center}

\subsection{Dual of the Unconstrained GP}
\textbf{Example:} Suppose we want to find the minimum of $f(x,y)=2xy+\frac{y}{x^2}+\frac{3x}{y}$.\\
We want $$2xy+\frac{y}{x^2}+\frac{3x}{y}\geq \left(\frac{2xy}{\delta_1}\right)^{\delta_1}\left(\frac{y}{\delta_2 x^2}\right)^{\delta_2}\left(\frac{3x}{\delta_3 y}\right)^{\delta_3}$$
which requires
\begin{enumerate}[(1)]
    \item \textbf{Power of $x$:} $\delta_1-2\delta_2+\delta_3=0$
    \item \textbf{Power of $y$:} $\delta_1+\delta_2-\delta_3=0$
    \item \textbf{Sum:} $\delta_1+\delta_2+\delta_3=1$
    \item \textbf{Positive:} $\delta_1,\delta_2,\delta_3>0$
\end{enumerate}
In general, we want to eliminate all $t_1,...,t_n$ is the RHS of the inequality, then the RHS can be transformed into constant $V(\delta)=\left(\frac{C_1}{\delta_1}\right)^{\delta_1}\left(\frac{C_2}{\delta_2}\right)^{\delta_2}\cdots\left(\frac{C_n}{\delta_n}\right)^{\delta_n}$ which is a lower bound of $g(\vec{t}),\vec{t}\in \mathbb{R}_{>0}^m$

\begin{center}
    \fcolorbox{black}{gray!10}{\parbox{.9\linewidth}{\textbf{Dual Geometric Problem}
    \begin{equation}
        \begin{aligned}
            \max_{\vec{\delta}\in \mathbb{R}^n_{>0}}\quad &V(\vec{\delta})=\left(\frac{C_1}{\delta_1}\right)^{\delta_1}\left(\frac{C_2}{\delta_2}\right)^{\delta_2}\cdots\left(\frac{C_n}{\delta_n}\right)^{\delta_n}
        \end{aligned}
        \nonumber
    \end{equation}
    \begin{equation}
        \begin{aligned}
            &s.t.\quad& \delta_1\alpha_{1,1}+\delta_2\alpha_{2,1}+\cdots+\delta_n\alpha_{n,1}&=0\quad \text{(power of $t_1$)}\\
            &&\vdots\\
            &&\delta_1\alpha_{1,m}+\delta_2\alpha_{2,m}+\cdots+\delta_n\alpha_{n,m}&=0\quad \text{(power of $t_m$)}\\
            &&\delta_1+\cdots+\delta_n&=1\\
            &&\delta_1,\delta_2,...,\delta_n&>0
        \end{aligned}
        \nonumber
    \end{equation}}}
\end{center}

Suppose $\vec{\delta}^*$ is the solution to the dual GP. $$\sum_{i=1}^n\delta^*_i\left(\frac{{Term}_i(\vec{t})}{\delta^*_i}\right)\geq \left(\frac{{Term}_1(\vec{t})}{\delta^*_1}\right)^{\delta^*_1}\cdots \left(\frac{{Term}_n(\vec{t})}{\delta^*_n}\right)^{\delta^*_n}$$
The inequality holds only if
\begin{equation}
    \begin{aligned}
        \frac{{Term}_1(\vec{t})}{\delta^*_1}=\frac{{Term}_2(\vec{t})}{\delta^*_2}=\cdots=\frac{{Term}_m(\vec{t})}{\delta^*_m}=V(\vec{\delta}^*)
    \end{aligned}
    \nonumber
\end{equation}
where $V(\vec{\delta}^*)$ is a function only related to $\vec{\delta}^*$.

\underline{\textbf{Note:} It is possible that system of equations for $\vec{t}$ has no solution.}

\subsection*{Dual $\Rightarrow$ Primal}
\begin{theorem}
        Given a feasible point $\vec{\delta}^*$ of the dual program. If the equations
        $$\frac{{Term}_1(\vec{t})}{\delta^*_1}=\frac{{Term}_2(\vec{t})}{\delta^*_2}=\cdots=\frac{{Term}_m(\vec{t})}{\delta^*_m}=V(\vec{\delta}^*)$$
        have a solution $\vec{t}^*$ with $t^*_i>0,i=1,2,...,m$, then $\vec{t}^*$ is a primal solution, $\vec{\delta}^*$ is a dual solution, and $g(\vec{t}^*)=V(\vec{\delta}^*)$
\end{theorem}
\begin{proof}
    If a solution $\vec{t}^*$ with $t^*_i>0,i=1,2,...,m$ exists, then $g(\vec{t}^*)=V(\vec{\delta}^*)$ (by AM-GM inequality).\\
    Suppose there exists another solution $\vec{t}'$ to the primal problem. Because $V(\vec{\delta}^*)$ is a lower bound of $g(\vec{t})$, $g(\vec{t}')\geq V(\vec{\delta}^*)=g(\vec{t}^*)$ $\Rightarrow$ $\vec{t}^*$ is an optimal solution minimizing $g$.\\
    Suppose there exists feasible $\vec{\delta}'$, $V(\vec{\delta}')$ is a lower bound of $g(\vec{t})$ $\Rightarrow$ $V(\vec{\delta}^*)=g(\vec{t}^*)\geq V(\vec{\delta}')$ $\Rightarrow$ $\vec{t}^*$ is also optimal maximizing $V$.
\end{proof}

\subsection*{Primal $\Rightarrow$ Dual}
\begin{theorem}
        If $\vec{t}^*$ is an optimal primal solution, then $$\vec{\delta}^*=\left(\frac{{Term}_1(\vec{t}^*)}{g(\vec{t}^*)},\frac{{Term}_2(\vec{t}^*)}{g(\vec{t}^*)},\cdots,\frac{{Term}_n(\vec{t}^*)}{g(\vec{t}^*)}\right)$$
        is an optimal dual solution and $g(\vec{t}^*)=V(\vec{\delta}^*)$.
\end{theorem}
\begin{proof}
    If $\vec{t}^*$ is an optimal primal solution, it's a critical point and $\nabla g(\vec{t}^*)=\vec{0}$. Recall $g(\vec{t}^*)=\sum_{i=1}^m{Term}_i(\vec{t}^*)=\sum_{i=1}^mC_i{t_1^*}^{\alpha_{i,1}}{t_2^*}^{\alpha_{i,2}}\cdots {t_m^*}^{\alpha_{i,m}}$\\
    $\nabla g(\vec{t}^*)=\vec{0}$ implies, for each $1\leq j\leq n$,
    \begin{equation}
        \begin{aligned}
            \frac{\partial g(\vec{t}^*)}{\partial t_j}=\sum_{i=1}^n\frac{\alpha_{i,j}}{t_j}{Term}_i(\vec{t}^*)=0
        \end{aligned}
        \nonumber
    \end{equation}
    Then, we can check $\vec{\delta}^*=\left(\frac{{Term}_1(\vec{t}^*)}{g(\vec{t}^*)},\frac{{Term}_2(\vec{t}^*)}{g(\vec{t}^*)},\cdots,\frac{{Term}_n(\vec{t}^*)}{g(\vec{t}^*)}\right)$ is feasible in dual problem and can get the equality in AM-GM inequality.
\end{proof}


\chapter{Polynomial Interpolation}
Suppose we are given a collection of points $\{(x_1,y_1),(x_2,y_2),...,(x_k,y_k)\}$. We want to find \textbf{polynomial} $f$ that passes through all the points.

We have two methods to solve this problem:
\begin{enumerate}[(1)]
    \item Set up and solve the system $M\vec{a}=\vec{y}$
    \item Use the Lagrange interpolation formula.
\end{enumerate}

\section{Method 1: $M\vec{a}=\vec{y}$}
If $f(x)=a_nx^n+a_{n-1}x^{n-1}+\cdots+a_1x+a_0$ passes through all $(x_i,y_i)$, then
\begin{equation}
    \begin{aligned}
        a_nx_1^n+\cdots+a_1x_1+a_0&=y_1\\
        a_nx_2^n+\cdots+a_1x_2+a_0&=y_2\\
        \vdots\quad &\\
        a_nx_k^n+\cdots+a_1x_k+a_0&=y_k
    \end{aligned}
    \nonumber
\end{equation}
in matrix form as $M\vec{a}=\vec{y}$
\begin{equation}
    \begin{aligned}
        \begin{bmatrix}
            1&x_1&x_1^2&\cdots&x_1^n\\
            1&x_2&x_2^2&\cdots&x_2^n\\
            \vdots&&\ddots&&\vdots\\
            1&x_k&x_k^2&\cdots&x_k^n\\
        \end{bmatrix}
        \begin{bmatrix}a_0\\a_1\\ \vdots\\a_n\end{bmatrix}=\begin{bmatrix}y_1\\y_2\\ \vdots\\y_k\end{bmatrix}
    \end{aligned}
    \nonumber
\end{equation}

In order to avoid no solution or multi solutions, we would let to set $n=k-1$.

\begin{theorem}
    If $x_i\neq x_j$ for all $1\leq i<j\leq k$, then there is a \underline{unique} polynomial of degree at most $k-1$ that passes through the points $\{(x_1,y_1),(x_2,y_2),\cdots,(x_k,y_k)\}$.
\end{theorem}

\section{Method 2: Lagrange Interpolation Formula}
Suppose $$l_i(x)=\prod_{j\neq i}\frac{x-x_j}{x_i-x_j}\ ,i=1,2,...,k$$
The $l_1,\cdots,l_k$ has the following properties:
\begin{enumerate}[$\bullet$]
    \item Each has degree $k-1$
    \item $l_i(x_i)=1$
    \item $l_i(x_j)=0$ when $i\neq j$
\end{enumerate}

Then the \underline{Lagrange Interpolation Formula} is
\begin{equation}
    \begin{aligned}
        f(x)=y_1l_1(x)+y_2l_2(x)+\cdots+y_kl_k(x)
    \end{aligned}
    \nonumber
\end{equation}
where $f(x_i)=y_1\cdot 0+y_2\cdot 0+\cdots +y_i\cdot 1+\cdots+y_k\cdot 0=y_i$

\section{Lines of Best Fit}
Suppose we use a linear function $y=ax+b$ to fit the collection of points $(x_1,y_1),...,(x_k,y_k)$.

We use error to measure the accuracy of line's accuracy of fitting.
\begin{enumerate}
    \item $Error(a,b)=|(ax_1+b)-y_1|+\cdots+|(ax_k+b)-y_k|$\\
    Pros: 1. Convex; 2. Minimizing problem is linear.\\
    Cons: not differentiable
    \item $Error(a,b)=[(ax_1+b)-y_1]^2+\cdots+[(ax_k+b)-y_k]^2=\|a\vec{x}+b\vec{1}-\vec{y}\|^2$
    which is also convex
    \begin{equation}
        \begin{aligned}
            \frac{\partial Error(a,b)}{\partial a}&=2\sum_{i=1}^n(ax_i+b-y_i)x_i=2(a\vec{x}+b\vec{1}-\vec{y})^T\cdot\vec{x}\\
            \frac{\partial Error(a,b)}{\partial b}&=2\sum_{i=1}^n(ax_i+b-y_i)=2(a\vec{x}+b\vec{1}-\vec{y})^T\cdot\vec{1}
        \end{aligned}
        \nonumber
    \end{equation}
    The critical point is the global minimizer
\end{enumerate}

\section{Least-Square Problem (Overconstrainted $A \vec{x}= \vec{b}$)}
For $A\in \mathbb{R}^{m\times n}$, $\vec{y}\in \mathbb{R}^m, m>n$. We want to solve $A\vec{x}=\vec{y}$. However, this equation system is overconstrained, we can only find a $\vec{x}$ to minimize the error between $A\vec{x}$ and $\vec{y}$.

The least-square error problem can be written as find $\vec{x}\in \mathbb{R}^n$ to minimize $\|A\vec{x}-\vec{y}\|$.

If $A$ is an $m\times n$ matrix, then the set $V=\{A\vec{x}:\vec{x}\in \mathbb{R}^n\}$ is a subspace of $\mathbb{R}^m$.

Then, "minimizing $\|A\vec{x}-\vec{y}\|$" means "finding the point of $V$ closest to $\vec{y}\in \mathbb{R}^m$".

\subsection{Lemma: closest point $\Leftrightarrow$ $(A\vec{x}^*-\vec{y})\bot \vec{a},\ \forall\vec{a}\in V$}
\begin{lemma}
    If $V=\{A\vec{x}:\vec{x}\in \mathbb{R}^n\}$, then the point $A\vec{x}^*\in V$ is the closest point of $V$ to $\vec{y}\in \mathbb{R}^m$ if and only if $$(A\vec{x}^*-\vec{y})\bot \vec{a},\ \forall\vec{a}\in V$$
\end{lemma}
\begin{proof}
We look at restrictions to lines. $\vec{x}^*$ is the global minimizer of $f(\vec{x})=\|A\vec{x}-\vec{y}\|^2$. $\vec{u}\in \mathbb{R}^n$ is an arbitrary element of $V$. Let $\vec{a}=A\vec{u}$.
\begin{equation}
    \begin{aligned}
        \phi_{\vec{u}}(t)&=\|A(\vec{x}^*+t\vec{u})-\vec{y}\|^2\\
        &=\|A\vec{x}^*-\vec{y}+t\vec{a}\|^2\\
        &=\|A\vec{x}^*-\vec{y}\|^2+2t(A\vec{x}^*-\vec{y})^T\vec{a}+t^2\|\vec{a}\|^2
    \end{aligned}
    \nonumber
\end{equation}
Since $\vec{x}^*$ is the global minimizer of $f(\vec{x})$, $t=0$ is the global minimizer of $\phi_{\vec{u}}$. For $C_1t^2+C_2t+C_3$, $t=0$ is global minimizer when $C_1\geq 0, C_2=0$ $\Rightarrow (A\vec{x}^*-\vec{y})^T\vec{a}=0$
\end{proof}

\subsection{Theorem: $\vec{x}^*=(A^TA)^{-1}A^T\vec{y}=A^+ \vec{y}$}
We can use this characterization to write down a \textbf{normal equation}. ("normal" is another word for "perpendicular")
\begin{theorem}
    A point $\vec{x}^*\in \mathbb{R}^n$ minimizes $\|A\vec{x}-\vec{y}\|$ if and only if $A^TA\vec{x}^*=A^T\vec{y}$
\end{theorem}
\begin{proof}
    Let $\vec{a}^{(i)}=A\vec{e}^{(i)}$ be the basis vector of $V$, then any vector $\vec{a}$ can be linear combination of $\{\vec{a}^{(i)}\}_{i=1,...,n}$. Hence, $\vec{x}^*$ is the global minimizer $\Leftrightarrow$ $(A\vec{x}^*-\vec{y})\bot \vec{a}^{(i)},i=1,...,n$ $\Leftrightarrow A^T(A\vec{x}^*-\vec{y})=0$
\end{proof}
We can solve the optimal $x^*$ as $$\vec{x}^*=(A^TA)^{-1}A^T\vec{y}=A^+ \vec{y}$$
we call $A^+=(A^TA)^{-1}A^T$ the \underline{pseudoinverse} of A because $A^+$ is a "pseudo-solution" to the overconstrained system.

\subsection{Def: Projection Matrix: $P=AA^+$; Projection of $\vec{y}$ on $V$: $A \vec{x}^*=P \vec{y}$}
\begin{definition}
    The matrix $P=AA^+$ is called a \textbf{projection matrix}. (It maps a point $\vec{y}\in \mathbb{R}^n$ to $P\vec{y}$, the closest point in $V$ to $\vec{y}$, $P\vec{y}$ is also called projection of $\vec{y}$ on $V$.)
\end{definition}
We have $P \vec{y}= A \vec{x}^*$, $(P \vec{y}-\vec{y})\bot \vec{a},\ \forall \vec{a}\in V$.

\textbf{Properties of $P$:}
\begin{enumerate}[$\bullet$]
    \item $P^2=P$ (i.e., $P$ is idempotent)
    \item $P^T=P$ (i.e., $P$ is symmetric)
\end{enumerate}

\subsection{Special Case: Projection on vector ${Proj}_{\vec{a}}(\vec{y})=\frac{(\vec{a}\cdot\vec{y})\vec{a}}{\|\vec{a}\|^2}$}

\textbf{A special case:} $A=\vec{a}\in \mathbb{R}^m$. $$P=\vec{a}(\vec{a}^T\vec{a})^{-1}\vec{a}^T=\frac{\vec{a}\vec{a}^T}{\|\vec{a}\|^2}$$
The projection of $\vec{y}$ onto $\vec{a}$ $${Proj}_{\vec{a}}(\vec{y})=P\vec{y}=\frac{\vec{a}\vec{a}^T}{\|\vec{a}\|^2}\vec{y}=\frac{\vec{a}\cdot\vec{y}}{\|\vec{a}\|^2}\vec{a}$$
where $\frac{\vec{a}\cdot\vec{y}}{\|\vec{a}\|^2}$ is a scalar to measure how much of $\vec{y}$ is “pointing in the same direction as” $\vec{a}$.

If we normalize $\vec{a}$ to $\vec{u}$, the projection of $\vec{y}$ onto vector $\vec{u}$ is
\begin{equation}
    \begin{aligned}
        {Proj}_{\vec{u}}(\vec{y})=(\vec{u}\cdot\vec{y})\vec{u}
    \end{aligned}
    \nonumber
\end{equation}

\subsection{Theorem: Projection Matrix $=$ Sum of outer products of orthonormal basis}
Let $\vec{u}^{(1)},...,\vec{u}^{(n)}$ be the \textbf{orthonormal basis} of $V\in \mathbb{R}^m$ (which requires $\|\vec{u}^{(i)}\|=1,i=1,...,n$ and $\vec{u}^{(i)}\cdot \vec{u}^{(j)}=0,\forall i\neq j$)

\begin{theorem}
    Suppose that $V\subseteq \mathbb{R}^m$ with an orthonormal basis $\{\vec{u}^{(1)},...,\vec{u}^{(n)}\}$. Then the projection matrix onto $V$ is given by the formula
    $$P=\vec{u}^{(1)}(\vec{u}^{(1)})^T+\vec{u}^{(2)}(\vec{u}^{(2)})^T+\cdots+\vec{u}^{(n)}(\vec{u}^{(n)})^T$$
\end{theorem}
\textbf{Note:} sometimes $\vec{u}\vec{v}^T$ is called the \textbf{outer product} of $\vec{u}$ and $\vec{v}$.
\begin{proof}
    For any $\vec{y}$, let $\vec{x}=P\vec{y}$ and $\vec{z}=\vec{y}-\vec{x}$, we know $\vec{z}\bot V$.

    Because $\vec{x}\in V$, we can write $$\vec{x}=a_1 \vec{u}^{(1)} +\cdots+a_n \vec{u}^{(n)}$$
    So we can write $$\vec{y}=\vec{x}+\vec{z}=a_1 \vec{u}^{(1)} +\cdots+a_n \vec{u}^{(n)}+\vec{z}$$
    For any $i=1,...,n$, we can compute by orthonormal property and
    \begin{equation}
        \begin{aligned}
            \vec{u}^{(i)}\cdot \vec{y}= a_i
        \end{aligned}
        \nonumber
    \end{equation}
    Then,
    \begin{equation}
        \begin{aligned}
            \vec{x}&=(\vec{u}^{(1)}\cdot \vec{y}) \vec{u}^{(1)} +\cdots+(\vec{u}^{(n)}\cdot \vec{y}) \vec{u}^{(n)}\\
            &=\vec{u}^{(1)}(\vec{u}^{(1)})^T\vec{y}+\cdots+\vec{u}^{(n)}(\vec{u}^{(n)})^T\vec{y}\\
            &=\left[\vec{u}^{(1)}(\vec{u}^{(1)})^T+\vec{u}^{(2)}(\vec{u}^{(2)})^T+\cdots+\vec{u}^{(n)}(\vec{u}^{(n)})^T\right]\vec{y}\\
            &=P\vec{y}
        \end{aligned}
        \nonumber
    \end{equation}
\end{proof}

\subsection{Corollary: $Q$ has orthonormal columns $\Rightarrow$ $\vec{x}^*=Q^T \vec{y}$. ($Q^+=Q^T$)}
\begin{corollary}
    When columns of $Q$ are orthonormal, the vector $\vec{x}^*$ that minimizes $\|Q \vec{x}-\vec{y}\|$ can be computed as $\vec{x}^*=Q^T \vec{y}$.
\end{corollary}
\begin{proof}
    $Q^+=(Q^TQ)^{-1}Q^T=I Q^T=Q^T$
\end{proof}

\subsection{The Gram-Schmidt process}
Now we know that if $Q$ has orthonormal columns, then we get a much nicer formula for the projection matrix and for the least-squares minimization problem. How do we make $Q$ have orthonormal columns?

One method for doing this is the Gram-Schmidt process. We want to input vectors $\vec{a}^{(1)},...,\vec{a}^{(n)}$ and output orthonormal vectors $\vec{u}^{(1)},...,\vec{u}^{(n)}$.

We assume $\vec{a}^{(1)},...,\vec{a}^{(n)}$ are linearly independent. In the algorithm, we will produce $\vec{v}^{(1)},...,\vec{v}^{(n)}$ that are orthogonal but not orthonormal, then we get $\vec{u}^{(1)},...,\vec{u}^{(n)}$ by $\vec{u}^{(i)}=\frac{\vec{v}^{(i)}}{\|\vec{v}^{(i)}\|}$.

\textbf{Gram-Schmidt process:}
\begin{enumerate}[(1)]
    \item Let $\vec{v}^{(1)}=\vec{a}^{(1)}$ and $\vec{u}^{(1)}=\frac{\vec{v}^{(1)}}{\|\vec{v}^{(1)}\|}$
    \item For $j=1,...,n$
    \begin{equation}
        \begin{aligned}
            \vec{v}^{(j)}&=\vec{a}^{(j)}-\sum_{i=1}^{j-1}(\vec{u}^{(i)}\cdot \vec{a}^{(j)})\vec{u}^{(i)}\\
            &=\vec{a}^{(j)}-\sum_{i=1}^{j-1}\frac{(\vec{v}^{(i)}\cdot \vec{a}^{(j)})\vec{v}^{(i)}}{\|\vec{v}^{(i)}\|^2}\\
            &=\vec{a}^{(j)}-\sum_{i=1}^{j-1}{Proj}_{\vec{v}^{(i)}}(\vec{a}^{(j)})
        \end{aligned}
        \nonumber
    \end{equation}
    and $$\vec{u}^{(j)}=\frac{\vec{v}^{(j)}}{\|\vec{v}^{(j)}\|}$$
\end{enumerate}
\textbf{Note:} In the general case, where $\vec{a}^{(1)},...\vec{a}^{(n)}$ are not linearly independent, step (2) will sometimes give us $\vec{v}^{(j)}=0$. In that case, we omit the $j^{th}$ vector: what this tells us is that $\vec{a}^{(j)}$ is not necessary to span the subspace.

\section{Minimum-norm problems (Underconstrainted $A \vec{x}= \vec{b}$)}
Consider systems of equations $A \vec{x}=\vec{b}$ with infinitely many solutions. We want to find the solution $\vec{x}$ with the smallest norm.
\begin{equation}
    \begin{aligned}
        \min_{\vec{x}\in \mathbb{R}^n}\quad &\|\vec{x}\|\\
        s.t.\quad  & A \vec{x}=\vec{b}
    \end{aligned}
    \nonumber
\end{equation}
i.e., we want to find the projection of $\vec{0}$ on $S=\{\vec{x}\in \mathbb{R}^n: A \vec{x}=\vec{b}\}$

\subsection{Applying the least-squares technique}
The solution set $S=\{\vec{x}\in \mathbb{R}^n: A \vec{x}=\vec{b}\}$ which is an example  of an \textbf{affine subspace}, i.e., a vector subspace of $\mathbb{R}^n$ that does not contain $\vec{0}$.

We may write $S=S'+\vec{x}_0$, where:
\begin{enumerate}[$\bullet$]
    \item $\vec{x}_0$ is an arbitrary element of $S$ (s.t. $A \vec{x}_0=\vec{b}$)
    \item $S'=\{\vec{y}\in \mathbb{R}^n: A \vec{y}=\vec{0}\}$
\end{enumerate}
\begin{center}
    \fcolorbox{black}{gray!10}{\parbox{.9\linewidth}{"Finding the point in $S$ closest to $\vec{0}$" is equivalent to "Finding the point in $S'$ closest to $-\vec{x}_0$"}}
\end{center}

\textbf{Example:}
\begin{equation}
    \begin{aligned}
        \min_{\vec{x}\in \mathbb{R}^4}\quad \|\vec{x}\|\\
        s.t.\quad2x_1-x_2+x_3-x_4=3\\
        x_2-x_3-x_4=1
    \end{aligned}
    \nonumber
\end{equation}
$x_2=1+x_3+x_4$ $\Rightarrow$ $2x_1-1-x_3-x_4+x_3-x_4=3 \Rightarrow x_1-x_4=2$.

The general solution has the form
\begin{equation}
    \begin{aligned}
        \vec{x}=\begin{bmatrix}
            2+x_4\\1+x_3+x_4\\x_3\\x_3
        \end{bmatrix}=\begin{bmatrix}
            2\\1\\0\\0
        \end{bmatrix}+\begin{bmatrix}
            0\\1\\1\\0
        \end{bmatrix}x_3+\begin{bmatrix}
            1\\1\\0\\1
        \end{bmatrix}x_4
    \end{aligned}
    \nonumber
\end{equation}
Let $\vec{x}_0=[2,1,0,0]^T$, the solution set can be $S=S'+\vec{x}_0$ where $S'$ is the set of linear combinations of $[0,1,1,0]^T$ and $[1,1,0,1]^T$.

Then, the problem becomes
\begin{equation}
    \begin{aligned}
        \min_{(x_3,x_4)\in \mathbb{R}^2}\left\|\begin{bmatrix}
            0&1\\1&1\\1&0\\0&1
        \end{bmatrix}\begin{bmatrix}
            x_3\\x_4
        \end{bmatrix}-\begin{bmatrix}
            -2\\-1\\0\\0
        \end{bmatrix}\right\|
    \end{aligned}
    \nonumber
\end{equation}
Let $A=\begin{bmatrix}
    0&1\\1&1\\1&0\\0&1
\end{bmatrix}$. From what we have already known, we can solve this problem by solving
\begin{equation}
    \begin{aligned}
        A^TAX=A^T(-x_0)\Leftrightarrow \begin{bmatrix}
            2&1\\
            1&3
        \end{bmatrix}X=\begin{bmatrix}
            1\\
            -3
        \end{bmatrix}
    \end{aligned}
    \nonumber
\end{equation}
Then, we can solve $(x_3,x_4)=(0,-1)$, then the optimal solution is $(1,0,0,-1)$.

\subsection{The short cut method}
We can find that $S=\{\vec{x}\in \mathbb{R}^n: A \vec{x}=\vec{b}\}$ and $S'=\{\vec{y}\in \mathbb{R}^n: A \vec{y}=\vec{0}\}$ are parallel. Then the vector $\vec{x}^*-\vec{0}$ (i.e. $\vec{x}^*$) should be perpendicular to $S'$.
\begin{lemma}
    A vector $\vec{x}^*$ satisfying $A \vec{x}^*=\vec{b}$ is the minimum-norm solution to the system of equations $A\vec{x}=\vec{b}$ if and only if $\vec{x}^*\cdot \vec{y}=0$ for all solutions $\vec{y}$ of the homogeneous system $A \vec{y}= \vec{0}$.
\end{lemma}
Obviously, \textbf{all} vectors in null space ($N(A)=\{\vec{y}:A \vec{y}=\vec{0}\}$) are orthogonal to a vector \textbf{if and only if} it is a linear combination of $A$'s rows. $\vec{x}^*=A^T \vec{w}$, for some $\vec{w}\in \mathbb{R}^n$. ($\vec{x}^* \cdot (\vec{y})= (\vec{x}^*)^T \vec{y}=\vec{w}^T A \vec{y}=0$)
\begin{theorem}
    A vector $\vec{x}^*$ satisfying $A \vec{x}^*=\vec{b}$ is the minimum-norm solution to the system of equations $A\vec{x}=\vec{b}$ if and only if it can be written as $\vec{x}^*=A^T \vec{w}$ for some $\vec{w}\in \mathbb{R}^n$
\end{theorem}
\begin{center}
    \fcolorbox{black}{gray!10}{\parbox{.9\linewidth}{Hence, we cam find the minimum-norm solution $\vec{x}^*$ by solving $$AA^T \vec{w}=\vec{b}$$ for $\vec{w}$ and then computing $\vec{x}^*=A^T \vec{w}$.}}
\end{center}
\textbf{Same Example:}
\begin{equation}
    \begin{aligned}
        \min_{\vec{x}\in \mathbb{R}^4}\quad \|\vec{x}\|\\
        s.t.\quad2x_1-x_2+x_3-x_4=3\\
        x_2-x_3-x_4=1
    \end{aligned}
    \nonumber
\end{equation}
Solve the system
\begin{equation}
    \begin{aligned}
        AA^T \vec{w}&=\vec{b}\\
        \begin{bmatrix}
            2&-1&1&-1\\
            0&1&-1&-1
        \end{bmatrix}
        \begin{bmatrix}
            2&0\\-1&1\\1&-1\\-1&-1
        \end{bmatrix}
        \begin{bmatrix}
            w_1\\w_2
        \end{bmatrix}&=
        \begin{bmatrix}
           3\\1
        \end{bmatrix}\\
        \begin{bmatrix}
            7&-1\\-1&3
        \end{bmatrix}
        \begin{bmatrix}
            w_1\\w_2
        \end{bmatrix}&=
        \begin{bmatrix}
           3\\1
        \end{bmatrix}
    \end{aligned}
    \nonumber
\end{equation}
We can solve $(w_1,w_2)=(\frac{1}{2},\frac{1}{2})$, then $\vec{x}^*=A^T \vec{w}=[1,0,0,-1]^T$

\subsection{The short cut method with $H$-norm}
\begin{definition}
    Given a positive definite (symmetric) matrix $H$, let the associated inner product be $\vec{x}\cdot_H \vec{y}= \vec{x}^T H \vec{y}$ and the associated norm be $\|\vec{x}\|_H=\sqrt{\vec{x}^T H \vec{x}}$
\end{definition}
Solve the optimization problem
\begin{equation}
    \begin{aligned}
        \min_{\vec{x}\in \mathbb{R}^n}\quad &\|\vec{x}\|_H^2=\vec{x}^T H \vec{x}\\
        s.t. \quad&A \vec{x}=\vec{b}
    \end{aligned}
    \nonumber
\end{equation}

\begin{lemma}
    A point $\vec{x}^*$ that satisfies $A \vec{x}^*= \vec{b}$ is the \textbf{minimum-$H$-norm solution} to $A \vec{x}= \vec{b}$ if and only if $$\vec{x}^*\cdot_H \vec{y}=0$$
    for all $\vec{y}$ for which $A \vec{y}=\vec{0}$
\end{lemma}
\begin{proof}
We proved that the $\vec{x}^*$ being perpendicular with $\{y\in \mathbb{R}^n: A \vec{y}= \vec{0}\}$ is equivalent to $\vec{x}^*$ is the minimum-norm solution to $A \vec{x}^*= \vec{b}$
\end{proof}
A vector that is orthogonal to $\{y\in \mathbb{R}^n: A \vec{y}= \vec{0}\}$ if and only if $\vec{x}^*=H^{-1}A^T \vec{w}$, for some $\vec{w}$. ($\vec{x}^*\cdot_H \vec{y}=(\vec{x}^*)^T H \vec{y}= \vec{w}^TAH^{-1}H\vec{y}=0$)

The same as the short cut method we can solve the $\vec{x}^*$ by computing $AH^{-1}A^T \vec{w}= \vec{b}$

\begin{theorem}
        The \textbf{minimum-$H$-norm solution} $\vec{x}^*$ of the Underconstrainted system $A \vec{x}=\vec{b}$ can be found by sovling $$A H^{-1}A^T \vec{w}= \vec{b}$$
        for $\vec{w}$ and then computing $\vec{x}^*= H^{-1}A^T \vec{w}$
\end{theorem}

\begin{example}
    \begin{equation}
        \begin{aligned}
            \min_{(x,y)\in \mathbb{R}^2} &3x^2+2xy+2y^2\\
            s.t.\quad & 3x-y=3
        \end{aligned}
        \nonumber
    \end{equation}
    $3x^2+2xy+2y^3$ is the square of the $H$-norm of the point $\begin{bmatrix}x\\y\end{bmatrix}$ for the matrix $H=\begin{bmatrix}
        3&1\\
        1&2
    \end{bmatrix}$. We can also compute $H^{-1}=\begin{bmatrix}
        0.4&-0.2\\
        -0.2&0.6
    \end{bmatrix}$.
    
    Then, we can solve $\vec{x}^*$ by computing
    \begin{equation}
        \begin{aligned}
            A H^{-1}A^T \vec{w}&= \vec{b}\\
            \begin{bmatrix}
                3&-1
            \end{bmatrix}
            \begin{bmatrix}
                0.4&-0.2\\
                -0.2&0.6
            \end{bmatrix}
            \begin{bmatrix}
                3\\
                -1
            \end{bmatrix}\vec{w}&=3
        \end{aligned}
        \nonumber
    \end{equation}
    We can compute $\vec{w}=\frac{5}{9}$, then the optimal solution is $\begin{bmatrix}
        x\\
        y
    \end{bmatrix}=H^{-1}A^T \vec{w}=\begin{bmatrix}
        \frac{7}{9}\\
        -\frac{2}{3}
    \end{bmatrix}$
\end{example}













\end{document}