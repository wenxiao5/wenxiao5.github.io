\documentclass[11pt]{elegantbook}
\usepackage{graphicx}
%\usepackage{float}
\definecolor{structurecolor}{RGB}{40,58,129}
\linespread{1.6}
\setlength{\footskip}{20pt}
\setlength{\parindent}{0pt}
\usetikzlibrary{arrows.meta}
\newcommand{\argmax}{\operatornamewithlimits{argmax}}
\newcommand{\argmin}{\operatornamewithlimits{argmin}}
\elegantnewtheorem{proof}{Proof}{}{Proof}
\elegantnewtheorem{claim}{Claim}{prostyle}{Claim}
\DeclareMathOperator{\col}{col}
\title{Causal Inference}
\author{Wenxiao Yang}
\institute{Haas School of Business, University of California Berkeley}
\date{2025}
\setcounter{tocdepth}{2}
\extrainfo{All models are wrong, but some are useful.}

\cover{cover.png}

% modify the color in the middle of titlepage
\definecolor{customcolor}{RGB}{32,178,170}
\colorlet{coverlinecolor}{customcolor}
\usepackage{cprotect}


\bibliographystyle{apalike_three}

\begin{document}
\maketitle

\frontmatter
\tableofcontents

\mainmatter



\chapter{Causal Inference}
The fundamental problem of causal inference:
\begin{enumerate}[(a).]
    \item Never see the same person treated and untreated
    \item Missing data problem
    \item "Solve" by finding a comparison group
\end{enumerate}

\begin{definition}[Notations and Estimands]
    \normalfont
    \begin{enumerate}[$\circ$]
        \item Treatment: $T\in\{0,1\}$
        \item Potential Outcome with treatment $Y(1), Y(0)$
        \item Other Variable $X$
        \item Individual Treatment Effect (ITE) $= Y_i(1)-Y_i(0)$
        \item Conditional Average Treatment Effect (CATE) $=\mathbb{E}[Y(1)-Y(0)|X=x]:=\tau(x)$
        \item Average Treatment Effect (ATE) $=\mathbb{E}[Y(1)-Y(0)]:=\tau$
        \item Average Treatment Effects on Treated (ATT) $=\mathbb{E}[Y(1)-Y(0)\mid T=1]$
    \end{enumerate}
\end{definition}


\subsection*{Difference in Means}
\begin{equation}
    \begin{aligned}
        \hat{\tau}=\bar{Y}_1-\bar{Y}_0=\frac{1}{n_1}\sum_{i=1}^n Y_i T_i - \frac{1}{n_0}\sum_{i=1}^n Y_i(1-T_i)
    \end{aligned}
    \nonumber
\end{equation}

By the Law of Large Numbers,
\begin{equation}
    \begin{aligned}
        \lim_{n \rightarrow \infty}\frac{1}{n_1}\sum_{i=1}^n Y_i T_i&=\lim_{n \rightarrow \infty}\frac{n}{n_i}\frac{1}{n}\sum_{i=1}^n Y_i T_i\\
        &=\left(P[T=1]\right)^{-1} \mathbb{E}[YT]\\
        &= \left(P[T=1]\right)^{-1}\mathbb{E}[YT\mid T=1]P[T=1]\\
        &=\mathbb{E}[YT\mid T=1]\\
        \bar{Y}_1 &\stackrel{P}{\longrightarrow} \mathbb{E}[YT\mid T=1]
    \end{aligned}
    \nonumber
\end{equation}

\subsubsection*{Causal Effect}
\begin{assumption}\quad
    \begin{enumerate}[(1).]
        \item SUTVA: Only your treatment matters;
        \item Consistency: Observed outcome matches treatment "assignment": $Y=TY(1)+(1-T)Y(0)$.
    \end{enumerate}
\end{assumption}

Only yields $\hat{\tau}=\bar{Y}_1-\bar{Y}_0\stackrel{P}{\longrightarrow} \mathbb{E}[Y(1)\mid T=1]-\mathbb{E}[Y(0)\mid T=0]$

\begin{equation}
    \begin{aligned}
        &\mathbb{E}[Y(1)\mid T=1]-\mathbb{E}[Y(0)\mid T=0]\\
        =&\underbrace{\mathbb{E}[Y(1)\mid T=1]-\mathbb{E}[Y(0)\mid T=1]}_{\textnormal{ATT}}+\underbrace{\mathbb{E}[Y(0)\mid T=1]-\mathbb{E}[Y(0)\mid T=0]}_{\textnormal{selection bias}}\\
    \end{aligned}
    \nonumber
\end{equation}
To get the ATT (eliminate the selection bias), we need exclusion/independence: Randomization.


Assume $Y(t)=\mu(t)+\epsilon_t$ (SUTVA). Consider the consistency assumption:
\begin{equation}
    \begin{aligned}
        Y&=TY(1)+(1-T)Y(0)\\
        &=Y(0)+T(Y(1)-Y(0))\\
        &=\underbrace{\mu_0}_\alpha+T\underbrace{(\mu_1-\mu_0)}_{\beta^T}+\underbrace{\epsilon_0+T(\epsilon_1-\epsilon_0)}_{\epsilon}
    \end{aligned}
    \nonumber
\end{equation}

Consider the covariate between $T$ and $X$. Why important?

\begin{equation}
    \begin{aligned}
        &\mathbb{E}[Y|X=1,T=1]-\mathbb{E}[Y|X=1,T=0]\\
        =& \underbrace{\mathbb{E}[Y(1)|X(1)=1]-\mathbb{E}[Y(0)|X(1)=1]}_{\textnormal{ATE}|X(1)=1} + \underbrace{\mathbb{E}[Y(0)|X(1)=1]-\mathbb{E}[Y(0)|X(0)=1]}_{\textnormal{selection bias}}\\
    \end{aligned}
    \nonumber
\end{equation}


$Y(t)=\mu(t,X)+\epsilon_t$. Then,
\begin{equation}
    \begin{aligned}
        Y&=Y(1)T+Y(0)(1-T)\\
        &=\underbrace{\mu(0,X)}_{\alpha(X)} + T\underbrace{(\mu_1(X)-\mu_0(X))}_{\beta(X)}+\epsilon
    \end{aligned}
    \nonumber
\end{equation}



\chapter{STAT 256}
\section{Lecture Zero: Causal Inference vs Association}{Amanda Coston}

\subsection{Causal Inference on the Effect of Red Wine on Health}\label{sec:intro}

\subsection{Is red wine good for heart health?}
For many years consensus held that red wine improved cardiovascular health. The evidence behind this was largely from studies on people of drinking age that compared the health outcomes of those who self-reported that they drank wine to those who self-reported that they did not. Findings under such a design showed that people who drink  red wine have better cardiovascular outcomes than those who don't drink alcohol.

What are potential problems with this study design? You may be thinking that people who drink wine systematically differ from those who don't in ways that matter for health outcomes. In fact, sociologist Kaye Middleton Fillmore showed that the statistical significance of these findings hinged on the inclusion of \emph{previous} drinkers in the ``non-drinkers" category. That is, some people who said they did not drink red wine previously drank alcohol. Problematically, a common reason these people gave up alcohol was poor health. Therefore the definition of  ``non-drinkers" selected for people who had poorer health outcomes. Fillmore showed that redefining ``non-drinkers" to be ``never-drinkers" eliminated any supposed advantage of drinking wine.

The debate here was one of causal inference -- looking at the cause and effect of red wine on health.

For more details, see \cite{fillmore2007moderate}.

\subsection{Syllabus}

See course syllabi here: \href{https://stat156.berkeley.edu/fall-2024/syllabus.html}{https://stat156.berkeley.edu/fall-2024/syllabus.html}.

\subsection{Association vs Causation}

Association is the focus of much of statistics but in causal inference our focus is, of course, on causation! As a starting point, we will today consider common measures of association and discuss why they may not capture causation. In the next lecture we will see causal analogues of these measures.
aaa
We first consider the setting where the outcome $Y$ and treatment $Z$ are both binary. 

\begin{definition}[Risk Difference]
    The associative risk difference (RD) is $E[Y \mid Z = 1] - E[Y \mid Z = 0]$.
\end{definition}


\begin{definition}[Risk Ratio]
    The associative risk ratio (RR) is $P[Y =1 \mid Z = 1]/P[Y =1 \mid Z = 0]$.
\end{definition}

\begin{definition}[Odds Ratio]
    The associative odds ratio (OR) is $\frac{P[Y =1 \mid Z = 1]}{P[Y =0 \mid Z = 1]}/\frac{P[Y=1 \mid Z = 0]}{P[Y=0 \mid Z = 0]}$.
\end{definition}

Which measure one chooses depends on their particular setting -- what question they are interested in and what data they have available (odds ratios can be estimated in with outcome-dependent sampling whereas the risk difference and risk ratio generally cannot). The measures are related to each other as follows:

\begin{enumerate}
    \item $Z \perp Y  \Longleftrightarrow \mathrm{RD} = 0 \Longleftrightarrow \mathrm{RR} = 1 \Longleftrightarrow \mathrm{OR} = 1$
    \item $\mathrm{RD} > 0 \Longleftrightarrow \mathrm{RR} > 1 \Longleftrightarrow \mathrm{OR} > 1$ assuming all conditional probabilities are non-zero.
    \item $\mathrm{RR} \approx \mathrm{OR}$ when $P(Y=1)$ is small.
\end{enumerate}

Next we consider measures of association that can accommodate non-binary outcomes. 

\subsection{Correlation and linear regression}
Suppose we are now interested in the outcome blood pressure as a measure of cardiovascular health. A natural starting point is to model the relationship between blood pressure ($Y$) and whether one drinks red wine ($Z$) as 

\begin{align*}
    Y = \beta Z + \alpha + \epsilon
\end{align*}

where $E[\epsilon] = 0 $ and $E[\epsilon Z ] = 0$.

Recall that we can relate $\beta$ to the Pearson correlation coefficient $\rho$ as follows

\begin{align*}
    \beta = \rho \frac{\mathrm{var}(Z)}{\mathrm{var}(Y)}.
\end{align*}

The coefficient $\beta$ describes the change in $Y$ associated with whether one drinks red wine. More generally, the coefficient $\beta$ describes the change in $Y$ associated with one unit increase in $Z$.
Sometimes people refer to $\beta$ as the ``effect" of $Z$ on $Y$ but this is generally misleading (without further assumptions). We have simply modeled an associative relationship; we can't claim anything causal yet! Next time we will introduce a new language, potential outcomes, so that we can make causal claims.


\section{Lecture One: Potential Outcomes Framework}{Aryan Shafat, Frederik Stihler, Mika Lee (Revisions)}

\subsection{Review of last lecture}\label{sec:review}
Review of Risk Difference, Risk Ratio and Odds Ratio for setting where the outcome $Y$ and treatment $Z$ are both binary. (See lecture notes of Lecture 0 for details)

Explanation of the last equivalence in the following statement:
\begin{center}
$\mathrm{RD} > 0 \Longleftrightarrow \mathrm{RR} > 1 \Longleftrightarrow \mathrm{OR} > 1$ assuming all conditional probabilities are non-zero
\end{center}

We know that:

\begin{center}
    $\mathbb{P}[Y = 1 \mid Z = 1] + \mathbb{P}[Y = 0 \mid Z = 1] = 1$ and
    $\mathbb{P}[Y = 1 \mid Z = 0] +$ $\mathbb{P}[Y = 0 \mid Z = 0] = 1$. 
\end{center}    
    $\mathrm{RR} > 1$ implies that $\mathbb{P}[Y = 1 \mid Z = 1] > \mathbb{P}[Y = 1 \mid Z = 0]$, and hence $\mathbb{P}[Y = 0 \mid Z = 0] > \mathbb{P}[Y = 0 \mid Z = 1]$. This leads us to $\mathrm{OR} > 1$.

\subsection{Potential Outcome Framework}\label{sec:pof}

\subsubsection{Types of questions we are interested in}

\begin{enumerate}
    \item Does journaling reduce the risk of depression?
    \item Do personalized AI tutors improve a student's grade?
    \item Does smoking cigarettes cause cancer?
\end{enumerate}

We can summarize these questions in the following table: \par
\begin{center}
\begin{tabular}{ccc}
  \hline
  & \textbf{$Z$} & \textbf{$Y$} \\
  \hline
  1 & Journal & Depression \\
  \hline
  2 & AI tutor & Grade  \\
  \hline
  3 & Cigarettes & Cancer  \\
  \hline
\end{tabular}
\end{center}

For all the examples, we want to form a causal estimand using potential outcomes. The potential outcome framework was developed by \cite{neyman1923} in 1923 over one hundred years ago and later revitalized and repopularized by \cite{rubin1980} in 1980. 

\subsubsection{Potential Outcomes}

We are interested in potential (hypothetical) outcomes when we are thinking about causal questions.

\begin{definition}[Potential Outcome]
    The potential outcome Y is a function of a particular treatment value $Y(Z=z)$.
\end{definition}

E.g. $Y(Z=1) = Y(1)$ (the potential outcome under the intervention that assigns treatment) vs. $Y(Z=0) = Y(0)$ (the potential outcome under the intervention that assigns no treatment).

In our particular examples, these quantities are represented by the following hypothetical questions: \par

(1) Would someone have depression if they journalled $\approx Y(1)$ \\
(2) Would a  student get a particular grade if they had an AI tutor $\approx Y(1)$ \\
(3) Would someone get cancer if they smoked cigarettes $\approx Y(1)$

\subsubsection{Causal Estimands}

Next we define the causal versions of our measures of association (called  causal estimands).

\begin{definition}[Causal Risk Difference]
    The causal risk difference (on a population level) is $E[Y(1) - Y(0)]$.
\end{definition}

\begin{definition}[Causal Risk Ratio]
    The causal risk ratio is $\frac{E[Y(1)]}{E[Y(0)]} = \frac{P(Y(1) = 1)}{P(Y(0) = 1)}$.
\end{definition}

\begin{definition}[Causal Odds Ratio]
    The causal odds ratio is $\frac{P(Y(1) = 1)}{P(Y(1) = 0)}/\frac{P(Y(0) = 1)}{P(Y(0) = 0)} = \\ 
    \frac{E[Y(1)]}{E[1-Y(1)]}/\frac{E[Y(0)]}{E[1-Y(0)]}$.
\end{definition}

\subsubsection{Hidden Assumptions}

As mentioned, \cite{rubin1980} repopularized this framework by clarifying some important hidden assumptions:

\begin{assumption}[Consistency]
The treatment levels are well-defined (there are no other versions of the treatment).
\end{assumption}

\begin{assumption}[No Interference]
The treatment assigned to other units does not affect the potential outcomes for unit $i$ (no spillover).
\end{assumption}

These 2 assumptions are called \textbf{Stable Unit Treatment Value Assumption (SUTVA).}

For example, our 3rd question/example about cigarette smoking violates both assumptions.

The question isn't well-defined (does smoking entail  smoking 1 cigarette a day or smoking a whole pack a day) and thus violates Assumption 1.

It also violates Assumption 2, through non-smokers who might end up passively smoking by being around smokers.

\subsection{Causal Estimands}
Causal Effects are functions of potential outcomes. For example, the causal risk difference is essentially the 'average treatment effect'.

\begin{itemize}
    \item Unit 'i' has 2 potential outcomes: $Y_i$(1) and $Y_i$(0)
\end{itemize}

\begin{itemize}
    \item \textbf{Individualized treatment effect:} $Y_i$(1) - $Y_i$(0)
\end{itemize}

\textbf{Fundamental Problem of Causal Inference (1986 Holland):}
Never observe both potential outcomes.

\subsubsection{Add a time-element}
We could include a time element - write into our personal journal/smoke a cigarette/receive treatment one day and then go without treatment the next day, to see the 'causal effect' of the treatment.

However, the effect of the treatment might be long-lasting. Thus, one could 'experience'/observe their causal effects even on the days without treatment. Additionally, say for a time-period of 2 days, we actually end up having 4 potential outcomes:

\begin{itemize}
    \item $Y_{i, day 1}$(1) vs $Y_{i, day 1}$(0)
\end{itemize}

\begin{itemize}
    \item $Y_{i, day 2}$(1) vs $Y_{i, day 2}$(0)
\end{itemize}


Going back to the potential outcomes framework, we basically end up 'observing' one of the potential outcomes framework.

\begin{itemize}
    \item The factual/observed outcome is, $Y_i = 
\begin{cases}
Y_i(1) & \text{if } Z_i = 1 \\
Y_i(0) & \text{if } Z_i = 0
\end{cases}
$ 
\end{itemize}

Equivalently, 

$Y_i = Z_i*Y_i(1) + (1-Z_i)*Y_i(0)$

\begin{itemize}

\item The (unobserved) counterfactual, or missing potential outcome is given by:

$Y_i^{\text{mis}} = Z_i*Y_i(0) + (1-Z_i)*Y_i(1)$

\end{itemize}

\subsection{Simpson's Paradox}
Based on the class poll, we saw that the effect of the hint wasn't that strong. There was a confounding variable (row number) that was dampening the effect of the hint (since most of the hints were given to people in the rows at the back). This was attributed to \textbf{Simpson's Paradox}, which might have made it seem like the hints had an \textbf{effect reversal} (as if those who got the hints actually ended up doing worse on the poll).

This is mathematically shown as:
$\mathbb{P}(Y=1|Z=0) > \mathbb{P}(Y=1|Z=1)$. However, we can easily counter this by conditioning on the confounding variable (X):

\begin{center}
    $\mathbb{P}(Y=1|Z=0, X=x) < \mathbb{P}(Y=1|Z=1, X=x) \quad  \forall x \in X $
\end{center}

\subsubsection{Sources of the paradox:}
\begin{itemize}
    \item Confounding variables/factors
    \item Non-collapsibility
\end{itemize}

\section{Lecture Three: Randomized Experiments}{Daisy Wang \& Mika Lee (Revisions)}

\subsection{Last Lecture: Simpson's Paradox}
    Simpson's paradox is when the data may originally appear to have one trend, but not when grouped. In stats terms:
    
\begin{center}
    $\mathbb{E}[Y \mid Z=1] - \mathbb{E}[Y \mid Z=0] > 0$ 
\end{center}
but is actually 
\begin{center}
    $\mathbb{E}[Y \mid Z=1, X=x] - \mathbb{E}[Y \mid Z=0, X=x] <0$ 
\end{center} 
when we condition on the confounding variable X. This is caused by confounding and non-collapsibility.

    \textbf{Note on notation:} Potential outcomes $Y(Z=1) = Y(1)$ and in Hernan, $Y^{Z=1} = Y^1$

\subsection{Randomized Experiments}
    Why is randomization so powerful? 
    \begin{itemize}
        \item\textit{\textbf{Ignorable}} treatment assignment
        \item Groups are \textit{\textbf{exchangeable}}, in that groups could swap assignments and still have the same result
    \end{itemize}

\subsubsection{Exchangeability}
$\boldsymbol{\Pr{Y(1) = 1 \mid Z = 1}}= \Pr{Y(1) = 1 \mid Z = 0}$

$\Pr{Y(0) = 1 \mid Z = 1} = \Pr{Y(0) = 1 \mid Z = 0} =$
$ 
\boldsymbol{\Pr(Y = 1 \mid Z = 0)}$

where the two bolded equations are identifiable from the data. 

    Additionally $Y(Z) \perp Z$  $\forall  z = 0, 1$ which means treatment assignment is independent of potential outcome. In general, treatment assignment is exchangeable which implies ignorable which implied exogenous.

    In an \textbf{ideal random experiment}, association would be equal to causation:
\begin{center}$\mathbb{E}[Y \mid Z=1] = \mathbb{E}[Y(1)\mid Z=1]$ \end{center}
    Additionally:
\begin{center} Risk Difference $\mathbb{E}[Y \mid Z=1] - \mathbb{E}[Y\mid Z=0]$ = Causal Risk Difference $\mathbb{E}[Y(1)] - \mathbb{E}[Y(0)]$\end{center}
In tandem with the previous part about the independence of random treatment assignment, it is important to note that $Z \perp 
 Y(Z) \neq Z\perp Y$,  in which the relationship on the left side of the inequality contains random treatment assignment and the one on the right side references no treatment assignments.

    Finally a \textbf{randomized experiment} is where treatments are assigned in a known and probabilistic ("random") manner i.e. Bernoulli random experiments.

 \subsubsection{Completely Randomized Experiment}
    We denote those who get the treatment as $n_1$ , and those who get the control as $n_0 = n - n_1$. The treatment assignment is denoted as $\mathbb{Z}= (z_1, ... z_n)$.  Then the probability that $\Pr(\mathbb{Z}=z ) = \frac{1}{\binom{n}{n_1}}$ where $\mathbb{Z}$ is such that $\sum_{i=1}^{n} z_i = n_1$. Also keep in mind that $\mathbb{Y}(1)$ and $\mathbb{Y}(0)$ are fixed in this situation.

\subsubsection{Fisher (1935) Fisher's Sharp Null}
    $H_0: Y_i(1) = Y_i(0)$ for all $i = 1,...n$ where $\mathbb{Y} = \mathbb{Y}(1) = \mathbb{Y}(0)$ and the test statistic is $T(Z, \mathbb{Y})$. Here the $Z$ is random and $\mathbb{Y}$ is fixed. This is also known as Neyman's Null Hypothesis (Weak Null).
    
    Under the null, $\{T(z^1, \mathbb{Y})...T(z^{\binom{n}{n_1}}, \mathbb{Y}\}$ is uniform, representing the randomization distribution. Thus we can find the p-value:
    \begin{center} 
    $p = \frac{1}{\binom{n}{n_1}} \sum_{m=1}^{\binom{n}{n_1}} \mathbf{1}\{T(z^m,y) \geq T(Z, Y)\}$ 
    \end{center}
    We can also use the Monte Carlo method to approximate the p-value:
     \begin{center} 
     $\frac{1}{R} \sum_{r=1}^{R} \mathbb{1}\{T(z^r,y) \geq T(Z, Y)\}$ 
     \end{center}
     where $T(z^r,y) $ is a particular fixed value and $T(Z, Y)$ is an observed random variable.
     All of this is the \textbf{Fisher's randomization test}, aka \textbf{permutation test}.
     
 \subsubsection{Choices for the Test-Statistic}
 \paragraph{Difference-in-means statistic} 
$\hat{\tau} = \hat{\bar{Y}}(1) - \hat{\bar{Y}}(0)$ where $\hat{\bar{Y}}(1) = \frac{1}{n_1}\sum_{z_i=1}Y_i = \frac{1}{n_1}\sum_{i=1}z_iY_i $ 

\textbf{Note:} this statistic is easily ruined by outliers

\paragraph{Wilcoxon Rank Sum}
    Unlike the difference-in-means statistic, the Wilcoxon rank sum test statistic is robust to outliers. This is because it is defined as follows: 
    \begin{center} 
    $R_i:$ the rank of $Y_i = \#\{j: Y_j \leq Y_i\}$, and so the test statistic itself is  $W=\sum_{i=1}^{n}z_iR_i$ 
    \end{center}
    The test statistic is more broad and may miss distributional differences. It can also be viewed as the difference-in-means of the rank under treatment vs control. 

\paragraph{Kolmogorov-Smirnov Statistic}
The empirical CDF of treated units is $\hat{F_1}(y) = \frac{1}{n_1}\sum_{i=1}^n z_1 \mathbf{1}\{Y_i \leq y\}$  and the control is $\hat{F_0}(y) = \frac{1}{n_0}\sum_{i=1}^n (1-z_1) \mathbf{1}\{Y_i \leq y\}$. The test statistic is then
\begin{center} $D=\underset{y}{\max}|\hat{F_1}(y) - \hat{F_0}(y)|$ \end{center}






\section{Lecture Four: Continue on Randomized Experiments}{Yulin Zhang \& Aadya Agarwal}
\subsection{Last Lecture: Complete Randomized Experiments}
    \begin{itemize}
        \item Complete randomized experiment
        \begin{itemize}
            \item \textit{terms}: let $n$ denotes the total number of test units, $n_1$ denotes the number of units get treated, $n_0 = 1 - n_1$ denotes the number of units get control.
        \end{itemize}
        \item exchangability: $y(z) \perp z$, where $z$ denotes treatment, $y(z)$ denotes the outcome of the treatment.
        \item Fisher's sharp null
    \end{itemize}
    
\subsection{Neymanian estimation \& inference}

\subsubsection{Finite population counterfactual (potential) quantities}
Let $y_{i}(1)$ and $y_{i}(0)$ denote the potential outcomes of the $i$th unit under treatment and control. Define the \textit{individual casual effect} of the $i$th unit as $$\tau_{i} = y_{i}(1) - y_{i}(0)$$
\textbf{Theoretical population level statistics}:\\
$$\bar{y}(1)= \frac{1}{n}\sum_{i = 1}^{n}y_i(1)$$
$$\bar{y}(0)= \frac{1}{n}\sum_{i = 1}^{n}y_i(0)$$
$$S^{2}(1) = \frac{1}{n - 1}\sum_{i = 1}^{n}(y_i(1) - \bar{y}(1))^{2}$$
$$S^{2}(0) = \frac{1}{n - 1}\sum_{i = 1}^{n}(y_i(0) - \bar{y}(0))^{2}$$
$$S(1, 0) = \frac{1}{n - 1}\sum_{i = 1}^{n}(y_i(1) - \bar{y}(1))(y_i(0) - \bar{y}(0))$$
$$\tau = \frac{1}{n}\sum_{i = 1}^{n} \tau_{i} = \bar{y}(1) - \bar{y}(0)$$
$$S^{2}(\tau) = \frac{1}{n - 1}\sum_{i = 1}^{n}(\tau_{i} - \tau)^2$$

\textbf{Lemma}: $2S(1, 0) = S^{2}(1) + S^{2}(0) - S^{2}(\tau)$ \\

\textbf{In the data}:
$$\hat{\bar{y}}(1) = \frac{1}{n_1}\sum_{i = 1}^{n}z_{i}y_{i}$$
$$\hat{\bar{y}}(0) = \frac{1}{n_0}\sum_{i = 1}^{n}(1 - z_{i})y_{i}$$
$$\hat{S^{2}}(1) = \frac{1}{n_1 - 1}\sum_{i = 1}^{n}z_i(y_i - \hat{\bar{y}}(1))^{2}$$
$$\hat{S^{2}}(0) = \frac{1}{n_0 - 1}\sum_{i = 1}^{n}(1 - z_i)(y_i - \hat{\bar{y}}(0))^{2}$$
Note, since we could not observe $y_i(1)$ and $y_i(0)$ at the same time, no other theoretical statistics could be calculated at this point.

\subsubsection{Neyman (1993) theorm}
\begin{itemize}
    \item $\hat{\tau} = \hat{\bar{y}}(1) - \hat{\bar{y}}(0)$ satisfies:
    \begin{itemize}
        \item unbiasedness: $\mathbb{E}(\hat{\tau}) = \tau$
        \item has variance: $$\mathbb{V}(\hat{\tau}) = \frac{S^{2}(1)}{n_1} + \frac{S^{2}(0)}{n_0} - \frac{S^{2}(\tau)}{n} = \frac{n_0}{n_{1}n}S^{2}(1) + \frac{n_1}{n_{0}n}S^{2}(0) + \frac{2}{n}S(1, 0)$$
        \item Note in variance terms, $S^{2}(\tau)$ and $S(1, 0)$ are not identifiable from the data, this variance could not be calculated from the data directly.
    \end{itemize}
    \item Neyman's variance estimator: $$\hat{\mathbb{V}}(\hat{\tau}) = \frac{\hat{S^{2}}(1)}{n_1} + \frac{\hat{S^{2}}(0)}{n_0}$$
    \begin{itemize}
        \item Note: this is the usual variance estimator under the assumption that the population is infinite and random samples. However, this is not the case we have here: the population is finite and we do not have random samples (treatment and controls are usually mutually exclusive).
        \item In the paper, Neyman proved that $$\mathbb{E}(\hat{V}) - \mathbb{V}(\hat{\tau}) = \frac{S^{2}(\tau)}{n} \ge 0$$ where equality holds when $\tau_{i} = \tau $ $\forall i$. This shows that the estimator above is conservative for estimating the variance of $\hat{\tau}$. People call this a "minor miracle that the two mistakes cancel" (Freeman, Pisani \& Purues, 2006)
    \end{itemize}
    \item Regression analysis of the complete randomized experiment \\
    In OLS settings, we could get $(\hat{\alpha}, \hat{\beta}) = argmin_{a, b}\sum_{i = 1}^{n} (y_i - a - bz_i)^2$, where $\hat{\beta} = \hat{\tau}$
    \begin{itemize}
        \item Usual OLS variance: $\hat{\mathbb{V}}_{OLS}(\hat{\beta}) = \hat{\mathbb{V}}_{OLS}(\hat{\tau}) \approx \frac{\hat{S^{2}}(1)}{n_0} + \frac{\hat{S^{2}}(0)}{n_1}$, note this is not equal to the variance we want. 
        \item Eicker-Huber-White (EHW) Robust Variance Estimator: $\hat{\mathbb{V}}_{EWH}(\hat{\tau}) \approx \frac{\hat{S^{2}}(1)}{n_1} + \frac{\hat{S^{2}}(0)}{n_0}$
        \item HC2 variant of EHW $= \frac{\hat{S^{2}}(1)}{n_1} + \frac{\hat{S^{2}}(0)}{n_0}$
    \end{itemize}
\end{itemize}

\subsection{Stratified (Conditional) Random Experiments}
\subsubsection{A potential problem with completely randomized experiment}
Bias could be introduced if the different covariate strata have non-random treatments assigned. 
\begin{itemize}
    \item Let $x_i \in \{1..k\}$ denote a covariate that we are interested in.
    \item Covariate imbalance: bias would be introduced if the proportion of units in stratum $k$ are different across treatment and control groups. 
\end{itemize}
\subsubsection{Definition}
Let $n_{k, 1} = \# \{i: x_i = k, z_i = 1\}$, $n_{k, 0} = \# \{i: x_i = k, z_i = 0\}$. $n_{k} = \# \{i: x_i = k\}$ \\
Stratified random experiment (SRE) is to run $k$ independent complete random experiments (CRE) within k strata of discrete covariate $x$ for fixed $n_{k, 1}$, $n_{k, 0.}$ \\
We could view stratum as block, stratified randomization could also be called block randomization. \\
$$\tau_{k} = \frac{1}{n_k}\sum_{x_i = k} \tau_{i}$$
$$\tau = \frac{1}{n}\sum_{i = 1}^{n}\tau_i = \frac{1}{n}\sum_{k = 1}^{k}\sum_{x_i = k}\tau_i = \sum_{k = 1}^{k}\pi_{k}\tau_{k}, \pi_k = \frac{n_k}{n}$$
\textbf{Comparing SRE and CRE}:
\begin{itemize}
    \item Number of potential assignments: $SRE: \prod_{k = 1}^{k} \binom{n_k}{n_{k, 1}} < CRE: \binom{n}{n_1}$
    \item Propensity score (propensity of getting treatment): $e_k = \frac{n_{k, 1}}{n_k}$. For SRE, $e_k$ is fixed; for CRE, $e_k$ is random.
\end{itemize}

\subsubsection{Fisher Randomization Test}
$H_0: Y_i(0) = Y_i(1)$ for all i \\
Run Fisher Randomization test by permuting the treatment indicators within each strata X according to the fixed values of treatment assigned for each of those strata \\
This process is called the \textbf{Conditional Randomization Test or Conditional Permutation test}\\
\\
Test Statistic Choice: \\
\\
\textbf{1. Stratified Estimator\\}
$\hat{\tau}_s = \sum_{k=1}^{K} \pi_k \hat{\tau}_k$\\
\\
\textbf{2. Combined Wilcoxon-Rank Sum Statistic\\}
\\
$W_k$ : Wilcoxon rank-sum statistic in stratum k
\\$W_s = \sum_{k=1}^{k}C_k W_k$
\\$C_k = \frac{1}{n_k,_1 n_k,_0}$ OR $C_k = \frac{1}{n_k + 1}$\\
\\Note: Works well only if k is small
\\
\\\textbf{3. Aligned Rank Statistic (Hodges and Lehmann)} \\
\\$\tilde{Y}_i = Y_i - \bar{Y}_k$ where $Y_k = \frac{1}{n-k} \sum_{X_i = k} Y_k$ (stratum specific mean)
\\Let $\tilde{Y}_i$ be the rank of $\tilde{Y}_i$
\\ $\tilde{W} = \sum_{i=1} ^ {n} Z_i \tilde{R}_i$

\subsubsection{Neymanian Inference for SRE}

$\mathbb{E}[\hat{\tau}_k] - {\tau}_k = 0$
\\$\hat{\tau}_k$ has variance:
Var($\hat{\tau}_k$) = $\frac{S_k^2 (1)}{n_k,1} + \frac{S_k^2 (0)}{n_k,0} - \frac{S_k^2 (\tau)}{n_k} $
\\where $S_k^2(1), S_k^2(0), S_k^2(\tau)$ are the stratum specific analogs (stratum specific variance of potential outcomes and causal effects)
\\We can combine these to get a stratified estimator: $\hat{\tau}_s = \sum_{k=1}^{k} \pi_k \hat{\tau}_k$ where $\pi_k = \frac{n_k}{n}$ and 
\begin{itemize}
    \item $\mathbb{E}[\hat{\tau}_s] - \tau = 0$ (unbiased)
    \item has variance var($\hat{\tau}_s$) = $\sum_{k=1}^{k} \pi_k^2 var(\hat{\tau}_k)$
\end{itemize}
If $n_k,1 >= 2$ and $n_k,0 >= 2$, we can compute $\hat{S_k^2}(1)$, $\hat{S_k^2}(0)$
\\$\hat{V_s} = \sum_{k=1}^{k} \pi_k^2 (\frac{\hat{S_k^2}(1)}{n_k,1} + \frac{\hat{S_k^2}(0)}{n_k,0})$
\\$\hat{\tau_s} +- Z_{1-\frac{\alpha}{2}} \sqrt{\hat{V_s}}$
\\$H_o$ : $\tau = 0 $ \quad $\epsilon_s = \frac{\hat{\tau_s}}{\sqrt{\hat{V_s}}}$

\subsubsection{SRE vs CRE}
$e_k = e$ for all k, $\hat{\tau} = \hat{\tau_s}$
\\$var_{CRE} (\hat{\tau}) - var_{SRE} (\hat{\tau_s}) = \sum_{k=1}^{k}\frac{\pi_k}{n} (\sqrt{\frac{n_0}{n_1}} (\bar{Y_k}(1) - \bar{Y}(1)) + \sqrt{\frac{n_1}{n_0}} (\bar{Y_k}(0) - \bar{Y}(0)))^2 >= 0 $
\\$\hat{\tau_s}$ is generally more efficient (has lower variance) than $\hat{\tau}_{CRE}$

\subsubsection{Post-Stratification}

We've already run a CRE
\\$\textbf{n} = \{n_{k,1}, n_{k,0}\}_{k=1}^{K}$
\\$P_{CRE} (\textbf{Z}=z | \textbf{n}) = \frac{P(\textbf{Z}=z, \textbf{n})}{P(\textbf{n})}$ where $\textbf{Z} \epsilon R^n = \{Z_1, ...., Z_n\}$
\\= $\frac{1}{\pi_{k=1}^k \binom{n_k}{n_k,1}}$
\\How to choose k?
\\$k=5$ empirically is a good choice OR $k = \frac{n}{2}$ which is the 'matched pairs experiment'


\section{Lecture Five: Rerandomization and Regression Adjustment}{Inigo Artiagoitia \& Sai Kolasani}
\subsection{Last Lecture: Discrete Covariate}
    \begin{itemize}
        \item Design: Stratification
        \end{itemize}
    \begin{itemize}
        \item Analysis: Post-stratification
    \end{itemize}
\subsection{Completely randomized experiment}
$n$ total experimental units

$n_1$ units randomly assigned to the treatment group

$n_0$ units randomly assigned to the control group

number of ways to assign $n_1$ units to treatment out of $n$ total units in a CRE:
$$\mathbb{Z}\binom{n}{n_1}$$
\subsubsection{Rerandomization}

In rerandomization, we draw $Z$ from the CRE and accept it if and only if the covariates are balanced across treatment and control groups.

Also, we have covariates. For example:
\begin{itemize}
    \item $x_i \in \mathbb{R}^k$ (e.g., age)
    \item Outcome $y$ might be fitness, while treatment $z$ could be a health intervention
\end{itemize}

We want to ensure these covariates are balanced between our treatment and control groups. To check this, we calculate the difference in means of covariates:

\[
\hat{\tau}_x = \frac{1}{n_1} \sum_{i=1}^n z_i x_i - \frac{1}{n_0} \sum_{i=1}^n (1-z_i) x_i
\]

Under CRE, $E[\hat{\tau}_x] = 0$, meaning the covariates are perfectly balanced on average. However, in practice, we often see $\hat{\tau}_x \neq 0$.

The covariance of this difference is given by:

\[
\text{cov}(\hat{\tau}_x) = \frac{1}{n_1} S_x^2 + \frac{1}{n_0} S_x^2 
\]

\[
= \frac{n}{n_1 n_0} S_x^2
\]

\[
S_x^2 = \frac{1}{n-1} \sum_{i=1}^{n} x_i x_i^T
\]

\subsubsection{Mahalanobis Distance}

The following Mahalanobis distance measures the difference between the treatment and control groups:

\[
M = \hat{\tau}_x^T \, \text{cov}(\hat{\tau}_x)^{-1} \hat{\tau}_x 
\]

\[
= \hat{\tau}_x^T \left( \frac{n}{n_i n_0} S_x^2 \right)^{-1} \hat{\tau}_x 
\]

\subsubsection{Rerandomization Procedure}

For this, we:
\begin{enumerate}
    \item Draw $Z$ from CRE
    \item Accept if and only if $M \leq a$
\end{enumerate}

The value of 'a' determines how strict we are about balance:
\begin{itemize}
    \item $a = \infty \rightarrow$ We accept any randomization (equivalent to CRE)
    \item $a = 0 \rightarrow$ We only accept perfect balance (usually impossible)
    \item $a = 0.001 \rightarrow$ A good threshold used.
\end{itemize}

This procedure is invariant to linear transformations of covariates.

\subsubsection{Inference}

\begin{itemize}
    \item We use FRT that simulates $Z$ under the constraint $M \leq a$
    \item For $a = \infty$: $\varepsilon$ be a univariate standard normal RV, and $\hat{\tau} - \tau \stackrel{D}{\sim} \sqrt{\text{var}(\hat{\tau})} \varepsilon$
    \item For $a$ close to zero: $\hat{\tau} - \tau \stackrel{D}{\sim} \sqrt{\text{var}(\hat{\tau})(1-R^2)} \varepsilon$ where $R^2$ is the squared correlation between $\hat{\tau}$ and $\hat{\tau}_x$
\end{itemize}

\subsection{Regression Adjustment (analysis)}

FRT:
\begin{enumerate}
    \item Regress $Y_i \sim X_i \rightarrow \hat{Y}_i$. 
    $\hat{\varepsilon}_i = \hat{Y}_i - Y_i$: pseudo outcome for test statistic.
    \item Regress $Y_i \sim (Z_i, X_i) \rightarrow \hat{\beta}$ (coefficient of $Z_i$): Test statistic.
    \item Estimation $\tau$:
\end{enumerate}

\subsubsection{Fisher's ANCOVA (1925)}

\begin{enumerate}
    \item Run an OLS Regression: $Y_i \sim (1, z_i, x_i)$
    \item Use $\beta$ (coefficient of $Z$) as our estimator for $\tau = \hat{\tau}_F$
\end{enumerate}

However, this method was criticized because:
\begin{enumerate}
    \item Bias: $E[\hat{\tau}_F] - \tau \neq 0$
    \item $\text{Var}(\hat{\tau}_F) > \text{var}(\hat{\tau})$ for $n_1 \neq n_0$
    \item Standard errors from OLS are incorrect
\end{enumerate}

\subsubsection{Lin (2013) Improvements}

Lin proposed some improvements to address these issues:
\begin{enumerate}
    \item The bias $E[\hat{\tau}_F] - \tau$ is small for large $n$ and approaches 0 as $n \rightarrow \infty$
    \item Run an OLS Regression: $y_i \sim (1, z_i, x_i, x_i * z_i) \rightarrow \hat{\tau}_L$ (coefficient of $z$)
    \begin{itemize}
        \item $\hat{\tau}_L$ is more efficient than $\hat{\tau}_F$
    \end{itemize}
    \item EHW standard error gives conservative estimate for $\hat{\tau}_L$.
\end{enumerate}


\subsubsection{Difference-in-Differences}

\begin{itemize}
    \item A special case of covariate adjustment, where $X$ is the \textbf{lagged outcome before the treatment}.
    
    \item The estimator for the average treatment effect is given by:
    \[
    \hat{\tau}_{\text{DiD}} = \frac{1}{n_{1}} \sum_{i=1}^{n} z_i (Y_{i} - x_i) - \frac{1}{n_{0}} \sum_{i=1}^{n} (1 - z_i)(Y_{i} - x_i)
    \]
    where $n_{1}$ is the number of treated units and $n_{0}$ is the number of control units. 
    
    This expression represents the \textbf{difference in treated and control outcomes after adjusting for pre-treatment covariates (the lagged outcomes $x_i$).}
    
    \item This can be rewritten as:
    \[
    \hat{\tau}_{\text{DiD}} = \left( \hat{Y}(1) - \hat{Y}(0) \right) - \left( \hat{X}(1) - \hat{X}(0) \right)
    \]
    where:
    \begin{itemize}
        \item $\hat{Y}(1)$ and $\hat{Y}(0)$ are the mean outcomes for the treated and control groups, respectively.
        \item $\hat{X}(1)$ and $\hat{X}(0)$ are the mean pre-treatment covariates for the treated and control groups, respectively.
    \end{itemize}
    
    \item This estimator is \textbf{unbiased}, and the mean of the treatment and control groups are:
    \[
    \hat{g}(C1) = \frac{1}{n_{1}} \sum_{i=1}^{n} z_i g_i
    \]
    \[
    \hat{g}(C0) = \frac{1}{n_{0}} \sum_{i=1}^{n} (1 - z_i) g_i
    \]
    where $g_i = Y_i - X_i$ represents the difference between the outcome and the lagged pre-treatment outcome for unit $i$.
    
    \item The variance of the estimator is:
    \[
    \hat{V} = \frac{1}{n_{1}(n_{1} - 1)} \sum_{i=1}^{n} z_i \left( g_i - \hat{g}(1) \right)^2 + \frac{1}{n_{0}(n_{0} - 1)} \sum_{i=1}^{n} (1 - z_i) \left( g_i - \hat{g}(0) \right)^2
    \]
    
\end{itemize}

\subsection{Matched Pairs Experiment}
\subsubsection{Notation and Setup}

\begin{itemize}
    
    \item Each pair contains one treated unit and one control unit. This ensures that comparisons between treatment and control are made within pairs, reducing the influence of extraneous variables.
    
    \item There are $2n$ experimental units, where:
    \[
    Z_i = \begin{cases}
    1 & \text{if the 1st unit receives treatment} \\
    0 & \text{if the 2nd unit receives treatment}
    \end{cases}
    \]
    
    \item $(i, j)$ indexes the $j$-th unit of pair $i$, with:
    \[
    i = 1, 2, \dots, n \quad j = 1, 2
    \]
    
    \item \textbf{Potential Outcomes Framework}: Each unit $(i, j)$ has potential outcomes $Y_{ij}(1)$ and $Y_{ij}(0)$, where:
    \begin{itemize}
        \item $Y_{ij}(1)$ is the outcome if unit $(i,j)$ receives treatment.
        \item $Y_{ij}(0)$ is the outcome if unit $(i,j)$ does not receive treatment.
    \end{itemize}
    The goal is to estimate the treatment effect by comparing the potential outcomes between treated and control units.
    
    \item The treatment assignment $Z_i$ is drawn \textit{i.i.d.} (independently and identically distributed) from a Bernoulli distribution with probability $\frac{1}{2}$:
    \[
    Z_i \overset{iid}{\sim} \text{Bernoulli}\left( \frac{1}{2} \right)
    \]
    This means that for each pair, there is an equal chance that either the first unit or the second unit will be assigned to treatment. This randomization ensures that treatment assignment is unbiased.
    
    \item The outcomes for an individual pair $(i)$ are defined as:
    \[
    y_{i1} = z_i y_{i1}(1) + (1 - z_i) y_{i1}(0)
    \]
    \[
    y_{i2} = z_i y_{i2}(0) + (1 - z_i) y_{i2}(1)
    \]
    These equations capture the realized outcomes for the units in the experiment. Depending on the treatment assignment $z_i$, we observe either the treated or control potential outcome.
    
\end{itemize}

\subsubsection{Fisher's Randomization Test (FRT)}

\begin{itemize}
    
    \item \textbf{Null Hypothesis ($H_0$)}: The null hypothesis in FRT states that there is no treatment effect, meaning the potential outcomes are the same regardless of whether the unit receives treatment or not. This implies:
    \[
    Y_{ij}(1) = Y_{ij}(0) \quad \text{for all} \quad i = 1, \dots, n \quad j = 1, 2
    \]
    Under $H_0$, the assignment of treatment is purely random, and any observed differences between the treated and control units are due to chance.
    
    \item \textbf{Pairwise Treatment Effect ($\hat{\tau}_i$)}: For each pair $i$, the treatment effect is estimated as the difference between the outcomes of the two units. The treatment effect for pair $i$ is given by:
    \[
    \hat{\tau}_i = y_{i1} - y_{i2} \quad \text{if} \quad z_i = 1
    \]
    \[
    \hat{\tau}_i = y_{i2} - y_{i1} \quad \text{if} \quad z_i = 0
    \]
    Here, $\hat{\tau}_i$ represents the observed treatment effect within pair $i$.
    
    \item \textbf{Average Treatment Effect ($\hat{\tau}$)}: The overall treatment effect, $\hat{\tau}$, is the average treatment effect across all pairs. This is calculated as:
    \[
    \hat{\tau} = \frac{1}{n} \sum_{i=1}^{n} \hat{\tau}_i
    \]
    The goal is to determine whether this observed treatment effect is significantly different from what would be expected under the null hypothesis.
    
    \item \textbf{Distribution of $\hat{\tau}$ under $H_0$}: Under the null hypothesis, the expected value of $\hat{\tau}$ is zero:
    \[
    \mathbb{E}[\hat{\tau}] = 0
    \]
    The variance of $\hat{\tau}$ is calculated as:
    \[
    \text{Var}(\hat{\tau}) = \frac{1}{n^2} \sum_{i=1}^{n} \text{Var}(\hat{\tau}_i)
    \]
    Given that $\hat{\tau}_i = Y_{i1} - Y_{i2}$, the variance can be expanded as:
    \[
    \text{Var}(\hat{\tau}) = \frac{1}{n^2} \sum_{i=1}^{n} \text{Var}(S_i) \left( Y_{i1} - Y_{i2} \right)^2
    \]
    Finally, the variance is simplified to:
    \[
    \text{Var}(\hat{\tau}) = \frac{1}{n^2} \sum_{i=1}^{n} \hat{\tau}_i^2
    \]
    
    \item \textbf{Test Statistic}: The test statistic for FRT is constructed by standardizing the observed treatment effect. Under the null hypothesis, the standardized test statistic follows a standard normal distribution:
    \[
    \frac{\hat{\tau}}{\sqrt{\frac{1}{n^2} \sum_{i=1}^{n} \hat{\tau}_i^2}} \sim N(0, 1)
    \]
    This allows us to evaluate whether the observed treatment effect $\hat{\tau}$ is significantly different from zero, which would indicate a treatment effect.
    
\end{itemize}

\subsubsection{Kolmogorov-Smirnov Type Statistic}

\begin{itemize}
    
    \item The observed treatment effects $\hat{\tau}_1, \dots, \hat{\tau}_n$ are fixed, and the test statistic $\Delta$ is defined as:
    \[
    \Delta = \sum_{i=1}^{n} \mathbb{I} \left( \hat{\tau}_i > 0 \right)
    \]
    where $\mathbb{I} \left( \hat{\tau}_i > 0 \right)$ is an indicator function that takes the value 1 if the treatment effect $\hat{\tau}_i$ is non-negative and 0 otherwise. This statistic counts the number of pairs where the treatment effect is positive.
    
    \item \textbf{Distribution of $\Delta$ under the Null Hypothesis ($H_0$)}: 
    Under the null hypothesis that there is no treatment effect, the indicator function $\mathbb{I} \left( \hat{\tau}_i > 0 \right)$ follows a Bernoulli distribution with probability $\frac{1}{2}$:
    \[
    \mathbb{I} \left( \hat{\tau}_i > 0 \right) \overset{iid}{\sim} \text{Bernoulli}\left( \frac{1}{2} \right)
    \]
    This means that under $H_0$, there is a 50\% chance that the treatment effect is positive in any given pair.

    \item \textbf{Binomial Distribution of $\Delta$}: Since the indicator function follows a Bernoulli distribution, the sum $\Delta$ follows a binomial distribution with $n$ trials and probability of success $\frac{1}{2}$:
    \[
    \Delta \sim \text{Binomial}(n, \frac{1}{2})
    \]
    
    \item \textbf{Binomial Test and Normal Approximation}: To test whether the observed number of positive treatment effects is significantly different from what we expect under $H_0$, we can either use the Binomial test with $p = \frac{1}{2}$ or, for large sample sizes, apply the normal approximation:
    \[
    \frac{\Delta - \frac{n}{2}}{\sqrt{\frac{n}{4}}} \sim N(0, 1)
    \]
    
\end{itemize}

\subsubsection{McNemar's Statistic}

\textbf{Binary Outcome}

\[
\begin{array}{|c|c|c|c|c|c|}
\hline
\text{Pair} \, i & Y_{i1} & Y_{i2} & z_i & \text{Treated Outcome} & \text{Control Outcome} \\
\hline
1 & 1 & 1 & 1 & 1 & 1 \\
2 & 0 & 0 & 0 & 0 & 0 \\
\vdots & \vdots & \vdots & \vdots & \vdots & \vdots \\
3 & 1 & 0 & 1 & 1 & 0 \\
4 & 1 & 0 & 0 & 0 & 1 \\
\hline
\end{array}
\]

This will be continued in the following lecture...


\section{Lecture Six: Observational Studies}{Yongsi Wu, Hanyang Li, Mika Lee (Revisions)}

\subsection{Review of Last Lecture: Matched Pairs Experiment}

One treated unit, one control unit per stratum (pair)

\underline{Example:} assign $1$ student to watch recordings online, $1$ to attend in-person where they are comparable in terms of GPA, major, year

\begin{itemize}
    \item SRE with $\frac{n}{2}$ strata
\end{itemize}
Unit $(i, j)$ is the $j$th unit of $i$th pair for $i = 1, 2, \dots, n$ and $j = 1, 2$ has potential outcomes $Y_{ij}(1)$ and $Y_{ij}(0)$

\begin{itemize}
    \item Define
        $Z_i = \left\{\begin{array}{rl}
        1 & \text{if $1$st unit is treated}\\
        0 & \text{if $2$nd unit is treated}
        \end{array}\right.$
\end{itemize}

\underline{\textbf{FRT}}\\
Null: $H_0$: $Y_{ij}(1) = Y_{ij}(0)$ $\forall i,j$

Let $\hat\tau_i$ be the outcome under treatment - outcome under control in pair $i$, i.e.,
    \[
        \hat\tau_i = S_i (Y_{i1} - Y_{i2}) \quad\text{with }S_i = 2Z_i - 1
    \]

We have introduced the following three statistics:

\begin{enumerate}
    \item \textbf{Sign test statistic}
    
    \item \textbf{McNemar's statistic:} Binary Outcome $Y$
    \[
    \begin{array}{|c|c|c|c|c|c|}
    \hline
    \text{Pair} \, i & Y_{i1} & Y_{i2} & z_i & \text{Treated Outcome} & \text{Control Outcome} \\
    \hline
    1 & 1 & 1 & 1 & 1 & 1 \\
    2 & 0 & 0 & 0 & 0 & 0 \\
    3 & 1 & 0 & 1 & 1 & 0 \\
    4 & 1 & 0 & 0 & 0 & 1 \\
    \vdots & \vdots & \vdots & \vdots & \vdots & \vdots \\
    n & & & & &\\
    \hline
    \end{array}
    \]

    Denote the number of rows where $Y_{i1} \neq Y_{i2}$ by $q$. Then the number of times $Z_i=1$ among $q$ discordant rows is given by
    \[
        m_{10} \sim \text{Binomial}(q, 1/2).
    \]
    This statistic can be computed using the mcnemar.test() function in R.

    \item \textbf{Kolmogorov-Smirnov-type statistic:}\\
    Under $H_0$, $|\hat\tau_i|$ is fixed and its sign is random with mean $0$ and variance $1$. Thus, $(\hat\tau_1, \cdots, \hat\tau_n)$ and -$(\hat\tau_1, \cdots, \hat\tau_n)$ should have the same distribution.\\
    Usual CDF:
    \begin{equation}\label{eq:D_tau}
        \widehat F(t) = \frac{1}{n} \sum^n_{i=1} \mathbf{1} \{\hat\tau_i \leq t\} \quad\text{for }(\hat\tau_1, \cdots, \hat\tau_n).
    \end{equation}
    Recalling the CDF for -$X$:
    \[
        F_{-X}(t) = \mathbb{P}(-X \leq t) = \mathbb{P}(X > -t) = 1 - F_X(-t),
    \]
    we then get the empirical distribution of $-(\hat\tau_1, \cdots, \hat\tau_n)$:
    \begin{equation}\label{eq:D_-tau}
        1 - \widehat F(-t-) = \frac{1}{n} \sum^{n}_{i=1} \mathbf{1} \{-\hat\tau_i \leq t\},
    \end{equation}
    where $\widehat F(-t-)$ is the left limit of the function $\widehat F(\cdot)$ at $-t$. Combining the two pieces \eqref{eq:D_tau} and \eqref{eq:D_-tau}, a Kolmogorov-Smirnov-type statistic is
    \[
        D = \max_t \left|\widehat F(t) + \widehat F(-t-) - 1\right|.
    \]
\end{enumerate}



\subsection{Neymanian Inference}
\subsubsection{Estimate average casual effect}
$$\tau = \frac{1}{n} \sum^{n}_{i=1}\tau_i$$
$$\tau_i = \frac{1}{2}(y_{i1}(1) + y_{i2}(1) - y_{i1}(0) - y_{i2}(0))$$
$$\mathbb{E}[\hat{\tau_i}] = \tau_i$$
\begin{center}
    $\hat{\tau} = \frac{1}{n}\sum^n_{i=1}\hat{\tau_i}$ and $\mathbb{E}[\hat{\tau}] = \tau$
\end{center}
$$\hat{V} = \frac{1}{n(n-1)}\sum^n_{i=1}(\hat{\tau_i} - \hat{\tau})^2$$

\subsubsection{Theorem}
$$\mathbb{E}[\hat{V}] - \text{Var} (\hat{\tau}) = \frac{1}{n(n-1)}\sum^n_{i=1}(\hat{\tau_i} - \hat{\tau}) \geq 0 \quad\textit{(conservative estimate)}$$
where OLS regress $(\hat{\tau_1}, ..., \hat{\tau_n})^T \thicksim$ intercept,
fitted intercept $= \hat{\tau}$,
variance estimate $= \hat{V}$.

\subsubsection{Covariate adjustment}
$\bullet$ Covariate-adjusted FRT \\
$\bullet$ Regression adjustment

\subsubsection{General Matched Experiment}
\begin{itemize}
    \item One unit gets treated in each stratum 
    \item $m_i$ units get control in set $i$
    \item Define $Z_{ij}$ for unit $i,j$
$$Y_{ij} = Z_{ij}Y_{ij}(1) + (1-Z_{ij})Y_{ij}(0)$$
    \item Within set average casual effect
$$\tau_i = \frac{1}{m_i+1} \sum^{m_i + 1}_{j=1} (Y_{ij}(1) - Y_{ij}(0))$$
$$\hat{\tau_i} = \sum^{m_i + 1}_{j=1} Z_{ij}Y_{ij} - \frac{1}{m_i} \sum^{m_i + 1}_{j=1} (1 - Z_{ij})Y_{ij} $$
$$\mathbb{E}[\hat{\tau_i}] - \tau_i = 0$$
\end{itemize}

\subsubsection{Estimation}

\begin{align}
    \tau &= \frac{1}{N} \sum^n_{i=1} \sum^{m_i + 1}_{j=1}(Y_{ij}(1) - Y_{ij}(0))\\
    &= \sum^n_{i=1} w_i\tau_i 
\end{align}
where $N$ is the number of experiment units, $n$ is the number of sets, and $w_i = \frac{1+m_i}{N}$.\\
Then the stimator is $\hat{\tau} = \sum^n_{i=1} w_i\hat{\tau_i}$, which is unbiased.

$\Rightarrow$ How are $m_i$'s determined?

\subsection{Observational Studies}
Is caffeine bad for babies in uterus?\\
Women who consume caffeine more likely to have a miscarriage?
\subsubsection{Observation Study}
\textit{\textbf{Why?}}
\begin{enumerate}
    \item Caffeine is bad.
    \item Selected for women without nausea.
    \begin{itemize}
        \item Nausea is predictive of good outcomes.
    \end{itemize}
\end{enumerate}
\textit{\textbf{A lot of casual inference problems:}}
\begin{enumerate}
    \item Choose target parameter (scientific).
    \item Identification
    \item Estimation and inference
\end{enumerate}

\paragraph{Example:}
In this example, denote $Y=$ miscarriage, $Z =$ drinks more than 2 cups of coffee.
\begin{itemize}
    \item $\frac{\mathbb{E}[Y(1)]}{\mathbb{E}[Y(0)]}$
    \item $\mathbb{E}[Y(1)] - \mathbb{E}[Y(0)]$
    \item $\mathbb{E}[Y(1) - Y(0) \mid \text{high-risk}]$
\end{itemize}
Step 1) $\mathbb{E}[Y(1)] - \mathbb{E}[Y(0)]$\\
Step 2) Identification

\subsubsection{Definition}
A parameter $\tau$ is identifiable if it can be written as a function of the observed data distribution.

\paragraph{Continue with example}
$X = $ confounding factor(nausea) \textrightarrow\  affect both $Y$ and $Z$.\\
$\mathbb{E}[Y(1) - Y(0) \mid\ X = 1]$ vs. $\mathbb{E}[Y \mid\ X = 1, Z = 1] - \mathbb{E}[Y \mid\ X = 1, Z = 0]$ holds when $Y(Z) \perp Z \mid\ X$ \\
\textbf{Proof:}
\begin{align}
    \mathbb{E}[Y(1) - Y(0)] &= \mathbb{E}[\mathbb{E}[Y(1) - Y(0) \mid\ X]]\\
    &= \mathbb{E}[\mathbb{E}[Y(1) \mid\ X] - \mathbb{E}[ Y(0) \mid\ X]]\\
    &= \mathbb{E}[\mathbb{E}[Y(1) \mid\ Z = 1, X] - \mathbb{E}[Y(0) \mid\ X, Z = 0]]\\
    &= \mathbb{E}[\mathbb{E}[Y \mid\ Z = 1, X] - \mathbb{E}[Y \mid\ Z = 0, X]]
\end{align}

$Y(z) \perp Z \mid\ X$:
\begin{itemize}
    \item Exchangeability
    \item Ignorability
    \item No unmeasured confounding
    \item Selection on observable
    \item Unconfoundedness
    \item Conditional independence
    \item Conditional exchangeability
\end{itemize}

$\bullet$ Untestable:
$$\mathbb{P} (Y(1) = 1 \mid\ Z = 1, X) = \mathbb{P} (Y(1) = 1 \mid\ Z = 0, X)$$\\
We can compute the RHS but not the LHS. \\
$\bullet$ Have we measured in $X$ all factors that affect both $Z$ and potential outcomes?
$$Y(1) = g_1 (x, w)$$
$$Y(0) = g_0 (x, u)$$
$$ z = g (x, v)$$
\begin{enumerate}
    \item exchangeability if $(w, u) \perp v$

eg: $V$ = preference for coffee 

If $Y(z) \not\perp Z \mid\ X$ but we pretend it is, then we get \textbf{omitted variable bias}.\\
Another assumption: 
    \item positivity / overlap: 
$$\mathbb{P}(0 < \mathbb{P} (Z = 1 \mid\ X) < 1) = 1$$ * doesn't require every factor to be accounted for\\
This equation is testable.
    \item SUTVA
\end{enumerate} 
\subsection{Target trial emulation}
What is the hypothetical ideal randomized experiment that our observational study is trying to emulate?

To be continued...




\section{Lecture Seven: Observational Studies (cont.)}{Youyou Xu \& Zhiwei Xiao}

\subsection{Introduction}

We continue our discussion to observational studies. Recall that observational studies are used instead of the ``ideal'' randomized experiments in several situations, primarily due to ethical, practical, or logistical constraints. 

\subsubsection{Example}
Denote outcome $Y$ as miscarriage, and treatment assignment $Z$ represent pregnant women that are coffee drinker or not. We are interested in the causal effect, that is the miscarriage rate if all women drink coffee versus if all women do not drink coffee: $\tau = \mathbb{E}[Y(1)-Y(0)].$

\subsubsection{Trial Emulation}
To target trial emulation, we want to answer what is the hypothetical randomized experiment that our observational study is trying to emulate?

\begin{enumerate}[label=\textbf{\arabic*}]
        \item Specify target trial
        \item Justify how the observational data can be used to emulate the trial
\end{enumerate}


\subsubsection{Another Example}
We want to study the effect of race in criminal justice system. Let outcome $Y$ represent ``stopped by the police'', and let the $Z$ represent race. Again, we are interested in estimating the causal effect $\tau = \mathbb{E}[Y(1)-Y(0)],$ where we can view $Y(1)$ as the response for one race and $Y(0)$ as the response for another. See that it is very tricky to estimate in this case, as race is a very complex social construct. It's very hard for researchers to intervene on this attribute. And it's difficult to well-define what constitutes as a treatment in this case. 

\subsection{Target Trial Emulation}
Target trial emulation helps us make interventions well-defined so that our causal questions are well-defined. Related readings on ``no causation without manipulation'' : Rubin(1975), Holland (1986).

\subsubsection{Steps}

\begin{enumerate}[label=\textbf{\arabic*}]
        \item Define causal estimand
        \item Identification 
        \item Estimation. 
\end{enumerate}

\subsubsection{Example of Estimation}
One approach is the outcome regression estimator (so called plug-in estimator, outcome modeling), motivated by the identification for $\tau$ we've seen before. The target estimand is again $\tau = \mathbb{E}[Y(1)-Y(0)].$ First, see that $\mathbb{E}[\mathbb{E}[Y|X,Z= 1]] = \mathbb{E}[Y|Z=1]$, which follows from the tower property. Then we have that 
$$\tau = \mathbb{E}[\underbrace{\mathbb{E}[Y|X,Z=1]}_{\mu_1(x)} - \underbrace{\mathbb{E}[Y|X,Z=0]}_{\mu_0(x)}] \quad \text{where } \mu_z \text{ is the nuisance function.}$$

Hence, $\tau = \mathbb{E}[\mu_1(x) - \mu_0(x)]$, and the estimator for the average treatment effect is $\hat{\tau}_{reg} = \frac{1}{n}\sum_{i = 1}^{n}[\hat{\mu_1}(x) - \hat{\mu_0}(x)]$ where $\hat{\mu_1}(x)$ is a regression model of $Y \sim X|Z=1$ and $\hat{\mu_0}(x)$ is a regression model of $Y \sim X|Z=0$. Keep in mind that different regression models all have their own pros and cons.

\textbf{Example}: suppose we have a binary outcome, with logistic regression, $$\P(Y =1 | Z, X) = \frac{e^{\beta_0 + \beta_z z + \beta_x^{T} x}}{1 + e^{\beta_0 + \beta_z z + \beta_x^{T} x}}.$$

Now can we take the $\beta_z$ as our causal effect estimate? No, there are two problems with this. One unavoidable problem is that $$\hat{\tau}_{reg} = \frac{1}{n}\sum_{i = 1}^{n}\{\frac{e^{\hat{\beta_0} + \hat{\beta_z} + \hat{\beta_x}^{T} x_i}}{1 + e^{\hat{\beta_0} + \hat{\beta_z} + \hat{\beta_x}^{T} x_i}} - \frac{e^{\hat{\beta_0} + \hat{\beta_x}^{T} x_i}}{1 + e^{\hat{\beta_0} + \hat{\beta_x}^{T} x_i}}\}.$$

If the logistic outcome model is correct, $\beta_z$ would be equal to the log odds ratio and $\exp(\beta_z) = \frac{\text{odds}(Y(1) = 1 | X)}{\text{odds}(Y(0) = 1 | X)}$. Another problem is that the strong parametric assumption can be misspecified. 

Alternatives: random forest, kernel regression, etc.


\subsection{Alternative Modeling}

Alternative to the logistic regression model which introduces parametric assumptions, there are other non-parametric models to estimate \(\beta_z\). Two examples are random forest model and kernel regression.

\begin{table}[h!]
\centering
\resizebox{\textwidth}{!}{%
\begin{tabular}{|c|l|l|}
\hline
\textbf{Model Type} & \textbf{Pros} & \textbf{Cons} \\
\hline
\textbf{Parametric Models} & 
\begin{tabular}[c]{@{}l@{}}- Easier to estimate standard errors \\ - Easier to use in high-dimensional settings\end{tabular} & 
\begin{tabular}[c]{@{}l@{}}- Parametric assumptions may be misspecified \\ - High bias\end{tabular} \\
\hline
\textbf{Non-Parametric Models} & 
\begin{tabular}[c]{@{}l@{}}- Low bias given no parametric assumptions\end{tabular} & 
\begin{tabular}[c]{@{}l@{}}- Harder to estimate standard errors \\ - Curse of high dimensionality\end{tabular} \\
\hline
\end{tabular}%
}
\caption{Comparison of Parametric and Non-Parametric Models}
\end{table}

Notably, the performance of non-parametric models worsens exponentially as dimensionality increases.

\subsection{Propensity Scores}
Another estimation approach involves propensity scores, which reweighs the sample groups to create a "pseudo-population".

\subsubsection{Question Setup}
We can estimate \(\mathbb{E}[Y(1)|Z=1] = \frac{1}{n_1}\sum^n_{i=1}z_i y_i\) and \(\mathbb{E}[Y(0)|Z=0] = \frac{1}{n_0}\sum^n_{i=1}(1-z_i)y_i\). Recall that the estimand is \(\tau = \mathbb{E}[Y(1)-Y(0)]\). Then, we reweigh the sample such that it creates a "pseudo-population" where the treatment assignment is as if it were random (unconfounded), given the covariates \(X\). The weights are defined by the inverse of propensity scores of getting treatment or control condition. The propensity score \(e(X)\) is:

\begin{equation}
    e(X) = \mathbb{P}(Z=1 | X)
\end{equation}
defined by Rosenbaum and Rubin (1983), and it follows that
\begin{equation}
    \tau = \mathbb{E}[Y(1)-Y(0)] = \mathbb{E}[\frac{Z\cdot Y}{e(X)}-\frac{(1-Z)\cdot Y }{1-e(X)}]
\end{equation}

\subsubsection{Proof of Eq.(1.2)}
We first prove that \(\mathbb{E}[Y(1)] = \mathbb{E}[\frac{Z\cdot Y}{e(X)}]\), and we proceed by leveraging the law of iterated expectations and properties of propensity scores. 

By definition, \(\mathbb{E}[Y(1)]\) represents the expected outcome if every individual were treated (\(Z = 1\)). Using \textbf{the law of iterated expectations}, we can write:
\[
\mathbb{E}\left[\frac{Z \cdot Y}{e(X)}\right] = \mathbb{E}\left[\mathbb{E}\left[\frac{Z \cdot Y}{e(X)} \bigg| X\right]\right].
\]
We now evaluate an expectation conditional on \(X\). Given the assumption of \textbf{ignorability}, conditioning on \(X\), we can express the expectation as:
\[
\mathbb{E}\left[\mathbb{E}\left[\frac{Z \cdot Y}{e(X)} \bigg| X\right]\right] = \mathbb{E}\left[\frac{1}{e(X)} \cdot \mathbb{E}[Z \cdot Y | X]\right].
\]
Since \(Z\) is a binary variable that takes the value 1 when the individual is treated, we know:
\[
\mathbb{E}[Z \cdot Y | X] = e(X) \cdot \mathbb{E}[Y(1) | X],
\]
because \(Z = 1\) with probability \(e(X)\) (the propensity score), and when \(Z = 1\), the expected outcome is \(Y(1)\).
Thus, the expectation simplifies to:
\[
\mathbb{E}\left[\frac{1}{e(X)} \cdot e(X) \cdot \mathbb{E}[Y(1) | X]\right] = \mathbb{E}\left[\mathbb{E}[Y(1) | X]\right] = \mathbb{E}[Y(1)].
\]
Therefore, we have shown that:
\[
\mathbb{E}\left[\frac{Z \cdot Y}{e(X)}\right] = \mathbb{E}[Y(1)].
\]
This completes the proof for the first part. Similarly, the proof for \(\mathbb{E}[Y(0)] = \mathbb{E}\left[\frac{(1-Z) \cdot Y}{1-e(X)}\right]\) follows the same steps. Given these two equations, we are successful in proving \textbf{Eq.(1.2)}.

\subsubsection{An IPW Estimator}
To account for the fact that we rarely have access to the true propensity score values in practice, we estimate the propensity scores, \( e(X) = \mathbb{P}(Z = 1 | X) \), using a model, typically a logistic regression or other classification model. Once we have the estimated propensity scores \(\hat{e}(X)\), we can use them to construct an Inverse Probability Weighting (IPW) estimator, also known as the Horvitz-Thompson estimator.
\begin{equation}
\tau_{\text{IPW}} = \frac{1}{n} \sum_{i=1}^{n} \left( \frac{Z_i \cdot Y_i}{\hat{e}(X_i)} - \frac{(1-Z_i) \cdot Y_i}{1 - \hat{e}(X_i)} \right)
\end{equation}
The empirical value of the score \(\hat{e}(X)\) can be obtained via regressing \(Z\) on \(X\).

\subsubsection{Challenges to Propensity Scores}
There are several challenges to the model of propensity scores:
\begin{enumerate}
    \item Is the propensity score model correct?
    \item Propensity scores are not invariant to local transformations. One solution to this issue is normalizing the weights, thereby proposing an alternative \textbf{Hajek estimator}:
    \begin{equation}
        \tau_{\text{Hajek}} = \frac{\sum_{i=1}^{n} \left( \frac{Z_i \cdot Y_i}{\hat{e}(X_i)} \right)}{\sum_{i=1}^{n} \frac{Z_i}{\hat{e}(X_i)}} - \frac{\sum_{i=1}^{n} \left( \frac{(1-Z_i) \cdot Y_i}{1 - \hat{e}(X_i)} \right)}{\sum_{i=1}^{n} \frac{1-Z_i}{1 - \hat{e}(X_i)}}
    \end{equation}
    \item Empirically, \(e(X)\) often come close to 0 or Two solutions exist for this issue:
    \begin{enumerate}
        \item \textbf{Truncate propensity scores.} In truncation, we replace the extreme values of the propensity scores with \(\alpha_L\) (lower bound) or \(\alpha_U\) (upper bound). The truncated propensity score \(e_{\text{trunc}}(X_i)\) is:
            \[
            e_{\text{trunc}}(X_i) = \max\left( \alpha_L, \min\left( e(X_i), \alpha_U \right)\right)
            \]
        \item \textbf{Trim observations.} In trimming, we remove observations where the estimated propensity score falls outside the range \([\alpha_L, \alpha_U]\). That is, we trim any observation \(i\) for which:
            \[
            e(X_i) < \alpha_L \quad \text{or} \quad e(X_i) > \alpha_U
            \]
        For possible upper and lower bound values, Crump et al. (2009) has used \(\{\alpha_L = 0.1, \alpha_U = 0.9\}\), and Klumb et al. (2005) has used \(\{\alpha_L = 0.05, \alpha_U = 0.95\}\).
    \end{enumerate}
\end{enumerate}

\subsubsection{Alternative Definition of Propensity Scores}
An alternative definition to \textbf{Eq.(1.1)} that incorporates potential outcomes is
\begin{equation}
    e(X, Y(1), Y(0)) = \mathbb{P}(Z=1|X,Y(1),Y(0))
\end{equation}
The propensity score defined by this equation is sometime called the "true propensities" while Eq.(1.1) is "nominal propensities".

Note that the two definitions are equal when:
\[
\mathbb{P}(Z=1|X,Y(1),Y(0)) = \mathbb{P}(Z=1|X) \quad \text{if} \quad Y(1),Y(0) \perp Z|X
\]

The next lecture will explore the great features of the propensity score model, including the fact that it offers a great dimensionality reduction tool, which is motivated by:

\[
\textbf{Thereom:} \quad \text{if} Z \perp (Y(0),Y(1)) | X \quad \text{then} Z \perp (Y(0),Y(1)) | e(X) 
\]

\section{Lecture Eight: Propensity Scores (cont.)}
{Dudu Tang \& Brian Fernando \& Lucas Costa}

\subsection{Final Project \& Logistics}

Group assignment for STAT 156 final project has been released. Please reach out to your group members and start discussing your project topics. Please review the Group Project Guideline (on course website) for more details. 

\subsection{Questions From Last Lecture}
\subsubsection{Target Trial Emulation}
    What is the relationship between Target Trial Emulation (TTE) and Stable Unit Treatment Values Assumption (SUTVA)?\\
    TTE encompasses a broader set of assumptions than SUTVA. TTE relies on SUTVA for the causal effects estimated in a target trial emulation to be valid. \\\\
    Assumptions:
    \begin{itemize}
        \item There is no interference between units
        \item There is a consistent level of treatment
    \end{itemize}

\subsubsection{Nominal Propensity Scores vs. True Propensity Scores}

\textbf{Nominal Propensity Score:} 
$$e_{n}(x) = P(Z = 1 | X = x)$$

\textbf{True Propensity Score:}
$$e_{p}(x, Y(1), Y(0)) = P(Z = 1 | X = x, Y(1), Y(0))$$

Let X denote the confounding factor: Morning Sickness.\\
Let Z denote the treatment assignment: Drinking Caffeine\\
Let Y denote the outcome: Miscarriage\\
We can see that X can affect both Y and Z. However, they may also be an unobserved confounding factor U that can affect both Y and Z. We can denote U as being cautious.

\begin{center}
\begin{tikzpicture}[scale=0.2]
\tikzstyle{every node}+=[inner sep=0pt]
\draw [black] (21.5,-21.3) circle (3);
\draw (21.5,-21.3) node {$X$};
\draw [black] (37.4,-21.3) circle (3);
\draw (37.4,-21.3) node {$Z$};
\draw [black] (37.4,-32.6) circle (3);
\draw (37.4,-32.6) node {$Y$};
\draw [black] (21.5,-32.6) circle (3);
\draw (21.5,-32.6) node {$U$};
\draw [black] (21.5,-32.6) circle (2.4);
\draw [black] (24.5,-21.3) -- (34.4,-21.3);
\fill [black] (34.4,-21.3) -- (33.6,-20.8) -- (33.6,-21.8);
\draw [black] (24.1,-23.2) -- (34.95,-30.87);
\fill [black] (34.95,-30.87) -- (34.59,-30) -- (34.01,-30.82);
\draw [black] (23.9,-30.8) -- (34.95,-23.03);
\fill [black] (34.95,-23.03) -- (34,-23.08) -- (34.58,-23.9);
\draw [black] (24.6,-32.6) -- (34.4,-32.6);
\fill [black] (34.4,-32.6) -- (33.6,-32.1) -- (33.6,-33.1);
\end{tikzpicture}
\end{center}

\textit{Example Nominal Propensity Score:}
$e_n(x) = P(Z = 1 | Morning Sickness = Yes)$\\\\
In this case, even if $X \perp Z | Y(0), Y(1)$, $e_n(x)$ is not sufficient for ignorability due to the unobserved confounding factor U. Therefore, adjusting for propensity scores may not remove all the bias in estimating the treatment effect.

\subsection{More on Propensity Scores}

\subsubsection{A Dimension Reduction Perspective on Propensity Scores}
From the dimension reduction viewpoint, propensity scores help simplify the dimensional problem of covariate adjustment by condensing multiple high-dimensional covariates into a single propensity score — the probability of receiving the treatment given the covariates X. The theorem below reduces the dimension of covariates while maintaining ignorability.
\\\\
\textbf{Theorem:}
\begin{center}
    If $Z \perp (Y(1), Y(0)) \thinspace|\thinspace X$
then $Z \perp (Y(1), Y(0)) \thinspace|\thinspace e(X)$
\end{center}

\subsubsection{Propensity Score Stratification}
We now introduce a method for estimating causal effect called Propensity Score Stratification.
\begin{enumerate}
    \item Estimate propensity score by regressing $Z \sim X \rightarrow \hat{e}(x)$
    \item Discretize $\hat{e}(x)$ into k quantities $\rightarrow \hat{e}'(x) = e_k$
    \item Analyze as SRE (Stratified Random Experiment)
\end{enumerate}

We see that $Z \perp (Y(1), Y(0)) \thinspace|\thinspace \hat{e}'(x) = e_k$ approximately holds for $k = 1,2,...,k$\\\\
Empirically, it often appears that getting the correct ordering of propensity scores is more important than obtaining the exact values. The reason lies in how propensity scores are used in practice: their primary function is to balance covariates between treatment and control groups. This balancing typically relies more on the relative ranking of the propensity scores than their precise numerical values.


\textbf{Key Questions:}
\begin{enumerate}
    \item How to choose k?
    \begin{itemize}
        \item If k is too small, then ignorability will be violated
        \item If k is too large, then we cannot analyze as SRE
        \item In 1985, Rosenbaum and Rubin published a widely regarded paper that recommended to use $k=5$
        \item We can increase k as long as each stratum has enough control and treated units
        \item We can decide k after looking at X and Z, however, we should fix k before looking at the outcome Y
    \end{itemize}
    \item How to estimate standard errors when propensity score is estimated?
    \begin{enumerate}
        \item \underline{Use SRE as a conservative estimate}; The estimate is conservative because estimated propensity scores decrease asymptotic variance for estimating causal effect.
        \item \underline{Bootstrap (include propensity score estimation)}; This theory is unclear due to discreteness of the estimator.
    \end{enumerate}
\end{enumerate}

\subsubsection{A Covariate Balancing Perspective on Propensity Scores}
The covariate balancing perspective focuses on the role of propensity scores in balancing covariates between treated and untreated groups. The idea is that once individuals are matched or stratified based on their propensity scores, the distribution of covariates should be similar between the two groups, as would be the case in a randomized controlled trial.

$$ Z \perp X | e(X)$$

\begin{itemize}
    \item Within the same level of $e(x)$, covariate distributions are balanced in expectation across treatment and control
    \item In practice, we can check for covariate balance as a check on how good two propensity score model is
\end{itemize}

\textbf{When to use propensity score models vs. outcome models?}
\begin{enumerate}
    \item Depends on how well you think you can estimate one vs. the other
    \item Do both and see if their causal estimates align:
    \begin{itemize}
        \item If their causal estimates differ, this suggests misspecification
        \item If estimates align, maybe we got the right causal estimate...
    \end{itemize}
    \item Can we combine the propensity score model estimates and outcome regression estimate to get a better estimate of the causal effect?
\end{enumerate}

\subsection{Doubly-robust Estimators}
Doubly-robust estimators are also known as:
\begin{enumerate}
    \item Bias-corrected plugin estimators (\emph{plugin} is the other word for \emph{outcome regression})
    \item Model-assisted Horvitz-Thompson estimators \emph{or} Model-assisted IPW estimators
    \item Semi-parametric estimators \emph{or} Semi-parametric efficient estimators
    \item Augmented IPW (AIPW)
    \item Double machine learning estimators
    \item Debiased machine learning estimators
    \item Orthogonal machine learning estimators
\end{enumerate}

\subsubsection{Introducing $\hat{\tau}_{\text{DR}}$}
Recall the formulas for the \textbf{IPW estimator} and the \textbf{outcome regression estimator} are:
\begin{equation}
\hat{\tau}_{\text{IPW}} = \frac{1}{n} \sum_{i=1}^{n} \frac{Z_i \cdot Y_i}{\hat{e}(X_i)} - \frac{1}{n}\sum_{i=1}^{n}\frac{(1-Z_i) \cdot Y_i}{1 - \hat{e}(X_i)}
\end{equation}
\begin{equation}
\hat{\tau}_{\text{Reg}} = \frac{1}{n} \sum_{i=1}^{n} \left( \hat{\mu}_1 \left(X_i\right) - \hat{\mu}_0 \left(X_i\right) \right)
\end{equation}

One of the reasons why we are using the \textbf{outcome regression estimator} is that some of the potential outcomes are missing. Thus we have this formula where each term in the summation is an estimate of the treatment effect for that particular unit. 

However, when we have already observed one potential outcome for every unit, why are we still using the predicted outcomes instead of using whichever we've observed? Again, the observed potential outcome is an observation of the conditional mean. We can reduce variance by directly estimating the conditional mean and plugging in that estimate here. However, we cannot estimate the conditional mean perfectly, so we are trading off the lower variance for a bias in estimation.

Here comes another question: is there really a way that we can use the observed outcomes but still get some nice bias-variance trade-offs? To solve this question, we introduce the \textbf{Doubly-robust estimator} where we use the errors in the estimates of the outcome regression model as a bias correction.

\subsubsection{Formula \& Interpretation}
Formula for the \textbf{Doubly-robust estimator} is:
\begin{equation}
\hat{\tau}_{\text{DR}} = \frac{1}{n} \sum_{i=1}^{n} \{ 
\hat{\mu}_1\left(X_i\right) + \frac{Z_i\left(Y_i - \mu_i\right)}{\hat{e}\left(X_i\right)} - \hat{\mu}_0\left(X_i\right) - \frac{\left(1 - Z_i\right)\left(Y_i - \hat{\mu}_0\left(X_i\right)\right)}{1 - \hat{e}\left(X_i\right)}
\}
\end{equation}

Here we can see that $\hat{\mu}_1\left(X_i\right)$ and $\hat{\mu}_0\left(X_i\right)$ are just the outcome regression estimator. 

The second and fourth terms, $\frac{Z_i\left(Y_i - \mu_i\right)}{\hat{e}\left(X_i\right)}$ and $\frac{\left(1 - Z_i\right)\left(Y_i - \hat{\mu}_0\left(X_i\right)\right)}{1 - \hat{e}\left(X_i\right)}$, display that $Z_i$ are selecting units that received treatments and they are inversely weighing by their propensity score of treatment and control, respectively. 

You can view the two terms as the IPW-weighed, bias-correction terms that are plugged in to reduce the bias of the \textbf{outcome regression model}. This explains why the \textbf{doubly-robust estimator} is also known as the \textbf{bias-corrected plugin estimator}. 

We can also rearrange the terms of the equation $\hat{\tau}_{\text{DR}}$:
\begin{equation}
\hat{\tau}_{\text{DR}} = \frac{1}{n} \sum_{i=1}^{n} \{ 
\frac{Z_i Y_i}{\hat{e}\left(X_i\right)} - \frac{Z_i - \hat{e}\left(X_i\right)}{\hat{e}\left(X_i\right)}\hat{\mu}_1\left(X_i\right) - \frac{\left(1 - Z_i\right) Y_i}{1 - \hat{e}\left(X_i\right)} - \frac{\left(\hat{e}\left(X_i\right) - Z_i\right)}{1 - \hat{e}\left(X_i\right)}\hat{\mu}_0\left(X_i\right)
\}
\end{equation}

where the first and third terms are usual IPW terms. The second and fourth terms here are the \emph{augmentation terms} that are augmenting the IPW by incorporating the outcome regression model estimates. This explains why the \textbf{doubly-robust estimator} is also known as the \textbf{augmented IPW (AIPW) estimator}. 


\textbf{Relationship with Machine Learning}

It's popular to use Machine Learning to estimate functions such as propensity scores, so people refer to $\hat{\tau}_{\text{DR}}$ also as the \textbf{double machine learning} and \textbf{debiased machine learning estimators} (bias-corrected as proved above).





\section{Lecture Nine: Doubly Robust Estimator (cont.)}
{Taejun Lee \& Zach Rewolinski \& Reet Mishra}

\subsection{Doubly Robust Estimator}
From last lecture:

\[\hat{\tau}_{\text{DR}} = \frac{1}{n} \sum\limits_{i=1}^{n} \left\{\hat{\mu}_1(X_i) + \frac{Z_i(Y_i - \hat{\mu}_1(X_i))}{\hat{e}(X_i)} - \hat{\mu}_0(X_i) - \frac{(1 - Z_i)(Y_i - \hat{\mu}_0(X_i))}{1 - \hat{e}(X_i)}\right\}\]

$\hat{\tau}_{DR}$ is flexible, avoids parametric assumptions, and gets confidence intervals. We call the doubly robust estimator ``double machine learning" because we use machine learning to estimate the nuisance functions $e(X)$, $\mu_0(X)$, and $\mu_1(X)$.

\subsubsection{Advantages of Doubly Robust Estimator (in Parametric Setting)}
\begin{itemize}
    \item Robust to misspecification in either $e(X)$ or $(\mu_1(X),\mu_0(X))$ (hence ``doubly robust").
    \item Can quantify confidence intervals on causal effect.
\end{itemize}

\subsubsection{Proof of Robustness to Misspecification}
\[\hat{\tau}_{\text{DR}} = \frac{1}{n} \sum_{i=1}^{n} \left\{\frac{Z_i(Y_i - \hat{\mu}_1(X_i))}{\hat{e}(X_i)} - \frac{(1 - Z_i)(Y_i - \hat{\mu}_0(X_i))}{1 - \hat{e}(X_i)} + \hat{\mu}_1(X_i) - \hat{\mu}_0(X_i)\right\}\]
\[\mathbb{E}\left[\hat{\tau}_{\text{DR}}\right] = \mathbb{E}\left[\frac{Z(Y - \hat{\mu}_1(X))}{\hat{e}(X)} - \frac{(1 - Z)(Y - \hat{\mu}_0(X))}{1 - \hat{e}(X)} + \hat{\mu}_1(X) - \hat{\mu}_0(X)\right]\]

\textbf{Case 1: $\hat{\mu}_1(X) = \mu_1(X)$ and $\hat{\mu}_0(X) = \mu_0(X)$}
\begin{align*}
    \mathbb{E}\left[\hat{\tau}_{\text{DR}}\right] &= \mathbb{E}\left[\frac{Z(Y - \mu_1(X))}{\hat{e}(X)} - \frac{(1 - Z)(Y - \mu_0(X))}{1 - \hat{e}(X)} + \mu_1(X) - \mu_0(X)\right] \\
    &= \mathbb{E}\left[\mathbb{E}\left[\frac{Z(Y - \mu_1(X))}{\hat{e}(X)} - \frac{(1 - Z)(Y - \mu_0(X))}{1 - \hat{e}(X)} + \mu_1(X) - \mu_0(X) \, \middle| \, X, Z\right]\right] \\
    &= \mathbb{E}\left[\frac{Z}{\hat{e}(X)}\mathbb{E}[Y \, | \, X, Z = 1] - \frac{Z}{\hat{e}(X)}\mu_1(X)\right. \\
    &\quad \left. - \left(\frac{1 - Z}{1 - \hat{e}(X)}\mathbb{E}[Y \, | \, X, Z = 0] - \frac{1 - Z}{1 - \hat{e}(X)}\mu_0(X)\right) + \mu_1(X) - \mu_0(X)\right] \\
    &= \mathbb{E}[\mu_1(X) - \mu_0(X)] \\
    &= \tau
\end{align*}

\textbf{Case 2: $\hat{e}(X) = e(X)$}
\begin{align*}
    \mathbb{E}\left[\hat{\tau}_{\text{DR}}\right] &= \mathbb{E}\left[\frac{Z}{\hat{e}(X)}(\mu_1(X) - \hat{\mu}_1(X)) - \frac{(1 - Z)}{1 - \hat{e}(X)}(\mu_0(X) - \hat{\mu}_0(X)) + \hat{\mu}_1(X) - \hat{\mu}_0(X) \, \middle| \, X\right] \\
    &= \mathbb{E}\left[\frac{e(X)}{e(X)}(\mu_1(X) - \hat{\mu}_1(X)) - \frac{1-e(X)}{1-e(X)}(\mu_0(X) - \hat{\mu}_0(X)) + \hat{\mu}_1(X) - \hat{\mu}_0(X)\right] \\
    &= \mathbb{E}[\mu_1(X) - \mu_0(X)] \\
    &= \tau
\end{align*}

\underline{\textbf{Theorem: $\hat{\tau}_{\text{DR}} - \mathbb{E}[\hat{\tau}_{\text{DR}}] = 0$ if 1) $\hat{e}(X) = e(X)$ OR 2) $\hat{\mu}_1(X) = \mu_1(X)$ and $\hat{\mu}_0(X) = \mu_0(X)$}}

\subsubsection{A Few Perspectives on the Doubly Robust Estimator}

\begin{enumerate}
    \item Reduces the bias of the outcome regression estimator.

    Assume estimand is $\mathbb{E}[Y(1)]$. Then we have $\hat{\tau}_{reg}^1=\frac{1}{n}\sum_{i=1}^n\hat{\mu}_1(X_i)$, which has a bias of $\mathbb{E}\left[\hat{\tau}^1_{reg}-Y(1)\right]=\mathbb{E}\left[\hat{\mu}_1(X)-Y(1)\right]$. Notice that we cannot calculate this quantity because we cannot observe the potential outcomes. Thus, we instead do the following estimate of bias:

    \[\mathbb{E}\left[\frac{Z(\hat{\mu}_1(X)-Y)}{\hat{e}(X)}\right]=B.\]

    The resulting de-biased estimator is \[\frac{1}{n}\sum\limits_{i=1}^n\hat{\mu}_1(X_i)-\frac{1}{n}\sum\limits_{i=1}^n\frac{Z_i(\hat{\mu}_1(X_i)-Y_i)}{\hat{e}(X_i)}.\]

    \item Reduces the variance of the IPW estimator.

    Recall that \[\tau_{IPW}=\frac{1}{n}\sum\limits_{i=1}^n\frac{Z_iY_i}{\hat{e}(X_i)}-\frac{(1-Z_i)Y_i}{1-\hat{e}(X_i)}\] can have high variance because we may be dividing by something which is close to zero.

    Instead, we can divide the residual term by $\hat{e}(X_i)$ so that it blows up less in the event that $\hat{e}(X_i)$ is close to zero or one:

    \[\hat{\tau}^1_{DR}=\frac{1}{n}\sum\limits_{i=1}^n\frac{Z_i(Y_i-\hat{\mu}_1(X_i))}{\hat{e}(X_i)}+\hat{\mu}_1(X).\]
\end{enumerate}

\subsubsection{Confidence Intervals}

\begin{enumerate}
    \item Nonparametric bootstrap
    \item Wald-type normal approximation under certain conditions:
    \begin{itemize}
        \item Sample-splitting: We use one half of the sample to estimate $e(X), \mu_0(X), \mu_1(X)$ and use the other half to estimate $\hat{\tau}_{DR}$ given $\hat{e}(X), \hat{\mu}_0(X), \hat{\mu}_1(X)$. Repeat the process starting with the second half to estimate the nuisance functions and average the results.
        \item Flexible nonparametric methods to estimate nuisance functions $e(X), \mu_0(X), \mu_1(X)$.
    \end{itemize}
    $\hat{\tau}_{DR} - \hat{\tau}$ is asymptotically normal with variance $\frac{var(\phi(X,Y,Z))}{n}$ where

    \[\phi(X,Y,Z)=\frac{Z}{e(X)}(Y-\mu_1(X))-\frac{(1-Z)}{1-e(X)}(Y-\mu_0(X))+\mu_1(X)-\mu_0(X).\]
    
\end{enumerate}

Caution: doubly fragile in a parametric setting

Bias: $\hat{\tau}^1_{DR}-\mathbb{E}[Y(1)]=\mathbb{E}\left[\frac{e(X)-\hat{e(X)}}{e(X)}(\mu_1(X)-\hat{\mu}_1(X))\right]$

Kang \& Schafer (2005) showed this product can make errors large when both $\hat{e}(X)$ and $\hat{\mu}_1(X)$ are misspecified.

\subsection{Causal Estimands Beyond $E[Y(1) - Y(0)]$}

Another estimand of interest is $\tau_T=\mathbb{E}[Y(1)-Y(0)|Z=1]$
\begin{itemize}
    \item ``Average causal effect on the treated".
    \item Allows us to learn about effects of removing an exposure.
    \item Relevant when its infeasible to assign treatment to everyone.
    \begin{itemize}
        \item Example: lottery to attend magnet school.
        \item Example: invasive surgery for high-risk patients
    \end{itemize}
    \item Advantage: requires weaker identifying assumptions.
\end{itemize}

We now need to identify $\tau_T$, specifically the term in \textcolor{red}{red} below:

\[\tau_T=\mathbb{E}[Y\mid Z=1]-\textcolor{red}{\mathbb{E}[Y(0)\mid Z=1]}.\]

Assumptions:
\begin{itemize}
    \item One-sided ignorability: $Z\perp Y(0)\mid X$.
    \item One-sided positivity: $e(X)<1$.
\end{itemize}

\subsubsection{Theorem}

Under one-sided ignorability and positivity,
\begin{enumerate}
    \item $\tau_T=\mathbb{E}[Y\mid Z=1]-\mathbb{E}\left[\mathbb{E}[Y\mid Z=0,X]\mid Z=1\right]$
    \item $\tau_T=\mathbb{E}[Y\mid Z=1]-\mathbb{E}\left[\frac{e(X)}{P(Z=1)}*\frac{(1-Z)Y}{1-e(X)}\right]$
\end{enumerate}

\subsubsection{Proof of Theorem 1}

Part 1:

We want to show that $\mathbb{E}[Y\mid Z=1]=\mathbb{E}\left[\mathbb{E}[Y\mid Z=0,X]\mid Z=1\right]$.
\begin{proof}
\begin{align*}
    \mathbb{E}[Y\mid Z=1]&=\mathbb{E}\left[\mathbb{E}[Y(0)\mid Z=1,X]\mid Z=1\right]\\
    &=\mathbb{E}\left[\mathbb{E}[Y(0)\mid Z=0,X]\mid Z=1\right]\\
    &=\mathbb{E}\left[\mathbb{E}[Y\mid Z=0,X]\mid Z=1\right]\\
\end{align*}
\end{proof}

END OF LECTURE.


\section{Lecture Ten: Matching in Observational Studies}
{Nikhil Shanbhag \& Boyu Fan \& Yichen Xu}

\subsection{Estimator for Average Causal Effect on the Treated}

\subsubsection{Assumptions}

\begin{itemize}
    \item One-sided ignorability: $Z \perp Y(0) | X$
    \item One-sided positivity: $P(e(X) > 0) = 1$
\end{itemize}

\subsubsection{Definition}

Define estimator for the average causal effect on the treated group to be $$\tau_{T} = \mathbb{E}[Y(1) - Y(0) \mid Z = 1] = E[Y \mid Z = 1] - E[Y \mid Z = 0].$$

\subsubsection{Theorem}

$$\tau_T = \mathbb{E}[Y \mid Z = 1] - \mathbb{E}[\mathbb{E}[Y \mid Z = 0, X] \mid Z = 1] = \mathbb{E}[Y \mid Z = 1] - \mathbb{E}\left[\frac{e(X)}{P(Z=1)} \cdot \frac{(1-Z)}{(1-e(X))} Y \mid Z = 1\right].$$

\subsubsection{Proof}

One-sided ignorability implies $Y(0) \perp Z \mid X$ and one-sided positivity implies $P(e(X) > 1) = 1.$

\noindent To show the first line of the theorem, $$\mathbb{E}[Y(0) \mid Z = 1] = \mathbb{E}[\mathbb{E}[Y(0) \mid Z = 1, X] \mid Z = 1] = $$
$$\mathbb{E}[\mathbb{E}[Y(0) \mid Z = 0, X] \mid Z = 1] = \mathbb{E}[\mathbb{E}[Y \mid Z = 0, X] \mid Z = 1].$$

\noindent Notice we can condition on $X$ to rewrite: 
$$\mathbb{E}\left[\frac{ZY(0)}{e}\right] = \frac{\mathbb{E}\left[\mathbb{E}[ZY(0) \mid X]\right]}{e} = \frac{\mathbb{E}\left[\mathbb{E}[Z \mid X]\mathbb{E}[Y(0) \mid X]\right]}{e} = \frac{\mathbb{E}\left[e(X)\mathbb{E}[Y(0) \mid X]\right]}{e}.$$

\noindent The RHS can be written as $$\mathbb{E}\left[\frac{e(X)}{e} \cdot \frac{(1-Z)Y}{1-e(X)} \mid X\right] = \mathbb{E}\left[\frac{e(X)}{e} \cdot \frac{\mathbb{E}[(1-Z)Y \mid X]}{1-e(X)}\right] = \mathbb{E}\left[\frac{e(X)}{e} \mathbb{E}[Y \mid Z=0, X]\right] = $$

$$\mathbb{E}\left[\frac{e(X)}{e} \mathbb{E}[Y(0) \mid X]\right]$$ by using the Tower property. This completes the proof. 

\subsection{Regression estimator}

\subsubsection{Definition}

Consider $$\tau_T = \mathbb{E}[Y \mid Z = 1] - \mathbb{E}[\mathbb{E}[Y \mid Z = 0, X] \mid Z = 1].$$

Define $$\hat{\tau}_T^{\text{reg}} = \frac{1}{n_1} \sum_{i=1}^{n} Y_i Z_i - \frac{1}{n_1} \sum_{i=1}^{n} Z_i \hat{\mu}_0(X_i),$$ where $\hat{\mu}_0(X)$ is the outcome model for $E[Y|Z=0,X]$ learned by regressing $Y \sim X \mid Z=0.$

In order to compute $$\frac{1}{n_1} \sum_{i=1}^{n} Z_i \left(Y_i - \hat{\mu}_0(X_i)\right),$$ recall the IPW estimators and specifically the 2nd identification, which is $$\tau_{T} = E[Y \mid Z=1] = \mathbb{E}\left[\frac{e(X)}{e} \cdot \frac{(1-Z)Y}{1-e(X)}\right].$$ 

\subsubsection{Derivation of Other Estimators}
\noindent This allows us to derive three estimators: (1) Horvitz-Thompson, (2) Hajek, and (3) Odds Ratio. They can written as follows: 

$$\hat{\tau}_T^{\text{ht}} = \frac{1}{n_1} \sum_{i=1}^{n} Z_i Y_i - \frac{1}{n_1} \sum_{i=1}^{n} \frac{\hat{e}(X_i)(1-Z_i) Y_i}{1 - \hat{e}(X_i)}$$

$$\hat{\tau}^{\text{Hajek}} = \frac{1}{n_1} \sum_{i=1}^{n} Z_i Y_i - \frac{\sum_{i=1}^{n} \frac{\hat{e}(X_i)(1-Z_i)Y_i}{1 - \hat{e}(X_i)}}{\sum_{i=1}^{n} \frac{\hat{e}(X_i)(1-Z_i)}{1 - \hat{e}(X_i)}}$$

$$\tau^{\text{OR}} = \frac{1}{n} \sum_{i=1}^{n} Z_i Y_i + \text{DR}(\mathbb{E}[Y(0) \mid Z = 1]) = \frac{1}{n_1} \sum_{i=1}^{n} \left\{ \frac{\hat{e}(X_i)}{1 - \hat{e}(X_i)} (1 - Z_i)(Y_i - \hat{\mu}_0(X_i)) + Z_i \hat{\mu}_0(X_i) \right\},$$ where DR represents the doubly robust estimator.

\subsubsection{Theorem}

Under one-sided ignorability and overlap, if either $\hat{e}(X) = e(X)$ or $\hat{\mu_0}(X) = \mu_0(X)$, then this estimator is unbiased.

\subsubsection{Proof}

The following is an alternative way to rewrite the doubly robust estimator: $$\hat{\tau}_T^{\text{DR}} = \hat{\tau}_T^{\text{Reg}} - \frac{1}{n_1} \sum_{i=1}^{n} \left[\frac{\hat{e}(X_i)}{1 - \hat{e}(X_i)} (1-Z_i)(Y_i - \hat{\mu}_0(X_i))\right] = \hat{\tau}_T^{\text{ht}} - \frac{1}{n_1} \sum_{i=1}^{n} \left[\frac{\hat{e}(X_i)}{1 - \hat{e}(X_i)} (1-Z_i) Z_i \right]\hat{\mu}_0(X_i).$$

The causal effect on the overlap population is then determined by the estimand 
$$\tau_0 = \frac{\mathbb{E}[e(X)(1-e(X))\tau(X)]}{\mathbb{E}[e(X)(1-e(X))]},$$ where $\tau(X)$ represents largest weights for units with $e(X) = \frac{1}{2}.$

Recall: 
$$\tau = \mathbb{E}[Y(1) - Y(0)]$$
$$\tau_{T} = \mathbb{E}[Y(1)-Y(0) \mid Z=1]$$
$$\tau_0 = \frac{\mathbb{E}[e(X)(1-e(X))\tau(X)]}{\mathbb{E}[e(X)(1-e(X))]}.$$

\subsection{Heterogeneous Causal Effects and Effect Modification}

\subsubsection{Definitions}

Let $V$ be a covariate, e.g., male/female. The heterogeneous causal effect is defined as:
\[
\tau_{HCE} = \mathbb{E}[Y(1) - Y(0) | V = v]
\]

A covariate $V$ is called an \emph{effect modifier} if:
\[
\mathbb{E}[Y(1) - Y(0) | V = v] \neq \mathbb{E}[Y(1) - Y(0)]
\]
and $V$ is not affected by the treatment $Z$. If $V$ is affected by treatment, this would instead be called a mediator.

The value of $\tau_{HCE}$ depends on the causal estimand.

\subsubsection{Example}

\[
P(Y(0) = 1 | V = 1) = 0.8
\]
\[
P(Y(1) = 1 | V = 1) = 0.9
\]
\[
P(Y(0) = 1 | V = 0) = 0.1
\]
\[
P(Y(1) = 1 | V = 0) = 0.2
\]

The heterogeneous causal effect is:
\[
\tau_{HCE} = 0.1
\]

Now consider:
\[
\frac{P(Y(1) = 1 | V = 1)}{P(Y(0) = 1 | V = 1)} = \frac{0.9}{0.8} = \frac{9}{8}
\]
\[
\frac{P(Y(1) = 1 | V = 0)}{P(Y(0) = 1 | V = 0)} = 2
\]

We have ``effect measure modifier" and ``effect heterogeneity"

\subsubsection{Why Do We Care?}

\begin{enumerate}
    \item $\tau$ could be zero, but $\tau_{HCE} \neq 0$, meaning the effect differs for different subpopulations.
    \item We want to identify who benefits from the treatment.
\end{enumerate}

\subsubsection{Examples}

\paragraph{Moving to Opportunity (MTO) Experiment:} A randomized experiment gave vouchers to families in public housing to move to richer neighborhoods. One outcome was mental health. The results showed that mental health improved for females but not for males, indicating an effect modification based on gender.

\paragraph{Direct Cash Transfers in Kenya:} The effect of cash transfers on reducing childhood mortality was particularly strong for families who gave birth during the program. This suggests targeting families accordingly in future interventions. Additionally, the overall effect could be muted in geographies with fewer people of child-bearing age.

Effect modifiers are also referred to as \emph{effect-measure modifiers} or \emph{effect heterogeneity}.

\subsubsection{Definition}

Transportability refers to the extrapolation of causal effects computed in one population to a second. Lack of transportability corresponds to a lack of external validity.

\subsubsection{Example: Transportability Issue}

An example where transportability was questioned is the study by Smith and Pell (2003), which found no effect modifiers for the effect of parachutes on high-altitude jumping, showing that the reduction in death after a jump may not be generalizable across populations or situations.

\subsection{RCT vs. Observational Studies}

\subsubsection{Randomized Controlled Trials (RCT)}

\textbf{Advantages:}
\begin{itemize}
    \item High confidence in causal effect for a particular context.
\end{itemize}

\textbf{Disadvantages:}
\begin{itemize}
    \item Expensive to scale.
\end{itemize}

\subsubsection{Observational Studies}

\textbf{Advantages:}
\begin{itemize}
    \item Easy to scale.
    \item Allows investigation across different contexts.
\end{itemize}

\textbf{Disadvantages:}
\begin{itemize}
    \item Make strong assumptions.
\end{itemize}

END OF LECTURE.



\section{Lecture Eleven: Matching in Observational Studies \& Causal DAGs}{Grace Yin \& Haodong Ling \& Zhengxing Cheng}

\subsection{Last Time:}
- RCT shows conflicting results. So the solution is using a meta-analysis of RCTs and observational studies clarified that certain outcomes, like pulmonary embolism, increased under HRT, while others, such as myocardial infarction, decreased.


\subsection{Matching in Observational Studies}
\begin{itemize}
    \item \textbf{Goal}: Construct a subset of the population in which covariates have the same distribution in the treated and control groups.
    \item Intuitive and interpretable.
\end{itemize}

\subsubsection{Matching with Many More Control Units}

%\subsubsection{Ideal Settings}
\begin{center}
\begin{tikzpicture}

% Circles for treated and control groups
\draw[thick] (-3,0) circle [radius=1.6]; % treated group
\draw[thick] (3,0) circle [radius=2.5]; % control group

% Labels for treated group
\node at (-3,2) {treated};
\node at (-3,0.8) {$X_1$};
\node at (-3,0.3) {$X_2$};
\node at (-3,-0.2) {$X_3$};
\node at (-3,-0.8) {$\vdots$};
\node at (-3,-1.3) {$X_n$};

% Labels for control group
\node at (3,2.8) {control};
\node at (3,1.6) {$X_{M(1)}$};
\node at (3,0.8) {$X_{M(2)}$};
\node at (3,-0.4) {$\vdots$};
\node at (3,-1.6) {$X_{M(n)}$};

% Matching lines
\draw[->] (-2.7,0.8) to[out=30,in=150] (2.5,1.7); % X1 to Xm(1)
\draw[->] (-2.7,0.3) to[out=30,in=150] (2.5,0.9); % X2 to Xm(2)
\draw[->] (-2.7,-1.3) to[out=-30,in=180] (2.5,-1.6); % Xn to Xm(n)

\end{tikzpicture}
\end{center}

where $X_i$ is matched to $X_{M(i)}$.

\begin{itemize}
\item \textbf{Ideal Settings:} we would find exact matches: $X_i = X_{M(i)} \Rightarrow e(X_i) = e(X_{M(i)})$.
    \begin{itemize}
        \item Matches pair experiment assuming $Y(0), Y(1) \perp Z | X$.
        \item Positivity holds in matching by construction.
    \end{itemize}
\item \textbf{More Realistic Settings: Imperfect Matching }
    \begin{itemize}
    \item Almost always the case in observational studies because the covariate space can be large (e.g., continuous features).
    \[
    m(i) = \arg \min_{k: Z_k = 0} d(X_i, X_k)
\]
where $d(X_i, X_k)$ is a distance metric.
    \end{itemize}
\end{itemize}

%\subsubsection{Realistic Settings: Imperfect Matching}
%\begin{itemize}
%    \item Ideally, we would find exact matches: $X_i = X_{m(i)} \Rightarrow e(X_i) = e(X_{m(i)})$.
%    \item Matched pair experiment assuming $Y(0), Y(1) \perp Z \mid X$.
%    \item Positivity holds in matching by construction.
%    \item More realistic settings involve \textbf{imperfect matching}:
 %   \begin{itemize}
 %       \item Almost always the case in observational studies because the covariate space can be large (e.g., continuous features).
 %   \end{itemize}
%end{itemize}

\subsubsection{Distance Metrics for Matching}
 Two Distance Metrics examples are:
\begin{itemize}
    \item \textbf{Euclidean}: 
    \[
    d(X_i, X_k) = (X_i - X_k)^\top (X_i - X_k)
    \]
    \item \textbf{Mahalanobis}: 
    \[
    d(X_i, X_k) = (X_i - X_k)^\top \Sigma^{-1} (X_i - X_k)
    \]
    where $\Sigma$ is the sample covariance matrix of $X$'s in the population.
    \begin{itemize}
        \item Accounts for differences in scale across covariates and correlations between them.
    \end{itemize}
\end{itemize}

\subsubsection{Covariate Adjustment}
Additionally, use covariate adjustment in analysis to correct for the residual covariate imbalance.

\subsubsection{Handling High-Dimensional Covariates}
\textbf{Problem}: When $X$ is high-dimensional, for some $X_i$, $\min\limits_{k: Z_k = 0} d(X_i, X_k)$ is too large.

\textbf{Solutions}: 

\begin{enumerate}
    \item Discard these $X_i$'s:
    \begin{itemize}
        \item Effectively changes the study population.
    \end{itemize}
    \item Dimension reduction before matching:
    \begin{itemize}
        \item e.g., Propensity score matching.
        \item $m(i) = \arg \min\limits_{k: Z_k = 0} |\hat{e}(X_i) - \hat{e}(X_{m(i)})|$
    \end{itemize}
\end{enumerate}


\subsubsection{Algorithm for Matching Multiple Control Units for Treated Unit}
Allow for $M_i$ Control Units for Treated Unit $X_i$:
\begin{itemize}
    \item Input to the matching algorithm: Specify the number of control units, as well as lower or/and upper bounds on $M_i$.

     \item Matching algorithm outputs matched sets, $M_i$ chosen to minimize:
    \[
    \frac{1}{M_i} \sum_{j \in J_i} d(X_i, X_j)
    \]
    where $d(X_i, X_j)$ is a distance function and $J_i$ is the set of indices for control units matched to treated unit $X_i$.
\end{itemize}

\subsubsection{Assessing the Distance Metric}
\begin{itemize}
    \item \textbf{Recap of Ignorability}: $Y(0), Y(1) \perp Z \mid X$ is impossible to empirically verify.
    \item \textbf{How to assess whether the distance metric is good?}
    \begin{itemize}
        \item Analyze covariate distributions in treated vs. matched controls using visualizations like boxplots, or by computing moments of empirical distributions.
    \end{itemize}
\end{itemize}

\subsubsection{Matching Paradigm and Estimand}
\begin{itemize}
    \item Matching paradigm connects to our earlier discussion of the estimand: $\tau_T$ (average causal effect among treated).
    \[
    \tau_T = \mathbb{E}[Y(1) - Y(0) \mid Z = 1] = \mathbb{E}[Y \mid Z = 1] - \mathbb{E}[\mathbb{E}[Y \mid Z = 0, X] \mid Z = 1]
    \]
    \item \textbf{Matching Estimator}:
    \[
    \hat{\tau}_T = \frac{1}{n} \sum_{i=1}^{n} Z_i Y_i - Z_i \hat{\mu}(X_i)
    \]
    where $\hat{\mu}(X_i) = \frac{1}{m_i} \sum_{j \in J_i} Y_j$.

\end{itemize}
This concludes our discussion on matching...

\subsection{Causal DAGs}

From now on, we begin our discussion on Causal Directed Acyclic Graphs (DAGs), which are powerful tools in causal inference. A famous quote, "\textit{Draw your assumptions before your conclusions,}" emphasizes the importance of understanding underlying assumptions in any causal analysis. 

In the computer science (CS) and artificial intelligence (AI) literatures, causal DAGs are extensively used to model relationships between variables. Pioneering work by researchers like Judea Pearl, Peter Spirtes, and Clark Glymour. 

\begin{example}
    Consider the following scenario: $X$, a confounder (e.g., nausea); $Z$, a treatment (e.g., drinking coffee); and $Y$, an outcome (e.g., miscarriage). In the following diagram, the DAG illustrate how the confounder $X$ might influence both the treatment $Z$ and the outcome $Y$. 

\begin{center}
\begin{tikzpicture}
    % Nodes
    \node (X) at (0,0) {$X$};
    \node (Z) at (3,0) {$Z$};
    \node (Y) at (6,0) {$Y$};
    
    % Arrows
    \draw[->] (X) -- (Z);
    \draw[->] (Z) -- (Y);
    \draw[->] (X) to[out=50,in=130] (Y);
    
    % Time arrow
    \draw[->] (-1,-1.5) -- (7,-1.5) node[anchor=north] {time};
\end{tikzpicture}
\end{center}
\end{example}


Here are the meanings of some common notations in causal DAGs:

\begin{itemize}
    \item $V \rightarrow W$ means $V$ has a direct causal effect on $W$ (an effect not mediated by another variable in the graph) for at least one individual.
    \item Lack of an arrow encodes the assumption that there is no direct causal effect of $V$ on $W$ for anyone.
\end{itemize}

Now, let's formally introduce Causal Directed Acyclic Graphs (Causal DAGs), which are essential tools for representing and analyzing causal relationships between variables in a structured, visual manner. Let's break down the key components:

\begin{itemize}
    \item \textbf{Directed}: Causal DAGs include directed arrows between variables to indicate causal effects. For example, $X \rightarrow Z$ means $X$ has a direct causal effect on $Z$.
    \item \textbf{Acyclic}: The graph is acyclic, meaning it cannot contain cycles like $X \rightleftarrows Z$ (where $X$ causes $Z$ and $Z$ causes $X$), as such relationships are not allowed.
\end{itemize}


Some key definitions are listed below:

\begin{definition}[Causal DAG] 
A DAG that satisfies the Causal Markov Assumption.
\end{definition}

\begin{definition}[Causal Markov Assumption] 
Conditional on its direct causes, any variable in a causal DAG is independent of any other variable that it does not cause.
\end{definition}

Causal DAGs represent causal relationships through properly ordered arrows (e.g., $X \rightarrow Y \rightarrow Z$) and associations through improperly ordered arrows (e.g., $X \rightarrow Z \leftarrow Y$).

\subsubsection{Backdoor Path}

A backdoor path is a non-causal path between two variables that can create spurious correlations. Backdoor paths are important because they can make it difficult to determine if an association between two variables is a result of a causal effect or a backdoor path. 

\begin{example}
    $Z$ and $Y$ are linked by a backdoor path, so that are associated. This type of path creates confounding and needs to be accounted for when estimating causal effects.

\begin{center}
\begin{tikzpicture}

    % Nodes
    \node (X) at (0,0) {$X$};
    \node (Z) at (3,0) {$Z$};
    \node (Y) at (6,0) {$Y$};
    
    % Arrows for the causal diagram
    \draw[->, thick, blue] (X) -- (Z);
    \draw[->, thick, red] (Z) -- (Y) node[midway, below] {};
    \draw[->, thick, blue, out=120, in=60] (X) to (Y);

    % Labels
    \node[red,scale=0.8] at (5, -0.5) {Causal Effect of $Z$ on $Y$};
    
\end{tikzpicture}
\end{center}

\begin{center}
\begin{tikzpicture}

    % Nodes
    \node (Z) at (0,0) {$Z$};
    \node (X) at (3,0) {$X$};
    \node (Y) at (6,0) {$Y$};
    
    % Arrows for the backdoor path (going against arrowhead direction)
    \draw[->, thick, blue] (X) -- (Y);
    \draw[<-, thick, blue] (Z) -- (X);
    

    % Label
    \node[blue, scale=0.8] at (3,-0.5) {Backdoor Path goes against arrowhead};
    
\end{tikzpicture}
\end{center}
\end{example}


\begin{example}
    Suppose variable $Z$ is Carries lighter, $Y$ means Lung cancer, and $X$ means cigarette smoker. $Z$ and $Y$ have an association through $X$ by a backdoor path, but no direct causal effect.

\begin{center}
\begin{tikzpicture}

    % Nodes
    \node (X) at (0,0) {$X$};
    \node (Z) at (3,0) {$Z$};
    \node (Y) at (6,0) {$Y$};
    
    % Arrows for the causal diagram
    \draw[->, thick, black] (X) -- (Z);
    \draw[->, thick, black, out=120, in=60] (X) to (Y);

\end{tikzpicture}
\end{center}
\end{example}

\subsubsection{Colliding}

In causal DAGs, a variable is a collider when it is causally influenced by two or more variables. For example, in the following diagram, $L$ is a collider, as it is causally influenced by $Z$ and $Y$.

\begin{center}
\begin{tikzpicture}
    % Nodes
    \node (Z2) at (0,0) {$Z$};
    \node (Y2) at (3,0) {$Y$};
    \node (L) at (6,0) {$L$};
    
    % Arrows for association through a shared effect

    \draw[->] (Y2) -- (L);
    
    % Curve from Z to L
    \draw[->] (Z2) .. controls (1.5,1.5) and (4.5,1.5) .. (L);
\end{tikzpicture}
\end{center}

\begin{definition} [collider]
Common effect is a \textbf{collider}.
\end{definition}

\begin{definition}
    Graphical structure of a collider is a \textbf{V-structure}, shown in the diagram below.
\end{definition} 

\begin{center}
\begin{tikzpicture}
    % Nodes
    \node (Z) at (0,2) {$Z$};
    \node (Y) at (2,2) {$Y$};
    \node (L) at (1,0) {$L$};
    
    % Arrows for the V-structure
    \draw[->] (Z) -- (L);
    \draw[->] (Y) -- (L);
\end{tikzpicture}
\end{center}

In the V-structure, we have:
\begin{itemize}
    \item No association unconditionally between $Z$ and $Y$: $Z \perp\!\!\!\perp Y$
    \item $Z$ and $Y$ are not independent when conditioning on $L$: $Z \not\!\perp\!\!\!\perp Y \mid L$
\end{itemize}

\begin{example}
    Consider $Z$ means poor time management, $Y$ means family emergency and $L$ means missed class. The relationship of $Z$, $Y$ and $L$ are shown as the V-structure, then $Z \perp\!\!\!\perp Y$. However, if I know that a student missed class, then if you tell me whether the student has poor time management skills, it is relevant for my guess about a family emergency occurring.  This phenomenon is what we called  \textit{"explaining away"}.
\end{example}

\subsubsection{Dependency in Causal DAGs}

Now we introduce how we represent conditioning in causal DAGs. 

\begin{example}
    Consider the following diagram. Suppose $Z$ means taking aspirin, $M$ means platelet aggregation and $Y$ means heart disease, 
\begin{center}
\begin{tikzpicture}
    % Nodes
    \node (Z) at (0,0) {$Z$};
    \node (M) at (2,0) {$M$};
    \node (Y) at (4,0) {$Y$};
    
    % Arrows for causal pathway
    \draw[->] (Z) -- (M);
    \draw[->] (M) -- (Y);
\end{tikzpicture}
\end{center}
Is there an association between $Z$ and $Y$ within levels of $M$ (e.g., conditional on $M$)? To study this question, we use the following diagram, where the box around $M$ indicates conditioning on $M$.

\begin{center}
\begin{tikzpicture}
    % Nodes
    \node (Z2) at (0,0) {$Z$};
    \node (M2) at (2,0) {$M$};
    \node (Y2) at (4,0) {$Y$};
    
    % Arrows for blocked pathway
    \draw[->] (Z2) -- (M2);
    \draw[->] (M2) -- (Y2);
    
    % Box around M to indicate conditioning
    \draw[thick] (1.5,-0.5) rectangle (2.5,0.5);
\end{tikzpicture}
\end{center}

\end{example}


\section{Lecture 12: Paths of association and d-separation}{James Bowden}

\subsection{Last Time: Causal DAG definitions}

\begin{example}
    Consider the following diagram. Suppose \( Z \): taking aspirin, \( M \): platelet aggregation, and \( Y \): heart disease.
\begin{center}
\begin{tikzpicture}
    % Nodes
    \node (Z) at (0,0) {$Z$};
    \node (M) at (2,0) {$M$};
    \node (Y) at (4,0) {$Y$};
    
    % Arrows for causal pathway
    \draw[->] (Z) -- (M);
    \draw[->] (M) -- (Y);
\end{tikzpicture}
\end{center}
Is there an association between \( Z \) and \( Y \) when conditioned on \( M \)? To study this question, we use the following diagram, where the box around $M$ indicates conditioning on $M$.

\begin{center}
\begin{tikzpicture}
    % Nodes
    \node (Z2) at (0,0) {$Z$};
    \node (M2) at (2,0) {$M$};
    \node (Y2) at (4,0) {$Y$};
    
    % Arrows for blocked pathway
    \draw[->] (Z2) -- (M2);
    \draw[->] (M2) -- (Y2);
    
    % Box around M to indicate conditioning
    \draw[thick] (1.5,-0.5) rectangle (2.5,0.5);
\end{tikzpicture}
\end{center}
\end{example}

New objects introduced in the context of causal graphs include confounders, backdoor paths (non-causal), colliders (variables causally influenced by two or more variables), and mediators (which yield causal independence when conditioned upon).


\subsection{Revisiting common cause structure}

\begin{example}
In this diagram, there's a direct causal path, $Z\rightarrow Y$, and an associative path, $Z \leftarrow X \rightarrow Y$. $X$ is called a \textbf{confounder}.

\begin{center}
\begin{tikzpicture}
    % Nodes
    \node (X) at (0,0) {$X$};
    \node (Z) at (3,0) {$Z$};
    \node (Y) at (6,0) {$Y$};
    
    % Arrows
    \draw[->] (X) -- (Z);
    \draw[->] (Z) -- (Y);
    \draw[->] (X) to[out=50,in=130] (Y);
    
\end{tikzpicture}
\end{center}

We can also denote blocking the associative path with a box, meaning $Z \perp Y(z) | X$. In this case, $X$ no longer confounds the relationship between $Z$ and $Y$; that is, we will only observe how changes in $Z$ not induced by $X$ affect $Y$.

\begin{center}
\begin{tikzpicture}

    % Nodes
    \node (X) at (0,0) {$X$};
    \node (Z) at (3,0) {$Z$};
    \node (Y) at (6,0) {$Y$};
    
    % Arrows
    \draw[->] (X) -- (Z);
    \draw[->] (Z) -- (Y);
    \draw[->] (X) to[out=50,in=130] (Y);
    
    % Box around X to indicate conditioning
    \draw[thick] (-0.5,-0.5) rectangle (0.5,0.5);
\end{tikzpicture}
\end{center}
\end{example}

\subsection{Common effect structure}

\begin{example}
Consider the following causal DAG: $L$ is wet grass, $Z$ is sprinklers ran overnight, $Y$ is that it rained.

\begin{center}
\begin{tikzpicture}
    % Nodes
    \node (Z) at (-1,1) {$Z$};
    \node (Y) at (3,1) {$Y$};
    \node (L) at (1,0) {$L$};
    
    % Arrows for the V-structure
    \draw[->] (Z) -- (L);
    \draw[->] (Y) -- (L);
\end{tikzpicture}
\end{center}
\end{example}

Here, $Z \perp Y$ but $Z \not\perp Y | L$. $L$ is called a \textbf{collider}. Paths of association do not flow through colliders (i.e., $Z \perp Y$). However, if we were to condition on $L$ (or some descendent thereof), denoted in a causal DAG as a box around $L$, then association would flow through $L$ (i.e., $Z \not\perp Y | L$). This is because when $L$ has been fixed as a known quantity, knowing something about $Z$ will provide information about $Y$ (and vice versa). One can see this through the simple relationship $Z+Y=L$ where $Z \perp Y$ but if $L$ is known, then knowing $Z$ yields $Y$ as $Y=L-Z$.

\subsection{Recap}
\subsubsection{Why might 2 variables be associated? Structural reasons:}

\begin{enumerate}
\item One causes the other
\item The two share a common cause (a confounder)
\item The two share a common effect, and our analysis looks at a certain level that effect (i.e., conditional on a value of $L=l$ in the diagram immediately above).
\end{enumerate}

\subsubsection{A causal DAG implies a set of structural equations}

\begin{center}
\begin{tikzpicture}
    % Nodes
    \node (X) at (0,0) {$X$};
    \node (Z) at (3,0) {$Z$};
    \node (Y) at (6,0) {$Y$};
    
    % Arrows
    \draw[->] (X) -- (Z);
    \draw[->] (Z) -- (Y);
    \draw[->] (X) to[out=50,in=130] (Y);
    
\end{tikzpicture}
\end{center}

What equations does this graph imply?
\begin{itemize}
\item $X \sim F_X(X)$
\item $Z \sim g_Z(X, \epsilon_Z)$
\item $Y \sim g_Y(X, Z, \epsilon_Y)$
\end{itemize}

\subsection{Exchangeability}

\subsubsection{Causal graphs + exchangeability}

If we know the true causal DAG, then we can determine whether there exists a set of variables $X$ s.t. $Z \perp Y(1), Y(0) | X$. This leads us to the backdoor criterion.

\begin{definition}[Backdoor criterion]
The backdoor criterion holds for a set of covariates $X$ when all backdoor paths (non-causal associative paths) between $Z$ and $Y$ are blocked by conditioning on $X$, and $X$ contains no variables that are descendents of $Z$. The backdoor criterion implies exchangeability conditional on $X$. 
\end{definition}

\begin{example}
In the below DAG, there is a backdoor path through $X$.
\begin{center}
\begin{tikzpicture}
    % Nodes
    \node (X) at (0,0) {$X$};
    \node (Z) at (3,0) {$Z$};
    \node (Y) at (6,0) {$Y$};
    
    \node (L) at (3,-1) {$L$};
    
    % Arrows
    \draw[->] (X) -- (Z);
    \draw[->] (Z) -- (Y);
    \draw[->] (X) to[out=50,in=130] (Y);
    \draw[->] (L) -- (Y);
    
\end{tikzpicture}
\end{center}

In order to make the backdoor criterion hold, we must condition on $X$, denoted by a box: 

\begin{center}
\begin{tikzpicture}
    % Nodes
    \node (X) at (0,0) {$X$};
    \node (Z) at (3,0) {$Z$};
    \node (Y) at (6,0) {$Y$};
    
    \node (L) at (3,-1) {$L$};
    
    % Arrows
    \draw[->] (X) -- (Z);
    \draw[->] (Z) -- (Y);
    \draw[->] (X) to[out=50,in=130] (Y);
    \draw[->] (L) -- (Y);

    % Box around X to indicate conditioning
    \draw[thick] (-0.5,-0.5) rectangle (0.5,0.5);
    
\end{tikzpicture}
\end{center}

Now there are no backdoor paths, so the backdoor criterion is fulfilled. 
\end{example}

\begin{example}
Let $Z$ be aspirin, $Y$ is stroke, and $X$ is doctor diagnosis of previous underlying health conditions.

\begin{center}
\begin{tikzpicture}
    % Nodes
    \node (X) at (0,0) {$X$};
    \node (Z) at (3,0) {$Z$};
    \node (Y) at (6,0) {$Y$};
    
    \node (U) at (3,-1) {$U$};
    
    % Arrows
    \draw[->] (X) -- (Z);
    \draw[->] (Z) -- (Y);
    % \draw[->] (X) to[out=50,in=130] (Y);
    \draw[->] (U) -- (Y);
    \draw[->] (U) -- (X);
    
\end{tikzpicture}
\end{center}

What paths go from $Z$ to $Y$? There's the direct causal path $Z \rightarrow Y$ as well as a backdoor path, $Z \leftarrow X \leftarrow U \rightarrow Y$. 
As such, $U$ is a \textbf{common cause}. 

To fulfill the backdoor criterion, we can block the backdoor path in two different ways: via conditioning on either $U$ or $X$.

\begin{center}
\begin{tikzpicture}
    % Nodes
    \node (X) at (0,0) {$X$};
    \node (Z) at (3,0) {$Z$};
    \node (Y) at (6,0) {$Y$};
    
    \node (U) at (3,-1) {$U$};
    
    % Arrows
    \draw[->] (X) -- (Z);
    \draw[->] (Z) -- (Y);
    % \draw[->] (X) to[out=50,in=130] (Y);
    \draw[->] (U) -- (Y);
    \draw[->] (U) -- (X);

    % Box around X to indicate conditioning
    \draw[thick] (2.5,-1.5) rectangle (3.5,-0.5);
    
\end{tikzpicture}
\end{center}

\begin{center}
\begin{tikzpicture}
    % Nodes
    \node (X) at (0,0) {$X$};
    \node (Z) at (3,0) {$Z$};
    \node (Y) at (6,0) {$Y$};
    
    \node (U) at (3,-1) {$U$};
    
    % Arrows
    \draw[->] (X) -- (Z);
    \draw[->] (Z) -- (Y);
    % \draw[->] (X) to[out=50,in=130] (Y);
    \draw[->] (U) -- (Y);
    \draw[->] (U) -- (X);

    % Box around X to indicate conditioning
    \draw[thick] (-0.5,-0.5) rectangle (0.5,0.5);
    
\end{tikzpicture}
\end{center}
\end{example}

\subsubsection{Recap}
Causal graphs are useful for thinking about confounding because:
\begin{itemize}
    \item We make assumptions about the data generating process explicit.
    \item They enable us to find the set of confounders (if possible) to achieve exchangeability under those assumptions.
\end{itemize}

\subsubsection{Other methods for assessing exchangeability}

Recall that exchangeability, $Z \perp Y(1), Y(0) | X$ is untestable. This is because it implies
$$P(Y(1)=1|Z=1,X) = P(Y(1)=1|Z=0,X)$$
and while the former term is observable, the latter is the counterfactual and is not ever known. \textit{Note that we can write the analogous equation for $Y(0)$ and observe the same issue}. 

As such, we cannot ever test this assumption formally. We can, however, assess the strength of evidence in support of exchangeability to validate (or call into question) our assumption.

\begin{enumerate}
    \item Use data on other outcomes: "negative outcomes".
    
    Suppose we observe $\Tilde{Y}$. Assume:
    \begin{itemize}
        \item The confounding of $\Tilde{Y}$ is the same as the confounding for $Y$ w.r.t. $Z$
        \item $Z$ has no effect on $\Tilde{Y}$
    \end{itemize}
    \begin{center}
    \begin{tikzpicture}
        % Nodes
        \node (X) at (0,0) {$X$};
        \node (Yt) at (3,0) {$\Tilde{Y}$};
        \node (Z) at (0,3) {$Z$};
        \node (Y) at (3,3) {$Y$};
        
        % Arrows
        \draw[->] (X) -- (Z);
        \draw[->] (X) -- (Yt);
        \draw[->] (X) -- (Y);
        \draw[->] (Z) -- (Y);
        
    \end{tikzpicture}
    \end{center}
    In terms of potential outcomes, we would write:
    \begin{itemize}
        \item $Z \perp Y(Z) | X$ and $Z \perp \Tilde{Y(Z)} | X$
        \item $\mathbb{E}[\Tilde{Y}|Z=1]-\mathbb{E}[\Tilde{Y}|Z=0] \neq 0$ if we have confounding 
    \end{itemize}
    \begin{example}
        From Cornfield (1959): $Y$ is lung cancer, $Z$ is cigarette smoking, $\Tilde{Y}$ is car accidents. 

        He found evidence that $\mathbb{E}[\Tilde{Y}|Z=1] = \mathbb{E}[\Tilde{Y}|Z=0]$.

        To the extent that there are common confounders for car accidents and lung cancer, then this supports the claim that the association between lung cancer and smoking is causal. It is noted that there's no real intuition for why car accidents and lung cancer are related, but this example may be illustrative anyway, as an extreme case.
    \end{example}
    \begin{example}
        From Jackson et al. (year?): Let $Y$ be hospitalization during flu season, $\Tilde{Y}$ hospitalization before flu season, $Z$ getting the flu vaccine.

        They found evidence that $\mathbb{E}[\Tilde{Y}|Z=1] - [\Tilde{Y}|Z=0]$ was very large. From this, we can conclude that there were unmeasured confounders, and that the assumption of exchangeability was violated / not reasonable.
    \end{example}
    \item To be continued...
\end{enumerate}





\section{Lecture 13: Negative Outcomes and Instrumental Variables}{Jean Lee}

\subsection{Last Time}

\begin{itemize}
    \item Negative outcomes were used to assess the exchangeability assumption.
    \item Negative outcomes are also known as negative outcome controls or placebo tests. 
    \end{itemize}
    
\textbf{Recall example:}
\begin{itemize}
    \item \( y \): hospitalization during flu season 
    \item \( \tilde{y} \): hospitalization before flu season
    \item \( z \): vaccine
    
\end{itemize}
$\mathbb{E}[\tilde{Y}|Z=1] - \mathbb{E}[\tilde{Y}|Z=0]$ large: suggests unmeasured confounding


\subsubsection{Another Example: lagged outcomes (Imbens and Rubin, 2015)}
\begin{itemize}
    \item Use outcome right before treatment as a negative outcome
    \item Treatment can't affect something that happened before the treatment
    \item The confounding structure may be similar for lagged outcomes
\end{itemize}
Assume the confounding follows:
\[
\mathbb{E}[Y(0) | Z = 1] - \mathbb{E}[Y(0) | Z = 0] = \mathbb{E}[\tilde{Y} | Z = 1] - \mathbb{E}[\tilde{Y}| Z = 0]
\]
This leads to a difference-in-differences model.

\subsection{Assessing Confounding Using Negative Exposures}
Another method to assess confounding is by using data on other treatments/exposures: "Negative Exposures"
\begin{itemize}
    \item $\tilde{Z}$: treatment/exposure variable that shares the same confounding as \(Z\) w.r.t \(Y\)
\end{itemize}
\begin{center}
  \begin{tikzpicture}
    % Nodes
    \node (X) at (0,0) {$X$};
    \node (Z_tilde) at (2,0) {$\tilde{Z}$};
    \node (Z) at (2,-2) {$Z$};
    \node (Y) at (2,-4) {$Y$};
    
    % Arrows for causal pathway
    \draw[->] (X) -- (Z_tilde);
    \draw[->] (X) -- (Z);
    \draw[->] (Z) -- (Y);
    \draw[->] (X) -- (Y);
\end{tikzpicture}
\end{center}
Assume: $Z \perp\!\!\!\perp Y(Z) \mid X \text{ and } \tilde{Z} \perp\!\!\!\perp Y(Z) \mid X$

\[
\mathbb{E}[Y(\tilde{Z} = 1) - Y(\tilde{Z} = 0)] = 0 
\]

\subsubsection{Example: Sanderson et al 2017}
\begin{itemize}
    \item Let \( Z \) be maternal exposure during pregnancy
    \item Let $\tilde{Z}$ be paternal exposure during pregnancy
    \item Let \( Y \) be the child's BMI or autism spectrum disorder
\end{itemize}

\subsection{Proximal Causal Inference}
Proximal causal inference: when we have both negative outcomes and negative exposures
\begin{itemize}
    \item Conditions under which we can nonparametrically identify the casual effect in the presence of unmeasured confounding 
    \begin{itemize}
    \item Example: 
    
    u: unobserved confounder (discrete)
    
    $\tilde{y}$: negative outcome (discrete)
    
    $\tilde{z}$: negative exposure (discrete)
    \end{itemize}
    \item when $\tilde{y}$ and $\tilde{z}$ have as many levels as u, then we can identify the causal effect
\end{itemize}

\subsection{Colliders and Over-Adjustment Problems}
Recall collider L:
\begin{center}
  \begin{tikzpicture}
    % Nodes
    \node (Z) at (0,0) {$Z$};
    \node (L) at (2,0) {$L$};
    \node (Y) at (0,-1) {$Y$};
    
    % Arrows for causal pathway
    \draw[->] (Z) -- (L);
    \draw[->] (Y) -- (L);
\end{tikzpicture}
\end{center}
Rule of d-separation: 

$\rightarrow \leftarrow \text{closed}$

$\rightarrow \boxed{L} \leftarrow \quad \text{opens } Z \perp\!\!\!\perp Y \mid L$

\subsubsection{Problem 1: M-Bias}
\textbf{M-bias:}
\begin{center}
  \begin{tikzpicture}
    % Nodes
    \node (U1) at (0,0) {$U_1$};
    \node (U2) at (2,0) {$U_2$};
    \node (X) at (1,-1) {$X$};
    \node (Z) at (0,-2) {$Z$};
    \node (Y) at (2,-2) {$Y$};
    
    % Arrows for causal pathway
    \draw[->] (U1) -- (X);
    \draw[->] (U1) -- (Z);
    \draw[->] (U2) -- (X);
    \draw[->] (U2) -- (Y);
\end{tikzpicture}
\end{center}

\[
\mathbb{E}[Y \mid Z = 1] - \mathbb{E}[Y \mid Z = 0] = 0
\]

\begin{itemize}
    \item valid unbiased estimator for causal effect because there's no confounding
\end{itemize}

If "over-adjust" by condition on X: $Z \perp\!\!\!\perp Y(Z) \mid X$

\[
\mathbb{E}[\mathbb{E}[Y \mid Z=1, X] - \mathbb{E}[Y \mid Z=0, X]] \neq 0
\]

\begin{center}
    \begin{tikzpicture}
      % Nodes
      \node (U1) at (0,0) {$\boxed{U_1}$};
      \node (U2) at (2,0) {$U_2$};
      \node (X) at (1,-1) {$\boxed{X}$};
      \node (Z) at (0,-2) {$Z$};
      \node (Y) at (2,-2) {$Y$};
      
      % Arrows for causal pathway
      \draw[->] (U1) -- (X);
      \draw[->] (U1) -- (Z);
      \draw[->] (U2) -- (X);
      \draw[->] (U2) -- (Y);
    \end{tikzpicture}%
  \begin{minipage}{0.4\textwidth}
    \centering
    \text{blocks path of association}
  \end{minipage}
\end{center}

\subsubsection{Problem 2: Z-bias}
\textbf{Z-bias:}
\begin{center}
  \begin{tikzpicture}
    % Nodes
    \node (Z) at (0,0) {$Z$};
    \node (D) at (2,0) {$D$};
    \node (Y) at (4,0) {$Y$};
    \node (U) at (3,1) {$U$};
    
    % Arrows for causal pathway
    \draw[->] (Z) -- (D);
    \draw[->] (D) -- (Y);
    \draw[->] (U) -- (D);
    \draw[->] (U) -- (Y);
\end{tikzpicture}
\end{center}

\textbf{Key Assumptions:}
\begin{itemize}
\item Instrument Independence: \( Z \perp\!\!\!\perp Y \)
\item Instrument Relevance: \( Z \not\perp\!\!\!\perp D \)
\item Z only affects y through D
\end{itemize}


\begin{itemize}
    \item Generally conditioning on Z will make our estimates more biased
    \begin{itemize}
    \item D: treatment received
    \item Y: outcome
    \item Ch. 16 of A first course example with linear models
    \end{itemize}
    \item Z is not a confounder, but U is and we don't observe U 
    \item Conditioning on Z makes D less random and amplifies the role of U in the remaining randomness of D 
    \item Z: instrumental variable 
    \item Z-bias = instrumental variable bias 
\end{itemize}
Example: 
\begin{itemize}
\item \(D \): prison sentence length 
\item \( Y \) : recommit an offense (re-arrest) after release
\item \( Z \): random assignment of cases to judges 
\item \( U \): personal characteristics, family support
\end{itemize}

What covariates should we adjust for in observational studies?

\begin{center}
  \begin{tikzpicture}
    % Nodes
    \node (XR) at (0,0) {$X_R$};
    \node (XZ) at (-2,-1) {$X_Z$};
    \node (X) at (0,-1) {$X$};
    \node (XY) at (2,-1) {$X_Y$};
    \node (Z) at (-1,-2) {$X$};
    \node (Y) at (1,-2) {$Y$};
    \node (XL) at (0,-3) {$X_l$};
    \node (XQ) at (-2,-3) {$X_Q$};
    \node at (0,-4) {\text{$X_l$: post-treatment variable}}; % Add this line

    % Underline colors
    \draw[green, thick] (XR.south west) -- (XR.south east);
    \draw[purple, thick] (XZ.south west) -- (XZ.south east);
    \draw[blue, thick] (X.south west) -- (X.south east);
    \draw[orange, thick] (XY.south west) -- (XY.south east);
    \draw[purple, thick] (Z.south west) -- (Z.south east);
    \draw[purple, thick] (Y.south west) -- (Y.south east);
    
    % Arrows for causal pathway
    \draw[->] (XZ) -- (Z);
    \draw[->] (X) -- (Z);
    \draw[->] (X) -- (Y);
    \draw[->] (XY) -- (Y);
    \draw[->] (Z) -- (Y);
    \draw[->] (Z) -- (XL);
    \draw[->] (Z) -- (XQ);
\end{tikzpicture}
\end{center}

\begin{center}
\begin{tabular}{|c|c|c|} % Specify three centered columns with borders
\hline
\textbf{Necessary for Identification} & \textbf{Helpful to Reduce Variance} & \textbf{Harmful} \\
\hline
$X$ & $X_Y$ & $X_R$ \\
"confounding" & "effect modifier" & $X_Z$\\
 & & $X_l$\\
\hline
\end{tabular}
\end{center}

\textbf{What to do when it's unreasonable to assume exchangeability}

Today: instrumental variables 

Next week: sensitivity analysis + bounds + partial identification
\subsection{Instrumental Variables}
In experiments, often participants may not adhere/comply to treatment assignment 

\begin{itemize}
    \item \( Z \): treatment assignment
    \item \( D \): the treatment taken, "adherence"
    \item\( U \) : confounders that affect \( Y \) and \( D \)
    \item \( Y \): outcome
\end{itemize}

\begin{center}
  \begin{tikzpicture}
    % Nodes
    \node (Z) at (0,0) {$Z$};
    \node (D) at (2,0) {$D$};
    \node (Y) at (4,0) {$Y$};
    \node (U) at (3,1) {$U$};
    
    % Arrows for causal pathway
    \draw[->] (Z) -- (D);
    \draw[->] (D) -- (Y);
    \draw[->] (U) -- (D);
    \draw[->] (U) -- (Y);
\end{tikzpicture}
\end{center}

\textbf{Example}

\begin{itemize}
    \item
    \begin{itemize}
    \item \( D \): aspirin taken
    \item \( Y \): stroke
    \item \(  Z \): whether patient was assigned aspirin
    \item \( U \): behavioral factors
    \end{itemize}
    \item Challenge: can't measure everything in \( U \)
    \item Solution: leverage randomness in instrument \( Z \)
\end{itemize}

\textbf{Definition:} A \textbf{instrumental variable} is a random variable that meets 3 condition: 
\begin{enumerate}
    \item \tikz[baseline=(X.base)]{
        \node[inner sep=0pt] (X) {\underline{\textbf{Relevance}:}};
        \draw[green, thick] (X.south west) -- (X.south east);
    } $Z \not \perp\!\!\!\perp D$
    \begin{itemize}
    \item \( Z \) is associated with the treatment \( D \)
    \end{itemize}
    \item  \tikz[baseline=(X.base)]{
        \node[inner sep=0pt] (X) {\underline{\textbf{Exclusion Restriction}:}};
        \draw[blue, thick] (X.south west) -- (X.south east);
    }\( Z \) only affects \( Y \) only through \( D \)
    \begin{itemize}
    \item No direct effect of \( Z \) on \( Y \)
    \item e.g., holds double-blind experiment 
    \end{itemize}
 \item \tikz[baseline=(X.base)]{
        \node[inner sep=0pt] (X) {\underline{\textbf{Exchangeable / Unconfounded IV}}}; % Fixed missing closing brace
        \draw[red, thick] (X.south west) -- (X.south east); % Dotted line
    } \( Z \) and \( Y \) don't share unmeasured confounders
\end{enumerate}

\begin{center}
  \begin{tikzpicture}
    % Nodes
    \node (X) at (0,0) {$X$};
    \node (Z) at (2,0) {$Z$};
    \node (D) at (4,0) {$D$};
    \node (Y) at (6,0) {$Y$};
    \node (U) at (4,-1) {$U$};
    
    % Arrows for causal pathway
    \draw[->] (X) -- (Z);
    \draw[->] (X) to[bend left] (Y);
    \draw[->] (Z) -- (D);
    \draw[->,  green, thick] (Z) -- (D);
    \draw[->] (D) -- (Y);
    \draw[->] (X) to[bend left] (D);
    \draw[->] (U) -- (X);
    \draw[->] (U) -- (Y);
    \draw[ -> , dotted, blue, thick] (Z) to[bend right] (Y);
    \draw[ -> , dotted, red, thick] (U) to[bend left] (Z);
\end{tikzpicture}
\end{center}
* dotted arrow: not an edge

\textbf{Example:} Minneapolis Domestic Violence Experiment (1980s)

\begin{itemize}
\item randomly assigned penalties:
\begin{enumerate}
\item arrest (1/3)
\item counseling (1/3)
\item separation (1/3)

(arrest, counseling, separation are all \( D \))
\end{enumerate}
\item \( Y \) : re-offense
\item Officers did not always comply
\end{itemize}








\section{Lecture 14: Instrumental Variables Continued}{Jaeeun Park \& Zekai Wang \& Claire Hsu \& Jiarong Zhou(revisions)}
\subsection{Review from Last Lecture}
Last lecture we introduced the notion of instrumental variable and gave a high level definition for it. Assume \(Z\) is the binary instrumental variable (e.g., treatment assigned), \(D\) is the binary treatment a unit actually received, \(Y\) is the outcome, and \(U\) are (unmeasured) confounders that affect both the treatment received and the outcome. 
\begin{center}
    \begin{tikzpicture}
      \node (Z) at (0,0) {$Z$};
      \node (D) at (2,0) {$D$};
      \node (Y) at (4,0) {$Y$};
      \node (U) at (3,1) {$U$};
      
      \draw[->] (Z) -- (D);
      \draw[->] (D) -- (Y);
      \draw[->] (U) -- (D);
      \draw[->] (U) -- (Y);
  \end{tikzpicture}
\end{center}

\begin{definition}
    Informally, the Instrumental Variable (IV) \(Z\) is valid if it meets the following three conditions:
    \begin{enumerate}
        \item Relevance: \(Z\) has a direct effect on \(D\). This can be seen from the \(Z \to D\) path in the causal DAG. 
        \item Exclusion Restriction: \(Z\) only affects \(Y\) through \(D\). This can be seen from the lack of a \(Z \to Y\) path in the causal DAG. 
        \item Exchangeability (Unconfoundness): \(Z\) and \(Y\) don't share unmeasured confounders. This can be seen from the lack of a \(U \to Z\) path in the causal DAG. 
    \end{enumerate}
\end{definition}

As a concrete example, in the Minneapolis Domestic Violence Experiment, police officers are randomly assigned to one of three penalties (arrest, counseling, and separation from partner) when dealing with domestic violence perpetrators. However, officers are not required to strictly follow the assignment. In this example, \(Z\) is the random assignment of the three penalties, \(D\) is the actual penalty given, \(Y\) is the indicator for re-offense, and \(U\) is some latent cause (e.g., propensity to commit violence, severity of the offense). \textit{Relevance} holds since the officers are encouraged to follow the penalty assignment, \textit{Exclusion Restriction} holds since it is reasonable to assume the assignment does not directly affect the chances of re-offense, and \textit{Exchangeability} holds by design since \(Z\) is randomly assigned in the experiment. However, as we will see later, it is usually hard to identify a valid IV in observational studies since doing so involves making untestable assumptions.


\subsection{Instrumental Variables}

\subsubsection{Formal Definition of IV}
To rigorously analyze IVs, we first need to introduce some additional notations. 
\begin{itemize}
    \item \(D(Z = z)\) is the potential outcomes for \(D\), i.e., what treatment a unit would actually receive if the IV takes value $z$. In the Minneapolis Domestic Violence Experiment, this is what penalty would be chosen if the officer is suggested to give penalty \(z\). In the case of binary outcome, we often abbreviate \(D(Z = 1) = D(1), D(Z = 0) = D(0)\).
    \item \(Y(D = d)\) is the potential outcomes for \(Y\) given the unit actually receives treatment \(d\). In the Minneapolis Domestic Violence Experiment, this is the indicator for re-offense if the officer gives penalty \(d\) to the perpetrator. 
    \item \(Y(Z = z, D = d)\) is the potential outcomes for \(Y\) given the unit has IV value \(z\) and actually receives treatment \(d\). In the Minneapolis Domestic Violence Experiment, this is the indicator for re-offense if the officer is suggested to give penalty \(z\) and actually gives penalty \(d\). 
    \item \(Y(Z = z) = Y(Z = z, D = D(Z = z))\) is the potential outcomes we would observe if a unit has IV value $z$. In the Minneapolis Domestic Violence Experiment, this is the indicator for re-offense if the officer is suggested to give penalty \(z\) but we don't have information about what penalty the officer actually gives. 
\end{itemize}
With these, we are ready to formally define IVs. 
\begin{definition}
    A variable \(Z\) is a valid Instrumental Variable (IV) if it satisfies the following three conditions:  
    \begin{enumerate}
        \item \textbf{Relevance}: \(Z \not \perp\!\!\!\perp D\), meaning \(Z\) is correlated with \(D\) and thus has a measurable effect on the treatment variable \(D\).  
        \item \textbf{Exclusion Restriction}: The potential outcome \(Y\) depends on the treatment \(D\) but not directly on \(Z\). Formally, \(Y(Z = z, D = d) = Y(D = d)\) for all possible values of \(z\) and \(d\).  
        \item \textbf{Exchangeability (Unconfoundedness)}: The IV \(Z\) is independent of the potential outcomes of \(D\) and \(Y\). Specifically, \(Z \perp (D(Z = z), Y(Z = z))\). In the case of a binary IV, this implies \(Z \perp (D(Z = 0), D(Z = 1), Y(Z = 0), Y(Z = 1))\). If covariates \(X\) are included, this condition becomes \(Z \perp (D(Z = z), Y(Z = z)) \mid X\).  
    \end{enumerate}
\end{definition}

The \textit{Relevance} assumption has a testable implication: \(\Pr{D = 1 \mid Z = 1} - \Pr{D = 1 \mid Z = 0} \neq 0\). Note when its value is small but nonzero, \(Z\) is considered to be a weak instrument. However, the other two assumptions are generally untestable.

\subsubsection{Some Estimands}
Sadly, if we only assume \(Z\) is a valid IV (i.e., we only assume \textit{Relevance, Exclusion Restriction, and Exchangeability}), we are not able to identify the Average Treatment Effect (ATE) \(\mathbb{E}[Y(D = 1) - Y(D = 0)]\). Nevertheless, below are some estimands.
\begin{itemize}
    \item Intention-to-treat effect: \(\tau_{ITT} = \mathbb{E}[Y \mid Z = 1] - \mathbb{E}[Y \mid Z = 0]\)
    \item As-treated estimand: \(\tau_{AT} = \mathbb{E}[Y \mid D = 1] - \mathbb{E}[Y \mid D = 0]\)
    \item Per protocol estimand: \(\tau_{PP} = \mathbb{E}[Y \mid D = Z = 1] - \mathbb{E}[Y \mid D = Z = 0]\)
\end{itemize}
Unfortunately, the As-treated estimand and the Per protocol estimand do not have a clear causal meaning. For the Intention-to-treat effect, under \textit{Relevance} and \textit{Exchangeability}, it equals the causal effect of instrument \(Z\) on \(Y\): 
\begin{align*}
    \tau_{ITT} &= \mathbb{E}[Y(Z = 1) \mid Z = 1] - \mathbb{E}[Y(Z = 0) \mid Z = 0] \\
    &= \mathbb{E}[Y(Z = 1) - Y(Z = 0)]
\end{align*}
If we additionally assume \textit{Exclusion Restriction}, 
\begin{align*}
    \tau_{ITT} &= \mathbb{E}[Y(Z = 1) - Y(Z = 0)] \\
    &= \mathbb{E}[Y(Z = 1, D = D(1)) - Y(Z = 0, D = D(0))] \\
    &= \mathbb{E}[Y(D = D(1)) - Y(D = D(0))]
\end{align*}
Where \(Y(D = D(0)), Y(D = D(1))\) are called stochastic intervention since the actual treatment received is a random variable (here they are \(D(0)\) and \(D(1)\) respectively). 

Recall Fisher's Sharp Null is \(H_0: Y(D = 1) = Y(D = 0)\) (for all units). It is stronger than the ``Neyman's weak null,'' which only asserts $\mathbb{E}[Y(D = 1) - Y(D = 0)] = 0$ in expectation. Under Fisher's Sharp Null and assuming binary instrument and treatment,
\begin{align*}
    \tau_{ITT} &= \mathbb{E}[Y(D = D(1)) - Y(D = D(0))] \\
    &= \mathbb{E}[(Y(D = 1) - Y(D = 0) \mathbf{1}\{D(1) > D(0)\}) + \\
    &~~~~~~~(Y(D = 0) - Y(D = 1) \mathbf{1}\{D(1) < D(0)\})] \\ 
    &= \mathbb{E}[(Y(D = 1) - Y(D = 0)) \times \\
    &~~~~~~~(\mathbf{1}\{D(1) > D(0)\} - \mathbf{1}\{D(1) < D(0)\})] \\
    &= 0
\end{align*}
One limitation of the Intention-to-treat effect is that it might not answer scientific questions about the effect of taking treatment (which is measured by the ATE). 

\subsubsection{Dealing with Noncompliance}
In the case of binary instrument and treatment, we can view the instrument as the treatment assigned. At a high level, we aim to stratify the population based on how each unit complies with the assigned treatment. We begin by defining a latent (unobservable) variable \(U_i\) (this is not the confounder \(U\)) for each unit \(i\) according to the table below. 

\begin{table}[h]
    \centering
    \begin{tabular}{|c|c|c|c|}
        \hline
        \(D_i(1)\) & \(D_i(0)\) & Label & \(U_i\) \\
        \hline
        \( 1 \) & \( 1 \) & Always Taker & a \\
        \( 1 \) & \( 0 \) & Complier & c \\
        \( 0 \) & \( 1 \) & Defier & d \\
        \( 0 \) & \( 0 \) & Never Taker & n \\
        \hline
    \end{tabular}
\end{table}

Usually we have to make the additional assumption of \textit{Monotonicity}.
\begin{assumption}
    Monotonicity means there is no defier in the population: for all units \(i\), \[D_i(1) \geq D_i(0)\]
\end{assumption}
\textit{Monotonicity} also has a testable implication that \(\Pr{D = 1 \mid Z = 1} \geq \Pr{D = 1 \mid Z = 0}\). When control units don't have access to treatment (e.g., when treatment is a new drug unavailable on the market), \(Z = 0\) would imply \(D = 0\), i.e., we have one-sided noncompliance (there can be compliers and never takers, but there cannot be any defier or always taker). One-sided noncompliance further implies \textit{Monotonicity}.

Under \textit{Relevance, Exclusion Restriction, Exchangeability} and if we assume one-sided noncompliance (meaning \(Z = 0\) would imply \(D = 0\)), we can identify the effect of removing treatment on the whole population or among the treated.
\begin{align*}
    \mathbb{E}[Y - Y(D = 0)] &= \mathbb{E}[Y] - \mathbb{E}[Y(D = 0) \mid Z = 0]\\
    &= \mathbb{E}[Y] - \mathbb{E}[Y \mid Z = 0] \\
    \mathbb{E}[Y - Y(D = 0) \mid D = 1] &= \frac{\mathbb{E}[Y] - \mathbb{E}[Y \mid Z = 0]}{\Pr{D = 1}}
\end{align*}
Where the last equation relies on the fact that \(\mathbb{E}[Y - Y(D = 0) \mid D = 0] = 0\). We can estimate these values using various techniques, for example the Regression, Inverse Probability Weighted, and Double Robust estimators.

\section{Lecture 15: Compiler/Local Average Treatment Effect (CATE/LATE) }{Tobias Kreiman \& 
Zixun Tan Jiarong Zhou(revisions)}

\subsection{Last time:}
\textbf{Monotonicity:} \(D_i(1) \geq D_i(0) \quad \text{for all } i\)
\begin{itemize}
    \item Holds when there is one-sided non-compliance:
\end{itemize}
\[
\Pr(D = 1 \mid Z = 1) \geq \Pr(D = 1 \mid Z = 0)
\]
\begin{itemize}
    \item Why \(\geq\) not \(>\)?
\end{itemize}
\[
\text{If} \quad \Pr(D = 1 \mid Z = 1) = \Pr(D = 1 \mid Z = 0), \quad \text{then the relevance assumption is violated}
\]

\subsection{CATE/LATE}
\[
\tau_c = \mathbb{E}\left[ Y(D = 1) - Y(D = 0) \mid U = c \right]
\]
\begin{itemize}
    \item \(U\) is the unobservable adherence random variable, \(c\) is the complier.
\end{itemize}
\[
= \mathbb{E}\left[ Y(D = 1) - Y(D = 0) \mid D(1) \geq D(0) \right]
\]
\[
= \mathbb{E}\left[ Y(D = 1) - Y(D = 0) \mid D(1) = 1, D(0) = 0 \right]
\]
\begin{itemize}
    \item Identifiable under monotonicity + 3 standard IV assumptions (Imbens and Angrist, 1994).
\end{itemize}

\[
\tau_c = \frac{\mathbb{E}[Y \mid Z = 1] - \mathbb{E}[Y \mid Z = 0]}{\mathbb{E}[D \mid Z = 1] - \mathbb{E}[D \mid Z = 0]}
\]

\textbf{Proof of identification:}

\text{Numerator:}
\begin{align*}
\mathbb{E}[Y \mid Z = 1] - \mathbb{E}[Y \mid Z = 0] &= \mathbb{E}[Y(Z = 1) \mid Z = 1] - \mathbb{E}[Y(Z = 0) \mid Z = 0] \quad \text{(consistency)} \\
&= \mathbb{E}[Y(Z = 1)] - \mathbb{E}[Y(Z = 0)] \quad \text{(exchangeability)} \\
&= \mathbb{E}[Y(Z = 1, D = D(1)) - Y(Z = 0, D = D(0))] \\
&= \mathbb{E}[Y(D = D(1)) - Y(D = D(0))]
\end{align*}

\[
\mathbb{E}\left[ Y(D = 1) - Y(D = 0) \mid D(1), D(0) \right] =
\begin{cases}
    \mathbb{E}\left[ Y(D = 1) - Y(D = 0) \right] & \text{if } D(1) = 1, D(0) = 0 \\
    \mathbb{E}\left[ Y(D = 0) - Y(D = 1) \right] & \text{if } D(1) = 0, D(0) = 1
\end{cases}
\]

\begin{align*}
\mathbb{E}[D \mid Z = 1] - \mathbb{E}[D \mid Z = 0] &= \mathbb{E}[D(Z = 1) \mid Z = 1] - \mathbb{E}[D(Z = 0) \mid Z = 0] \\
&= \mathbb{E}[D(Z = 1) - D(Z = 0)] \\
&= P(D(Z = 1) = 1) - P(D(Z = 0) = 1) \\
&= P(D(1) > D(0)) + P(D(1) = D(0) = 1) \\
&\quad - P(D(1) < D(0)) - P(D(1) = D(0) = 1) \\
&= P(D(1) > D(0)) - P(D(1) < D(0))
\end{align*}

\[
\frac{\text{Numerator}}{\text{Denominator}} 
= \frac{\mathbb{E}\left[ (Y(D = 1) - Y(D = 0)) \left( \mathbf{1}\{D(1) > D(0)\} - \mathbf{1}\{D(1) < D(0)\} \right) \right]}{P(D(1) > D(0)) - P(D(1) < D(0))}
\]

\[
\text{Apply monotonicity:} \quad \mathbf{1}\{D(1) < D(0)\} = 0 \quad \Rightarrow \quad P(D(1) < D(0)) = 0
\]


\[
= \frac{\mathbb{E}\left[ (Y(D = 1) - Y(D = 0)) \, \mathbf{1}\{D(1) > D(0)\} \right]}{P(D(1) > D(0))}
\]

\[
= \mathbb{E}\left[ (Y(D = 1) - Y(D = 0)) \mid D(1) > D(0) \right]
\]
where \(D(1) > D(0)\) is an unidentifiable subgroup.

\bigskip

\textbf{In the context of the Minneapolis Domestic Violence Experiment:}
\begin{itemize}
    \item \(Y = \text{re-offense}\)
    \item \(Z = \text{random assignment to arrest or counseling}\)
    \item \(D = 1\): Arrest, occurs with probability \(1/3\)
    \item \(D = 0\): Counseling/Separation, occurs with probability \(2/3\)
\end{itemize}

\[
\tau_c = \frac{\mathbb{E}[Y \mid Z = 1] - \mathbb{E}[Y \mid Z = 0]}{\mathbb{E}[D \mid Z = 1] - \mathbb{E}[D \mid Z = 0]}
\]

\begin{tikzpicture}[->, node distance=2cm, thick]
    % Nodes
    \node (Z) at (0, 0) {\(Z\)};
    \node (D) at (2, 0) {\(D\)};
    \node (Y) at (4, 0) {\(Y\)};
    \node (U) at (3, -1.5) {\(U\)};

    % Arrows
    \draw[->] (Z) -- (D);
    \draw[->] (D) -- (Y);   
    \draw[->] (U) -- (D);
    \draw[->] (U) -- (Y);
\end{tikzpicture}

\text{Among compliers,} $Z \to Y \quad \text{and} \quad Z \to D \to Y$

\subsection{Classical IV Models}

\textbf{Pre-1990 Parametric Two-Stage Model}

\begin{align*}
    D &= \alpha_0 + \alpha_1 Z + \gamma \tag{A} \\
    Y &= \beta_0 + \beta_1 D + \epsilon \tag{A}
\end{align*}

where \(\epsilon\) and \(\gamma\) are random error terms.

\(Z\) is the IV that follows:
\begin{enumerate}
    \item \text{Relevance:} 
    \[
    \text{Cov}(D, Z) \neq 0 \Rightarrow \alpha_1 \neq 0
    \]
    \item \text{Exclusion Restriction:} \(Z\) does not appear in the second-stage OLS regression, meaning \(Z\) affects \(Y\) only through \(D\).

    \item \text{Unconfoundedness:} 
    \[
    \text{Cov}(Z, Y) = 0 \quad \text{and} \quad \text{Cov}(Z, \epsilon) = 0
    \]
\end{enumerate}

With covariates, the models become:
\begin{align*}
    D &= \alpha_0 + \alpha_1 Z + \alpha_2 X^\top + \gamma \\
    Y &= \beta_0 + \beta_1 D + \beta_2 X^\top + \epsilon
\end{align*}
with the assumptions:
\[
\text{Cov}(Z, \epsilon \mid X) = 0 \quad \text{and} \quad \text{Cov}(Z, \gamma \mid X) = 0
\]
These models are causal if the structural assumptions hold.

\textbf{Structural Equation Model}

\begin{align*}
    D(Z) &= \alpha_0 + \alpha_1 Z + \gamma \tag{B} \\
    Y(Z, D) &= \beta_0 + \beta_1 D + \epsilon \tag{B}
\end{align*}

\textbf{Theorem:} Under the assumptions of SUTVA (Stable Unit Treatment Value Assumption) and the standard IV conditions:
\begin{enumerate}
    \item \text{Relevance:} \(Z \not\perp\!\!\!\perp D\) 
    \item \text{Exclusion Restriction:} \(Y(Z = z, D = d) = Y(D = d)\)
    \item \text{Exchangeable IV:} \(Z \perp\!\!\!\perp (D(Z), Y(Z))\).
\end{enumerate}

The structural equation model (B) implies the observed model (A) under the following assumptions:
\begin{enumerate}
    \item \text{Relevance:} \(\text{Cov}(D, Z) \neq 0\).
    \item \text{Exclusion Restriction:} \(Z\) does not appear in the second-stage OLS regression.
    \item \text{Unconfounded IV:} \(\text{Cov}(Z, Y) = 0\) and \(\text{Cov}(Z, \epsilon) = 0\).
\end{enumerate}

Under the structural model, $\beta_1 = \mathbb{E}[Y(D=1) - Y(D=0)]$. Additionally, under IV, $\beta_1 = \frac{\text{Cov}(Y,Z)}{\text{Cov}(D,Z)}$ (see the "Wald Estimand" from Wald (1940)). The proof of this is as follows:
\begin{proof}
$$D = \alpha_0 + \alpha_1 Z + \nu$$
$$Y = \beta_0 + \beta_1 D + \varepsilon$$
$$\text{Cov}(Y, Z) = \text{Cov} (\beta_0 + \beta_1 D + \varepsilon, Z) = \beta_1 \text{Cov}(D, Z) + \text{Cov}(\varepsilon, Z) = \beta_1 \text{Cov}(D, Z),$$
where in the last step we used the unfounded IV assumption that $\text{Cov}(\varepsilon, Z) = 0$. This gives us the result that (by dividing by $\text{Cov}(D,Z)$):
$$\beta_1 = \frac{\text{Cov}(Y,Z)}{\text{Cov}(D,Z)},$$
which is valid due to relevance since $\text{Cov}(D, Z) \neq 0$.
\end{proof}

\subsubsection{2 Stage Estimate of $\beta_1$}
We now describe the 2-stage least squares estimate of $\beta_1$.

\begin{enumerate}
    \item OLS regress $D \sim Z$. This gives predictions for $\hat{D} = \hat{\alpha}_1 Z$.
    \item OLS regress $Y \sim \hat{D}$, where $\hat{D} = \hat{\alpha}_1Z$. This gives a predicted $\hat{\beta}_1$.
\end{enumerate}

This procedure leads to the following theorem: Under the 2 stage model with $\text{Cov}(D,Z) \neq 0$ and $\text{Cov}(Z,\nu) = \text{Cov}(Z,\varepsilon) = 0$, $\beta_1$ is the coefficient in a population regression of Y on the predicted value from a population regression of $D$ on $Z$.

As a side note, we can also control for covariates by conditioning on them:
$$\beta_1 = \frac{\text{Cov}(Y, Z | X) }{\text{Cov}(D, Z | X)}$$
The two stage approach with covariates becomes:
\begin{enumerate}
    \item OLS regress $D \sim Z, X$, giving predicted $\hat{D} = \hat{\alpha}_1 Z + \hat{\alpha_2}^T X$
    \item OLS regress $Y \sim \hat{D}$ yielding $\hat{\beta}_1$. 

\end{enumerate}
This 2 stage model is easy to implement, but it has strong parametric assumptions, such as linearity, additivity, and constant effects.

Putting all together through the lens of CACE, the average causal effect under compliers is:
$$\beta_1 = \frac{\text{Cov}(Y, Z ) }{\text{Cov}(D, Z )} = \frac{\mathbb{E} [Y | Z = 1] - \mathbb{E} [Y | Z = 0]}{\mathbb{E} [D | Z = 1] - \mathbb{E} [D | Z = 0]} = $$
And the right hand side can be written as (focusing on compliers):
$$= \mathbb{E} [Y(D=1) - Y(D=0) | D(1) > D(0)]$$.
We reiterate that this assumes monotonicity. Additionally, remember that this is untestable, since we cannot observe whether $D_i(1) \geq D_i(0)$.

\subsubsection{Some Practical Considerations}  
In practice, several challenges arise when using preference-based instruments. For example, consider the random assignment of cases to judges (\(Z\)) as an instrument. Here, \(D\) represents whether there is a pre-trial detention, and \(Y\) indicates whether there is a future offense. If \(Z = 1\), it implies that the case was randomly assigned to a harsh judge. However, it is not always possible to assume that \(D_i(1) \geq D_i(0) \ \forall i\), as there may be exceptions. For instance, a harsh judge might show leniency in specific cases due to personal biases or "soft spots" for particular defendants.  

Another practical issue is the limited policy relevance of analyzing effects for unidentified subgroups. Eaton (2010) critiques this approach, stating:  
\begin{quote}  
"This goes beyond the story of looking for an object where the light is strong enough to see. Rather, we have control over the light, but we choose to let it fall where it may and then proclaim that whatever the light illuminates is what we were looking for all along."  
\end{quote}  
This highlights the importance of a structured approach: (1) defining the estimand, (2) ensuring identification, and (3) performing estimation.  

In contrast, a policy-relevant analysis focuses on identifiable subgroups. For example, Kennedy et al. (2019) suggest that predicting which individuals are likely to comply has practical significance. Formally, this can be expressed as:  
\[
\Pr(D(1) > D(0)) = \mathbb{E}[D | X, Z = 1] - \mathbb{E}[D | X, Z = 0].  
\]  
Such an approach ensures that findings align with actionable insights, making them more useful for policymaking.  

\section{Lecture 16: CACE, IV Methods, and Overlap Violations in RDD }{Minji, Hyemin Park, Jiarong Zhou(revisions))}



\subsubsection{Key Definitions:}
\begin{itemize}
    \item \( Z_i \): Treatment assigned to unit \( i \) (1 for treatment, 0 for control).
    \item \( D_i \): Treatment received by unit \( i \) (1 for treatment, 0 for control).
    \item \( Y_i \): Outcome of interest for unit \( i \).
\end{itemize}

\subsection{CACE (Complier Average Causal Effect)}
 The average causal effect of the treatment for individuals who comply with the assigned treatment. 

\subsubsection{Estimation of CACE}
To estimate the CACE under IV assumptions and monotonicity, the formula is given as:


\[
\mathbb{E}\left[ y(D=1) - y(D=0) \mid D(1) > D(0) \right] =
\]
This can be broken down as follows: 
\[
= \frac{\mathbb{E}[Y \mid Z=1] - \mathbb{E}[Y \mid Z=0]}{\mathbb{E}[D \mid Z=1] - \mathbb{E}[D \mid Z=0]}=  \frac{\mathbb{E}\left[ \mathbb{E}[Y \mid X, Z=1] - \mathbb{E}[Y \mid X, Z=0] \right]} {\mathbb{E}\left[ \mathbb{E}[D \mid X, Z=1] - \mathbb{E}[D \mid X, Z=0] \right]}
\]

\subsubsection{How to estimate?}
\begin{itemize}
    \item regression: $\mu_Z(x) = \mathbb{E}[y \mid X, Z=z]$
    \item weighting: $\lambda_Z(x) = \mathbb{E}[D \mid X, Z=z]$
    \item doubly-robust: $e(X) = P(Z=1 \mid X)$
\end{itemize}

\[
\hat{\mu}_2(x) \text{ by regress } y \sim X \mid Z=z
\]
\[
\hat{\lambda}_z(x) \text{ by regress } D \sim X \mid Z=z
\]

\subsection{Estimators}

\begin{itemize}
    \item \textbf{Regression Estimator}:
    \[
\hat{\tau}_{i, reg} = \frac{\sum_{i=1}^{n} \left\{ \hat{\mu}_1(X_i) - \hat{\mu}_0(X_i) \right\}}{\sum_{i=1}^{n} \left\{ \hat{\lambda}_1(X_i) - \hat{\lambda}_0(X_i) \right\}}
\]
    This estimator uses the regression estimates for the potential outcomes and the treatment assignment, providing a way to compute the average causal effect.

    \item \textbf{Inverse Probability Weighting (IPW) Estimator}:
    \[
\hat{\tau}_{i, IPW} = \frac{\sum_{i=1}^{n} \left\{ \frac{Z_i Y_i}{\hat{e}(X_i)} - \frac{(1-Z_i) Y_i}{1 - \hat{e}(X_i)} \right\}}{\sum_{i=1}^{n} \left\{ \frac{Z_i D_i}{\hat{e}(X_i)} - \frac{(1-Z_i) D_i}{1 - \hat{e}(X_i)} \right\}}
\]
    This estimator accounts for bias by weighting the observed outcomes according to the propensity scores.

    \item \textbf{Doubly Robust Estimator}:
  \[
\hat{\tau}_{i, DR} = \frac{\sum_{i=1}^{n} \left\{ \left( \frac{Z_i}{\hat{e}(X_i)} - \frac{(1-Z_i)}{1 - \hat{e}(X_i)} \right) \left( Y_i - \hat{\mu}_z(X_i) \right) + \hat{\mu}_1(X_i) - \hat{\mu}_0(X_i) \right\}}{\sum_{i=1}^{n} \left\{ \left( \frac{Z_i}{\hat{e}(X_i)} - \frac{(1-Z_i)}{1 - \hat{e}(X_i)} \right) \left( D_i - \hat{\lambda}_z(X_i) \right) + \hat{\lambda}(X_i) - \hat{\lambda}_0(X_i) \right\}}
\]
    This estimator combines both regression and weighting approaches, ensuring validity even if one model is misspecified.
\end{itemize}


\subsubsection{Question to Consider}


Should you always use the same covariate to condition on \(D\) and \(Y\)?


\subsubsection{Graphical Interpretation}

The graph outlines the relationships where:

- \(X_1\): Represents measured confounders that affect both \(Z\) and \(Y\).

- \(X_2\): Influences treatment \(D\) but not directly associated with the outcome.

- \(X_3\): Affects outcome \(Y\) without influencing treatment \(D\).

\begin{center}
\begin{tikzpicture}[
  node distance=1.5cm and 2cm, 
  >={Stealth[round]}, 
  thick, 
  observed/.style={fill=white!20, draw=none, font=\large},
  unobserved/.style={draw, circle, minimum size=0.8cm, font=\small},
  square/.style={draw, minimum size=0.8cm, font=\small}
]

% Nodes
\node[observed, square] (X1) at (0,0) {$X_1$ (measured confounder)};
\node[unobserved] (U) at (4,6) {$U$};
\node[observed, square] (X2) at (2,6) {$X_2$};
\node[observed, square] (X3) at (8,6) {$X_3$};
\node[observed] (Z) at (0,4) {$Z$};
\node[observed] (D) at (4,4) {$D$};
\node[observed] (Y) at (8,4) {$Y$};

% Arrows
\draw[->] (X1) -- (Z);
\draw[->] (X2) -- (D);
\draw[->] (X3) -- (Y);
\draw[->] (Z) -- (D);
\draw[->] (D) -- (Y);
\draw[->] (U) -- (D);
\draw[->] (U) -- (Y);
\draw[->] (X1) -- (Y);
\end{tikzpicture}
\end{center}

\begin{itemize}
    \item We should always include \(X_1\) in \(\lambda\) and \(\mu\) regressions to achieve exchangeability.
    \item You can additionally condition on \(X_2\) in \(\lambda\) regress for efficiency.
    \item You can condition on \(X_3\) in \(\mu\) regression for efficiency.
    \item If we condition on \(X_1\), \(X_2\), and \(X_3\) in both \(\lambda\) and \(\mu\), that is fine.
\end{itemize}

\subsubsection{Question}
Which among the regression, IPW, and DR estimators should we use?

\begin{itemize}
    \item If \(e(X)\) is known: Use IPW or DR.
    
    \item If \(e(X)\) is unknown but \(X\) is discrete and low-dimensional, any estimator is good. They are numerically equivalent and optimally efficient.
    
    \item If \(e(X)\) is unknown and \(X\) includes continuous components or is high-dimensional, choose the doubly robust (DR) estimator.
\end{itemize}

\textbf{Additional Explanation:} 

In experimental settings where the instrument propensity \(e(X)\) is known (as determined by the experimental design), using the IPW estimator is generally robust. The robust estimator is also suitable because it utilizes the true \(e(X)\) while also incorporating estimated models.

In cases where \(e(X)\) is unknown but \(X\) is discrete and low-dimensional, we can estimate nuisance functions using empirical distributions, making any estimator effective.

However, when \(e(X)\) is unknown and involves continuous components or is high-dimensional, the doubly robust estimator is preferable. It offers better theoretical properties under milder assumptions, allowing it to perform well even if only some of the nuisance functions are estimated accurately.


\subsection{Beyond the CACE
: Partial Identification of Average Causal Effect}

The average causal effect \(\tau\) is defined as:
\[
\tau = E[Y(D=1)] - E[Y(D=0)]
\]
This represents the expectation if we treated everyone versus treating no one.

Under the monotonicity and exclusion restriction, we can establish bounds:
\[
\tau_L \leq \tau \leq \tau_U
\]

Where:
\[
\tau_L = E[Y | Z=1] - E[Y | (1-D) + D | Z=0]
\]
\[
\tau_U = \tau_L + 1 - E[D | Z=1] + E[D | Z=0]
\]

The length of the bound is determined by the proportion of non-compliers:
\[
1 - (E[D | Z=1] - E[D | Z=0]) = P(D(1) \leq D(0))
\]

This formulation shows that we can partially identify the average causal effect, giving us lower and upper bounds based on observable quantities that can be estimated from the data. 


\newpage
\subsection{Examples of IVs (Instrumental Variables)}

\subsubsection{Challenge: Identifying and Justifying the IV}
One of the challenges with using instrumental variables (IVs) is identifying a valid instrument and justifying its use. The instrument must satisfy key assumptions, including the exclusion restriction, which states that it affects the outcome only through the treatment.

\subsubsection{Experiments with Non-compliance}  
\textbf{Example: Minneapolis Domestic Violence Experiment}  
- \textbf{\(Z\)} (Instrument): The penalty recommendation by officers responding to a domestic violence incident. This recommendation was randomly assigned.  
- \textbf{\(D\)} (Treatment): Whether the officer adhered to the recommended penalty.  
- \textbf{\(Y\)} (Outcome): Whether the individual re-offended.  

This experimental setup involves non-compliance, as officers had the discretion to deviate from the randomly assigned recommendation. In such cases, Instrumental Variable (IV) methods are particularly useful to estimate the causal effect of the recommendation on the likelihood of re-offending.  

\subsubsection{Distance-based Measures}  
\textbf{Example: Family Visitation and Re-offense Rates (Mauro et al., 2008)}  
- \textbf{\(Z\)} (Instrument): The proximity of an inmate’s family to the jail.  
- \textbf{\(D\)} (Treatment): Whether the family visited the inmate.  
- \textbf{\(Y\)} (Outcome): The re-offense rate after release from jail.  

This approach assumes that the distance between the family’s residence and the jail affects the likelihood of visitation but does not directly influence the re-offense rate except through its impact on visitation. By using distance as an instrument, it becomes possible to estimate the causal effect of family visits on post-release behavior.  

\subsubsection{Preference-based Measures}
\textbf{Example :} \\
\textbf{Z} (Instrument): Doctor’s preference for prescribing Paxlovid (an antiviral drug). \\
\textbf{D} (Treatment): Whether the patient was prescribed Paxlovid. \\
\textbf{Y} (Outcome): Potential liver damage as a side effect. \\

The instrument is the variation in doctors' prescribing behavior in this case. Doctors with different preferences might prescribe Paxlovid differently, and this variation can be exploited as an IV to study the effect of Paxlovid on liver damage.

\subsubsection{Time-based Measures (Change in Policy)}
\textbf{Example :} \\
\textbf{Z} (Instrument): Expansion in the mandated reporter law in 2014, which increased the number of people required to report cases of child abuse or neglect. \\
\textbf{D} (Treatment): Whether a child welfare investigation was opened. \\
\textbf{Y} (Outcome): Educational outcomes of children in the family. \\

Here, a policy change creates variation in reporting, which serves as an instrument for the likelihood of child welfare investigations. This variation allows researchers to study the causal effect of an investigation on children's education.

$$X \rightarrow Z \rightarrow D \rightarrow Y$$
Where $X$ are covariates that may affect both the treatment ($D$) and the outcome ($Y$), while $Z$ (the instrument) affects $D$ but only affects $Y$ indirectly through $D$.

\begin{center}
\begin{tikzpicture}[
  node distance=1.5cm and 2cm, 
  >={Stealth[round]}, 
  thick, 
  observed/.style={fill=white!20, draw=none, font=\large},
  unobserved/.style={draw, circle, minimum size=0.8cm, font=\small},
  square/.style={draw, minimum size=0.8cm, font=\small}
]

% Nodes
\node[observed] (Z) {Z};
\node[observed, right=of Z] (D) {D};
\node[observed, right=of D] (Y) {Y};
\node[unobserved, above=of D] (U) {U};
\node[square, below left=of Z] (X) {X};

% Edges
\draw[->] (X) -- (Z);
\draw[->] (X) -- (Y);
\draw[->] (Z) -- (D);
\draw[->] (D) -- (Y);
\draw[->] (U) -- (D);
\draw[->] (U) -- (Y);
\end{tikzpicture}
\end{center}


\subsubsection{Genes: Mendelian Randomization}
\begin{itemize}
    \item \textbf{Mendel's 2nd law:} The law of random assortment states that the inheritance of one trait is independent of other traits.
    \item Mendelian randomization uses genetic variation as an IV to study the causal effects of exposure (e.g., cholesterol levels) on an outcome (e.g., cancer).
\end{itemize}
\textbf{Example :} \\
\textbf{Z} (Instrument): Genes \\
\textbf{D} (Treatment): Cholesterol levels \\
\textbf{Y} (Outcome): Likelihood of developing cancer \\

In Katan (1986), apolipoprotein E genes were used as an instrument to study the effect of cholesterol levels on cancer risk. The assumption here is that genes affect cholesterol levels but do not directly affect cancer, satisfying the exclusion restriction.

\begin{center}
\begin{tikzpicture}[
  node distance=1.5cm and 2cm,
  >={Stealth[round]}, 
  thick, 
  observed/.style={draw=none, font=\small},
  unobserved/.style={draw, circle, minimum size=1cm, font=\small}
]

% Nodes
\node[observed] (G1) {$G_1$};
\node[observed, below=0.7cm of G1] (G2) {$G_2$};
\node[observed, below=0.7cm of G2] (Gdots) {$\vdots$};
\node[observed, below=0.7cm of Gdots] (Gp) {$G_p$};
\node[observed, right=of Gdots, xshift=1cm] (D) {$D$};
\node[unobserved, above right=of D] (U) {$U$};
\node[observed, right=of D] (Y) {$Y$};

% Edges
\draw[->] (G1) -- (D);
\draw[->] (G2) -- (D);
\draw[->] (Gp) -- (D);
\draw[->] (U) -- (D);
\draw[->] (U) -- (Y);
\draw[->] (D) -- (Y);

\end{tikzpicture}
\end{center}

$$G_1, G_2, \dots, G_p \rightarrow D \rightarrow Y$$
Where $G_1, G_2, \dots, G_p$ represent different genetic variants (instruments), $D$ represents cholesterol levels, and $Y$ represents the outcome (cancer). 

$$G_1, G_2, \dots, G_p \rightarrow U$$
Where $U$ represents unobserved confounders.

\paragraph*{Standard Linear IV Model:}
$$Y = \beta_0 + \beta D + \beta_\mu U + \epsilon_Y$$
$$D = \gamma_0 + \gamma_1 G_1 + \dots + \gamma_p G_p + \gamma_U U + \epsilon_D$$
Here, $Y$ is the outcome, $D$ is the treatment, $G_1, G_2, \dots, G_p$ are the genetic instruments, $U$ is an unobserved confounder, and $\epsilon_Y$ and $\epsilon_D$ are error terms.

\paragraph*{ Reduced Form:}
$$Y = \beta_0 + \beta_D \gamma_0 + \beta_D \gamma_1 G_1 + \dots + \beta_D \gamma_p G_p + (\beta_U + \beta_D \lambda_U) U + \epsilon_Y$$

\paragraph*{Apply Two-Stage Least Squares Estimators to Estimate $\beta$:}
See Chapter 25 of \textit{Ding} for further details.


\subsubsection*{ Critiques of Mendelian Randomization Analyses}
Mendelian Randomization is a powerful tool, but there are several potential issues to consider:
\begin{itemize}
    \item \textbf{SUTVA (Stable Unit Treatment Value Assumption)} or \textbf{target trial emulation} may not be satisfied
    \begin{itemize}
        \item For example, treatment variables like BMI or cholesterol levels may not satisfy SUTVA.
        \item There may be situations where the treatment status of one individual affects the outcome of another, or where the treatment is not well-defined.
    \end{itemize}
    \item \textbf{Exclusion restriction} may be violated
    \begin{itemize}
        \item There may be alternative pathways through which the genes affect the outcome. For example, genes may directly affect cancer development, in addition to affecting cholesterol levels, which would violate the exclusion restriction.
    \end{itemize}
    \item \textbf{Linearity assumption in the IV model} may be violated
    \begin{itemize}
        \item Mendelian Randomization typically assumes a linear relationship between the treatment and the outcome, but in reality, the relationship may be nonlinear.
    \end{itemize}
    \item \textbf{Measurement error} may exist
    \begin{itemize}
        \item The treatment variable (e.g., cholesterol levels) or the outcome variable (e.g., cancer diagnosis) may not be measured accurately, which could lead to biased results.
    \end{itemize}
\end{itemize}


\subsection{Violations of Overlap / Positivity}
\subsubsection{Overlap Condition:}
$$P(0 < e(X) < 1) = 1$$
The overlap or positivity assumption requires that all units have some probability of receiving either treatment or control (i.e., no deterministic treatment assignment). Here. $e(X)$ represents the propensity score.

\subsubsection{Tension between Overlap and Exchangeability:}
There is a tension between the overlap assumption and the exchangeability assumption. As we condition on more covariates to satisfy exchangeability, we may reduce the randomness in the treatment assignment, which can lead to a violation of overlap. \\
\\
\textbf{Example:  Child Welfare Investigation} \\
\textbf{X}: Administrative records, allegations. \\
\textbf{D}: Child welfare investigation \\

In this context, administrative records and the natural language of allegations are used to determine whether a child welfare investigation is opened. Conditioning on these covariates may improve exchangeability but can reduce overlap if these features deterministically affect the treatment decision.


\subsection{Conceptual Challenge: Deterministic Treatment Assignment and Counterfactuals}

In settings where overlap is violated (i.e., units deterministically receive treatment or control), conceptual challenges arise. For instance, what does it mean to consider a counterfactual outcome for a unit that always receives treatment? This raises questions about the definition of potential outcomes for such deterministic units.

To address this, overlap requires that propensity scores be bounded away from 0 and 1:
$$\epsilon < e(X) < 1 - \epsilon$$
where $\epsilon$ is a small positive constant. This boundedness assumption is necessary for many causal inference techniques, particularly in high-dimensional settings.

\subsubsection{Handling Overlap Violations: Trimming}
One common approach to handle overlap violations is trimming, where observations with extreme propensity scores are removed from the analysis. The downside is that this changes the estimand, meaning we are no longer estimating the causal effect for the entire population but rather for a subset where overlap holds.

\subsection{Regression Discontinuity Design (RDD)}

When overlap is violated, an alternative approach is to use a \textbf{regression discontinuity design (RDD)}. RDD is useful in settings with a threshold-based decision rule, where the decision threshold can be considered somewhat arbitrary. By examining outcomes near the threshold, we can assume that the groups on either side are exchangeable, and we can estimate causal effects by comparing outcomes just above and below the threshold.

\subsubsection{Example: COVID-19 Treatment and Oxygen Levels}

Consider a new antiviral COVID-19 treatment. We want to measure its impact on oxygen levels (a continuous outcome) one week after treatment. During the pandemic, this treatment was restricted to high-risk patients, specifically those aged 65 and older.

\begin{itemize}
    \item \textbf{Treatment Group}: Patients aged 65 and older.
    \item \textbf{Control Group}: Patients under 65.
\end{itemize}

Since there is no randomness in who receives the treatment (i.e., there is no overlap), we cannot apply typical causal inference methods. However, we can use RDD by assuming continuity of the outcome regression functions around the threshold (age 65).

\subsubsection{Assumptions for RDD:}
The key assumption in RDD is that the outcome regression functions are continuous at the threshold:
$$\lim_{X \to 65^-} E[Y | X] = \lim_{X \to 65^+} E[Y | X]$$
where $X$ is the age variable. Under this assumption, the difference in outcomes around age 65 can be attributed to the causal effect of the treatment.

\textbf{Interpretation}
The causal effect of the treatment is estimated by the difference in the observed means just above and below the threshold:
$$\text{Causal Effect} = E[Y | X = 65^+] - E[Y | X = 65^-]$$

In this example, we would observe oxygen levels (Y) for patients just below 65 (control group) and just above 65 (treatment group) and interpret the difference in means as the causal effect of the COVID-19 treatment.

\subsection*{Next Class}
How to implement a regression discontinuity analysis in practice. \\
Reminder: there is a midterm exam scheduled for next Tuesday.

\section{Lecture 17: Sharp Regression Discontinuity Design (RDD)}{Carlos Guirado, Stella Jia, Kayla Sim(revisions)}

\subsection{Regression Discontinuity Design (RDD)}

\subsubsection{Introduction and Motivating Example}

In the last lecture, Regression Discontinuity (RD) was introduced through a COVID treatment example:
\begin{itemize}
    \item Treatment (D): New antiviral COVID treatment
    \item Outcome (Y): Patient oxygen levels
    \item Running variable (X): Age
    \item Policy: Treatment restricted to those 65 and older
\end{itemize}

This creates a sharp discontinuity in treatment probability:
\[P(D=1|X \geq 65) = 1\]
\[P(D=1|X < 65) = 0\]

\subsubsection{Exchangeability in RD}

A key feature of RD is that exchangeability holds by design. This can be understood by examining the causal structure:

\begin{enumerate}
    \item The running variable X (age) determines treatment D:
    \[D = \mathbf{1}[X \geq x_0]\]
    
    \item Traditional settings require:
    \begin{itemize}
        \item Both exchangeability and overlap
        \item D as a function of X and other variables V:
        \[D = f(X,V)\]
        \item V affects D but not Y for overlap to hold
    \end{itemize}
    
    \item In sharp RD:
    \begin{itemize}
        \item Overlap does not hold
        \item D is determined solely by X (i.e. deterministic)
        \item Conditioning on X eliminates all unobserved confounding
        \item Exchangeability holds automatically due to treatment assignment mechanism
    \end{itemize}
\end{enumerate}

\subsubsection{Examples of RD in Practice}

\begin{enumerate}
    \item Thistlethwaite \& Campbell (1960) - First RD study:
    \begin{itemize}
        \item D: Certificate of merit
        \item Treatment rule: \[D = \mathbf{1}[\text{test score} \geq 10.5]\]
        \item Y: Plans for graduate study and scientific research
        \item Demonstrated how seemingly arbitrary cutoffs can be leveraged for causal inference
    \end{itemize}

    \item Carpenter \& Dobkin (2009):
    \begin{itemize}
        \item D: Legal drinking status
        \item Treatment rule: \[D = \mathbf{1}[\text{age} \geq 21]\]
        \item Y: Death outcomes
        \item Examined different types of mortality:
        \begin{itemize}
            \item Overall mortality
            \item Alcohol-related deaths
            \item External vs. internal causes of death
        \end{itemize}
    \end{itemize}
\end{enumerate}

\subsubsection{Formal Treatment and Identification}

RD identifies the local average treatment effect at the threshold:
\[\tau(x_0) = \mathbb{E}[Y(1) - Y(0)|X=x_0]\]

The identification strategy differs from traditional approaches:

\begin{enumerate}
    \item For treated potential outcome:
    \[\mathbb{E}[Y(1)|X=x_0] = \lim_{\epsilon \to 0^+} \mathbb{E}[Y(1)|X=x_0+\epsilon]\]
    \[= \lim_{\epsilon \to 0^+} \mathbb{E}[Y(1)|D=1, X=x_0+\epsilon]\]
    \[= \lim_{\epsilon \to 0^+} \mathbb{E}[Y|D=1, X=x_0+\epsilon]\]

    \item For control potential outcome:
    \[\mathbb{E}[Y(0)|X=x_0] = \lim_{\epsilon \to 0^+} \mathbb{E}[Y|D=0, X=x_0-\epsilon]\]

    \item The treatment effect:
    \[\tau(x_0) = \lim_{\epsilon \to 0^+} \mathbb{E}[Y|D=1, X=x_0+\epsilon] - \lim_{\epsilon \to 0^+} \mathbb{E}[Y|D=0, X=x_0-\epsilon]\]
\end{enumerate}

\subsubsection{Local Linear Regression Approach}

Implementation typically uses local linear regression:

\begin{enumerate}
    \item Basic procedure:
    \begin{itemize}
        \item Select observations near threshold $x=x_0$
        \item Fit separate linear models on each side of the threshold
        \item Compare fitted values at threshold
    \end{itemize}

    \item Critical bandwidth choice considerations:
    \begin{itemize}
        \item Too small (e.g., 1 day from threshold):
        \begin{itemize}
            \item Few data points
            \item High variance in estimates
        \end{itemize}
        \item Too large (e.g., 2 years from threshold):
        \begin{itemize}
            \item Bias from comparing non-exchangeable individuals
            \item Other factors may vary over wider window
        \end{itemize}
    \end{itemize}

    \item Practical recommendations:
    \begin{itemize}
        \item Report estimates for multiple bandwidths
        \item Include confidence intervals
        \item Use \texttt{rdrobust} package in R
        \item Consider covariate adjustment for wider bandwidths
    \end{itemize}
\end{enumerate}

\subsubsection{Limitations and Potential Problems}

\begin{enumerate}
    \item Data sensitivity:
    \begin{itemize}
        \item Results vulnerable to leverage points near cutoff
        \item Individual observations can heavily influence slope estimates
    \end{itemize}

    \item Continuity assumption:
    \begin{itemize}
        \item Requires continuity in potential outcome regression functions
        \item Not directly testable as counterfactual outcomes unobservable
        \item $\mu_1(x)$ and $\mu_0(x)$ not fully observable
    \end{itemize}

    \item Multiple threshold effects:
    \begin{itemize}
        \item Other treatments/policies may change at threshold
        \item Example: Medicare eligibility at age 65
        \item Complicates isolation of specific treatment effect
        \item May affect interpretation of results
    \end{itemize}

    \item Bandwidth selection challenges:
    \begin{itemize}
        \item Tradeoff between bias and variance
        \item Results may be sensitive to choice
        \item Need to demonstrate robustness across choices
    \end{itemize}
\end{enumerate}

\subsubsection{Comparison to Traditional Experiments}

When considering RD designs, it's useful to think about the ideal experiment:
\begin{itemize}
    \item Traditional experiment might randomize treatment directly
    \item RD approximates experiment around threshold
    \item Treatment effect identified only for subpopulation near threshold
    \item Requires additional assumptions for broader generalization
\end{itemize}

\subsection*{Class Survey}
This lecture was shorter than usual as students were given time to complete a course evaluation survey.

\subsection*{Next Class}
Fuzzy regression  discontinuity design.

\section{Lecture 18: Fuzzy Regression Discontinuity Design (RDD)}{Harish Srinivasan, Aditya Vunnum, Simon Cha, Kayla Sim (revisions)}

\subsection{Wrapping Up Regression Discontinuity}

\subsubsection{Sharp RDD Recap}

In the previous lecture, we explored the concept of Sharp Regression Discontinuity Design (RDD), which uses a strict cutoff point in the running variable to determine treatment assignment. 
\[D = \mathbf{1}[X \geq x_0]\]

A key example was a COVID treatment policy:
\begin{itemize}
    \item \textbf{Treatment (D)}: Administration of a new antiviral COVID treatment.
    \item \textbf{Outcome (Y)}: Patient oxygen levels.
    \item \textbf{Running Variable (X)}: Age.
    \item \textbf{Policy Rule}: Treatment was provided only to patients aged 65 and older.
\end{itemize}

This setup created a sharp discontinuity in treatment probability:
\[
P(D=1|X \geq 65) = 1, \quad P(D=1|X < 65) = 0
\]
The sharp cutoff at age 65 made treatment assignment deterministic based on age alone, allowing for causal inference by comparing outcomes immediately on either side of the threshold.

\textbf{Exchangeability} holds by design in Sharp RDD. Since treatment is fully determined by the running variable, conditioning on the running variable (age in this case) controls for unobserved confounding. This setup does not require overlap between treatment and control groups, as assignment is non-random and based purely on the cutoff.

\subsubsection{Transition to Fuzzy RDD}

In practice, strict cutoff rules are not always feasible, leading to the development of Fuzzy RDD. In Fuzzy RDD, the probability of treatment changes at the threshold, but treatment is not strictly assigned based on the cutoff. Instead, the threshold increases the likelihood of treatment, introducing ``fuzziness."

\subsection{Fuzzy Regression Discontinuity Design (RDD)}

\subsubsection{Introduction to Fuzzy RDD}

Fuzzy RDD relaxes the strict assignment of treatment, allowing for a probabilistic jump in treatment assignment at the threshold. This is useful in settings where the treatment may be recommended but not enforced strictly based on the running variable.

Today’s example uses a treatment indicator \( Z \) defined as:
\[
Z = \mathbf{1}[X \geq x_0]
\]
where \( X \) is the running variable and \( x_0 \) is the cutoff. With Fuzzy RDD, the probability of treatment \( P(D=1 | X=x) \) has a jump at \( X = x_0 \), but treatment is not assigned deterministically.

\subsubsection{Examples of Fuzzy RDD in Practice}

\paragraph{Example 1: College Admissions}
\begin{itemize}
    \item A fuzzy cutoff is applied with SAT scores around 1300.
    \item Colleges may admit students who score below the threshold if they excel in other areas, and may reject students above the threshold if their grades are poor.
\end{itemize}

\paragraph{Example 2: Eligibility for Social Assistance}
\begin{itemize}
    \item A social assistance program uses a predicted income threshold \( X \leq x_0 \) to determine eligibility.
    \item Not all eligible families enroll, and other eligibility criteria may apply.
\end{itemize}

The estimand for the local complier average causal effect in this setup is:
\[
\tau_c(x_0) = \mathbb{E} \left[ Y(D=1) - Y(D=0) \mid D(1) > D(0), X = x_0 \right]
\]

\subsection{Theorem for Fuzzy RDD}

Under the following conditions:
\begin{enumerate}
    \item Monotonicity: \(D(1) \geq D(0)\)
    \item Exclusion restriction: \((Y(Z=z, D=d) = Y(D=d))\)
\end{enumerate}
the local complier average causal effect is given by:
\[
\tau_{c}(x_0) = \frac{\mathbb{E} \left[ Y(D=1) - Y(D=0) \mid X = x_0 \right]}{\mathbb{E} \left[ D(1) - D(0) \mid X = x_0 \right]}
\]

Under continuity conditions, and if \( P(D=1|X=x) \) has a jump at \( x_0 \), then \( \tau_c(x_0) \) is identified as:
\[
\tau_c(x_0) = (\lim_{\epsilon \to 0^+} \mathbb{E}[Y|Z=1, X=x_0+\epsilon] - \lim_{\epsilon \to 0^+} \mathbb{E}[Y|Z=0, X=x_0-\epsilon])
\]
\[
\div (\lim_{\epsilon \to 0^+} \mathbb{E}[D|Z=1, X=x_0+\epsilon] - \lim_{\epsilon \to 0^+} \mathbb{E}[D|Z=0, X=x_0-\epsilon])
\]

For more details, see Chapter 24 of \emph{Imbens \& Lemieux (2008)}.

\subsection{Causal Inference under Unobserved Confounding}

\subsubsection{Sensitivity to Ignorability Assumption}

It's essential to understand how sensitive results are to the assumption of ignorability.

\paragraph{Example: Smoking and Lung Cancer}
\begin{itemize}
    \item Doll and Hill (1950) found that the risk ratio for smoking on lung cancer was 9.
    \item However, Fisher (1957) proposed that a common genetic cause might be responsible for both smoking and lung cancer.
\end{itemize}

The sensitivity parameter describes the strength of this unobserved confounding.

\subsubsection{Sensitivity Parameters}

For binary \(Y\) and binary \(U\):
\[
Z \not\!\perp\!\!\!\perp \{Y(1), Y(0)\} \mid X
\]
\[
Z \perp \{Y(1), Y(0)\} \mid X, U
\]

Define two sensitivity parameters:
\begin{enumerate}
    \item \(\text{RR}_{ZU|X} = \frac{P(U=1 \mid Z=1, X=x)}{P(U=1 \mid Z=0, X=x)}\)
    \item \(\text{RR}_{YU|X} = \frac{P(Y=1 \mid U=1, X=x)}{P(Y=1 \mid U=0, X=x)}\)
\end{enumerate}

\subsubsection{Observed Risk Ratio}

The observed risk ratio is given by:
\[
\text{RR}_{ZY|X}^{obs} = \frac{P(Y=1|Z=1, X=x)}{P(Y=1|Z=0, X=x)}
\]
Under unmeasured confounding, the true risk ratio differs:
\[
\text{RR}_{ZY|X}^{true} = \frac{P(Y(1)=1|X=x)}{P(Y(0)=1|X=x)}
\]

\subsubsection{Theorem for Sensitivity Analysis}

The observed risk ratio under unobserved confounding is bounded by:
\[
\text{RR}_{ZY|X}^{\text{obs}} \leq \frac{\text{RR}_{ZU|X} \cdot \text{RR}_{YU|X}}{\text{RR}_{ZU|X} + \text{RR}_{YU|X} - 1}
\]

Assuming \( Z \perp Y \mid (X, U) \) and, without loss of generality, that:
\[
\text{RR}_{ZY|X}^{\text{obs}} > 1, \quad \text{RR}_{ZU|X} > 1, \quad \text{RR}_{YU|X} > 1.
\]

\subsubsection{Implications of the Theorem}

The theorem implies that:
\[
\text{RR}_{ZU|X} \geq \text{RR}_{ZY|X}^{\text{obs}}
\]
\[
\text{RR}_{YU|X} \geq \text{RR}_{ZY|X}^{\text{obs}}
\]
This is known as the \textbf{Cornfield inequality}. 

To explain away the observed relative risk, both confounding measures \(\text{RR}_{ZU|X}\) and \(\text{RR}_{YU|X}\) must be at least as large as \(\text{RR}_{ZY|X}^{\text{obs}}\).


\subsubsection{Bounding the Average Causal Effect without Sensitivity Parameters}

Assume \( \underline{Y} \leq Y \leq \overline{Y} \), meaning the outcome is bounded. The average causal effect can be bounded as:
\[
\tau = \mathbb{E}[Y(1) - Y(0)]
\]
\[
\tau \leq \overline{Y} - \underline{Y}
\]

If \(\tau\) is partially identified, multiple values of \(\tau\) are compatible with the observed data distribution, as discussed in \emph{Manski (1990, 2003)}. However, bounds typically cover zero.

More informative bounds can be obtained when combined with other assumptions, such as monotonicity:
\[
Y(1) \geq Y(0) \Rightarrow \epsilon_0 \leq 0
\]

Define the sensitivity parameters:
\[
\epsilon_1(X) = \frac{\mathbb{E}[Y(1) \mid Z=1, X]}{\mathbb{E}[Y(1) \mid Z=0, X]}
\]
\[
\epsilon_0(X) = \frac{\mathbb{E}[Y(0) \mid Z=1, X]}{\mathbb{E}[Y(0) \mid Z=0, X]}
\]

\subsubsection{Theorem}



\[
\mathbb{E}[Y(1) \mid Z=0] = \mathbb{E} \left[ \frac{\mathbb{E}[Y \mid Z=1, X]}{\epsilon_1(X)} \mid Z=0 \right]
\]
\[
\mathbb{E}[Y(0) \mid Z=1] = \mathbb{E} \left[ \mathbb{E}[Y \mid Z=0, X] \epsilon_0(X) \mid Z=1 \right]
\]

where \( \epsilon_1(X) \) and \( \epsilon_0(X) \) are sensitivity parameters.





\subsection{Estimators}


\subsubsection{Estimators}

Define the following estimators:

1. \(\hat{\tau}_{\text{ht}}\):
\[
\hat{\tau}_{\text{ht}} = \frac{1}{n} \sum_{i=1}^n \left( \frac{Z_i Y_i \left( \hat{e}(X_i) + (1 - \hat{e}(X_i)) / \epsilon_1(X_i) \right)}{\hat{e}(X_i)} - \frac{\hat{e}(X_i) \epsilon_0(X_i) + (1 - \hat{e}(X_i))(1 - Z_i) Y_i}{1 - \hat{e}(X_i)} \right)
\]


2. \(\hat{\tau}_{\text{haj}}\):
\[
\hat{\tau}_{\text{haj}} = \frac{\sum_{i=1}^n \frac{\hat{e}(X_i) \epsilon_0(X_i) + (1 - \hat{e}(X_i))(1 - Z_i) Y_i}{1 - \hat{e}(X_i)}}{\sum_{i=1}^n \frac{1 - Z_i}{1 - \hat{e}(X_i)}}
\]



3. \(\hat{\tau}_{\text{OR}}\):
\[
\hat{\tau}_{\text{OR}} = \hat{\tau}_{\text{HT}} - \frac{1}{n} \sum_{i=1}^n \left( Z_i - \hat{e}(X_i) \right) \left( \frac{\hat{\mu}_1(X_i)}{\hat{e}(X_i) \epsilon_1(X_i)} + \frac{\hat{\mu}_0(X_i) \epsilon_0(X_i)}{1 - \hat{e}(X_i)} \right)
\]



\subsection{Calibrating Sensitivity Parameters Using Observed Covariates}

Using a leave-one-out approach:
\[
\epsilon_z(X_{-j}) = \frac{\mathbb{E}[Y(z) | Z=1, X_{-j}]}{\mathbb{E}[Y(z) | Z=0, X_{-j}]}
\]

Under \(Z \perp Y(z) \mid X\), we have:
\[
\epsilon_z(X) = \exp(\alpha z + \beta^T z)
\]

\section{Lecture 19: Sensitivity Analysis}{Isabel Moreno,
Jane Chen,
Xuanlin Mao, \\ Kayla Sim(revisions)}

\subsection{Strategies for causal inference under unobserved confounding}

\subsubsection{Cornfield-style inequalities and Manski-style bounds: bound estimand using outcome bounds - recap}

In the previous lecture, the first two strategies were introduced:

\begin{itemize}
    \item 1. \textbf{Cornfield-style} inequalities provide a way to bound the \emph{relative risk} (RR) in the presence of unobserved confounding variables. These inequalities express a relationship between observed treatment effects and the potential impact of confounders.
\[RR^{obs}_{zy|x}\leq  min(RR_{zu|x},RR_{uy|x})\]

\item 2. \textbf{Manski-style bounds} are used to bound the estimand by placing bounds on the outcomes. These can be represented as:
\[
\underline{y} \leq y \leq \overline{y}
\]
For treatment outcomes:
\begin{itemize}
    \item
    \[
    \underline{y} \leq E[y(0) | z=1] \leq \overline{y}
    \]
    \[
    \underline{y} \leq E[y(1) | z=0] \leq \overline{y}
    \]
\end{itemize}
\end{itemize}

\subsection{Rosenbaum-Style Sensitivity Analysis}

Rosenbaum (1987) introduced a sensitivity analysis for one-to-one matched observational studies to test the sharp null hypothesis of no individual treatment effect.

\textbf{Example:} Consider the following variables:
\begin{itemize}
    \item \( Z \) = in-class lecture attendance (treatment),
    \item \( Y \) = grade (outcome),
    \item \( X \) = graduate student, statistics concentration (covariate),
    \item \( U \) = studiousness, passion for the subject, etc. (unobserved confounder).
\end{itemize}

The sharp null hypothesis asserts that attending the lecture in person has no effect on anyone’s grade.

\subsubsection{Notation and Setup}

Let \((i, j)\) index pair \(j\) in unit \(i\), where \( i = 1, 2, \dots, n \) and \( j = 1, 2 \).

\[
H_0^f: Y_{ij}(1) = Y_{ij}(0) \quad \text{for} \quad i = 1, 2, \dots, n, \quad j = 1, 2
\]
Let:
\[
z_{ij} = \text{treatment assigned to unit } (i,j),
\]
\[
e_{ij} = P(Z_{ij} = 1 | X_i, Y_{ij}(1), Y_{ij}(0)) \quad \text{(true propensity score)},
\]
\[
X_i = \text{covariates of the } i \text{th pair (assume perfect matching)}.
\]
\[
O_{ij} = \frac{e_{ij}}{1 - e_{ij}} \quad \text{(odds of treatment for unit } (i, j)\text{)}.
\]

\subsubsection{Rosenbaum Sensitivity Analysis Model}

The sensitivity model is defined as:
\[
\frac{\sigma_{i1}}{\sigma_{i2}} \leq \Gamma, \quad \frac{\sigma_{i2}}{\sigma_{i1}} \leq \Gamma \quad \text{for} \quad i = 1, 2, \dots, n,
\]
which is equivalent to:
\[
\frac{1}{1 + \Gamma} \leq \pi_{i1} \leq \frac{\Gamma}{1 + \Gamma} \quad \text{for} \quad i = 1, 2, \dots, n,
\]

Where: 
\[\pi_{i1}=P(Z_{i1}=1|X_i, Z_{i1}+Z_{i2}=1, Y_{i1}(0), Y_{i2}(0), Y_{i1}(1), Y_{i2}(1))\]

\[\pi_{i1}=\frac{e_{i1}(1-e_{i2})}{e_{i1}(1-e_{i2})+e_{i2}(1-e_{i1})}\]

where \(\pi_{i1}\) is the probability of receiving treatment \(Z_{i1} = 1\), conditioned on covariates and potential outcomes.

If there is no unobserved confounding, we have:
\[
e_{ij} = P(Z_{ij} = 1 | X_i),
\]
which leads to:
\[
e_{i1} = e_{i2}, \quad \pi_{i1} = \frac{1}{2}, \quad \Gamma = 1 = \frac{\sigma_{i1}}{\sigma_{i2}}.
\]

\subsubsection{Testing the Sharp Null Hypothesis}

To test the sharp null hypothesis, we examine the within-pair sign. Under the null hypothesis (\(H_0^f\)), the distribution of the sign\(S_i\) follows a Bernoulli(1/2):

Let 
\[\hat{\tau_i} = (2Z_{ij} - 1)(Y_{i1}- Y_{i2})\]

\[
S_i \sim \text{Bernoulli}(\pi_{i1}),
\]
where the worst-case scenario under the Rosenbaum model is:
\[
S_i \sim \text{Bernoulli}\left(\frac{\Gamma}{1 + \Gamma}\right).
\]

\paragraph{Test Statistics}

Two common test statistics are used in matching:
\begin{itemize}
    \item 1. \textbf{Sign Statistic:} 
   \[
   T = \sum_{i=1}^n S_i.
   \]
   \item  2. \textbf{Wilcoxon Signed-Rank Statistic:} 
   \[
   T = \sum_{i=1}^n S_i R_i,
   \]
   where \(R_i\) is the rank of the absolute value of \(T_i\).
\end{itemize}

\textbf{Generalization:}
   \[T = \sum S_i q_i\]
   
The distribution of \(T\) has the following properties:
\[
E_\Gamma(T) = \frac{\Gamma}{1 + \Gamma} \sum_{i=1}^n q_i,
\]
\[
\text{Var}_\Gamma(T) = \frac{\Gamma}{(1 + \Gamma)^2} \sum_{i=1}^n q_i^2.
\]
The normal approximation of \(T\) is given by:
\[
\frac{T - \frac{\Gamma}{1 + \Gamma} \sum_{i=1}^n q_i}{\sqrt{\frac{\Gamma}{(1 + \Gamma)^2} \sum_{i=1}^n q_i^2}} \to N(0, 1).
\]

\subsubsection{Generalizing the Rosenbaum Model Beyond 1-1 Matching}

In the general model, the assumption is:
\[
Z \not\perp\!\!\!\perp (Y(1), Y(0)) | X, \quad Z \perp\!\!\!\perp (Y(1), Y(0)) | X, U.
\]
The odds ratio model becomes:
\[
\frac{1}{\Gamma} \leq \frac{\text{odds}(Z=1 | X, U = u)}{\text{odds}(Z=1 | X, U = u')} \leq \Gamma \quad \text{for any} \quad u, u'.
\]

Where \(\Gamma\) is a measure of how far we are from the perfect unconfounded experiment

The marginal sensitivity model is:
\[
\frac{1}{\Gamma} \leq \frac{\text{odds}(Z=1 | X, U)}{\text{odds}(Z=1 | X)} \leq \Gamma,
\]
which is equivalent to:
\[
\frac{1}{\Gamma} \leq \frac{\frac{P(Z=1 | X, U=u)}{P(Z=0 | X, U=u)}}{\frac{P(Z=1 | X)}{P(Z=0 | X)}} \leq \Gamma.
\]

Once we specify \(\Gamma\), we can partially identify our estimand under the marginal sensitivity model (MSM) and proceed with the estimation of the partially identified target.

\section{Lecture 20: Principal Stratification and Mediation}{Yixin Feng, Keira Chiu, Allie Huang, Fangyuan Li(revisions)}
\subsection*{Announcements:}
\begin{itemize}
    \item No lecture Thursday - BSTARS 1-5
    \item 256: submit video on Gradescope
    \item Project signups
    \item Midterm grades released
\end{itemize}

\subsection*{Challenge example:}
\begin{itemize}
    \item Officers are as likely to shoot a white person they stop as a black person (Fryer 2019)
    \item Officers are more likely to use unjustified force on black people
    \item Example of post-treatment bias:
    \begin{itemize}
        \item Treatment: race
        \item Post-treatment variable: stop
        \item Outcome: shoot
    \end{itemize}
\end{itemize}

\subsection*{Post-treatment Variables}
\begin{itemize}
    \item \textbf{Police bias:}
    \begin{itemize}
        \item $Y$: use of force
        \item $Z$: race of driver/pedestrian
        \item $M$: stopped driver/pedestrian
    \end{itemize}
    \item \textbf{Job training:}
    \begin{itemize}
        \item $Z$: job training
        \item $Y$: wage
        \item $M$: employment status
    \end{itemize}
    \item \textbf{Clinical trial:}
    \begin{itemize}
        \item $Z$: HIV treatment
        \item $Y$: 30-year survival
        \item $M$: surrogate endpoints
        \item e.g., CD4 cell counts
    \end{itemize}
\end{itemize}

\subsection*{Completely Randomized Experiment}
\[
Z = \{ Y(0), Y(1), M(0), M(1) \}
\]

\textbf{Goal: Compare $P(Y(1) | M = m)$ vs $P(Y(0) | M = m)$}

Can we use $P(Y|Z=1, M=m)$ vs $P(Y|Z=0, M=m)$?

By CRE:
\[
P(Y|Z=1, M=m) = P(Y(1)|Z=1, M(1) = m) = P(Y(1)| M(1) = m)
\]
\[
P(Y|Z=0, M=m) = P(Y(0)|Z=0, M(0) = m) = P(Y(0)| M(0) = m)
\]
In the challenge example, comparing \( P(Y \mid Z=1, M=m) \) and \( P(Y \mid Z=0, M=m) \) involves two different populations. Specifically, \( Y(1) \) represents the outcome for all black drivers, while \( M(1) = 1 \) corresponds to black drivers who are stopped. Similarly, \( Y(0) \) represents the outcome for all white drivers, and \( M(0) = 0 \) corresponds to white drivers who are stopped. Since these comparisons condition on two completely different populations, it is not possible to determine whether the causal effect is due to the causal relationship between \( Y \) and \( Z \), or if it arises because we are not comparing equivalent groups. Comparing causal effects using different populations can introduce bias, as the comparisons may not accurately reflect the true causal relationships.


\textbf{Another example:} truncation by death \newline
    \quad \( Z \): severe disease treatment \newline
    \( \quad Y \): quality of life \newline
    \( \quad M \): survival status \newline
    
Suppose treatment affects survival status, so treatment can save more weak patients than control. 
\[P(Y(1)|M(1)=1) \text{ vs } P(Y(0)|M(0)=1)\]
where the first includes weaker patients, and the second does not.

\subsection{Principal Stratification: conditioning on potential}

\subsubsection{Values of post-treatment variables (Frangakis and Rubin 2002):}

Goal: compare 
\(P(Y(1)|M(1)=m_1, M(0) = m_0) \text{ vs } P(Y(0)|M(1)=m_1, M(0) = m_0)\) for some \(m_1, m_0\).

\begin{itemize}
    \item \(M(1)\), \(M(0)\) are pre-treatment covariates:
    \item Principal strata defined by potential values of post-treatment variable

\end{itemize}

Ex: Binary M

Principal strata, severe disease example:

\(M(1)=1\), \(M(0)=1\) survive always \newline
\(M(1)=1\), \(M(0)=0\) need treatment to survive \newline
\(M(1)=0\), \(M(0)=1\) harmed by treatment \newline
\(M(1)=0\), \(M(0)=0\) doomed \newline

\textbf{Define:}
\[\tau(m_1, m_0) = E[Y(1) - Y(0) \text{ }| \text{ } M(1) = m_1, M(0) = m_0]\] as the Principal Stratification Average Causal Effect for subgroup with \(M(1) = m_1\) and \(M(0) = m_0\).

For binary M, we have four effects:
\[\tau(1, 1) = E[Y(1) - Y(0) \text{ }| \text{ } M(1) = 1, M(0) = 1]\]
\[\tau(1, 0) = E[Y(1) - Y(0) \text{ }| \text{ } M(1) = 1, M(0) = 0]\]
\[\tau(0, 1) = E[Y(1) - Y(0) \text{ }| \text{ } M(1) = 0, M(0) = 1]\]
\[\tau(0, 0) = E[Y(1) - Y(0) \text{ }| \text{ } M(1) = 0, M(0) = 0]\]
where \(\tau(1, 1)\) and \(\tau(0,0)\) are measures of dissociative effects, and \(\tau(1, 0)\) and \(\tau(0,1)\) are measures of associative effects.

\textbf{Example:} experiment with noncompliance \newline
    \( Z \): treatment assigned \newline
    \( Y \): treatment received (we used D for this previously) \newline
    \( M \): outcome

\begin{center}
\begin{tikzpicture}[
  node distance=1.5cm and 2cm, 
  >={Stealth[round]}, 
  thick, 
  observed/.style={fill=white!20, draw=none, font=\large},
  unobserved/.style={draw, circle, minimum size=0.8cm, font=\small},
  square/.style={draw, minimum size=0.8cm, font=\small}
]

% Nodes
\node[observed] (M) {M};
\node[observed, right=of M] (Y) {Y};
\node[unobserved, above=of M] (U) {U};
\node[observed, below left=of M] (Z) {Z};

% Edges
\draw[->] (Z) -- (M);
\draw[->] (Z) -- (Y);
\draw[->] (M) -- (Y);
\draw[->] (U) -- (M);
\draw[->] (U) -- (Y);
\end{tikzpicture}
\end{center}

\(\tau(1,1)\) always takers effect \newline
\(\tau(1,1)\) compliers effect \newline
\(\tau(1,1)\) defiers effect \newline
\(\tau(1,1)\) never takers effect \newline

\textbf{Example:} severe disease
    \begin{itemize}
        \item \(Y(M=0)\) is not well defined (\(Y\) is quality of life).
        \item Only \(\tau(1,1)\) is well defined.
        \begin{itemize}
            \item Survivor average causal effect (Rubin 2006) 
        \end{itemize}

    \end{itemize}

\textbf{Example: Unemployment}
\begin{itemize}
    \item \( Y \): wage
    \item \( Z \): job training
    \item \( M \): employment status
\end{itemize}

\textbf{Note:} Only \( \tau(1, 1) \) is well-defined.

\textbf{Example: Clinical Trial}
\begin{itemize}
    \item A good surrogate meets two criteria:
    \begin{enumerate}
        \item If treatment doesn’t affect the surrogate, then it doesn’t affect the outcome (\textbf{causal necessity}).
        \item If treatment affects the surrogate, then it affects the outcome too (\textbf{causal sufficiency}).
    \end{enumerate}
    \item \textbf{Causal necessity} requires:
    \[
    \tau(1,1) \text{ and } \tau(0,0) = 0.
    \]
    \item \textbf{Causal sufficiency} requires:
    \[
    \tau(1,0) \text{ and } \tau(0,1) \neq 0.
    \]
\end{itemize}

\subsection{Identification of $\tau(m_1, m_0)$}
To identify \( \tau(m_1, m_0) = E[Y(1) - Y(0) \mid M(1) = m_1, M(0) = m_0] \), we rely on the following assumptions:

\subsubsection{Assumptions}
\begin{enumerate}
    \item \textbf{Completely Randomized Experiment (CRE)}: \[ Z \perp \{ Y(1), Y(0), M(1), M(0) \} \]
    \item \textbf{Monotonicity}: \( M(1) \geq M(0) \). \newline
    Example: no defiers in compliance settings.
\end{enumerate}

\subsubsection{Latent Strata Proportions}
Under these assumptions, we can identify the proportions of three latent strata:
\begin{align*}
    \pi(1, 1) &= P(M(1) = 1, M(0) = 1) \\
    \pi(0, 0) &= P(M(1) = 0, M(0) = 0) \\
    \pi(1, 0) &= P(M(1) = 1, M(0) = 0).
\end{align*}
\begin{itemize}
    \item By monotonicity, \( P(M(1) < M(0)) = 0 \).
\end{itemize}

\textbf{Example:} Severe disease and truncation by death. \newline
Only \( \tau(1, 1) \) is well-defined because \( Y(M = 0) \) (e.g., quality of life for non-survivors) is not observable.

\subsubsection{Key Derivations}
To identify \( E[Y(1) \mid M(1) = 1, M(0) = 1] \):
\[
E[Y(1) \mid M(1) = 1, M(0) = 1] = E[Y \mid Z = 1, M = 1] 
\]
However, due to CRE, this cannot directly identify \( E[Y(1) \mid M(1) = 1, M(0) = 1] \), as:
\[
E[Y \mid Z = 1, M = 1] \neq E[Y(1) \mid M(1) = 1, M(0) = 1].
\]

Instead:
\begin{align*}
E[Y(1) \mid M(1) = 1, M(0) = 1] &= E[Y(1) \mid M(1) = 1, M(0) = 1] \cdot P(M(0) = 1 \mid M(1) = 1) \\
&\quad + E[Y(1) \mid M(1) = 1, M(0) = 0] \cdot P(M(0) = 0 \mid M(1) = 1).
\end{align*}

\subsubsection{Partial Identification}
As there is one equation with two unknowns, we cannot point-identify \( E[Y(1) \mid M(1) = 1, M(0) = 0] \). \newline
\textbf{Solution:} Partial identification techniques can provide bounds for the unobservable quantities.

\section{Lecture 21: Guest Lecture about Panel Causal Model}{Lingxi Zhong, Dominic Fannjiang, Fangyuan Li(revisions)}

\subsection*{Note:}

\begin{itemize}
    \item This lecture is a guest lecture offer by professor Avi Feller
\end{itemize}

\subsection*{Panel Data Overview}

\begin{itemize}
    \item \textbf{Cross-sectional data:} Observe a single snapshot in time
        \begin{itemize}
            \item Standard observational data - \textbf{assume away unmeasured confounding.}
            \item Most useful when we have lots of Xs and large N
        \end{itemize}

    \item \textbf{Panel Data:} Observe same units over time. Usually in settings with a small number of units.
        \begin{itemize}
            \item Almost all analysis is model-based, not design-based. Therefore, in order to believe the results, one needs to be convinced about the underlying model.
            \item Can we do better than assuming away unmeasured confounding?
        \end{itemize}

    \item Assume \textbf{unmeasured confounding is "stable" over time}
\end{itemize}

\subsection*{Longitudinal Designs}

\begin{itemize}
    \item A) Pre/post
    \item B) Interrupted time-series
    \item C) Difference-in-difference
    \item D) Comparative interrupted time-series

    \item \textbf{WARNING}
        \begin{itemize}
            \item Some people call every panel data setting \textbf{diff-in-diff}
            \item Elsewhere, common to reserve \textbf{diff-in-diff} for \textbf{2 period case only}. More general setting is Comparative Interrupted Time-Series
        \end{itemize}
\end{itemize}

\subsection*{Panel Data By Any Other Name}

\begin{itemize}
    \item \textbf{Treated unit only:} Pre-post, Interrupted Time-Series(ITS)
    \item \textbf{+Comparison unit:} Difference-in-difference(DiD), Comparative Interrupted Time-Series(CITS), Two-Way Fixed Effect(TWFE)
    \item \textbf{Variation:} Event Study, Difference-in-Difference-in-Difference(DDD), Synthetic Control Method.
\end{itemize}

\subsection*{Part 1: BEFORE-AFTER(Pre-Post)}

\begin{itemize}
    \item Pre-Post: Basic Setup
        \begin{itemize}
            \item We observe many schools: Albany HS, Berkeley HS, Cerrito HS, ...
            \item \textbf{All} schools adopt a new tutoring program
            \item Observe test scores before/after
        \end{itemize}
        
    \item Pre/Post
        \begin{itemize}
            \item Outcome levels at two time points
            \item \textbf{Treated unit/group only}
            \item Attribute all changes to intervention
            
            \[\widehat{\text{Pre-Post}}= \bar{Y}_{T,\text{Post}} - \bar{Y}_{T,\text{Pre}}\]
        \end{itemize}
\end{itemize}

\subsection*{Part 2: Canonical 2x2 DiD, the basics}

\begin{itemize}
    \item 2x2 DiD: The Setup
        \begin{itemize}
            \item We observe many schools: Albany HS, Berkeley HS, Cerrito HS, ...
            \item \textbf{Some} schools adopt a new tutoring program
                \begin{itemize}
                    \item Treatment = Z
                \end{itemize}
            \item School-average assessments before and after
                \begin{itemize}
                    \item Scores = Pre and Post
                \end{itemize}
            \item 2x2 Differences-in-Differences(DiD) setting
                \begin{itemize}
                    \item Two groups and two time periods
                \end{itemize}
        \end{itemize}

    \item Differences-in-Differences
        \begin{itemize}
            \item Other units comparison
            \item Relative change in levels
            \item \textbf{Assumed counterfactual:} outcome would have changed as much as the non-treatment group if not for the intervention. This usually involves an assumption of  "parallel trends", which is that if the treated unit were not treated, its different in potential outcomes would be the same as that of the control unit.
            \[
\widehat{\text { DiD }}=\underbrace{\left(\bar{Y}_{T, \text { Post }}-\bar{Y}_{T, \text { Pre }}\right)}_{\text {Gain }_T}-\underbrace{\left(\bar{Y}_{C, \text { Post }}-\bar{Y}_{C, \text { Pre }}\right)}_{\text {Gain }_c}
\]

\[
\widehat{\text{DiD}} = (\bar{Y}_{T,\text{Post}} - \bar{Y}_{C,\text{Post}}) - (\bar{Y}_{T,\text{Pre}} - \bar{Y}_{C,\text{Pre}})
\]
        \end{itemize}
\end{itemize}

\subsection*{Part 3: Interrupted time-series}
One timeseries, but you compare multiple data points of before and after the time of treatment. 

Emphasis on the fact that multiple points in time can be used to define the estimand of interest. Note that this is different from regression discontinuity precisely because the estimand of interest is not merely what happens around the threshold (which, in this case, is the time point where the treatment occured).

\subsection*{Part 4: Comparative interrupted time-series}
The same as part 3, but with multiple time series, where some are considered control and other are considered treated. 
ex) You have a time series of gun-related deaths in each state where some states implemented some law related to gun control, and other have not.
Often times, the goal is to use the "control" timeseries to be an example of what would have happened to the treated state had it not been treated.


\section{Lecture 23: Post-Treatment Variables}{Max Medina, Shana Kim, Jake Derr,\\ Fangyuan Li(revisions)}

\subsection*{Recap}
Recall that \textbf{Principal Stratification} is conditioning on potential outcomes of post treatment variables. For example: with $Z$ a treatment for a disease, $Y$ a health-related outcome and $M$ a post-treatment variable, like the quality of life, we define the ATE in each strata as $\tau(m_1,m_0)=E(Y(1)-Y(0)|M(1)=m_1,M(0)=m_0)$. Is $\tau(1,1)$ well-defined? Partially:
\begin{itemize}
    \item Under CRE + monotonicity: $E(Y(0)|M(1)=1,M(0)=1) = E(Y|Z=0,M=1)$, all observables
    \item However $\mathbb{E}[Y(1) \mid M(1) = 1, M(0) = 1]$
    cannot be identified. Specifically:
    \[\begin{split}
    \mathbb{E}[Y \mid Z = 1, M = 1] &= \Psi \cdot \mathbb{P}(M(1) = 1 \mid M(0) = 1) \\
    &+ E(Y(1) \mid M(1) = 1, M(0) =0 ] P(M(1)=1 \mid M(0)=0),
    \end{split}
    \]
    where the last term is unobservable. 
\end{itemize}

\subsection*{Mediation Analysis}

\subsubsection*{Why Mediation Analysis?}
\begin{itemize}
    \item Reveals mechanisms
    \item Easier to intervene on $M$ than $Z$ (e.g., $Z$: smoking, $M$: blood lipid levels, $Y$: cardiovascular disease).
    \item $Y$ may be expensive or time-consuming to observe; $M$ serves as a surrogate (proxy) outcome.
\end{itemize}

\textbf{Notation:}
\begin{itemize}
    \item $M(Z)$: Potential outcome of $M$ given $Z$.
    \item $Y(Z)$: Potential outcome of $Y$ given $Z$.
    \item $Y(Z, M)$: Potential outcome of $Y$ given $Z$ and $M$.
    \item $Y(Z, M(Z'))$: Counterfactual outcome of $Y$ under $Z$ and $M$ had $Z$ been $Z'$.
\end{itemize}

Example (binary $Z, M$):  
$Y(1, M_0)$ and $Y(0, M_1)$ are unobservable \textit{cross-world counterfactuals}.

\subsubsection*{Total Effect Decomposition}
Why care about cross-world counterfactuals?
\[
\text{Total Effect (TE)} = \mathbb{E}[Y(1) - Y(0)].
\]
\[
\text{Natural Direct Effect (NDE)} = \mathbb{E}[Y(1, M_0) - Y(0, M_0)].
\]
\[
\text{Natural Indirect Effect (NIE)} = \mathbb{E}[Y(0, M_1) - Y(0, M_0)].
\]

\subsection*{Identification (Pearl, 2001)}

\textbf{Mediation Formula:} let $X$ denote any observed covariate.
\begin{enumerate}
    \item $Z \perp Y(z, m) \mid X$ 
    \item $M \perp Y(z, m) \mid X, Z$
    \item $Z \perp M(z) \mid X$
    \item $Y(Z, M) \perp M(Z') \mid X$ for all $z, z'$ (Cross-world assumption, untestable)
\end{enumerate}
Note that 1 and 2 together are called \textit{sequential ignorability} and is equivalent to joint randomization $(Z,M)\perp Y(z,m) | X$.

\textbf{Under 1--4:}
\[
\text{NDE}(x) = \mathbb{E}[Y(1, m_0) - Y(0, m_0) \mid X = x]
\]
\[
= \sum_m \Big\{(\mathbb{E}[Y \mid Z = 1, M = m, X = x] 
- \mathbb{E}[Y \mid Z = 0, M = m, X = x] 
\Big\} \mathbb{P}(M = m \mid Z = 0, X = x),
\]

\[
\text{NIE}(x) = \sum_m \mathbb{E}[Y \mid Z = 1, M = m, X = x] 
 \Big\{\mathbb{P}(M = m \mid Z = 1, X = x) - \mathbb{P}(M = m \mid Z = 0, X = x)\Big\}.
\]

Finally:
\[
\text{NDE} = \int \text{NDE}(x) p(x) dx, \quad \text{NIE} = \int \text{NIE}(x) p(x) dx.
\]

\subsection*{Baron-Kenny Method}

Assume linear relationships and a binary $M$. We adopt a parametric model and use the regressions $M\sim Z,M$ and $Y\sim Z,M,X$ to estimate the direct and indirect effects. Specifically:

\begin{equation}
\mathbb{E}[M \mid Z, X] = \beta_0 + \beta_1 Z + \beta_2^T X,    
\end{equation}
and:
\begin{equation}
\mathbb{E}[Y \mid M, Z, X] = \theta_0 + \theta_1 Z + \theta_2 M + \theta_3^T X.    
\end{equation}

The coefficients in this regressions can be identified in the edges of the following causal diagram.

\begin{figure}[h]
    \centering
    \begin{tikzpicture}
    \node[state] (z) {$Z$};
    \node[state] (m) [right =of z] {$M$};
    \node[state] (y) [right =of m] {$Y$};
    \node[state] (x) [above =of m] {$X$};

    \path (z) edge node[above] {$\beta_1$} (m);
    \path (m) edge node[above] {$\theta_2$} (y);
    \path (x) edge node[el,above] {} (z);
    \path (x) edge node[el,above] {$\beta_2$} (m);
    \path (x) edge node[el,above] {$\theta_3$} (y);
    \path (z) edge[bend right=30] node[below] {$\theta_1$} (y);
\end{tikzpicture}
    \caption{Causal relationships between $Z,X,M,Y$}
    \label{fig:enter-label}
\end{figure}






\textbf{Effects:}
\begin{itemize}
    \item $\text{NDE} = \theta_1$
    \item $\text{NIE} = \theta_2 \beta_1$
\end{itemize}

These can be estimated using normal OLS regression, with mean and variance.  
\textbf{Pros:} Running regressions is straightforward.  
\textbf{Cons:} Relies on strong parametric assumptions, which may be unstable.

\subsection*{Controlled Effects}

\textbf{Controlled Direct Effect (CDE):}
\[
\text{CDE}(m) = \mathbb{E}[Y(1, m) - Y(0, m)].
\]

\textbf{Controlled Indirect Effect (CIE):}
\[
\text{CIE}(m, m') = \mathbb{E}[Y(z, m) - Y(z, m')].
\]
\subsection*{Identification of CDE}

\subsubsection*{Assumptions}
\begin{enumerate}
    \item $Z \perp\!\!\!\perp Y(Z, M) \mid X$
    \item $M \perp\!\!\!\perp Y(Z, M) \mid Z, X$
\end{enumerate}

\subsubsection*{For Binary $Z$ and $M$}
\begin{itemize}
    \item Define $\mu_{zm} = \mathbb{E}[Y(Z, M)]$, so $\text{CDE}(m) = \mu_{1m} - \mu_{0m}$
    \item $\mu_{zm}(X) = \mathbb{E}[Y \mid Z = z, M = m, X = x]$
    \item $e_{zm}(X) = P(Z = z, M = m \mid X = x) = P(Z = z \mid X = x) \cdot P(M = m \mid Z = z, X = x)$
\end{itemize}

\subsubsection*{Under $(Z, M) \perp\!\!\!\perp Y(Z, M) \mid X$}
\begin{align*}
    \mu_{zm} &= \mathbb{E}[\mu_{zm}(X)] \\
    \mu_{zm} &= \mathbb{E} \left[ \frac{\mathbb{I}\{Z = z, M = m\} Y}{e_{zm}(X)} \right]
\end{align*}

\subsubsection*{Controlled Direct Effect (CDE)}
\[
\text{CDE}(m) = \mu_{1m} - \mu_{0m}
\]

\subsubsection*{Doubly Robust Formula}
Given the following working models
\begin{itemize}
    \item $e_{zm}(X, \alpha)$ for $e_{zm}(X)$,
    \item $\mu_{zm}(X, \beta)$ for $\mu_{zm}(X)$,
\end{itemize}
we can employ a DR approach as follows:
\begin{align*}
    \mu_{zm} &= \mathbb{E} \left[\mu_{zm}(X, \beta)\right] + \mathbb{E} \left[\frac{\mathbb{I}\{Z = z, M = m\} (Y - \mu_{zm}(X, \beta))}{e_{zm}(X, \alpha)} \right] \\
    &= \mu_{zm} \text{ if either } e_{zm}(X, \alpha) = e_{zm}(X) \text{ or } \mu_{zm}(X, \beta) = \mu_{zm}(X).
\end{align*}
\subsection*{Recap on Post-Treatment Variables}
\begin{tabular}{|c|c|c|}
    \hline
    Framework & Direct Effect & Indirect Effect \\ \hline
    Principal Stratification & $Y(1,1), Y(0,0)$ & - \\ \hline
    Mediation Analysis & $\text{NDE}$ & $\text{NIE}$ \\ \hline
    Controlled Effects & $\text{CDE}(m)$ & $\text{CIE}(m, m') \mid Z$ \\ \hline
\end{tabular}

\section{Lecture 24: Time-Varying Treatment and Confounders}{Will Rathgeb, Qinhan Zhou, Ziyuan Jin}

\subsection*{Recap}
\textbf{Mediation Analysis} via Natural and Controlled Effects
\begin{itemize}
    \item M: Mediator
    \begin{tikzpicture}[->, node distance=2cm, thick]
        % Nodes
        \node (Z) at (0, 0) {\(Z\)};
        \node (M) at (2, 0) {\(D\)};
        \node (Y) at (4, 0) {\(Y\)};
    
        % Arrows
        \draw[->] (Z) -- (M);
        \draw[->] (M) -- (Y);   
        \draw[->] (Z) to[bend left] (Y);
    \end{tikzpicture}
    \item Direct Effect: $\mathbb{E}[(Y(1, M=m) - Y(0,M=m)]$
    \begin{description}
        \item[Natural effects:] $m = M(1) \text{ or } m = M(0)$
        \item[Controlled effects:] $m \in \mathcal{M}$ (set of all possible values of M)
    \end{description}
    \item Indirect Effect: $\mathbb{E}[(Y(Z, m_1) - Y(Z,m_0)]$
    \begin{description}
        \item[Natural effects:] $m_1 = M(1),~ m_0 = M(0)$
        \item[Controlled effects:] $m_0, m_1 \in \mathcal{M}$
    \end{description}
\end{itemize}

\subsection*{Time-Varying Treatment + Confounding}
ex: HIV patients take antiretroviral medicines on + off over time
ex: candidates adjust their campaign strategy based on polls + opponent's behavior

Suppose we have two time points.\\

\textbf{Temporal order:}
\[
X_0 \to Z_1 \overset{a}{\to} X_1 \to Z_2 \to Y
\]
\begin{itemize}
    \item $X_0$: baseline covariates
    \item $Z_1$: treatment at time point 1 (binary): with 1-4 hours
    \item $X_1$: time-varying covariates observed between treatments
    \item $Z_2$: treatment at time point 2 (binary): with 4-8 hours
    \item $Y$: Outcome
\end{itemize}
4 potential outcomes: Y(0,0), Y(0,1), Y(1,0), Y(1,1)
Observed outcome: 

\textbf{Estimates:}

E[Y($Z_1$, $Z_2$)]
\begin{itemize}
    \item E[Y(1,0) - Y(0,0)]
    \item E[Y(0,1) - Y(0,0)]
    \item E[Y(1,1) - Y(0,0)]
\end{itemize}

Our choice of estimand depends on policy/ science question \\ \\
Suppose our estimate is E[Y($Z_1$, $Z_2$)]

\textbf{Identifications:}
Assumption: Sequential ignorability: treatments are sequentially randomized given observed history 
1. $Z_1$ is randomized given $C(X_0)$.
\[
Z_1 \perp\!\!\!\perp Y(Z_1, Z_0) \mid Z_0, 
\]
for $z_1, z_2 \in \{0, 1\}$.
2. $Z_2$ is randomized given $C(Z_1, X_1, X_0)$.
\[
Z_2 \perp\!\!\!\perp Y(Z_1, Z_0) \mid Z_1, X_1, X_0
\]
for $z_1, z_2 \in \{0, 1\}$

Satisfied under a DAG
\begin{center}
\begin{tikzpicture}[->, node distance=1.5cm]
    % Nodes
    \node (Z1) {$Z_1$};
    \node (X1) [right of=Z1] {$X_1$};
    \node (Z2) [right of=X1] {$Z_2$};
    \node (Y) [right of=Z2] {$Y$};

    % Arrows
    \draw[->] (Z1) -- (X1);
    \draw[->] (X1) -- (Z2);
    \draw[->] (Z2) -- (Y);
    \draw[->, bend right=30] (Z1) to (Y);
    \draw[->, bend right=30] (X1) to (Y);
    \draw[->, bend left=30] (Z1) to (Z2);
\end{tikzpicture}
\end{center}
Also satisfied under OAG
\begin{center}
\begin{tikzpicture}[->, node distance=1.5cm]
    % Nodes
    \node (Z1) {$Z_1$};
    \node (X1) [right of=Z1] {$X_1$};
    \node (Z2) [right of=X1] {$Z_2$};
    \node (Y) [right of=Z2] {$Y$};
    \node (U) [below of=X1, yshift=-1cm] {$U$};

    % Arrows
    \draw[->] (Z1) -- (X1);
    \draw[->] (X1) -- (Z2);
    \draw[->] (Z2) -- (Y);
    \draw[->, bend left=30] (Z1) to (Y);
    \draw[->, bend left=30] (Z1) to (Z2);
    \draw[->] (U) -- (X1);
    \draw[->] (U) -- (Z2);
    \draw[->] (U) -- (Y);
\end{tikzpicture}
\end{center}

Recall identification under a single time-point setting:
$\mathbb{E}[Y(Z)] = \mathbb{E}[\mathbb{E}[Y|Z=z,X]]$
\begin{itemize}
    \item discrete X: $\sum_X\mathbb{E}[Y|Z=z,X=x]\mathbb{P}(X=x)$
    \item continuous X: $\int_X\mathbb{E}[Y|Z=z,X=x]p(x)dx$
\end{itemize}
\begin{itemize}
    \item For discrete $X$:
    \[
    \sum_{x} \mathbb{E}[Y \mid Z = z, X = x] P(X = x)
    \]
    \item For continuous $X$:
    \[
    \int_{x} \mathbb{E}[Y \mid Z = z, X = x] p(X) \, dx
    \]
\end{itemize}
\textbf{Theorem}

 Under sequential ignorability:
 $\mathbf{E}[Y(Z_1, Z_2)] = \mathbf{E}\left[\mathbf{E}\left[\mathbf{E}[Y \mid Z_2 = Z_2, Z_1 = Z_1, X_1, X_0] \mid Z_1 = Z_1, X_0\right]\right]$
 
"g-formula" - Jamie Robins:
\begin{itemize}
\item discrete $X_0, X_1$

$=\sum_{X_0}\sum_{X_1}\mathbb{E}[Y|z_2,z_1,X_1,X_0]\mathbb{P}(X_1|z_1,X_0)\mathbb{P}(X_0)$ 
\item continuous $X_0, X_1$

$=\int\int\mathbb{E}[Y|Z_2,Z_1,X_1,X_0]p(X_1|Z_1,X_0)p(X_0)dx_1dx_0$ 
\end{itemize}



\textbf{Proof}
\begin{itemize}
    \item 
\begin{align*}
\mathbb{E}[Y(Z_1,Z_2)] &= \mathbb{E}[\mathbb{E}[Y(Z_1,Z_2) \mid X_0]] \\
&= \mathbb{E}[\mathbb{E}[Y(Z_1,Z_2) \mid Z_1, X_0]] \quad \text{by sequential ignorability} \\
&= \mathbb{E}[\mathbb{E}[\mathbb{E}[Y(Z_1,Z_2) \mid Z_1, X_1, X_0] \mid Z_1, X_0]] \quad \text{by Tower Property} \\
&= \mathbb{E}[\mathbb{E}[\mathbb{E}[Y(Z_1,Z_2) \mid Z_2, Z_1, X_1, X_0] \mid Z_1, X_0]] \quad \text{by sequential ignorability} \\
&= \mathbb{E}[\mathbb{E}[\mathbb{E}[Y \mid Z_2, Z_1, X_1, X_0] \mid Z_1, X_0]]
\end{align*}
\end{itemize}

\subsection*{Estimators}

\subsubsection*{1. Outcome Modeling}

\begin{itemize}
    \item Recall outcome modeling in a single time-point setting:
    \[
    \mathbb{E}[Y(Z)] = \mathbb{E}[\mathbb{E}[Y \mid Z = z, X]]
    \]

    \item Steps:
    \begin{enumerate}
        \item Regress $Y \sim X \mid Z = z$ to obtain $\hat{Y}_i(z)$ for all units.
        \item Compute:
        \[
        \widehat{\mathbb{E}}[Y(Z)] = \frac{1}{n} \sum_{i=1}^n \hat{Y}_i(z)
        \]
    \end{enumerate}
    
    \item Now, in a time-varying setting:
    \begin{enumerate}
        \item Regress $Y \sim X_1, X_0 \mid Z_1 = z_1, Z_2 = z_2$ to get $\hat{Y}_{i2}(z_1, z_2)$.
        \item Regress $\hat{Y}_{i2}(z_1, z_2) \sim X_0 \mid Z_1 = z_1$ to get $\hat{Y}_i(z_1, z_2)$.
        \item Compute:
        \[
        \widehat{\mathbb{E}}[Y(Z_1, Z_2)] = \frac{1}{n} \sum_{i=1}^n \hat{Y}_i(z_1, z_2)
        \]
    \end{enumerate}
\end{itemize}

\subsection*{2. IPW Estimator}

\begin{itemize}
    \item Recall for a single time-point, the identification:
    \[
    \mathbb{E}[Y(Z)] = \mathbb{E}\left[\frac{\mathbb{I}\{Z = z\} Y}{P(Z = z \mid X)}\right]
    \]

    \item For a two-time-point setting:
    \begin{itemize}
        \item Define:
        \[
        e(z_1, X_0) = P(Z_1 = z_1 \mid X_0)
        \]
        \[
        e(z_2, z_1, X_1, X_0) = P(Z_2 = z_2 \mid Z_1 = z_1, X_1, X_0)
        \]
    \end{itemize}

    \item Theorem (under sequential ignorability):
    \[
    \mathbb{E}[Y(Z_1, Z_2)] = \mathbb{E}\left[\frac{\mathbb{I}\{Z_1 = z_1\} \mathbb{I}\{Z_2 = z_2\} Y}{e(z_1, X_0) e(z_2, z_1, X_1, X_0)}\right]
    \]

    \item Highlights the overlap assumption:
    \[
    0 < e(z_1, X_0) < 1, \quad 0 < e(z_2, z_1, X_1, X_0) < 1, \quad \forall z_1, z_2
    \]

    \item IPW Estimator:
    \begin{enumerate}
        \item Regress $Z_1 \sim X_0$ to estimate $\hat{e}(z_1, X_0)$.
        \item Regress $Z_2 \sim Z_1, X_1, X_0$ to estimate $\hat{e}(z_2, z_1, X_1, X_0)$.
        \item Compute:
        \[
        \widehat{\mathbb{E}}[Y(Z_1, Z_2)] = \frac{1}{n} \sum_{i=1}^n \frac{\mathbb{I}\{Z_{1i} = z_1\} \mathbb{I}\{Z_{2i} = z_2\} Y_i}{\hat{e}(z_1, X_{0i}) \hat{e}(z_2, z_1, X_{1i}, X_{0i})}
        \]
    \end{enumerate}
\end{itemize}



\subsection*{Limitations and Solutions}

\subsubsection*{Limitations}
For $K$ time points, we have $2^K$ treatment combinations. This approach may not work well for settings with more than a few time points.

\subsubsection*{Solution: Marginal Structural Model (MSM)}
References:
\begin{itemize}
    \item Robins 2000
    \item Hernán 2000
\end{itemize}

\subsubsection*{Detour:}
Point treatment, continuous, or multivariate/multi-dimensional treatments.

\subsubsection*{Examples:}
\begin{enumerate}
    \item \textbf{Dose:} Drug dose, e.g., mg of aspirin over time.
    \item \textbf{Duration:} e.g., effect of education duration on wages.
    \begin{itemize}
        \item Multidimensional: also degree level (undergrad, master’s, etc.).
    \end{itemize}
    \item \textbf{Frequency:} e.g., estimate effects of hospital nurse staffing on readmission outcomes.
    \begin{itemize}
        \item Nurse staffing = nurse hours/day.
    \end{itemize}
\end{enumerate}
                                    
\subsubsection{Identification:}
$\mathbb{E}[Y(Z=z)] = \mathbb{E}[\mathbb{E}[Y|X,Z=z]]$

\subsubsection{MSM Estimation:}
\begin{itemize}
    \item Assume a model:
    \[
    \mathbb{E}[Y(Z)] = g(z, \beta)
    \]
    where $g(z, \beta)$ is a known function up to a finite-dimensional parameter $\beta$.
    
    \item Example (Semiparametric model assumption):
    \[
    \mathbb{E}[Y(Z)] = \beta_0 + \beta_1 z + \beta_2 z^2
    \]
\end{itemize}

\subsection*{Semiparametric Approach and IPW Estimator under MSM}

\subsubsection*{Definition}
\[
\beta = \arg\min_\beta \int \left(\mathbb{E}[Y(Z)] - g(Z; \beta)\right)^2 P(Z) dZ
\]

\subsubsection*{Back to the Time-Varying Setting (2 Time Points)}
\paragraph*{Semiparametric Approach}
Assume:
\[
\mathbb{E}[Y(Z_1, Z_2)] = g(Z_1, Z_2; \beta)
\]

\paragraph*{Best Approximation Approach}
\[
\beta = \arg\min_\beta \sum_{z_1} \sum_{z_2} \left(\mathbb{E}[Y(Z_1, Z_2)] - g(Z_1, Z_2, X_0; \beta)\right)^2
\]

\subsubsection*{IPW under MSM with Sequential Ignorability}
\[
\beta = \arg\min_\beta \sum_{z_1} \sum_{z_2} \mathbb{E} \left[\frac{\mathbb{I}\{Z_1 = z_1\} \mathbb{I}\{Z_2 = z_2\}}{e(z_1, X_0) e(z_2, z_1, X_1, X_0)} \left(Y - g(Z_1, Z_2, X_0; \beta)\right)^2 \right]
\]

\subsubsection*{IPW Estimator under MSM}
\begin{enumerate}
    \item Estimate two propensity scores (same as IPW steps described above):
    \begin{itemize}
        \item $e(z_1, X_0) = P(Z_1 = z_1 \mid X_0)$
        \item $e(z_2, z_1, X_1, X_0) = P(Z_2 = z_2 \mid Z_1 = z_1, X_1, X_0)$
    \end{itemize}
    \item $\hat{\beta}$ is the coefficient of a Weighted Least Squares (WLS) fit of:
    \[
    Y_i \sim (1, Z_{1i}, Z_{2i}, X_{0i})
    \]
    with weights:
    \[
    w_i = \frac{1}{\hat{e}(Z_{1i}, X_{0i}) \hat{e}(Z_{2i}, Z_{1i}, X_{1i}, X_{0i})}.
    \]
\end{enumerate}























































































\bibliography{ref}

\end{document}