\documentclass[11pt]{elegantbook}
\definecolor{structurecolor}{RGB}{40,58,129}
\linespread{1.6}
\setlength{\footskip}{20pt}
\setlength{\parindent}{0pt}
\newcommand{\argmax}{\operatornamewithlimits{argmax}}
\newcommand{\argmin}{\operatornamewithlimits{argmin}}
\elegantnewtheorem{proof}{Proof}{}{Proof}
\elegantnewtheorem{claim}{Claim}{prostyle}{Claim}
\DeclareMathOperator{\col}{col}
\title{\textbf{Unsupervised Learning}}
\author{Wenxiao Yang}
\institute{Department of Mathematics, University of Illinois at Urbana-Champaign}
\date{}
\setcounter{tocdepth}{2}
\cover{cover.jpg}
\extrainfo{All models are wrong, but some are useful.}

% modify the color in the middle of titlepage
\definecolor{customcolor}{RGB}{32,178,170}
\colorlet{coverlinecolor}{customcolor}
\usepackage{cprotect}

\addbibresource[location=local]{reference.bib} % bib

\begin{document}

\maketitle
\frontmatter
\tableofcontents
\mainmatter

\chapter{Clustering}
General Goal of \textbf{Clustering Algorithm}:
\begin{enumerate}[$\circ$]
    \item the "similarity" of the objects in the same cluster is \underline{maximized} while
    \item the "similarity" of objects in different clusters is \underline{minimized}.
\end{enumerate}

\begin{definition}
    For a given set of objects $V = \{x_1, x_2, ... , x_m\}$, we call a \textbf{cluster $\mathbf{S_k}$} a subset of these objects, and we call a \textbf{clustering} the set of all $K$ clusters $\mathbf{\{S_1 ,S_2 , ... , S_K\}}$.
\end{definition}













\end{document}